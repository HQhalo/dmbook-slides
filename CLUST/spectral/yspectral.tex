\lecture{spectral}{spectral}

\date{Chapter 16: Spectral \& Graph Clustering}

\begin{frame}
\titlepage
\end{frame}


\begin{frame}{Graphs and Matrices: Adjacency Matrix}
Given a dataset $\bD = \{\bx_i \}_{i=1}^n$ consisting of $n$ points in $\setR^d$, let $\bA$
denote the $n \times n$ symmetric {\em similarity matrix} between
the points, given as
\begin{align*}
    \bA = \matr{
        a_{11} & a_{12} & \cdots & a_{1n}\\
        a_{21} & a_{22} & \cdots & a_{2n}\\
        \vdots & \vdots & \cdots & \vdots\\
        a_{n1} & a_{n2} & \cdots & a_{nn}\\
    }
\end{align*}
where $\bA(i,j) = a_{ij}$ denotes the similarity or aff\/{i}nity
between points $\bx_i$ and $\bx_{j}$. 

\medskip
We require the similarity to
be symmetric and non-negative, that is, $a_{ij} = a_{ji}$ and $a_{ij}
\ge 0$, respectively. 

\medskip
The matrix $\bA$ is 
the {\em weighted adjacency matrix} for the data graph.
If all aff\/{i}nities are 0 or 1, then $\bA$ represents
the regular adjacency relationship between the vertices.
\end{frame}


\begin{frame}{Iris Similarity Graph: Mutual Nearest Neighbors}
  \framesubtitle{$|V|=n=150$, $|E|=m=1730$}
\begin{figure}
    \centerline{
        \scalebox{0.75}{
        \psset{unit=0.75in,dotscale=2}
        \begin{pspicture}(5,5)
            \pnode(1.108338,3.858886){n0}
\pnode(2.527444,2.445348){n1}
\pnode(2.017666,2.841472){n2}
\pnode(0.6785566,0.8802483){n3}
\pnode(1.257860,2.892203){n4}
\pnode(0.8247274,0.631479){n5}
\pnode(4.581229,3.210751){n6}
\pnode(3.098514,4.866756){n7}
\pnode(4.591683,2.878327){n8}
\pnode(4,2.2){n9}
\pnode(2.888154,1.090116){n10}
\pnode(3.105909,0.7401545){n11}
\pnode(0.3962556,3.984627){n12}
\pnode(2.661068,4.46547){n13}
\pnode(1.439636,1.295117){n14}
\pnode(2.236885,1.274193){n15}
\pnode(2.024484,0.7152085){n16}
\pnode(0.6005647,3.371056){n17}
\pnode(1.814559,0.4981684){n18}
\pnode(4.08826,1.970573){n19}
\pnode(2.577236,4.740179){n20}
\pnode(3.773692,3.552994){n21}
\pnode(3.913548,0.4282107){n22}
\pnode(0.711105,2.899006){n23}
\pnode(1.232538,4.29378){n24}
\pnode(2.521529,1.412281){n25}
\pnode(0.9249148,2.893182){n26}
\pnode(0.5726195,1.294175){n27}
\pnode(3.759983,2.172351){n28}
\pnode(4.7622,2.335772){n29}
\pnode(3.665883,0.6603324){n30}
\pnode(1.422528,0.3128956){n31}
\pnode(2.557950,5){n32}
\pnode(0.1556468,2.998522){n33}
\pnode(1.403207,1.916558){n34}
\pnode(0.3174471,2.600607){n35}
\pnode(0.1307679,3.277233){n36}
\pnode(2.901075,1.575363){n37}
\pnode(2.55,3.97){n38}
\pnode(3.353207,4.403764){n39}
\pnode(0.9114814,4.353879){n40}
\pnode(1.686579,3.819467){n41}
\pnode(4.209552,3.016675){n42}
\pnode(3.891111,1.748183){n43}
\pnode(0.4179513,3.522051){n44}
\pnode(0.8306748,1.037910){n45}
\pnode(4.389942,2.082443){n46}
\pnode(0.1748816,2.281055){n47}
\pnode(3.273419,0.6181322){n48}
\pnode(3.66405,0.2172084){n49}
\pnode(0.5841016,3.811024){n50}
\pnode(2.24852,4.942186){n51}
\pnode(0.2341176,1.037886){n52}
\pnode(4.969442,3.079323){n53}
\pnode(2.864019,0){n54}
\pnode(5,2.809783){n55}
\pnode(3.97574,2.571348){n56}
\pnode(4.371287,1.534671){n57}
\pnode(1.967572,3.868347){n58}
\pnode(0,2.563469){n59}
\pnode(3.764707,4.352432){n60}
\pnode(3.737104,3.981957){n61}
\pnode(2.127025,0.3401537){n62}
\pnode(2.312403,1.030709){n63}
\pnode(0.2217693,3.701035){n64}
\pnode(0.6466549,4.297944){n65}
\pnode(3.364475,4.745668){n66}
\pnode(1.509306,0.9905115){n67}
\pnode(1.631241,0.08123197){n68}
\pnode(2.363231,0.03982867){n69}
\pnode(4.688611,1.662437){n70}
\pnode(4.32925,1.149353){n71}
\pnode(1.004251,3.3599){n72}
\pnode(3.170304,0.08031329){n73}
\pnode(1.092846,0.4743804){n74}
\pnode(0.957797,1.457980){n75}
\pnode(0.4145735,1.587721){n76}
\pnode(2.550006,4.241430){n77}
\pnode(0.8983155,1.243353){n78}
\pnode(1.560778,4.1743){n79}
\pnode(4.791386,3.470398){n80}
\pnode(1.183687,2.360412){n81}
\pnode(2.078577,3.201014){n82}
\pnode(4.135811,4.087185){n83}
\pnode(3.153883,3.140639){n84}
\pnode(0.6058247,2.060042){n85}
\pnode(1.891242,4.717356){n86}
\pnode(4.792091,2.606388){n87}
\pnode(3.635776,1.016917){n88}
\pnode(3.202631,4.133673){n89}
\pnode(1.765180,1.228386){n90}
\pnode(0.4301581,0.7624465){n91}
\pnode(4.292557,3.369169){n92}
\pnode(2.64,3.705){n93}
\pnode(2.869689,4.205452){n94}
\pnode(4.306428,1.798957){n95}
\pnode(0.9419618,2.164323){n96}
\pnode(2.802984,2.471939){n97}
\pnode(0.4990411,2.364772){n98}
\pnode(2.265986,3.781169){n99}
\pnode(2.886530,2.782621){n100}
\pnode(2.074548,4.120629){n101}
\pnode(3.578431,3.339056){n102}
\pnode(4.396008,2.731498){n103}
\pnode(2.161968,2.481101){n104}
\pnode(1.915954,1.686363){n105}
\pnode(1.073082,0.7635667){n106}
\pnode(4.351759,2.410157){n107}
\pnode(0.1632209,1.576404){n108}
\pnode(0.1964432,1.278737){n109}
\pnode(0.5000101,2.853756){n110}
\pnode(2.625767,0.8297721){n111}
\pnode(1.705716,3.114971){n112}
\pnode(2.796314,0.5814426){n113}
\pnode(0.3549037,3.121353){n114}
\pnode(3.949152,0.8549694){n115}
\pnode(4.562201,1.223617){n116}
\pnode(3.053188,3.925351){n117}
\pnode(4.318411,3.852927){n118}
\pnode(2.890681,3.735371){n119}
\pnode(2.880736,4.879838){n120}
\pnode(1.377059,0.7879707){n121}
\pnode(2.617372,2.90596){n122}
\pnode(3.046366,4.518936){n123}
\pnode(2.621703,0.4200536){n124}
\pnode(1.203850,1.499658){n125}
\pnode(0.9849086,4.079775){n126}
\pnode(3.254325,1.04113){n127}
\pnode(2.280932,2.838994){n128}
\pnode(4.117188,3.717776){n129}
\pnode(3.471599,1.285363){n130}
\pnode(4.500053,3.637197){n131}
\pnode(4.069486,1.411759){n132}
\pnode(4.11247,0.5910543){n133}
\pnode(1.252724,1.071603){n134}
\pnode(3.385955,0.3694225){n135}
\pnode(2.407195,0.4605674){n136}
\pnode(1.689756,4.453583){n137}
\pnode(3.989081,3.406712){n138}
\pnode(0.710692,2.532491){n139}
\pnode(4.873979,1.934726){n140}
\pnode(3.645946,2.645275){n141}
\pnode(1.606368,3.440222){n142}
\pnode(2.445606,3.296266){n143}
\pnode(2.274105,4.535248){n144}
\pnode(2.648264,0.1809270){n145}
\pnode(0.7976457,3.6483){n146}
\pnode(3.950723,2.918283){n147}
\pnode(4.776492,1.394465){n148}
\pnode(2.296867,0.7074797){n149}
\psset{linewidth=0.5pt,dotsep=2pt}
\ncline[linecolor=lightgray]{n4}{n93}
\ncline[linecolor=lightgray]{n9}{n46}
\ncline[linecolor=lightgray]{n9}{n28}
\ncline[linecolor=lightgray]{n9}{n60}
\ncline[linecolor=lightgray]{n9}{n95}
\ncline[linecolor=lightgray]{n19}{n57}
\ncline[linecolor=lightgray]{n19}{n116}
\ncline[linecolor=lightgray]{n19}{n140}
\ncline[linecolor=lightgray]{n9}{n28}
\ncline[linecolor=lightgray]{n28}{n57}
\ncline[linecolor=lightgray]{n34}{n85}
\ncline[linecolor=lightgray]{n43}{n71}
\ncline[linecolor=lightgray]{n43}{n140}
\ncline[linecolor=lightgray]{n9}{n46}
\ncline[linecolor=lightgray]{n46}{n140}
\ncline[linecolor=lightgray]{n19}{n57}
\ncline[linecolor=lightgray]{n57}{n140}
\ncline[linecolor=lightgray]{n57}{n71}
\ncline[linecolor=lightgray]{n28}{n57}
\ncline[linecolor=lightgray]{n57}{n70}
\ncline[linecolor=lightgray]{n57}{n148}
\ncline[linecolor=lightgray]{n60}{n94}
\ncline[linecolor=lightgray]{n9}{n60}
\ncline[linecolor=lightgray]{n65}{n137}
\ncline[linecolor=lightgray]{n70}{n116}
\ncline[linecolor=lightgray]{n57}{n70}
\ncline[linecolor=lightgray]{n70}{n140}
\ncline[linecolor=lightgray]{n70}{n132}
\ncline[linecolor=lightgray]{n70}{n71}
\ncline[linecolor=lightgray]{n57}{n71}
\ncline[linecolor=lightgray]{n43}{n71}
\ncline[linecolor=lightgray]{n71}{n95}
\ncline[linecolor=lightgray]{n71}{n148}
\ncline[linecolor=lightgray]{n70}{n71}
\ncline[linecolor=lightgray]{n34}{n85}
\ncline[linecolor=lightgray]{n4}{n93}
\ncline[linecolor=lightgray]{n60}{n94}
\ncline[linecolor=lightgray]{n9}{n95}
\ncline[linecolor=lightgray]{n71}{n95}
\ncline[linecolor=lightgray]{n19}{n116}
\ncline[linecolor=lightgray]{n116}{n148}
\ncline[linecolor=lightgray]{n70}{n116}
\ncline[linecolor=lightgray]{n116}{n140}
\ncline[linecolor=lightgray]{n132}{n148}
\ncline[linecolor=lightgray]{n70}{n132}
\ncline[linecolor=lightgray]{n65}{n137}
\ncline[linecolor=lightgray]{n46}{n140}
\ncline[linecolor=lightgray]{n57}{n140}
\ncline[linecolor=lightgray]{n19}{n140}
\ncline[linecolor=lightgray]{n140}{n148}
\ncline[linecolor=lightgray]{n70}{n140}
\ncline[linecolor=lightgray]{n116}{n140}
\ncline[linecolor=lightgray]{n43}{n140}
\ncline[linecolor=lightgray]{n116}{n148}
\ncline[linecolor=lightgray]{n140}{n148}
\ncline[linecolor=lightgray]{n57}{n148}
\ncline[linecolor=lightgray]{n132}{n148}
\ncline[linecolor=lightgray]{n71}{n148}
\ncline[linecolor=lightgray]{n14}{n34}
\ncline[linecolor=lightgray]{n15}{n34}
\ncline[linecolor=lightgray]{n25}{n34}
\ncline[linecolor=lightgray]{n34}{n37}
\ncline[linecolor=lightgray]{n34}{n54}
\ncline[linecolor=lightgray]{n34}{n63}
\ncline[linecolor=lightgray]{n34}{n67}
\ncline[linecolor=lightgray]{n34}{n75}
\ncline[linecolor=lightgray]{n34}{n78}
\ncline[linecolor=lightgray]{n34}{n105}
\ncline[linecolor=lightgray]{n34}{n106}
\ncline[linecolor=lightgray]{n34}{n121}
\ncline[linecolor=lightgray]{n34}{n125}
\ncline[linecolor=lightgray]{n34}{n134}
\ncline[linecolor=lightgray]{n34}{n145}
\ncline[linecolor=lightgray]{n63}{n96}
\ncline[linecolor=lightgray]{n96}{n105}
\ncline[linecolor=lightgray]{n12}{n24}
\ncline[linecolor=lightgray]{n17}{n24}
\ncline[linecolor=lightgray]{n23}{n24}
\ncline[linecolor=lightgray]{n24}{n26}
\ncline[linecolor=lightgray]{n24}{n32}
\ncline[linecolor=lightgray]{n24}{n35}
\ncline[linecolor=lightgray]{n24}{n40}
\ncline[linecolor=lightgray]{n24}{n44}
\ncline[linecolor=lightgray]{n24}{n50}
\ncline[linecolor=lightgray]{n24}{n51}
\ncline[linecolor=lightgray]{n24}{n64}
\ncline[linecolor=lightgray]{n24}{n65}
\ncline[linecolor=lightgray]{n24}{n110}
\ncline[linecolor=lightgray]{n24}{n126}
\ncline[linecolor=lightgray]{n24}{n139}
\ncline[linecolor=lightgray]{n24}{n146}
\ncline[linecolor=lightgray]{n4}{n108}
\ncline[linecolor=lightgray]{n23}{n108}
\ncline[linecolor=lightgray]{n26}{n108}
\ncline[linecolor=lightgray]{n33}{n108}
\ncline[linecolor=lightgray]{n34}{n108}
\ncline[linecolor=lightgray]{n35}{n108}
\ncline[linecolor=lightgray]{n36}{n108}
\ncline[linecolor=lightgray]{n47}{n108}
\ncline[linecolor=lightgray]{n59}{n108}
\ncline[linecolor=lightgray]{n81}{n108}
\ncline[linecolor=lightgray]{n85}{n108}
\ncline[linecolor=lightgray]{n96}{n108}
\ncline[linecolor=lightgray]{n98}{n108}
\ncline[linecolor=lightgray]{n108}{n110}
\ncline[linecolor=lightgray]{n108}{n114}
\ncline[linecolor=lightgray]{n108}{n139}
\ncline[linecolor=lightgray]{n14}{n24}
\ncline[linecolor=lightgray]{n15}{n24}
\ncline[linecolor=lightgray]{n24}{n25}
\ncline[linecolor=lightgray]{n24}{n27}
\ncline[linecolor=lightgray]{n24}{n37}
\ncline[linecolor=lightgray]{n24}{n54}
\ncline[linecolor=lightgray]{n24}{n63}
\ncline[linecolor=lightgray]{n24}{n67}
\ncline[linecolor=lightgray]{n24}{n75}
\ncline[linecolor=lightgray]{n24}{n78}
\ncline[linecolor=lightgray]{n24}{n88}
\ncline[linecolor=lightgray]{n24}{n90}
\ncline[linecolor=lightgray]{n24}{n105}
\ncline[linecolor=lightgray]{n24}{n121}
\ncline[linecolor=lightgray]{n24}{n125}
\ncline[linecolor=lightgray]{n24}{n134}
\ncline[linecolor=lightgray]{n24}{n145}
\ncline[linecolor=lightgray]{n3}{n108}
\ncline[linecolor=lightgray]{n5}{n108}
\ncline[linecolor=lightgray]{n14}{n108}
\ncline[linecolor=lightgray]{n27}{n108}
\ncline[linecolor=lightgray]{n45}{n108}
\ncline[linecolor=lightgray]{n52}{n108}
\ncline[linecolor=lightgray]{n67}{n108}
\ncline[linecolor=lightgray]{n75}{n108}
\ncline[linecolor=lightgray]{n78}{n108}
\ncline[linecolor=lightgray]{n91}{n108}
\ncline[linecolor=lightgray]{n106}{n108}
\ncline[linecolor=lightgray]{n108}{n109}
\ncline[linecolor=lightgray]{n108}{n121}
\ncline[linecolor=lightgray]{n108}{n125}
\ncline[linecolor=lightgray]{n108}{n134}
\ncline[linecolor=lightgray]{n24}{n108}
\ncline[linecolor=darkgray]{n0}{n12}
\ncline[linecolor=darkgray]{n0}{n17}
\ncline[linecolor=darkgray]{n0}{n64}
\ncline[linecolor=darkgray]{n0}{n142}
\ncline[linecolor=darkgray]{n0}{n72}
\ncline[linecolor=darkgray]{n0}{n40}
\ncline[linecolor=darkgray]{n0}{n41}
\ncline[linecolor=darkgray]{n0}{n126}
\ncline[linecolor=darkgray]{n0}{n44}
\ncline[linecolor=darkgray]{n0}{n137}
\ncline[linecolor=darkgray]{n0}{n112}
\ncline[linecolor=darkgray]{n1}{n2}
\ncline[linecolor=darkgray]{n1}{n104}
\ncline[linecolor=darkgray]{n1}{n128}
\ncline[linecolor=darkgray]{n2}{n112}
\ncline[linecolor=darkgray]{n2}{n82}
\ncline[linecolor=darkgray]{n2}{n142}
\ncline[linecolor=darkgray]{n1}{n2}
\ncline[linecolor=darkgray]{n2}{n41}
\ncline[linecolor=darkgray]{n2}{n97}
\ncline[linecolor=darkgray]{n2}{n58}
\ncline[linecolor=darkgray]{n2}{n84}
\ncline[linecolor=darkgray]{n2}{n99}
\ncline[linecolor=darkgray]{n3}{n76}
\ncline[linecolor=darkgray]{n3}{n106}
\ncline[linecolor=darkgray]{n4}{n33}
\ncline[linecolor=darkgray]{n4}{n35}
\ncline[linecolor=darkgray]{n4}{n36}
\ncline[linecolor=darkgray]{n4}{n139}
\ncline[linecolor=darkgray]{n4}{n26}
\ncline[linecolor=darkgray]{n4}{n38}
\ncline[linecolor=darkgray]{n4}{n98}
\ncline[linecolor=darkgray]{n4}{n47}
\ncline[linecolor=darkgray]{n5}{n52}
\ncline[linecolor=darkgray]{n6}{n8}
\ncline[linecolor=darkgray]{n6}{n103}
\ncline[linecolor=darkgray]{n6}{n29}
\ncline[linecolor=darkgray]{n6}{n42}
\ncline[linecolor=darkgray]{n6}{n107}
\ncline[linecolor=darkgray]{n6}{n87}
\ncline[linecolor=darkgray]{n6}{n83}
\ncline[linecolor=darkgray]{n6}{n138}
\ncline[linecolor=darkgray]{n6}{n53}
\ncline[linecolor=darkgray]{n6}{n118}
\ncline[linecolor=darkgray]{n6}{n55}
\ncline[linecolor=darkgray]{n6}{n92}
\ncline[linecolor=darkgray]{n6}{n56}
\ncline[linecolor=darkgray]{n7}{n123}
\ncline[linecolor=darkgray]{n7}{n13}
\ncline[linecolor=darkgray]{n7}{n144}
\ncline[linecolor=darkgray]{n7}{n66}
\ncline[linecolor=darkgray]{n7}{n77}
\ncline[linecolor=darkgray]{n7}{n39}
\ncline[linecolor=darkgray]{n7}{n89}
\ncline[linecolor=darkgray]{n7}{n86}
\ncline[linecolor=darkgray]{n7}{n94}
\ncline[linecolor=darkgray]{n6}{n8}
\ncline[linecolor=darkgray]{n8}{n103}
\ncline[linecolor=darkgray]{n8}{n56}
\ncline[linecolor=darkgray]{n8}{n147}
\ncline[linecolor=darkgray]{n8}{n92}
\ncline[linecolor=darkgray]{n8}{n131}
\ncline[linecolor=darkgray]{n8}{n55}
\ncline[linecolor=darkgray]{n8}{n129}
\ncline[linecolor=darkgray]{n8}{n141}
\ncline[linecolor=darkgray]{n8}{n46}
\ncline[linecolor=darkgray]{n8}{n53}
\ncline[linecolor=darkgray]{n9}{n129}
\ncline[linecolor=darkgray]{n9}{n83}
\ncline[linecolor=darkgray]{n9}{n131}
\ncline[linecolor=darkgray]{n10}{n149}
\ncline[linecolor=darkgray]{n10}{n135}
\ncline[linecolor=darkgray]{n10}{n37}
\ncline[linecolor=darkgray]{n11}{n62}
\ncline[linecolor=darkgray]{n11}{n130}
\ncline[linecolor=darkgray]{n11}{n30}
\ncline[linecolor=darkgray]{n11}{n88}
\ncline[linecolor=darkgray]{n11}{n37}
\ncline[linecolor=darkgray]{n0}{n12}
\ncline[linecolor=darkgray]{n12}{n126}
\ncline[linecolor=darkgray]{n12}{n36}
\ncline[linecolor=darkgray]{n12}{n40}
\ncline[linecolor=darkgray]{n12}{n114}
\ncline[linecolor=darkgray]{n12}{n33}
\ncline[linecolor=darkgray]{n12}{n146}
\ncline[linecolor=darkgray]{n12}{n72}
\ncline[linecolor=darkgray]{n12}{n65}
\ncline[linecolor=darkgray]{n12}{n142}
\ncline[linecolor=darkgray]{n13}{n89}
\ncline[linecolor=darkgray]{n13}{n123}
\ncline[linecolor=darkgray]{n13}{n99}
\ncline[linecolor=darkgray]{n13}{n117}
\ncline[linecolor=darkgray]{n13}{n101}
\ncline[linecolor=darkgray]{n7}{n13}
\ncline[linecolor=darkgray]{n13}{n120}
\ncline[linecolor=darkgray]{n13}{n41}
\ncline[linecolor=darkgray]{n13}{n79}
\ncline[linecolor=darkgray]{n13}{n102}
\ncline[linecolor=darkgray]{n13}{n21}
\ncline[linecolor=darkgray]{n15}{n105}
\ncline[linecolor=darkgray]{n15}{n125}
\ncline[linecolor=darkgray]{n16}{n31}
\ncline[linecolor=darkgray]{n16}{n75}
\ncline[linecolor=darkgray]{n0}{n17}
\ncline[linecolor=darkgray]{n17}{n23}
\ncline[linecolor=darkgray]{n17}{n126}
\ncline[linecolor=darkgray]{n17}{n72}
\ncline[linecolor=darkgray]{n17}{n35}
\ncline[linecolor=darkgray]{n17}{n146}
\ncline[linecolor=darkgray]{n17}{n40}
\ncline[linecolor=darkgray]{n17}{n59}
\ncline[linecolor=darkgray]{n19}{n46}
\ncline[linecolor=darkgray]{n19}{n43}
\ncline[linecolor=darkgray]{n19}{n95}
\ncline[linecolor=darkgray]{n19}{n29}
\ncline[linecolor=darkgray]{n19}{n107}
\ncline[linecolor=darkgray]{n20}{n86}
\ncline[linecolor=darkgray]{n20}{n123}
\ncline[linecolor=darkgray]{n20}{n51}
\ncline[linecolor=darkgray]{n20}{n77}
\ncline[linecolor=darkgray]{n20}{n89}
\ncline[linecolor=darkgray]{n20}{n117}
\ncline[linecolor=darkgray]{n20}{n32}
\ncline[linecolor=darkgray]{n20}{n66}
\ncline[linecolor=darkgray]{n21}{n138}
\ncline[linecolor=darkgray]{n21}{n147}
\ncline[linecolor=darkgray]{n21}{n92}
\ncline[linecolor=darkgray]{n21}{n118}
\ncline[linecolor=darkgray]{n21}{n84}
\ncline[linecolor=darkgray]{n21}{n42}
\ncline[linecolor=darkgray]{n21}{n129}
\ncline[linecolor=darkgray]{n21}{n56}
\ncline[linecolor=darkgray]{n21}{n83}
\ncline[linecolor=darkgray]{n21}{n131}
\ncline[linecolor=darkgray]{n21}{n89}
\ncline[linecolor=darkgray]{n21}{n141}
\ncline[linecolor=darkgray]{n13}{n21}
\ncline[linecolor=darkgray]{n22}{n115}
\ncline[linecolor=darkgray]{n22}{n133}
\ncline[linecolor=darkgray]{n22}{n49}
\ncline[linecolor=darkgray]{n22}{n130}
\ncline[linecolor=darkgray]{n17}{n23}
\ncline[linecolor=darkgray]{n23}{n110}
\ncline[linecolor=darkgray]{n23}{n114}
\ncline[linecolor=darkgray]{n23}{n146}
\ncline[linecolor=darkgray]{n23}{n98}
\ncline[linecolor=darkgray]{n23}{n36}
\ncline[linecolor=darkgray]{n23}{n50}
\ncline[linecolor=darkgray]{n23}{n64}
\ncline[linecolor=darkgray]{n23}{n59}
\ncline[linecolor=darkgray]{n25}{n136}
\ncline[linecolor=darkgray]{n25}{n62}
\ncline[linecolor=darkgray]{n25}{n54}
\ncline[linecolor=darkgray]{n25}{n37}
\ncline[linecolor=darkgray]{n26}{n33}
\ncline[linecolor=darkgray]{n26}{n98}
\ncline[linecolor=darkgray]{n26}{n44}
\ncline[linecolor=darkgray]{n26}{n146}
\ncline[linecolor=darkgray]{n26}{n110}
\ncline[linecolor=darkgray]{n26}{n114}
\ncline[linecolor=darkgray]{n4}{n26}
\ncline[linecolor=darkgray]{n26}{n47}
\ncline[linecolor=darkgray]{n27}{n90}
\ncline[linecolor=darkgray]{n28}{n95}
\ncline[linecolor=darkgray]{n28}{n46}
\ncline[linecolor=darkgray]{n28}{n43}
\ncline[linecolor=darkgray]{n6}{n29}
\ncline[linecolor=darkgray]{n29}{n46}
\ncline[linecolor=darkgray]{n29}{n42}
\ncline[linecolor=darkgray]{n29}{n53}
\ncline[linecolor=darkgray]{n29}{n56}
\ncline[linecolor=darkgray]{n29}{n55}
\ncline[linecolor=darkgray]{n29}{n140}
\ncline[linecolor=darkgray]{n19}{n29}
\ncline[linecolor=darkgray]{n30}{n48}
\ncline[linecolor=darkgray]{n11}{n30}
\ncline[linecolor=darkgray]{n30}{n115}
\ncline[linecolor=darkgray]{n30}{n88}
\ncline[linecolor=darkgray]{n30}{n73}
\ncline[linecolor=darkgray]{n30}{n130}
\ncline[linecolor=darkgray]{n30}{n133}
\ncline[linecolor=darkgray]{n30}{n49}
\ncline[linecolor=darkgray]{n16}{n31}
\ncline[linecolor=darkgray]{n31}{n45}
\ncline[linecolor=darkgray]{n31}{n78}
\ncline[linecolor=darkgray]{n31}{n124}
\ncline[linecolor=darkgray]{n31}{n149}
\ncline[linecolor=darkgray]{n32}{n66}
\ncline[linecolor=darkgray]{n32}{n123}
\ncline[linecolor=darkgray]{n20}{n32}
\ncline[linecolor=darkgray]{n32}{n39}
\ncline[linecolor=darkgray]{n32}{n94}
\ncline[linecolor=darkgray]{n33}{n72}
\ncline[linecolor=darkgray]{n33}{n50}
\ncline[linecolor=darkgray]{n33}{n139}
\ncline[linecolor=darkgray]{n26}{n33}
\ncline[linecolor=darkgray]{n12}{n33}
\ncline[linecolor=darkgray]{n33}{n64}
\ncline[linecolor=darkgray]{n4}{n33}
\ncline[linecolor=darkgray]{n33}{n146}
\ncline[linecolor=darkgray]{n33}{n98}
\ncline[linecolor=darkgray]{n34}{n81}
\ncline[linecolor=darkgray]{n34}{n96}
\ncline[linecolor=darkgray]{n35}{n36}
\ncline[linecolor=darkgray]{n35}{n44}
\ncline[linecolor=darkgray]{n17}{n35}
\ncline[linecolor=darkgray]{n35}{n47}
\ncline[linecolor=darkgray]{n4}{n35}
\ncline[linecolor=darkgray]{n35}{n146}
\ncline[linecolor=darkgray]{n35}{n59}
\ncline[linecolor=darkgray]{n35}{n110}
\ncline[linecolor=darkgray]{n35}{n36}
\ncline[linecolor=darkgray]{n36}{n44}
\ncline[linecolor=darkgray]{n12}{n36}
\ncline[linecolor=darkgray]{n36}{n50}
\ncline[linecolor=darkgray]{n36}{n72}
\ncline[linecolor=darkgray]{n36}{n64}
\ncline[linecolor=darkgray]{n23}{n36}
\ncline[linecolor=darkgray]{n36}{n146}
\ncline[linecolor=darkgray]{n36}{n126}
\ncline[linecolor=darkgray]{n36}{n139}
\ncline[linecolor=darkgray]{n4}{n36}
\ncline[linecolor=darkgray]{n37}{n127}
\ncline[linecolor=darkgray]{n37}{n88}
\ncline[linecolor=darkgray]{n10}{n37}
\ncline[linecolor=darkgray]{n11}{n37}
\ncline[linecolor=darkgray]{n25}{n37}
\ncline[linecolor=darkgray]{n37}{n105}
\ncline[linecolor=darkgray]{n37}{n130}
\ncline[linecolor=darkgray]{n38}{n119}
\ncline[linecolor=darkgray]{n4}{n38}
\ncline[linecolor=darkgray]{n38}{n112}
\ncline[linecolor=darkgray]{n38}{n58}
\ncline[linecolor=darkgray]{n38}{n94}
\ncline[linecolor=darkgray]{n39}{n61}
\ncline[linecolor=darkgray]{n39}{n77}
\ncline[linecolor=darkgray]{n39}{n119}
\ncline[linecolor=darkgray]{n39}{n123}
\ncline[linecolor=darkgray]{n39}{n117}
\ncline[linecolor=darkgray]{n7}{n39}
\ncline[linecolor=darkgray]{n39}{n120}
\ncline[linecolor=darkgray]{n39}{n94}
\ncline[linecolor=darkgray]{n32}{n39}
\ncline[linecolor=darkgray]{n40}{n50}
\ncline[linecolor=darkgray]{n40}{n79}
\ncline[linecolor=darkgray]{n0}{n40}
\ncline[linecolor=darkgray]{n12}{n40}
\ncline[linecolor=darkgray]{n40}{n64}
\ncline[linecolor=darkgray]{n40}{n44}
\ncline[linecolor=darkgray]{n17}{n40}
\ncline[linecolor=darkgray]{n40}{n41}
\ncline[linecolor=darkgray]{n40}{n146}
\ncline[linecolor=darkgray]{n40}{n86}
\ncline[linecolor=darkgray]{n40}{n137}
\ncline[linecolor=darkgray]{n41}{n82}
\ncline[linecolor=darkgray]{n41}{n99}
\ncline[linecolor=darkgray]{n41}{n112}
\ncline[linecolor=darkgray]{n0}{n41}
\ncline[linecolor=darkgray]{n41}{n144}
\ncline[linecolor=darkgray]{n41}{n143}
\ncline[linecolor=darkgray]{n41}{n126}
\ncline[linecolor=darkgray]{n41}{n137}
\ncline[linecolor=darkgray]{n41}{n77}
\ncline[linecolor=darkgray]{n13}{n41}
\ncline[linecolor=darkgray]{n2}{n41}
\ncline[linecolor=darkgray]{n40}{n41}
\ncline[linecolor=darkgray]{n42}{n107}
\ncline[linecolor=darkgray]{n42}{n103}
\ncline[linecolor=darkgray]{n42}{n147}
\ncline[linecolor=darkgray]{n42}{n138}
\ncline[linecolor=darkgray]{n6}{n42}
\ncline[linecolor=darkgray]{n21}{n42}
\ncline[linecolor=darkgray]{n42}{n129}
\ncline[linecolor=darkgray]{n42}{n131}
\ncline[linecolor=darkgray]{n42}{n141}
\ncline[linecolor=darkgray]{n29}{n42}
\ncline[linecolor=darkgray]{n42}{n46}
\ncline[linecolor=darkgray]{n42}{n118}
\ncline[linecolor=darkgray]{n42}{n92}
\ncline[linecolor=darkgray]{n19}{n43}
\ncline[linecolor=darkgray]{n43}{n116}
\ncline[linecolor=darkgray]{n43}{n46}
\ncline[linecolor=darkgray]{n28}{n43}
\ncline[linecolor=darkgray]{n43}{n132}
\ncline[linecolor=darkgray]{n43}{n57}
\ncline[linecolor=darkgray]{n36}{n44}
\ncline[linecolor=darkgray]{n44}{n114}
\ncline[linecolor=darkgray]{n35}{n44}
\ncline[linecolor=darkgray]{n26}{n44}
\ncline[linecolor=darkgray]{n40}{n44}
\ncline[linecolor=darkgray]{n0}{n44}
\ncline[linecolor=darkgray]{n44}{n65}
\ncline[linecolor=darkgray]{n44}{n110}
\ncline[linecolor=darkgray]{n31}{n45}
\ncline[linecolor=darkgray]{n45}{n52}
\ncline[linecolor=darkgray]{n45}{n76}
\ncline[linecolor=darkgray]{n19}{n46}
\ncline[linecolor=darkgray]{n46}{n107}
\ncline[linecolor=darkgray]{n29}{n46}
\ncline[linecolor=darkgray]{n46}{n95}
\ncline[linecolor=darkgray]{n42}{n46}
\ncline[linecolor=darkgray]{n28}{n46}
\ncline[linecolor=darkgray]{n46}{n103}
\ncline[linecolor=darkgray]{n43}{n46}
\ncline[linecolor=darkgray]{n46}{n56}
\ncline[linecolor=darkgray]{n8}{n46}
\ncline[linecolor=darkgray]{n46}{n57}
\ncline[linecolor=darkgray]{n46}{n87}
\ncline[linecolor=darkgray]{n47}{n139}
\ncline[linecolor=darkgray]{n35}{n47}
\ncline[linecolor=darkgray]{n47}{n59}
\ncline[linecolor=darkgray]{n47}{n114}
\ncline[linecolor=darkgray]{n26}{n47}
\ncline[linecolor=darkgray]{n4}{n47}
\ncline[linecolor=darkgray]{n30}{n48}
\ncline[linecolor=darkgray]{n48}{n88}
\ncline[linecolor=darkgray]{n48}{n149}
\ncline[linecolor=darkgray]{n48}{n54}
\ncline[linecolor=darkgray]{n48}{n145}
\ncline[linecolor=darkgray]{n49}{n135}
\ncline[linecolor=darkgray]{n49}{n145}
\ncline[linecolor=darkgray]{n30}{n49}
\ncline[linecolor=darkgray]{n49}{n73}
\ncline[linecolor=darkgray]{n49}{n115}
\ncline[linecolor=darkgray]{n49}{n133}
\ncline[linecolor=darkgray]{n22}{n49}
\ncline[linecolor=darkgray]{n49}{n130}
\ncline[linecolor=darkgray]{n33}{n50}
\ncline[linecolor=darkgray]{n36}{n50}
\ncline[linecolor=darkgray]{n40}{n50}
\ncline[linecolor=darkgray]{n50}{n114}
\ncline[linecolor=darkgray]{n50}{n146}
\ncline[linecolor=darkgray]{n50}{n72}
\ncline[linecolor=darkgray]{n50}{n79}
\ncline[linecolor=darkgray]{n50}{n65}
\ncline[linecolor=darkgray]{n23}{n50}
\ncline[linecolor=darkgray]{n20}{n51}
\ncline[linecolor=darkgray]{n51}{n66}
\ncline[linecolor=darkgray]{n51}{n144}
\ncline[linecolor=darkgray]{n51}{n86}
\ncline[linecolor=darkgray]{n51}{n65}
\ncline[linecolor=darkgray]{n5}{n52}
\ncline[linecolor=darkgray]{n45}{n52}
\ncline[linecolor=darkgray]{n52}{n78}
\ncline[linecolor=darkgray]{n52}{n74}
\ncline[linecolor=darkgray]{n53}{n55}
\ncline[linecolor=darkgray]{n53}{n87}
\ncline[linecolor=darkgray]{n6}{n53}
\ncline[linecolor=darkgray]{n29}{n53}
\ncline[linecolor=darkgray]{n8}{n53}
\ncline[linecolor=darkgray]{n53}{n80}
\ncline[linecolor=darkgray]{n54}{n135}
\ncline[linecolor=darkgray]{n54}{n149}
\ncline[linecolor=darkgray]{n48}{n54}
\ncline[linecolor=darkgray]{n54}{n63}
\ncline[linecolor=darkgray]{n25}{n54}
\ncline[linecolor=darkgray]{n55}{n92}
\ncline[linecolor=darkgray]{n53}{n55}
\ncline[linecolor=darkgray]{n55}{n87}
\ncline[linecolor=darkgray]{n8}{n55}
\ncline[linecolor=darkgray]{n55}{n103}
\ncline[linecolor=darkgray]{n6}{n55}
\ncline[linecolor=darkgray]{n29}{n55}
\ncline[linecolor=darkgray]{n55}{n131}
\ncline[linecolor=darkgray]{n55}{n107}
\ncline[linecolor=darkgray]{n56}{n141}
\ncline[linecolor=darkgray]{n56}{n147}
\ncline[linecolor=darkgray]{n56}{n107}
\ncline[linecolor=darkgray]{n8}{n56}
\ncline[linecolor=darkgray]{n56}{n103}
\ncline[linecolor=darkgray]{n21}{n56}
\ncline[linecolor=darkgray]{n56}{n138}
\ncline[linecolor=darkgray]{n56}{n102}
\ncline[linecolor=darkgray]{n46}{n56}
\ncline[linecolor=darkgray]{n29}{n56}
\ncline[linecolor=darkgray]{n56}{n92}
\ncline[linecolor=darkgray]{n56}{n87}
\ncline[linecolor=darkgray]{n6}{n56}
\ncline[linecolor=darkgray]{n56}{n118}
\ncline[linecolor=darkgray]{n57}{n95}
\ncline[linecolor=darkgray]{n57}{n116}
\ncline[linecolor=darkgray]{n46}{n57}
\ncline[linecolor=darkgray]{n43}{n57}
\ncline[linecolor=darkgray]{n57}{n132}
\ncline[linecolor=darkgray]{n58}{n79}
\ncline[linecolor=darkgray]{n58}{n142}
\ncline[linecolor=darkgray]{n58}{n126}
\ncline[linecolor=darkgray]{n58}{n119}
\ncline[linecolor=darkgray]{n58}{n143}
\ncline[linecolor=darkgray]{n58}{n112}
\ncline[linecolor=darkgray]{n2}{n58}
\ncline[linecolor=darkgray]{n58}{n117}
\ncline[linecolor=darkgray]{n38}{n58}
\ncline[linecolor=darkgray]{n58}{n128}
\ncline[linecolor=darkgray]{n59}{n114}
\ncline[linecolor=darkgray]{n47}{n59}
\ncline[linecolor=darkgray]{n59}{n64}
\ncline[linecolor=darkgray]{n17}{n59}
\ncline[linecolor=darkgray]{n35}{n59}
\ncline[linecolor=darkgray]{n59}{n110}
\ncline[linecolor=darkgray]{n23}{n59}
\ncline[linecolor=darkgray]{n59}{n139}
\ncline[linecolor=darkgray]{n59}{n98}
\ncline[linecolor=darkgray]{n60}{n83}
\ncline[linecolor=darkgray]{n60}{n123}
\ncline[linecolor=darkgray]{n60}{n117}
\ncline[linecolor=darkgray]{n61}{n129}
\ncline[linecolor=darkgray]{n39}{n61}
\ncline[linecolor=darkgray]{n61}{n83}
\ncline[linecolor=darkgray]{n61}{n138}
\ncline[linecolor=darkgray]{n61}{n89}
\ncline[linecolor=darkgray]{n61}{n131}
\ncline[linecolor=darkgray]{n61}{n118}
\ncline[linecolor=darkgray]{n61}{n117}
\ncline[linecolor=darkgray]{n61}{n123}
\ncline[linecolor=darkgray]{n61}{n84}
\ncline[linecolor=darkgray]{n61}{n102}
\ncline[linecolor=darkgray]{n61}{n147}
\ncline[linecolor=darkgray]{n11}{n62}
\ncline[linecolor=darkgray]{n25}{n62}
\ncline[linecolor=darkgray]{n62}{n74}
\ncline[linecolor=darkgray]{n54}{n63}
\ncline[linecolor=darkgray]{n63}{n145}
\ncline[linecolor=darkgray]{n63}{n121}
\ncline[linecolor=darkgray]{n63}{n105}
\ncline[linecolor=darkgray]{n0}{n64}
\ncline[linecolor=darkgray]{n64}{n114}
\ncline[linecolor=darkgray]{n40}{n64}
\ncline[linecolor=darkgray]{n64}{n126}
\ncline[linecolor=darkgray]{n33}{n64}
\ncline[linecolor=darkgray]{n36}{n64}
\ncline[linecolor=darkgray]{n64}{n65}
\ncline[linecolor=darkgray]{n59}{n64}
\ncline[linecolor=darkgray]{n23}{n64}
\ncline[linecolor=darkgray]{n64}{n146}
\ncline[linecolor=darkgray]{n64}{n72}
\ncline[linecolor=darkgray]{n64}{n110}
\ncline[linecolor=darkgray]{n65}{n126}
\ncline[linecolor=darkgray]{n50}{n65}
\ncline[linecolor=darkgray]{n64}{n65}
\ncline[linecolor=darkgray]{n44}{n65}
\ncline[linecolor=darkgray]{n12}{n65}
\ncline[linecolor=darkgray]{n65}{n146}
\ncline[linecolor=darkgray]{n51}{n65}
\ncline[linecolor=darkgray]{n65}{n110}
\ncline[linecolor=darkgray]{n51}{n66}
\ncline[linecolor=darkgray]{n7}{n66}
\ncline[linecolor=darkgray]{n66}{n120}
\ncline[linecolor=darkgray]{n32}{n66}
\ncline[linecolor=darkgray]{n20}{n66}
\ncline[linecolor=darkgray]{n68}{n74}
\ncline[linecolor=darkgray]{n68}{n91}
\ncline[linecolor=darkgray]{n68}{n109}
\ncline[linecolor=darkgray]{n69}{n145}
\ncline[linecolor=darkgray]{n69}{n135}
\ncline[linecolor=darkgray]{n70}{n148}
\ncline[linecolor=darkgray]{n71}{n132}
\ncline[linecolor=darkgray]{n71}{n116}
\ncline[linecolor=darkgray]{n33}{n72}
\ncline[linecolor=darkgray]{n0}{n72}
\ncline[linecolor=darkgray]{n17}{n72}
\ncline[linecolor=darkgray]{n72}{n112}
\ncline[linecolor=darkgray]{n36}{n72}
\ncline[linecolor=darkgray]{n50}{n72}
\ncline[linecolor=darkgray]{n12}{n72}
\ncline[linecolor=darkgray]{n64}{n72}
\ncline[linecolor=darkgray]{n73}{n145}
\ncline[linecolor=darkgray]{n30}{n73}
\ncline[linecolor=darkgray]{n49}{n73}
\ncline[linecolor=darkgray]{n62}{n74}
\ncline[linecolor=darkgray]{n68}{n74}
\ncline[linecolor=darkgray]{n52}{n74}
\ncline[linecolor=darkgray]{n75}{n121}
\ncline[linecolor=darkgray]{n16}{n75}
\ncline[linecolor=darkgray]{n75}{n105}
\ncline[linecolor=darkgray]{n3}{n76}
\ncline[linecolor=darkgray]{n45}{n76}
\ncline[linecolor=darkgray]{n76}{n78}
\ncline[linecolor=darkgray]{n39}{n77}
\ncline[linecolor=darkgray]{n77}{n123}
\ncline[linecolor=darkgray]{n77}{n117}
\ncline[linecolor=darkgray]{n77}{n101}
\ncline[linecolor=darkgray]{n77}{n119}
\ncline[linecolor=darkgray]{n20}{n77}
\ncline[linecolor=darkgray]{n77}{n143}
\ncline[linecolor=darkgray]{n77}{n79}
\ncline[linecolor=darkgray]{n41}{n77}
\ncline[linecolor=darkgray]{n77}{n86}
\ncline[linecolor=darkgray]{n7}{n77}
\ncline[linecolor=darkgray]{n77}{n120}
\ncline[linecolor=darkgray]{n31}{n78}
\ncline[linecolor=darkgray]{n52}{n78}
\ncline[linecolor=darkgray]{n76}{n78}
\ncline[linecolor=darkgray]{n79}{n126}
\ncline[linecolor=darkgray]{n58}{n79}
\ncline[linecolor=darkgray]{n40}{n79}
\ncline[linecolor=darkgray]{n79}{n144}
\ncline[linecolor=darkgray]{n50}{n79}
\ncline[linecolor=darkgray]{n77}{n79}
\ncline[linecolor=darkgray]{n79}{n112}
\ncline[linecolor=darkgray]{n79}{n82}
\ncline[linecolor=darkgray]{n79}{n137}
\ncline[linecolor=darkgray]{n13}{n79}
\ncline[linecolor=darkgray]{n79}{n143}
\ncline[linecolor=darkgray]{n80}{n102}
\ncline[linecolor=darkgray]{n80}{n87}
\ncline[linecolor=darkgray]{n80}{n118}
\ncline[linecolor=darkgray]{n53}{n80}
\ncline[linecolor=darkgray]{n81}{n85}
\ncline[linecolor=darkgray]{n34}{n81}
\ncline[linecolor=darkgray]{n81}{n98}
\ncline[linecolor=darkgray]{n82}{n122}
\ncline[linecolor=darkgray]{n41}{n82}
\ncline[linecolor=darkgray]{n82}{n112}
\ncline[linecolor=darkgray]{n2}{n82}
\ncline[linecolor=darkgray]{n82}{n143}
\ncline[linecolor=darkgray]{n82}{n137}
\ncline[linecolor=darkgray]{n82}{n142}
\ncline[linecolor=darkgray]{n82}{n101}
\ncline[linecolor=darkgray]{n79}{n82}
\ncline[linecolor=darkgray]{n61}{n83}
\ncline[linecolor=darkgray]{n83}{n131}
\ncline[linecolor=darkgray]{n6}{n83}
\ncline[linecolor=darkgray]{n21}{n83}
\ncline[linecolor=darkgray]{n60}{n83}
\ncline[linecolor=darkgray]{n9}{n83}
\ncline[linecolor=darkgray]{n84}{n122}
\ncline[linecolor=darkgray]{n21}{n84}
\ncline[linecolor=darkgray]{n84}{n100}
\ncline[linecolor=darkgray]{n84}{n102}
\ncline[linecolor=darkgray]{n84}{n117}
\ncline[linecolor=darkgray]{n84}{n89}
\ncline[linecolor=darkgray]{n84}{n138}
\ncline[linecolor=darkgray]{n61}{n84}
\ncline[linecolor=darkgray]{n84}{n97}
\ncline[linecolor=darkgray]{n2}{n84}
\ncline[linecolor=darkgray]{n84}{n99}
\ncline[linecolor=darkgray]{n84}{n118}
\ncline[linecolor=darkgray]{n81}{n85}
\ncline[linecolor=darkgray]{n85}{n96}
\ncline[linecolor=darkgray]{n85}{n98}
\ncline[linecolor=darkgray]{n20}{n86}
\ncline[linecolor=darkgray]{n86}{n137}
\ncline[linecolor=darkgray]{n86}{n99}
\ncline[linecolor=darkgray]{n77}{n86}
\ncline[linecolor=darkgray]{n51}{n86}
\ncline[linecolor=darkgray]{n40}{n86}
\ncline[linecolor=darkgray]{n7}{n86}
\ncline[linecolor=darkgray]{n86}{n120}
\ncline[linecolor=darkgray]{n87}{n103}
\ncline[linecolor=darkgray]{n55}{n87}
\ncline[linecolor=darkgray]{n6}{n87}
\ncline[linecolor=darkgray]{n53}{n87}
\ncline[linecolor=darkgray]{n87}{n92}
\ncline[linecolor=darkgray]{n56}{n87}
\ncline[linecolor=darkgray]{n87}{n147}
\ncline[linecolor=darkgray]{n80}{n87}
\ncline[linecolor=darkgray]{n46}{n87}
\ncline[linecolor=darkgray]{n87}{n141}
\ncline[linecolor=darkgray]{n48}{n88}
\ncline[linecolor=darkgray]{n37}{n88}
\ncline[linecolor=darkgray]{n11}{n88}
\ncline[linecolor=darkgray]{n88}{n127}
\ncline[linecolor=darkgray]{n30}{n88}
\ncline[linecolor=darkgray]{n88}{n130}
\ncline[linecolor=darkgray]{n88}{n115}
\ncline[linecolor=darkgray]{n13}{n89}
\ncline[linecolor=darkgray]{n89}{n119}
\ncline[linecolor=darkgray]{n89}{n117}
\ncline[linecolor=darkgray]{n89}{n123}
\ncline[linecolor=darkgray]{n61}{n89}
\ncline[linecolor=darkgray]{n89}{n144}
\ncline[linecolor=darkgray]{n89}{n143}
\ncline[linecolor=darkgray]{n21}{n89}
\ncline[linecolor=darkgray]{n84}{n89}
\ncline[linecolor=darkgray]{n20}{n89}
\ncline[linecolor=darkgray]{n7}{n89}
\ncline[linecolor=darkgray]{n89}{n120}
\ncline[linecolor=darkgray]{n27}{n90}
\ncline[linecolor=darkgray]{n90}{n124}
\ncline[linecolor=darkgray]{n68}{n91}
\ncline[linecolor=darkgray]{n92}{n102}
\ncline[linecolor=darkgray]{n92}{n147}
\ncline[linecolor=darkgray]{n21}{n92}
\ncline[linecolor=darkgray]{n55}{n92}
\ncline[linecolor=darkgray]{n8}{n92}
\ncline[linecolor=darkgray]{n92}{n131}
\ncline[linecolor=darkgray]{n42}{n92}
\ncline[linecolor=darkgray]{n92}{n129}
\ncline[linecolor=darkgray]{n92}{n103}
\ncline[linecolor=darkgray]{n92}{n118}
\ncline[linecolor=darkgray]{n92}{n141}
\ncline[linecolor=darkgray]{n87}{n92}
\ncline[linecolor=darkgray]{n6}{n92}
\ncline[linecolor=darkgray]{n56}{n92}
\ncline[linecolor=darkgray]{n93}{n119}
\ncline[linecolor=darkgray]{n93}{n94}
\ncline[linecolor=darkgray]{n94}{n119}
\ncline[linecolor=darkgray]{n94}{n123}
\ncline[linecolor=darkgray]{n93}{n94}
\ncline[linecolor=darkgray]{n39}{n94}
\ncline[linecolor=darkgray]{n32}{n94}
\ncline[linecolor=darkgray]{n7}{n94}
\ncline[linecolor=darkgray]{n94}{n120}
\ncline[linecolor=darkgray]{n38}{n94}
\ncline[linecolor=darkgray]{n46}{n95}
\ncline[linecolor=darkgray]{n19}{n95}
\ncline[linecolor=darkgray]{n28}{n95}
\ncline[linecolor=darkgray]{n57}{n95}
\ncline[linecolor=darkgray]{n95}{n116}
\ncline[linecolor=darkgray]{n95}{n132}
\ncline[linecolor=darkgray]{n34}{n96}
\ncline[linecolor=darkgray]{n85}{n96}
\ncline[linecolor=darkgray]{n97}{n104}
\ncline[linecolor=darkgray]{n97}{n100}
\ncline[linecolor=darkgray]{n2}{n97}
\ncline[linecolor=darkgray]{n97}{n128}
\ncline[linecolor=darkgray]{n84}{n97}
\ncline[linecolor=darkgray]{n23}{n98}
\ncline[linecolor=darkgray]{n26}{n98}
\ncline[linecolor=darkgray]{n33}{n98}
\ncline[linecolor=darkgray]{n98}{n114}
\ncline[linecolor=darkgray]{n59}{n98}
\ncline[linecolor=darkgray]{n4}{n98}
\ncline[linecolor=darkgray]{n98}{n110}
\ncline[linecolor=darkgray]{n81}{n98}
\ncline[linecolor=darkgray]{n85}{n98}
\ncline[linecolor=darkgray]{n99}{n137}
\ncline[linecolor=darkgray]{n41}{n99}
\ncline[linecolor=darkgray]{n13}{n99}
\ncline[linecolor=darkgray]{n99}{n122}
\ncline[linecolor=darkgray]{n86}{n99}
\ncline[linecolor=darkgray]{n99}{n144}
\ncline[linecolor=darkgray]{n99}{n101}
\ncline[linecolor=darkgray]{n99}{n128}
\ncline[linecolor=darkgray]{n99}{n104}
\ncline[linecolor=darkgray]{n2}{n99}
\ncline[linecolor=darkgray]{n84}{n99}
\ncline[linecolor=darkgray]{n100}{n143}
\ncline[linecolor=darkgray]{n84}{n100}
\ncline[linecolor=darkgray]{n97}{n100}
\ncline[linecolor=darkgray]{n100}{n128}
\ncline[linecolor=darkgray]{n100}{n104}
\ncline[linecolor=darkgray]{n100}{n119}
\ncline[linecolor=darkgray]{n101}{n142}
\ncline[linecolor=darkgray]{n77}{n101}
\ncline[linecolor=darkgray]{n101}{n143}
\ncline[linecolor=darkgray]{n101}{n144}
\ncline[linecolor=darkgray]{n13}{n101}
\ncline[linecolor=darkgray]{n101}{n117}
\ncline[linecolor=darkgray]{n101}{n126}
\ncline[linecolor=darkgray]{n82}{n101}
\ncline[linecolor=darkgray]{n99}{n101}
\ncline[linecolor=darkgray]{n101}{n123}
\ncline[linecolor=darkgray]{n101}{n119}
\ncline[linecolor=darkgray]{n101}{n112}
\ncline[linecolor=darkgray]{n92}{n102}
\ncline[linecolor=darkgray]{n84}{n102}
\ncline[linecolor=darkgray]{n102}{n147}
\ncline[linecolor=darkgray]{n102}{n118}
\ncline[linecolor=darkgray]{n102}{n138}
\ncline[linecolor=darkgray]{n102}{n141}
\ncline[linecolor=darkgray]{n56}{n102}
\ncline[linecolor=darkgray]{n13}{n102}
\ncline[linecolor=darkgray]{n61}{n102}
\ncline[linecolor=darkgray]{n80}{n102}
\ncline[linecolor=darkgray]{n8}{n103}
\ncline[linecolor=darkgray]{n42}{n103}
\ncline[linecolor=darkgray]{n6}{n103}
\ncline[linecolor=darkgray]{n87}{n103}
\ncline[linecolor=darkgray]{n56}{n103}
\ncline[linecolor=darkgray]{n103}{n118}
\ncline[linecolor=darkgray]{n103}{n138}
\ncline[linecolor=darkgray]{n55}{n103}
\ncline[linecolor=darkgray]{n46}{n103}
\ncline[linecolor=darkgray]{n92}{n103}
\ncline[linecolor=darkgray]{n103}{n129}
\ncline[linecolor=darkgray]{n103}{n131}
\ncline[linecolor=darkgray]{n103}{n147}
\ncline[linecolor=darkgray]{n104}{n112}
\ncline[linecolor=darkgray]{n104}{n143}
\ncline[linecolor=darkgray]{n97}{n104}
\ncline[linecolor=darkgray]{n100}{n104}
\ncline[linecolor=darkgray]{n1}{n104}
\ncline[linecolor=darkgray]{n99}{n104}
\ncline[linecolor=darkgray]{n75}{n105}
\ncline[linecolor=darkgray]{n15}{n105}
\ncline[linecolor=darkgray]{n63}{n105}
\ncline[linecolor=darkgray]{n37}{n105}
\ncline[linecolor=darkgray]{n3}{n106}
\ncline[linecolor=darkgray]{n42}{n107}
\ncline[linecolor=darkgray]{n56}{n107}
\ncline[linecolor=darkgray]{n46}{n107}
\ncline[linecolor=darkgray]{n6}{n107}
\ncline[linecolor=darkgray]{n107}{n147}
\ncline[linecolor=darkgray]{n107}{n141}
\ncline[linecolor=darkgray]{n55}{n107}
\ncline[linecolor=darkgray]{n19}{n107}
\ncline[linecolor=darkgray]{n107}{n140}
\ncline[linecolor=darkgray]{n68}{n109}
\ncline[linecolor=darkgray]{n23}{n110}
\ncline[linecolor=darkgray]{n44}{n110}
\ncline[linecolor=darkgray]{n26}{n110}
\ncline[linecolor=darkgray]{n110}{n139}
\ncline[linecolor=darkgray]{n35}{n110}
\ncline[linecolor=darkgray]{n59}{n110}
\ncline[linecolor=darkgray]{n64}{n110}
\ncline[linecolor=darkgray]{n98}{n110}
\ncline[linecolor=darkgray]{n110}{n146}
\ncline[linecolor=darkgray]{n65}{n110}
\ncline[linecolor=darkgray]{n111}{n135}
\ncline[linecolor=darkgray]{n2}{n112}
\ncline[linecolor=darkgray]{n112}{n143}
\ncline[linecolor=darkgray]{n104}{n112}
\ncline[linecolor=darkgray]{n82}{n112}
\ncline[linecolor=darkgray]{n72}{n112}
\ncline[linecolor=darkgray]{n41}{n112}
\ncline[linecolor=darkgray]{n79}{n112}
\ncline[linecolor=darkgray]{n101}{n112}
\ncline[linecolor=darkgray]{n58}{n112}
\ncline[linecolor=darkgray]{n0}{n112}
\ncline[linecolor=darkgray]{n112}{n122}
\ncline[linecolor=darkgray]{n38}{n112}
\ncline[linecolor=darkgray]{n44}{n114}
\ncline[linecolor=darkgray]{n23}{n114}
\ncline[linecolor=darkgray]{n50}{n114}
\ncline[linecolor=darkgray]{n59}{n114}
\ncline[linecolor=darkgray]{n64}{n114}
\ncline[linecolor=darkgray]{n12}{n114}
\ncline[linecolor=darkgray]{n114}{n139}
\ncline[linecolor=darkgray]{n47}{n114}
\ncline[linecolor=darkgray]{n98}{n114}
\ncline[linecolor=darkgray]{n26}{n114}
\ncline[linecolor=darkgray]{n22}{n115}
\ncline[linecolor=darkgray]{n30}{n115}
\ncline[linecolor=darkgray]{n115}{n133}
\ncline[linecolor=darkgray]{n88}{n115}
\ncline[linecolor=darkgray]{n49}{n115}
\ncline[linecolor=darkgray]{n43}{n116}
\ncline[linecolor=darkgray]{n57}{n116}
\ncline[linecolor=darkgray]{n71}{n116}
\ncline[linecolor=darkgray]{n95}{n116}
\ncline[linecolor=darkgray]{n117}{n119}
\ncline[linecolor=darkgray]{n89}{n117}
\ncline[linecolor=darkgray]{n77}{n117}
\ncline[linecolor=darkgray]{n13}{n117}
\ncline[linecolor=darkgray]{n61}{n117}
\ncline[linecolor=darkgray]{n101}{n117}
\ncline[linecolor=darkgray]{n84}{n117}
\ncline[linecolor=darkgray]{n39}{n117}
\ncline[linecolor=darkgray]{n20}{n117}
\ncline[linecolor=darkgray]{n58}{n117}
\ncline[linecolor=darkgray]{n117}{n144}
\ncline[linecolor=darkgray]{n117}{n138}
\ncline[linecolor=darkgray]{n60}{n117}
\ncline[linecolor=darkgray]{n21}{n118}
\ncline[linecolor=darkgray]{n118}{n129}
\ncline[linecolor=darkgray]{n61}{n118}
\ncline[linecolor=darkgray]{n102}{n118}
\ncline[linecolor=darkgray]{n42}{n118}
\ncline[linecolor=darkgray]{n103}{n118}
\ncline[linecolor=darkgray]{n118}{n131}
\ncline[linecolor=darkgray]{n6}{n118}
\ncline[linecolor=darkgray]{n92}{n118}
\ncline[linecolor=darkgray]{n56}{n118}
\ncline[linecolor=darkgray]{n84}{n118}
\ncline[linecolor=darkgray]{n80}{n118}
\ncline[linecolor=darkgray]{n117}{n119}
\ncline[linecolor=darkgray]{n89}{n119}
\ncline[linecolor=darkgray]{n39}{n119}
\ncline[linecolor=darkgray]{n119}{n123}
\ncline[linecolor=darkgray]{n94}{n119}
\ncline[linecolor=darkgray]{n77}{n119}
\ncline[linecolor=darkgray]{n119}{n143}
\ncline[linecolor=darkgray]{n101}{n119}
\ncline[linecolor=darkgray]{n58}{n119}
\ncline[linecolor=darkgray]{n93}{n119}
\ncline[linecolor=darkgray]{n38}{n119}
\ncline[linecolor=darkgray]{n100}{n119}
\ncline[linecolor=darkgray]{n120}{n123}
\ncline[linecolor=darkgray]{n13}{n120}
\ncline[linecolor=darkgray]{n120}{n144}
\ncline[linecolor=darkgray]{n66}{n120}
\ncline[linecolor=darkgray]{n77}{n120}
\ncline[linecolor=darkgray]{n39}{n120}
\ncline[linecolor=darkgray]{n89}{n120}
\ncline[linecolor=darkgray]{n86}{n120}
\ncline[linecolor=darkgray]{n94}{n120}
\ncline[linecolor=darkgray]{n75}{n121}
\ncline[linecolor=darkgray]{n63}{n121}
\ncline[linecolor=darkgray]{n84}{n122}
\ncline[linecolor=darkgray]{n82}{n122}
\ncline[linecolor=darkgray]{n122}{n143}
\ncline[linecolor=darkgray]{n99}{n122}
\ncline[linecolor=darkgray]{n112}{n122}
\ncline[linecolor=darkgray]{n7}{n123}
\ncline[linecolor=darkgray]{n120}{n123}
\ncline[linecolor=darkgray]{n20}{n123}
\ncline[linecolor=darkgray]{n89}{n123}
\ncline[linecolor=darkgray]{n13}{n123}
\ncline[linecolor=darkgray]{n77}{n123}
\ncline[linecolor=darkgray]{n119}{n123}
\ncline[linecolor=darkgray]{n123}{n144}
\ncline[linecolor=darkgray]{n39}{n123}
\ncline[linecolor=darkgray]{n101}{n123}
\ncline[linecolor=darkgray]{n94}{n123}
\ncline[linecolor=darkgray]{n61}{n123}
\ncline[linecolor=darkgray]{n32}{n123}
\ncline[linecolor=darkgray]{n60}{n123}
\ncline[linecolor=darkgray]{n124}{n127}
\ncline[linecolor=darkgray]{n31}{n124}
\ncline[linecolor=darkgray]{n90}{n124}
\ncline[linecolor=darkgray]{n124}{n135}
\ncline[linecolor=darkgray]{n15}{n125}
\ncline[linecolor=darkgray]{n79}{n126}
\ncline[linecolor=darkgray]{n12}{n126}
\ncline[linecolor=darkgray]{n17}{n126}
\ncline[linecolor=darkgray]{n65}{n126}
\ncline[linecolor=darkgray]{n0}{n126}
\ncline[linecolor=darkgray]{n58}{n126}
\ncline[linecolor=darkgray]{n64}{n126}
\ncline[linecolor=darkgray]{n41}{n126}
\ncline[linecolor=darkgray]{n101}{n126}
\ncline[linecolor=darkgray]{n36}{n126}
\ncline[linecolor=darkgray]{n126}{n142}
\ncline[linecolor=darkgray]{n37}{n127}
\ncline[linecolor=darkgray]{n124}{n127}
\ncline[linecolor=darkgray]{n127}{n136}
\ncline[linecolor=darkgray]{n88}{n127}
\ncline[linecolor=darkgray]{n128}{n142}
\ncline[linecolor=darkgray]{n100}{n128}
\ncline[linecolor=darkgray]{n97}{n128}
\ncline[linecolor=darkgray]{n99}{n128}
\ncline[linecolor=darkgray]{n1}{n128}
\ncline[linecolor=darkgray]{n58}{n128}
\ncline[linecolor=darkgray]{n61}{n129}
\ncline[linecolor=darkgray]{n129}{n138}
\ncline[linecolor=darkgray]{n118}{n129}
\ncline[linecolor=darkgray]{n42}{n129}
\ncline[linecolor=darkgray]{n21}{n129}
\ncline[linecolor=darkgray]{n8}{n129}
\ncline[linecolor=darkgray]{n92}{n129}
\ncline[linecolor=darkgray]{n9}{n129}
\ncline[linecolor=darkgray]{n129}{n147}
\ncline[linecolor=darkgray]{n103}{n129}
\ncline[linecolor=darkgray]{n11}{n130}
\ncline[linecolor=darkgray]{n30}{n130}
\ncline[linecolor=darkgray]{n88}{n130}
\ncline[linecolor=darkgray]{n37}{n130}
\ncline[linecolor=darkgray]{n49}{n130}
\ncline[linecolor=darkgray]{n22}{n130}
\ncline[linecolor=darkgray]{n83}{n131}
\ncline[linecolor=darkgray]{n61}{n131}
\ncline[linecolor=darkgray]{n8}{n131}
\ncline[linecolor=darkgray]{n42}{n131}
\ncline[linecolor=darkgray]{n131}{n138}
\ncline[linecolor=darkgray]{n21}{n131}
\ncline[linecolor=darkgray]{n92}{n131}
\ncline[linecolor=darkgray]{n118}{n131}
\ncline[linecolor=darkgray]{n131}{n147}
\ncline[linecolor=darkgray]{n103}{n131}
\ncline[linecolor=darkgray]{n9}{n131}
\ncline[linecolor=darkgray]{n55}{n131}
\ncline[linecolor=darkgray]{n71}{n132}
\ncline[linecolor=darkgray]{n43}{n132}
\ncline[linecolor=darkgray]{n95}{n132}
\ncline[linecolor=darkgray]{n57}{n132}
\ncline[linecolor=darkgray]{n115}{n133}
\ncline[linecolor=darkgray]{n30}{n133}
\ncline[linecolor=darkgray]{n22}{n133}
\ncline[linecolor=darkgray]{n49}{n133}
\ncline[linecolor=darkgray]{n54}{n135}
\ncline[linecolor=darkgray]{n135}{n145}
\ncline[linecolor=darkgray]{n10}{n135}
\ncline[linecolor=darkgray]{n69}{n135}
\ncline[linecolor=darkgray]{n124}{n135}
\ncline[linecolor=darkgray]{n49}{n135}
\ncline[linecolor=darkgray]{n111}{n135}
\ncline[linecolor=darkgray]{n25}{n136}
\ncline[linecolor=darkgray]{n127}{n136}
\ncline[linecolor=darkgray]{n99}{n137}
\ncline[linecolor=darkgray]{n86}{n137}
\ncline[linecolor=darkgray]{n82}{n137}
\ncline[linecolor=darkgray]{n41}{n137}
\ncline[linecolor=darkgray]{n79}{n137}
\ncline[linecolor=darkgray]{n0}{n137}
\ncline[linecolor=darkgray]{n137}{n144}
\ncline[linecolor=darkgray]{n40}{n137}
\ncline[linecolor=darkgray]{n21}{n138}
\ncline[linecolor=darkgray]{n61}{n138}
\ncline[linecolor=darkgray]{n129}{n138}
\ncline[linecolor=darkgray]{n42}{n138}
\ncline[linecolor=darkgray]{n131}{n138}
\ncline[linecolor=darkgray]{n102}{n138}
\ncline[linecolor=darkgray]{n6}{n138}
\ncline[linecolor=darkgray]{n103}{n138}
\ncline[linecolor=darkgray]{n56}{n138}
\ncline[linecolor=darkgray]{n84}{n138}
\ncline[linecolor=darkgray]{n117}{n138}
\ncline[linecolor=darkgray]{n138}{n147}
\ncline[linecolor=darkgray]{n33}{n139}
\ncline[linecolor=darkgray]{n47}{n139}
\ncline[linecolor=darkgray]{n114}{n139}
\ncline[linecolor=darkgray]{n36}{n139}
\ncline[linecolor=darkgray]{n110}{n139}
\ncline[linecolor=darkgray]{n59}{n139}
\ncline[linecolor=darkgray]{n4}{n139}
\ncline[linecolor=darkgray]{n29}{n140}
\ncline[linecolor=darkgray]{n107}{n140}
\ncline[linecolor=darkgray]{n56}{n141}
\ncline[linecolor=darkgray]{n42}{n141}
\ncline[linecolor=darkgray]{n102}{n141}
\ncline[linecolor=darkgray]{n21}{n141}
\ncline[linecolor=darkgray]{n8}{n141}
\ncline[linecolor=darkgray]{n92}{n141}
\ncline[linecolor=darkgray]{n107}{n141}
\ncline[linecolor=darkgray]{n87}{n141}
\ncline[linecolor=darkgray]{n0}{n142}
\ncline[linecolor=darkgray]{n101}{n142}
\ncline[linecolor=darkgray]{n128}{n142}
\ncline[linecolor=darkgray]{n58}{n142}
\ncline[linecolor=darkgray]{n82}{n142}
\ncline[linecolor=darkgray]{n142}{n143}
\ncline[linecolor=darkgray]{n2}{n142}
\ncline[linecolor=darkgray]{n12}{n142}
\ncline[linecolor=darkgray]{n126}{n142}
\ncline[linecolor=darkgray]{n100}{n143}
\ncline[linecolor=darkgray]{n122}{n143}
\ncline[linecolor=darkgray]{n112}{n143}
\ncline[linecolor=darkgray]{n82}{n143}
\ncline[linecolor=darkgray]{n104}{n143}
\ncline[linecolor=darkgray]{n101}{n143}
\ncline[linecolor=darkgray]{n41}{n143}
\ncline[linecolor=darkgray]{n77}{n143}
\ncline[linecolor=darkgray]{n119}{n143}
\ncline[linecolor=darkgray]{n142}{n143}
\ncline[linecolor=darkgray]{n89}{n143}
\ncline[linecolor=darkgray]{n58}{n143}
\ncline[linecolor=darkgray]{n79}{n143}
\ncline[linecolor=darkgray]{n89}{n144}
\ncline[linecolor=darkgray]{n79}{n144}
\ncline[linecolor=darkgray]{n101}{n144}
\ncline[linecolor=darkgray]{n41}{n144}
\ncline[linecolor=darkgray]{n123}{n144}
\ncline[linecolor=darkgray]{n7}{n144}
\ncline[linecolor=darkgray]{n99}{n144}
\ncline[linecolor=darkgray]{n120}{n144}
\ncline[linecolor=darkgray]{n51}{n144}
\ncline[linecolor=darkgray]{n137}{n144}
\ncline[linecolor=darkgray]{n117}{n144}
\ncline[linecolor=darkgray]{n69}{n145}
\ncline[linecolor=darkgray]{n135}{n145}
\ncline[linecolor=darkgray]{n145}{n149}
\ncline[linecolor=darkgray]{n48}{n145}
\ncline[linecolor=darkgray]{n73}{n145}
\ncline[linecolor=darkgray]{n63}{n145}
\ncline[linecolor=darkgray]{n49}{n145}
\ncline[linecolor=darkgray]{n17}{n146}
\ncline[linecolor=darkgray]{n23}{n146}
\ncline[linecolor=darkgray]{n50}{n146}
\ncline[linecolor=darkgray]{n36}{n146}
\ncline[linecolor=darkgray]{n12}{n146}
\ncline[linecolor=darkgray]{n33}{n146}
\ncline[linecolor=darkgray]{n26}{n146}
\ncline[linecolor=darkgray]{n40}{n146}
\ncline[linecolor=darkgray]{n64}{n146}
\ncline[linecolor=darkgray]{n65}{n146}
\ncline[linecolor=darkgray]{n35}{n146}
\ncline[linecolor=darkgray]{n110}{n146}
\ncline[linecolor=darkgray]{n21}{n147}
\ncline[linecolor=darkgray]{n56}{n147}
\ncline[linecolor=darkgray]{n42}{n147}
\ncline[linecolor=darkgray]{n92}{n147}
\ncline[linecolor=darkgray]{n8}{n147}
\ncline[linecolor=darkgray]{n102}{n147}
\ncline[linecolor=darkgray]{n131}{n147}
\ncline[linecolor=darkgray]{n129}{n147}
\ncline[linecolor=darkgray]{n107}{n147}
\ncline[linecolor=darkgray]{n61}{n147}
\ncline[linecolor=darkgray]{n103}{n147}
\ncline[linecolor=darkgray]{n138}{n147}
\ncline[linecolor=darkgray]{n87}{n147}
\ncline[linecolor=darkgray]{n70}{n148}
\ncline[linecolor=darkgray]{n76}{n108}
\ncline[linecolor=black]{n0}{n50}
\ncline[linecolor=black]{n0}{n79}
\ncline[linecolor=black]{n1}{n97}
\ncline[linecolor=black]{n1}{n122}
\ncline[linecolor=black]{n1}{n84}
\ncline[linecolor=black]{n1}{n100}
\ncline[linecolor=black]{n2}{n128}
\ncline[linecolor=black]{n2}{n143}
\ncline[linecolor=black]{n2}{n100}
\ncline[linecolor=black]{n2}{n104}
\ncline[linecolor=black]{n2}{n122}
\ncline[linecolor=black]{n3}{n27}
\ncline[linecolor=black]{n3}{n5}
\ncline[linecolor=black]{n3}{n91}
\ncline[linecolor=black]{n3}{n74}
\ncline[linecolor=black]{n3}{n75}
\ncline[linecolor=black]{n3}{n45}
\ncline[linecolor=black]{n3}{n52}
\ncline[linecolor=black]{n3}{n78}
\ncline[linecolor=black]{n3}{n125}
\ncline[linecolor=black]{n3}{n90}
\ncline[linecolor=black]{n3}{n14}
\ncline[linecolor=black]{n3}{n67}
\ncline[linecolor=black]{n3}{n134}
\ncline[linecolor=black]{n3}{n5}
\ncline[linecolor=black]{n5}{n27}
\ncline[linecolor=black]{n5}{n45}
\ncline[linecolor=black]{n5}{n74}
\ncline[linecolor=black]{n5}{n78}
\ncline[linecolor=black]{n5}{n75}
\ncline[linecolor=black]{n5}{n91}
\ncline[linecolor=black]{n5}{n106}
\ncline[linecolor=black]{n5}{n14}
\ncline[linecolor=black]{n5}{n31}
\ncline[linecolor=black]{n5}{n67}
\ncline[linecolor=black]{n5}{n134}
\ncline[linecolor=black]{n5}{n125}
\ncline[linecolor=black]{n6}{n131}
\ncline[linecolor=black]{n6}{n129}
\ncline[linecolor=black]{n7}{n120}
\ncline[linecolor=black]{n7}{n32}
\ncline[linecolor=black]{n7}{n51}
\ncline[linecolor=black]{n7}{n20}
\ncline[linecolor=black]{n8}{n87}
\ncline[linecolor=black]{n8}{n107}
\ncline[linecolor=black]{n8}{n42}
\ncline[linecolor=black]{n8}{n29}
\ncline[linecolor=black]{n10}{n11}
\ncline[linecolor=black]{n10}{n127}
\ncline[linecolor=black]{n10}{n136}
\ncline[linecolor=black]{n10}{n111}
\ncline[linecolor=black]{n10}{n48}
\ncline[linecolor=black]{n10}{n124}
\ncline[linecolor=black]{n10}{n113}
\ncline[linecolor=black]{n10}{n25}
\ncline[linecolor=black]{n10}{n62}
\ncline[linecolor=black]{n10}{n15}
\ncline[linecolor=black]{n10}{n11}
\ncline[linecolor=black]{n11}{n127}
\ncline[linecolor=black]{n11}{n48}
\ncline[linecolor=black]{n11}{n111}
\ncline[linecolor=black]{n11}{n136}
\ncline[linecolor=black]{n11}{n113}
\ncline[linecolor=black]{n11}{n124}
\ncline[linecolor=black]{n11}{n135}
\ncline[linecolor=black]{n12}{n50}
\ncline[linecolor=black]{n12}{n64}
\ncline[linecolor=black]{n12}{n17}
\ncline[linecolor=black]{n12}{n44}
\ncline[linecolor=black]{n13}{n144}
\ncline[linecolor=black]{n13}{n77}
\ncline[linecolor=black]{n13}{n20}
\ncline[linecolor=black]{n13}{n86}
\ncline[linecolor=black]{n14}{n67}
\ncline[linecolor=black]{n14}{n134}
\ncline[linecolor=black]{n14}{n106}
\ncline[linecolor=black]{n14}{n125}
\ncline[linecolor=black]{n14}{n45}
\ncline[linecolor=black]{n14}{n121}
\ncline[linecolor=black]{n14}{n18}
\ncline[linecolor=black]{n14}{n75}
\ncline[linecolor=black]{n14}{n78}
\ncline[linecolor=black]{n5}{n14}
\ncline[linecolor=black]{n14}{n27}
\ncline[linecolor=black]{n14}{n16}
\ncline[linecolor=black]{n14}{n31}
\ncline[linecolor=black]{n14}{n90}
\ncline[linecolor=black]{n3}{n14}
\ncline[linecolor=black]{n15}{n63}
\ncline[linecolor=black]{n15}{n16}
\ncline[linecolor=black]{n15}{n25}
\ncline[linecolor=black]{n15}{n149}
\ncline[linecolor=black]{n15}{n136}
\ncline[linecolor=black]{n15}{n90}
\ncline[linecolor=black]{n15}{n18}
\ncline[linecolor=black]{n15}{n113}
\ncline[linecolor=black]{n15}{n124}
\ncline[linecolor=black]{n10}{n15}
\ncline[linecolor=black]{n15}{n62}
\ncline[linecolor=black]{n15}{n69}
\ncline[linecolor=black]{n15}{n111}
\ncline[linecolor=black]{n16}{n149}
\ncline[linecolor=black]{n16}{n18}
\ncline[linecolor=black]{n16}{n124}
\ncline[linecolor=black]{n16}{n136}
\ncline[linecolor=black]{n15}{n16}
\ncline[linecolor=black]{n16}{n69}
\ncline[linecolor=black]{n16}{n90}
\ncline[linecolor=black]{n16}{n111}
\ncline[linecolor=black]{n16}{n113}
\ncline[linecolor=black]{n16}{n62}
\ncline[linecolor=black]{n14}{n16}
\ncline[linecolor=black]{n16}{n67}
\ncline[linecolor=black]{n16}{n134}
\ncline[linecolor=black]{n17}{n50}
\ncline[linecolor=black]{n17}{n44}
\ncline[linecolor=black]{n17}{n64}
\ncline[linecolor=black]{n12}{n17}
\ncline[linecolor=black]{n17}{n33}
\ncline[linecolor=black]{n17}{n114}
\ncline[linecolor=black]{n17}{n36}
\ncline[linecolor=black]{n16}{n18}
\ncline[linecolor=black]{n18}{n149}
\ncline[linecolor=black]{n18}{n31}
\ncline[linecolor=black]{n18}{n69}
\ncline[linecolor=black]{n18}{n124}
\ncline[linecolor=black]{n18}{n62}
\ncline[linecolor=black]{n18}{n136}
\ncline[linecolor=black]{n14}{n18}
\ncline[linecolor=black]{n18}{n67}
\ncline[linecolor=black]{n18}{n134}
\ncline[linecolor=black]{n15}{n18}
\ncline[linecolor=black]{n18}{n90}
\ncline[linecolor=black]{n18}{n111}
\ncline[linecolor=black]{n18}{n113}
\ncline[linecolor=black]{n18}{n106}
\ncline[linecolor=black]{n19}{n28}
\ncline[linecolor=black]{n13}{n20}
\ncline[linecolor=black]{n20}{n144}
\ncline[linecolor=black]{n7}{n20}
\ncline[linecolor=black]{n20}{n120}
\ncline[linecolor=black]{n21}{n102}
\ncline[linecolor=black]{n21}{n61}
\ncline[linecolor=black]{n23}{n26}
\ncline[linecolor=black]{n23}{n35}
\ncline[linecolor=black]{n23}{n44}
\ncline[linecolor=black]{n23}{n139}
\ncline[linecolor=black]{n23}{n33}
\ncline[linecolor=black]{n15}{n25}
\ncline[linecolor=black]{n25}{n63}
\ncline[linecolor=black]{n10}{n25}
\ncline[linecolor=black]{n23}{n26}
\ncline[linecolor=black]{n26}{n139}
\ncline[linecolor=black]{n26}{n35}
\ncline[linecolor=black]{n3}{n27}
\ncline[linecolor=black]{n27}{n75}
\ncline[linecolor=black]{n27}{n125}
\ncline[linecolor=black]{n5}{n27}
\ncline[linecolor=black]{n27}{n45}
\ncline[linecolor=black]{n27}{n78}
\ncline[linecolor=black]{n27}{n52}
\ncline[linecolor=black]{n27}{n76}
\ncline[linecolor=black]{n27}{n91}
\ncline[linecolor=black]{n14}{n27}
\ncline[linecolor=black]{n27}{n67}
\ncline[linecolor=black]{n27}{n134}
\ncline[linecolor=black]{n27}{n74}
\ncline[linecolor=black]{n27}{n106}
\ncline[linecolor=black]{n19}{n28}
\ncline[linecolor=black]{n29}{n107}
\ncline[linecolor=black]{n29}{n87}
\ncline[linecolor=black]{n29}{n103}
\ncline[linecolor=black]{n8}{n29}
\ncline[linecolor=black]{n30}{n135}
\ncline[linecolor=black]{n18}{n31}
\ncline[linecolor=black]{n31}{n106}
\ncline[linecolor=black]{n5}{n31}
\ncline[linecolor=black]{n31}{n62}
\ncline[linecolor=black]{n14}{n31}
\ncline[linecolor=black]{n31}{n67}
\ncline[linecolor=black]{n31}{n69}
\ncline[linecolor=black]{n31}{n134}
\ncline[linecolor=black]{n7}{n32}
\ncline[linecolor=black]{n32}{n120}
\ncline[linecolor=black]{n32}{n51}
\ncline[linecolor=black]{n33}{n114}
\ncline[linecolor=black]{n33}{n36}
\ncline[linecolor=black]{n17}{n33}
\ncline[linecolor=black]{n33}{n35}
\ncline[linecolor=black]{n33}{n44}
\ncline[linecolor=black]{n23}{n33}
\ncline[linecolor=black]{n35}{n139}
\ncline[linecolor=black]{n23}{n35}
\ncline[linecolor=black]{n33}{n35}
\ncline[linecolor=black]{n35}{n98}
\ncline[linecolor=black]{n35}{n114}
\ncline[linecolor=black]{n26}{n35}
\ncline[linecolor=black]{n33}{n36}
\ncline[linecolor=black]{n36}{n114}
\ncline[linecolor=black]{n17}{n36}
\ncline[linecolor=black]{n38}{n93}
\ncline[linecolor=black]{n39}{n89}
\ncline[linecolor=black]{n40}{n65}
\ncline[linecolor=black]{n40}{n126}
\ncline[linecolor=black]{n41}{n101}
\ncline[linecolor=black]{n41}{n79}
\ncline[linecolor=black]{n41}{n58}
\ncline[linecolor=black]{n41}{n142}
\ncline[linecolor=black]{n42}{n56}
\ncline[linecolor=black]{n8}{n42}
\ncline[linecolor=black]{n43}{n95}
\ncline[linecolor=black]{n17}{n44}
\ncline[linecolor=black]{n44}{n50}
\ncline[linecolor=black]{n44}{n146}
\ncline[linecolor=black]{n23}{n44}
\ncline[linecolor=black]{n33}{n44}
\ncline[linecolor=black]{n44}{n64}
\ncline[linecolor=black]{n12}{n44}
\ncline[linecolor=black]{n44}{n126}
\ncline[linecolor=black]{n45}{n106}
\ncline[linecolor=black]{n14}{n45}
\ncline[linecolor=black]{n45}{n67}
\ncline[linecolor=black]{n45}{n134}
\ncline[linecolor=black]{n45}{n78}
\ncline[linecolor=black]{n45}{n125}
\ncline[linecolor=black]{n5}{n45}
\ncline[linecolor=black]{n27}{n45}
\ncline[linecolor=black]{n3}{n45}
\ncline[linecolor=black]{n45}{n121}
\ncline[linecolor=black]{n45}{n75}
\ncline[linecolor=black]{n47}{n98}
\ncline[linecolor=black]{n48}{n135}
\ncline[linecolor=black]{n48}{n113}
\ncline[linecolor=black]{n11}{n48}
\ncline[linecolor=black]{n48}{n127}
\ncline[linecolor=black]{n10}{n48}
\ncline[linecolor=black]{n48}{n124}
\ncline[linecolor=black]{n48}{n136}
\ncline[linecolor=black]{n48}{n69}
\ncline[linecolor=black]{n48}{n111}
\ncline[linecolor=black]{n48}{n73}
\ncline[linecolor=black]{n49}{n88}
\ncline[linecolor=black]{n12}{n50}
\ncline[linecolor=black]{n17}{n50}
\ncline[linecolor=black]{n50}{n64}
\ncline[linecolor=black]{n44}{n50}
\ncline[linecolor=black]{n0}{n50}
\ncline[linecolor=black]{n50}{n126}
\ncline[linecolor=black]{n7}{n51}
\ncline[linecolor=black]{n51}{n120}
\ncline[linecolor=black]{n32}{n51}
\ncline[linecolor=black]{n52}{n76}
\ncline[linecolor=black]{n52}{n91}
\ncline[linecolor=black]{n52}{n109}
\ncline[linecolor=black]{n3}{n52}
\ncline[linecolor=black]{n27}{n52}
\ncline[linecolor=black]{n54}{n145}
\ncline[linecolor=black]{n54}{n69}
\ncline[linecolor=black]{n54}{n113}
\ncline[linecolor=black]{n54}{n73}
\ncline[linecolor=black]{n54}{n136}
\ncline[linecolor=black]{n55}{n80}
\ncline[linecolor=black]{n42}{n56}
\ncline[linecolor=black]{n58}{n101}
\ncline[linecolor=black]{n41}{n58}
\ncline[linecolor=black]{n21}{n61}
\ncline[linecolor=black]{n62}{n136}
\ncline[linecolor=black]{n62}{n111}
\ncline[linecolor=black]{n62}{n124}
\ncline[linecolor=black]{n16}{n62}
\ncline[linecolor=black]{n18}{n62}
\ncline[linecolor=black]{n62}{n69}
\ncline[linecolor=black]{n62}{n149}
\ncline[linecolor=black]{n62}{n113}
\ncline[linecolor=black]{n10}{n62}
\ncline[linecolor=black]{n15}{n62}
\ncline[linecolor=black]{n31}{n62}
\ncline[linecolor=black]{n15}{n63}
\ncline[linecolor=black]{n25}{n63}
\ncline[linecolor=black]{n12}{n64}
\ncline[linecolor=black]{n50}{n64}
\ncline[linecolor=black]{n17}{n64}
\ncline[linecolor=black]{n44}{n64}
\ncline[linecolor=black]{n40}{n65}
\ncline[linecolor=black]{n14}{n67}
\ncline[linecolor=black]{n67}{n134}
\ncline[linecolor=black]{n67}{n106}
\ncline[linecolor=black]{n67}{n125}
\ncline[linecolor=black]{n45}{n67}
\ncline[linecolor=black]{n67}{n121}
\ncline[linecolor=black]{n18}{n67}
\ncline[linecolor=black]{n67}{n75}
\ncline[linecolor=black]{n67}{n78}
\ncline[linecolor=black]{n5}{n67}
\ncline[linecolor=black]{n27}{n67}
\ncline[linecolor=black]{n16}{n67}
\ncline[linecolor=black]{n31}{n67}
\ncline[linecolor=black]{n67}{n90}
\ncline[linecolor=black]{n3}{n67}
\ncline[linecolor=black]{n69}{n113}
\ncline[linecolor=black]{n69}{n124}
\ncline[linecolor=black]{n69}{n149}
\ncline[linecolor=black]{n69}{n136}
\ncline[linecolor=black]{n16}{n69}
\ncline[linecolor=black]{n18}{n69}
\ncline[linecolor=black]{n62}{n69}
\ncline[linecolor=black]{n69}{n111}
\ncline[linecolor=black]{n54}{n69}
\ncline[linecolor=black]{n48}{n69}
\ncline[linecolor=black]{n69}{n73}
\ncline[linecolor=black]{n15}{n69}
\ncline[linecolor=black]{n31}{n69}
\ncline[linecolor=black]{n72}{n142}
\ncline[linecolor=black]{n72}{n114}
\ncline[linecolor=black]{n73}{n135}
\ncline[linecolor=black]{n54}{n73}
\ncline[linecolor=black]{n69}{n73}
\ncline[linecolor=black]{n48}{n73}
\ncline[linecolor=black]{n3}{n74}
\ncline[linecolor=black]{n5}{n74}
\ncline[linecolor=black]{n74}{n90}
\ncline[linecolor=black]{n74}{n91}
\ncline[linecolor=black]{n74}{n75}
\ncline[linecolor=black]{n27}{n74}
\ncline[linecolor=black]{n75}{n125}
\ncline[linecolor=black]{n27}{n75}
\ncline[linecolor=black]{n75}{n90}
\ncline[linecolor=black]{n3}{n75}
\ncline[linecolor=black]{n14}{n75}
\ncline[linecolor=black]{n67}{n75}
\ncline[linecolor=black]{n75}{n134}
\ncline[linecolor=black]{n5}{n75}
\ncline[linecolor=black]{n45}{n75}
\ncline[linecolor=black]{n74}{n75}
\ncline[linecolor=black]{n75}{n78}
\ncline[linecolor=black]{n75}{n106}
\ncline[linecolor=black]{n52}{n76}
\ncline[linecolor=black]{n27}{n76}
\ncline[linecolor=black]{n76}{n91}
\ncline[linecolor=black]{n76}{n109}
\ncline[linecolor=black]{n77}{n89}
\ncline[linecolor=black]{n13}{n77}
\ncline[linecolor=black]{n77}{n144}
\ncline[linecolor=black]{n78}{n106}
\ncline[linecolor=black]{n45}{n78}
\ncline[linecolor=black]{n78}{n125}
\ncline[linecolor=black]{n5}{n78}
\ncline[linecolor=black]{n27}{n78}
\ncline[linecolor=black]{n14}{n78}
\ncline[linecolor=black]{n67}{n78}
\ncline[linecolor=black]{n78}{n134}
\ncline[linecolor=black]{n3}{n78}
\ncline[linecolor=black]{n78}{n121}
\ncline[linecolor=black]{n75}{n78}
\ncline[linecolor=black]{n41}{n79}
\ncline[linecolor=black]{n79}{n101}
\ncline[linecolor=black]{n0}{n79}
\ncline[linecolor=black]{n79}{n142}
\ncline[linecolor=black]{n55}{n80}
\ncline[linecolor=black]{n80}{n92}
\ncline[linecolor=black]{n81}{n96}
\ncline[linecolor=black]{n82}{n99}
\ncline[linecolor=black]{n82}{n128}
\ncline[linecolor=black]{n82}{n104}
\ncline[linecolor=black]{n83}{n138}
\ncline[linecolor=black]{n83}{n118}
\ncline[linecolor=black]{n83}{n129}
\ncline[linecolor=black]{n1}{n84}
\ncline[linecolor=black]{n86}{n144}
\ncline[linecolor=black]{n13}{n86}
\ncline[linecolor=black]{n8}{n87}
\ncline[linecolor=black]{n87}{n107}
\ncline[linecolor=black]{n29}{n87}
\ncline[linecolor=black]{n49}{n88}
\ncline[linecolor=black]{n88}{n135}
\ncline[linecolor=black]{n77}{n89}
\ncline[linecolor=black]{n39}{n89}
\ncline[linecolor=black]{n75}{n90}
\ncline[linecolor=black]{n16}{n90}
\ncline[linecolor=black]{n15}{n90}
\ncline[linecolor=black]{n18}{n90}
\ncline[linecolor=black]{n74}{n90}
\ncline[linecolor=black]{n90}{n105}
\ncline[linecolor=black]{n90}{n125}
\ncline[linecolor=black]{n90}{n149}
\ncline[linecolor=black]{n3}{n90}
\ncline[linecolor=black]{n14}{n90}
\ncline[linecolor=black]{n67}{n90}
\ncline[linecolor=black]{n90}{n134}
\ncline[linecolor=black]{n90}{n111}
\ncline[linecolor=black]{n52}{n91}
\ncline[linecolor=black]{n3}{n91}
\ncline[linecolor=black]{n5}{n91}
\ncline[linecolor=black]{n27}{n91}
\ncline[linecolor=black]{n74}{n91}
\ncline[linecolor=black]{n76}{n91}
\ncline[linecolor=black]{n91}{n109}
\ncline[linecolor=black]{n80}{n92}
\ncline[linecolor=black]{n38}{n93}
\ncline[linecolor=black]{n43}{n95}
\ncline[linecolor=black]{n81}{n96}
\ncline[linecolor=black]{n1}{n97}
\ncline[linecolor=black]{n97}{n122}
\ncline[linecolor=black]{n98}{n139}
\ncline[linecolor=black]{n35}{n98}
\ncline[linecolor=black]{n47}{n98}
\ncline[linecolor=black]{n82}{n99}
\ncline[linecolor=black]{n2}{n100}
\ncline[linecolor=black]{n100}{n122}
\ncline[linecolor=black]{n1}{n100}
\ncline[linecolor=black]{n41}{n101}
\ncline[linecolor=black]{n58}{n101}
\ncline[linecolor=black]{n79}{n101}
\ncline[linecolor=black]{n21}{n102}
\ncline[linecolor=black]{n103}{n107}
\ncline[linecolor=black]{n29}{n103}
\ncline[linecolor=black]{n104}{n128}
\ncline[linecolor=black]{n2}{n104}
\ncline[linecolor=black]{n104}{n122}
\ncline[linecolor=black]{n82}{n104}
\ncline[linecolor=black]{n90}{n105}
\ncline[linecolor=black]{n45}{n106}
\ncline[linecolor=black]{n78}{n106}
\ncline[linecolor=black]{n14}{n106}
\ncline[linecolor=black]{n67}{n106}
\ncline[linecolor=black]{n106}{n134}
\ncline[linecolor=black]{n106}{n121}
\ncline[linecolor=black]{n106}{n125}
\ncline[linecolor=black]{n5}{n106}
\ncline[linecolor=black]{n31}{n106}
\ncline[linecolor=black]{n18}{n106}
\ncline[linecolor=black]{n27}{n106}
\ncline[linecolor=black]{n75}{n106}
\ncline[linecolor=black]{n29}{n107}
\ncline[linecolor=black]{n8}{n107}
\ncline[linecolor=black]{n87}{n107}
\ncline[linecolor=black]{n103}{n107}
\ncline[linecolor=black]{n52}{n109}
\ncline[linecolor=black]{n91}{n109}
\ncline[linecolor=black]{n76}{n109}
\ncline[linecolor=black]{n111}{n124}
\ncline[linecolor=black]{n62}{n111}
\ncline[linecolor=black]{n111}{n136}
\ncline[linecolor=black]{n16}{n111}
\ncline[linecolor=black]{n111}{n149}
\ncline[linecolor=black]{n111}{n113}
\ncline[linecolor=black]{n10}{n111}
\ncline[linecolor=black]{n11}{n111}
\ncline[linecolor=black]{n69}{n111}
\ncline[linecolor=black]{n18}{n111}
\ncline[linecolor=black]{n111}{n127}
\ncline[linecolor=black]{n48}{n111}
\ncline[linecolor=black]{n15}{n111}
\ncline[linecolor=black]{n90}{n111}
\ncline[linecolor=black]{n112}{n142}
\ncline[linecolor=black]{n112}{n128}
\ncline[linecolor=black]{n69}{n113}
\ncline[linecolor=black]{n113}{n124}
\ncline[linecolor=black]{n113}{n149}
\ncline[linecolor=black]{n113}{n136}
\ncline[linecolor=black]{n16}{n113}
\ncline[linecolor=black]{n48}{n113}
\ncline[linecolor=black]{n111}{n113}
\ncline[linecolor=black]{n113}{n135}
\ncline[linecolor=black]{n18}{n113}
\ncline[linecolor=black]{n54}{n113}
\ncline[linecolor=black]{n62}{n113}
\ncline[linecolor=black]{n113}{n145}
\ncline[linecolor=black]{n10}{n113}
\ncline[linecolor=black]{n15}{n113}
\ncline[linecolor=black]{n11}{n113}
\ncline[linecolor=black]{n113}{n127}
\ncline[linecolor=black]{n33}{n114}
\ncline[linecolor=black]{n17}{n114}
\ncline[linecolor=black]{n36}{n114}
\ncline[linecolor=black]{n35}{n114}
\ncline[linecolor=black]{n72}{n114}
\ncline[linecolor=black]{n115}{n130}
\ncline[linecolor=black]{n116}{n132}
\ncline[linecolor=black]{n117}{n123}
\ncline[linecolor=black]{n118}{n138}
\ncline[linecolor=black]{n83}{n118}
\ncline[linecolor=black]{n7}{n120}
\ncline[linecolor=black]{n32}{n120}
\ncline[linecolor=black]{n51}{n120}
\ncline[linecolor=black]{n20}{n120}
\ncline[linecolor=black]{n14}{n121}
\ncline[linecolor=black]{n67}{n121}
\ncline[linecolor=black]{n121}{n134}
\ncline[linecolor=black]{n106}{n121}
\ncline[linecolor=black]{n121}{n125}
\ncline[linecolor=black]{n45}{n121}
\ncline[linecolor=black]{n78}{n121}
\ncline[linecolor=black]{n1}{n122}
\ncline[linecolor=black]{n2}{n122}
\ncline[linecolor=black]{n104}{n122}
\ncline[linecolor=black]{n97}{n122}
\ncline[linecolor=black]{n100}{n122}
\ncline[linecolor=black]{n122}{n128}
\ncline[linecolor=black]{n117}{n123}
\ncline[linecolor=black]{n124}{n136}
\ncline[linecolor=black]{n111}{n124}
\ncline[linecolor=black]{n124}{n149}
\ncline[linecolor=black]{n69}{n124}
\ncline[linecolor=black]{n113}{n124}
\ncline[linecolor=black]{n16}{n124}
\ncline[linecolor=black]{n62}{n124}
\ncline[linecolor=black]{n18}{n124}
\ncline[linecolor=black]{n10}{n124}
\ncline[linecolor=black]{n48}{n124}
\ncline[linecolor=black]{n15}{n124}
\ncline[linecolor=black]{n11}{n124}
\ncline[linecolor=black]{n75}{n125}
\ncline[linecolor=black]{n14}{n125}
\ncline[linecolor=black]{n67}{n125}
\ncline[linecolor=black]{n125}{n134}
\ncline[linecolor=black]{n27}{n125}
\ncline[linecolor=black]{n121}{n125}
\ncline[linecolor=black]{n45}{n125}
\ncline[linecolor=black]{n78}{n125}
\ncline[linecolor=black]{n106}{n125}
\ncline[linecolor=black]{n3}{n125}
\ncline[linecolor=black]{n90}{n125}
\ncline[linecolor=black]{n5}{n125}
\ncline[linecolor=black]{n40}{n126}
\ncline[linecolor=black]{n50}{n126}
\ncline[linecolor=black]{n126}{n146}
\ncline[linecolor=black]{n44}{n126}
\ncline[linecolor=black]{n11}{n127}
\ncline[linecolor=black]{n10}{n127}
\ncline[linecolor=black]{n48}{n127}
\ncline[linecolor=black]{n111}{n127}
\ncline[linecolor=black]{n113}{n127}
\ncline[linecolor=black]{n127}{n135}
\ncline[linecolor=black]{n127}{n130}
\ncline[linecolor=black]{n104}{n128}
\ncline[linecolor=black]{n2}{n128}
\ncline[linecolor=black]{n112}{n128}
\ncline[linecolor=black]{n82}{n128}
\ncline[linecolor=black]{n128}{n143}
\ncline[linecolor=black]{n122}{n128}
\ncline[linecolor=black]{n129}{n131}
\ncline[linecolor=black]{n6}{n129}
\ncline[linecolor=black]{n83}{n129}
\ncline[linecolor=black]{n115}{n130}
\ncline[linecolor=black]{n127}{n130}
\ncline[linecolor=black]{n129}{n131}
\ncline[linecolor=black]{n6}{n131}
\ncline[linecolor=black]{n116}{n132}
\ncline[linecolor=black]{n14}{n134}
\ncline[linecolor=black]{n67}{n134}
\ncline[linecolor=black]{n106}{n134}
\ncline[linecolor=black]{n125}{n134}
\ncline[linecolor=black]{n45}{n134}
\ncline[linecolor=black]{n121}{n134}
\ncline[linecolor=black]{n18}{n134}
\ncline[linecolor=black]{n75}{n134}
\ncline[linecolor=black]{n78}{n134}
\ncline[linecolor=black]{n5}{n134}
\ncline[linecolor=black]{n27}{n134}
\ncline[linecolor=black]{n16}{n134}
\ncline[linecolor=black]{n31}{n134}
\ncline[linecolor=black]{n90}{n134}
\ncline[linecolor=black]{n3}{n134}
\ncline[linecolor=black]{n48}{n135}
\ncline[linecolor=black]{n113}{n135}
\ncline[linecolor=black]{n73}{n135}
\ncline[linecolor=black]{n11}{n135}
\ncline[linecolor=black]{n127}{n135}
\ncline[linecolor=black]{n30}{n135}
\ncline[linecolor=black]{n88}{n135}
\ncline[linecolor=black]{n124}{n136}
\ncline[linecolor=black]{n62}{n136}
\ncline[linecolor=black]{n69}{n136}
\ncline[linecolor=black]{n111}{n136}
\ncline[linecolor=black]{n113}{n136}
\ncline[linecolor=black]{n136}{n149}
\ncline[linecolor=black]{n16}{n136}
\ncline[linecolor=black]{n18}{n136}
\ncline[linecolor=black]{n10}{n136}
\ncline[linecolor=black]{n15}{n136}
\ncline[linecolor=black]{n11}{n136}
\ncline[linecolor=black]{n48}{n136}
\ncline[linecolor=black]{n54}{n136}
\ncline[linecolor=black]{n118}{n138}
\ncline[linecolor=black]{n83}{n138}
\ncline[linecolor=black]{n98}{n139}
\ncline[linecolor=black]{n35}{n139}
\ncline[linecolor=black]{n23}{n139}
\ncline[linecolor=black]{n26}{n139}
\ncline[linecolor=black]{n141}{n147}
\ncline[linecolor=black]{n112}{n142}
\ncline[linecolor=black]{n72}{n142}
\ncline[linecolor=black]{n41}{n142}
\ncline[linecolor=black]{n79}{n142}
\ncline[linecolor=black]{n2}{n143}
\ncline[linecolor=black]{n128}{n143}
\ncline[linecolor=black]{n13}{n144}
\ncline[linecolor=black]{n86}{n144}
\ncline[linecolor=black]{n77}{n144}
\ncline[linecolor=black]{n20}{n144}
\ncline[linecolor=black]{n54}{n145}
\ncline[linecolor=black]{n113}{n145}
\ncline[linecolor=black]{n44}{n146}
\ncline[linecolor=black]{n126}{n146}
\ncline[linecolor=black]{n141}{n147}
\psset{dotstyle=Bo}
\dotnode[](1.108338,3.858886){n0}
\dotnode[](2.527444,2.445348){n1}
\dotnode[](2.017666,2.841472){n2}
\dotnode[](0.6785566,0.8802483){n3}
\dotnode[](1.257860,2.892203){n4}
\dotnode[](0.8247274,0.631479){n5}
\dotnode[](4.581229,3.210751){n6}
\dotnode[](3.098514,4.866756){n7}
\dotnode[](4.591683,2.878327){n8}
\dotnode[](4,2.2){n9}
\dotnode[](2.888154,1.090116){n10}
\dotnode[](3.105909,0.7401545){n11}
\dotnode[](0.3962556,3.984627){n12}
\dotnode[](2.661068,4.46547){n13}
\dotnode[](1.439636,1.295117){n14}
\dotnode[](2.236885,1.274193){n15}
\dotnode[](2.024484,0.7152085){n16}
\dotnode[](0.6005647,3.371056){n17}
\dotnode[](1.814559,0.4981684){n18}
\dotnode[](4.08826,1.970573){n19}
\dotnode[](2.577236,4.740179){n20}
\dotnode[](3.773692,3.552994){n21}
\dotnode[](3.913548,0.4282107){n22}
\dotnode[](0.711105,2.899006){n23}
\dotnode[](1.232538,4.29378){n24}
\dotnode[](2.521529,1.412281){n25}
\dotnode[](0.9249148,2.893182){n26}
\dotnode[](0.5726195,1.294175){n27}
\dotnode[](3.759983,2.172351){n28}
\dotnode[](4.7622,2.335772){n29}
\dotnode[](3.665883,0.6603324){n30}
\dotnode[](1.422528,0.3128956){n31}
\dotnode[](2.557950,5){n32}
\dotnode[](0.1556468,2.998522){n33}
\dotnode[](1.403207,1.916558){n34}
\dotnode[](0.3174471,2.600607){n35}
\dotnode[](0.1307679,3.277233){n36}
\dotnode[](2.901075,1.575363){n37}
\dotnode[](2.55,3.97){n38}
\dotnode[](3.353207,4.403764){n39}
\dotnode[](0.9114814,4.353879){n40}
\dotnode[](1.686579,3.819467){n41}
\dotnode[](4.209552,3.016675){n42}
\dotnode[](3.891111,1.748183){n43}
\dotnode[](0.4179513,3.522051){n44}
\dotnode[](0.8306748,1.037910){n45}
\dotnode[](4.389942,2.082443){n46}
\dotnode[](0.1748816,2.281055){n47}
\dotnode[](3.273419,0.6181322){n48}
\dotnode[](3.66405,0.2172084){n49}
\dotnode[](0.5841016,3.811024){n50}
\dotnode[](2.24852,4.942186){n51}
\dotnode[](0.2341176,1.037886){n52}
\dotnode[](4.969442,3.079323){n53}
\dotnode[](2.864019,0){n54}
\dotnode[](5,2.809783){n55}
\dotnode[](3.97574,2.571348){n56}
\dotnode[](4.371287,1.534671){n57}
\dotnode[](1.967572,3.868347){n58}
\dotnode[](0,2.563469){n59}
\dotnode[](3.764707,4.352432){n60}
\dotnode[](3.737104,3.981957){n61}
\dotnode[](2.127025,0.3401537){n62}
\dotnode[](2.312403,1.030709){n63}
\dotnode[](0.2217693,3.701035){n64}
\dotnode[](0.6466549,4.297944){n65}
\dotnode[](3.364475,4.745668){n66}
\dotnode[](1.509306,0.9905115){n67}
\dotnode[](1.631241,0.08123197){n68}
\dotnode[](2.363231,0.03982867){n69}
\dotnode[](4.688611,1.662437){n70}
\dotnode[](4.32925,1.149353){n71}
\dotnode[](1.004251,3.3599){n72}
\dotnode[](3.170304,0.08031329){n73}
\dotnode[](1.092846,0.4743804){n74}
\dotnode[](0.957797,1.457980){n75}
\dotnode[](0.4145735,1.587721){n76}
\dotnode[](2.550006,4.241430){n77}
\dotnode[](0.8983155,1.243353){n78}
\dotnode[](1.560778,4.1743){n79}
\dotnode[](4.791386,3.470398){n80}
\dotnode[](1.183687,2.360412){n81}
\dotnode[](2.078577,3.201014){n82}
\dotnode[](4.135811,4.087185){n83}
\dotnode[](3.153883,3.140639){n84}
\dotnode[](0.6058247,2.060042){n85}
\dotnode[](1.891242,4.717356){n86}
\dotnode[](4.792091,2.606388){n87}
\dotnode[](3.635776,1.016917){n88}
\dotnode[](3.202631,4.133673){n89}
\dotnode[](1.765180,1.228386){n90}
\dotnode[](0.4301581,0.7624465){n91}
\dotnode[](4.292557,3.369169){n92}
\dotnode[](2.64,3.705){n93}
\dotnode[](2.869689,4.205452){n94}
\dotnode[](4.306428,1.798957){n95}
\dotnode[](0.9419618,2.164323){n96}
\dotnode[](2.802984,2.471939){n97}
\dotnode[](0.4990411,2.364772){n98}
\dotnode[](2.265986,3.781169){n99}
\dotnode[](2.886530,2.782621){n100}
\dotnode[](2.074548,4.120629){n101}
\dotnode[](3.578431,3.339056){n102}
\dotnode[](4.396008,2.731498){n103}
\dotnode[](2.161968,2.481101){n104}
\dotnode[](1.915954,1.686363){n105}
\dotnode[](1.073082,0.7635667){n106}
\dotnode[](4.351759,2.410157){n107}
\dotnode[](0.1632209,1.576404){n108}
\dotnode[](0.1964432,1.278737){n109}
\dotnode[](0.5000101,2.853756){n110}
\dotnode[](2.625767,0.8297721){n111}
\dotnode[](1.705716,3.114971){n112}
\dotnode[](2.796314,0.5814426){n113}
\dotnode[](0.3549037,3.121353){n114}
\dotnode[](3.949152,0.8549694){n115}
\dotnode[](4.562201,1.223617){n116}
\dotnode[](3.053188,3.925351){n117}
\dotnode[](4.318411,3.852927){n118}
\dotnode[](2.890681,3.735371){n119}
\dotnode[](2.880736,4.879838){n120}
\dotnode[](1.377059,0.7879707){n121}
\dotnode[](2.617372,2.90596){n122}
\dotnode[](3.046366,4.518936){n123}
\dotnode[](2.621703,0.4200536){n124}
\dotnode[](1.203850,1.499658){n125}
\dotnode[](0.9849086,4.079775){n126}
\dotnode[](3.254325,1.04113){n127}
\dotnode[](2.280932,2.838994){n128}
\dotnode[](4.117188,3.717776){n129}
\dotnode[](3.471599,1.285363){n130}
\dotnode[](4.500053,3.637197){n131}
\dotnode[](4.069486,1.411759){n132}
\dotnode[](4.11247,0.5910543){n133}
\dotnode[](1.252724,1.071603){n134}
\dotnode[](3.385955,0.3694225){n135}
\dotnode[](2.407195,0.4605674){n136}
\dotnode[](1.689756,4.453583){n137}
\dotnode[](3.989081,3.406712){n138}
\dotnode[](0.710692,2.532491){n139}
\dotnode[](4.873979,1.934726){n140}
\dotnode[](3.645946,2.645275){n141}
\dotnode[](1.606368,3.440222){n142}
\dotnode[](2.445606,3.296266){n143}
\dotnode[](2.274105,4.535248){n144}
\dotnode[](2.648264,0.1809270){n145}
\dotnode[](0.7976457,3.6483){n146}
\dotnode[](3.950723,2.918283){n147}
\dotnode[](4.776492,1.394465){n148}
\dotnode[](2.296867,0.7074797){n149}

        \end{pspicture}
        }
    }
\end{figure}
Edge weight given as
\begin{align*}
    a_{ij} =
    \exp \lB\{-{\norm{\bx_i - \bx_{j}}^2 \over 2\sigma^2} \rB\}
\end{align*}
using $\sigma=1$. An edge exists if both nodes are mutual nearest
neighbors among top 15 neighbors. Self-loops not shown.
\end{frame}



\begin{frame}{Graphs and Matrices: Degree Matrix}
For a vertex $\bx_i$, let $d_i$ denote the {\em degree} of the
vertex, def\/{i}ned as
\begin{align*}
    d_i = \sum_{j=1}^n a_{ij}
\end{align*}

We def\/{i}ne the {\em
degree matrix} $\bDelta$ of graph $G$ as the $n \times n$ diagonal
matrix:
\begin{align*}
    \bDelta = \matr{
      d_1 & 0 & \cdots & 0 \\
      0 & d_2 & \cdots & 0\\
      \vdots & \vdots & \ddots & \vdots\\
      0 & 0 &\cdots  & d_n \\
    } =
    \matr{
      \sum_{j=1}^n a_{1j} & 0 & \cdots & 0 \\
      0 & \sum_{j=1}^n a_{2j} & \cdots & 0\\
      \vdots & \vdots & \ddots & \vdots\\
      0 & 0 &\cdots  & \sum_{j=1}^n a_{nj} \\
    }
\end{align*}
$\bDelta$ can be compactly written as $\bDelta(i,i) = d_i$ for all
$1 \le i \le n$.
\end{frame}



\begin{frame}{Graphs and Matrices: Normalized Adjacency Matrix}
The normalized adjacency
matrix is obtained by dividing each row of the adjacency matrix by
the degree of the corresponding node. Given the weighted 
adjacency
matrix $\bA$ for a graph $G$, its normalized adjacency matrix is
def\/{i}ned as
\begin{align*}
    \bM =  \bDelta^{-1} \bA & =
    \matr{
        {a_{11}\over d_1} &  {a_{12}\over d_1} &
        \cdots &  {a_{1n}\over d_1}\\[1ex]
        {a_{21}\over d_2} &  {a_{22}\over d_2} &
        \cdots &  {a_{2n}\over d_2}\\[1ex]
        \vdots & \vdots & \ddots & \vdots\\
        {a_{n1}\over d_n} &  {a_{n2}\over d_n} & \cdots &
        {a_{nn}\over d_n}\\
    }
\end{align*}
Because $\bA$ is assumed to have non-negative elements, this implies
that each element of $\bM$, namely $m_{ij}$ is also non-negative,
as $m_{ij} = \tfrac{a_{ij}}{d_i} \ge 0$. 


\medskip
Each row in $\bM$ sums to $1$, which implies that $1$ is an
eigenvalue of $\bM$. In fact, $\lambda_1 = 1$ is the largest
eigenvalue of $\bM$, and the other eigenvalues satisfy the
property that $|\lambda_i| \le 1$. 
Because $\bM$
is not symmetric, its eigenvectors are not necessarily orthogonal.
\end{frame}


\begin{frame}{Example Graph: Adjacency and Degree Matrices}
\begin{figure}
    \centerline{
	\scalebox{0.7}{
        \psset{unit=0.75in,dotscale=2,fillcolor=lightgray,dotstyle=Bo}
        \begin{pspicture}(0,-0.25)(0,2.25)
            \psmatrix[mnode=circle]
                & [name=a] 1 & & & [name=f]6\\
                [name=b]2 & & [name=d]4 & [name=e]5 & [mnode=none]\\
                & [name=c]3  & & & [name=g]7
            \endpsmatrix
   %         \circlenode*(2,3){a}{1}
            %\circlenode*(1,2){b}{2}
            %\circlenode*(2,1){c}{3}
            %\circlenode*(3,2){d}{4}
            %\circlenode*(4,2){e}{5}
            %\circlenode*(5,3){f}{6}
            %\circlenode*(5,1){g}{7}
            \ncline{a}{b}
            \ncline{a}{d}
            \ncline{a}{f}
            \ncline{b}{c}
            \ncline{b}{d}
            \ncline{c}{d}
            \ncline{c}{g}
            \ncline{d}{e}
            \ncline{e}{f}
            \ncline{e}{g}
            \ncline{f}{g}
        \end{pspicture}
		}}
 \end{figure}
 \small
    Its adjacency and degree matrices are given as
    \begin{align*}
        \bA &= \matr{0 & 1 & 0 & 1 & 0 & 1 & 0\\
                    1 & 0 & 1 & 1 & 0 & 0 & 0\\
                    0 & 1 & 0 & 1 & 0 & 0 & 1\\
                    1 & 1 & 1 & 0 & 1 & 0 & 0\\
                    0 & 0 & 0 & 1 & 0 & 1 & 1\\
                    1 & 0 & 0 & 0 & 1 & 0 & 1\\
                    0 & 0 & 1 & 0 & 1 & 1 & 0} &
        \bDelta & = \matr{
                3 & 0 & 0 & 0 & 0 & 0 & 0\\
                0 & 3 & 0 & 0 & 0 & 0 & 0\\
                0 & 0 & 3 & 0 & 0 & 0 & 0\\
                0 & 0 & 0 & 4 & 0 & 0 & 0\\
                0 & 0 & 0 & 0 & 3 & 0 & 0\\
                0 & 0 & 0 & 0 & 0 & 3 & 0\\
                0 & 0 & 0 & 0 & 0 & 0 & 3}
    \end{align*}
\end{frame}


\begin{frame}{Example Graph: Normalized Adjacency Matrix}
\begin{figure}
    \centerline{
	\scalebox{0.7}{
        \psset{unit=0.75in,dotscale=2,fillcolor=lightgray,dotstyle=Bo}
        \begin{pspicture}(0,-0.25)(0,2.25)
            \psmatrix[mnode=circle]
                & [name=a] 1 & & & [name=f]6\\
                [name=b]2 & & [name=d]4 & [name=e]5 & [mnode=none]\\
                & [name=c]3  & & & [name=g]7
            \endpsmatrix
   %         \circlenode*(2,3){a}{1}
            %\circlenode*(1,2){b}{2}
            %\circlenode*(2,1){c}{3}
            %\circlenode*(3,2){d}{4}
            %\circlenode*(4,2){e}{5}
            %\circlenode*(5,3){f}{6}
            %\circlenode*(5,1){g}{7}
            \ncline{a}{b}
            \ncline{a}{d}
            \ncline{a}{f}
            \ncline{b}{c}
            \ncline{b}{d}
            \ncline{c}{d}
            \ncline{c}{g}
            \ncline{d}{e}
            \ncline{e}{f}
            \ncline{e}{g}
            \ncline{f}{g}
        \end{pspicture}
		}}
		\vspace{-0.2in}
 \end{figure}
 \small
    The normalized adjacency matrix is as follows:
    \begin{align*}
        \bM & = \bDelta^{-1}\bA =
        \amatr{r}{
        0& 0.33& 0& 0.33& 0& 0.33& 0\\
        0.33& 0& 0.33& 0.33& 0& 0& 0\\
        0& 0.33& 0& 0.33& 0& 0& 0.33\\
        0.25& 0.25& 0.25& 0& 0.25& 0& 0\\
        0& 0& 0& 0.33& 0& 0.33& 0.33\\
        0.33& 0& 0& 0& 0.33& 0& 0.33\\
        0& 0& 0.33& 0& 0.33& 0.33& 0\\
        }
    \end{align*}
    The eigenvalues of $\bM$ are:
    $\lambda_1   = 1 $,
    $\lambda_2  =0.483 $,
    $\lambda_3  =0.206$,
    $\lambda_4  = -0.045$,
    $\lambda_5  =-0.405 $,
    $\lambda_6  = -0.539 $,
    $\lambda_7 =-0.7$
  \end{frame}



\begin{frame}{Graph Laplacian Matrix}
The {\em
Laplacian matrix} of a graph is def\/{i}ned as
\begin{align*}
    \bL & = \bDelta - \bA \nonumber\\ %\label{eq:clus:spectral:L}\\
    & =     \matr{
      \sum_{j=1}^n a_{1j} & 0 & \cdots & 0 \\
      0 & \sum_{j=1}^n a_{2j} & \cdots & 0\\
      \vdots & \vdots & \ddots & \vdots\\
      0 & 0 &\cdots  & \sum_{j=1}^n a_{nj} \\
    }
    -
    \matr{
        a_{11} & a_{12} & \cdots & a_{1n}\\
        a_{21} & a_{22} & \cdots & a_{2n}\\
        \vdots & \vdots & \cdots & \vdots\\
        a_{n1} & a_{n2} & \cdots & a_{nn}\\
    } \notag\\
     & =    \matr{
    \sum_{j\ne1} a_{1j} & -a_{12} & \cdots & -a_{1n} \\
    -a_{21} & \sum_{j\ne2} a_{2j} & \cdots & -a_{2n}\\
      \vdots & \vdots & \cdots & \vdots\\
      -a_{n1} & -a_{n2} &\cdots  & \sum_{j\ne n} a_{nj} \\
    } 
\end{align*}

$\bL$ is a symmetric, positive
semidef\/{i}nite matrix.
This means that $\bL$ has $n$ real, non-negative eigenvalues,
which can be arranged in decreasing order as follows: $\lambda_1
\ge \lambda_2 \ge \cdots \ge \lambda_n \ge 0$.  Because $\bL$ is
symmetric, its eigenvectors are orthonormal. 
The rank of $\bL$ is at most $n-1$, and the
smallest eigenvalue is $\lambda_n = 0$.
\end{frame}



\begin{frame}{Example Graph: Laplacian Matrix}
\begin{figure}
    \centerline{
	\scalebox{0.7}{
        \psset{unit=0.75in,dotscale=2,fillcolor=lightgray,dotstyle=Bo}
        \begin{pspicture}(0,-0.25)(0,2.25)
            \psmatrix[mnode=circle]
                & [name=a] 1 & & & [name=f]6\\
                [name=b]2 & & [name=d]4 & [name=e]5 & [mnode=none]\\
                & [name=c]3  & & & [name=g]7
            \endpsmatrix
   %         \circlenode*(2,3){a}{1}
            %\circlenode*(1,2){b}{2}
            %\circlenode*(2,1){c}{3}
            %\circlenode*(3,2){d}{4}
            %\circlenode*(4,2){e}{5}
            %\circlenode*(5,3){f}{6}
            %\circlenode*(5,1){g}{7}
            \ncline{a}{b}
            \ncline{a}{d}
            \ncline{a}{f}
            \ncline{b}{c}
            \ncline{b}{d}
            \ncline{c}{d}
            \ncline{c}{g}
            \ncline{d}{e}
            \ncline{e}{f}
            \ncline{e}{g}
            \ncline{f}{g}
        \end{pspicture}
		}}
		\vspace{-0.2in}
 \end{figure}
\small
 The graph Laplacian is given as
    \begin{align*}
        \bL = \bDelta-\bA =
        \amatr{r}{
                 3 &  -1 &   0 &  -1 &   0 &  -1 &   0\\
                -1 &   3 &  -1 &  -1 &   0 &   0 &   0\\
                 0 &  -1 &   3 &  -1 &   0 &   0 &  -1\\
                -1 &  -1 &  -1 &   4 &  -1 &   0 &   0\\
                 0 &   0 &   0 &  -1 &   3 &  -1 &  -1\\
                -1 &   0 &   0 &   0 &  -1 &   3 &  -1\\
                 0 &   0 &  -1 &   0 &  -1 &  -1 &   3}
    \end{align*}
    The eigenvalues of $\bL$ are as follows:
    $\lambda_1= 5.618$,
    $\lambda_2=4.618$,
    $\lambda_3= 4.414 $,
    $\lambda_4=3.382$,
    $\lambda_5=2.382 $,
    $\lambda_6=1.586 $,
    $\lambda_7= 0$  
\end{frame}


\begin{frame}{Normalized Laplacian Matrices}
  \small
The {\em normalized
symmetric Laplacian matrix} of a graph is def\/{i}ned as
\begin{align*}
    \bL^s & = \bDelta^{-1/2}\bL\bDelta^{-1/2}
          =    \matr{
     {\sum_{j\ne1} a_{1j} \over \sqrt{d_1d_1}} & -{a_{12} \over
     \sqrt{d_1d_2}} & \cdots & -{a_{1n} \over \sqrt{d_1d_n}}
     \\[1ex]
     -{a_{21} \over \sqrt{d_2d_1}} & {\sum_{j\ne2}a_{2j} \over
     \sqrt{d_2d_2}} & \cdots & -{a_{2n} \over \sqrt{d_2d_n}} \\
      \vdots & \vdots & \ddots & \vdots\\
      -{a_{n1} \over \sqrt{d_nd_1}} & -{a_{n2} \over
      \sqrt{d_nd_2}} & \cdots & {\sum_{j\ne n}a_{nj} \over
      \sqrt{d_nd_n}} \\
      } 
\end{align*}
$\bL^s$ is a symmetric, positive semidef\/{i}nite matrix, with 
rank at most $n-1$. The
smallest eigenvalue $\lambda_n = 0$.

The {\em normalized asymmetric Laplacian} matrix is def\/{i}ned as
\begin{align*}
    \bL^a &= \bDelta^{-1} \bL
     = \matr{
    {\sum_{j\ne1} a_{1j} \over d_1} & -{a_{12} \over d_1} &
    \cdots & -{a_{1n} \over d_1} \\[1ex]
    -{a_{21}\over d_2} & {\sum_{j\ne2} a_{2j} \over d_2}
    & \cdots & -{a_{2n} \over d_2}\\
      \vdots & \vdots & \ddots & \vdots\\
      -{a_{n1} \over d_n} & -{a_{n2} \over d_n} &\cdots
      & {\sum_{j\ne n} a_{nj} \over d_n} \\
    }
\end{align*}
$\bL^a$ is also a
positive semi-def\/{i}nite matrix with $n$ real eigenvalues $\lambda_1
\ge \lambda_2 \ge \dots \ge \lambda_n = 0$.
\end{frame}



\begin{frame}{Example Graph: Normalized Symmetric Laplacian Matrix}
\begin{figure}
    \centerline{
	\scalebox{0.6}{
        \psset{unit=0.75in,dotscale=2,fillcolor=lightgray,dotstyle=Bo}
        \begin{pspicture}(0,-0.25)(0,2.25)
            \psmatrix[mnode=circle]
                & [name=a] 1 & & & [name=f]6\\
                [name=b]2 & & [name=d]4 & [name=e]5 & [mnode=none]\\
                & [name=c]3  & & & [name=g]7
            \endpsmatrix
            \ncline{a}{b}
            \ncline{a}{d}
            \ncline{a}{f}
            \ncline{b}{c}
            \ncline{b}{d}
            \ncline{c}{d}
            \ncline{c}{g}
            \ncline{d}{e}
            \ncline{e}{f}
            \ncline{e}{g}
            \ncline{f}{g}
        \end{pspicture}
		}}
		\vspace{-0.2in}
 \end{figure}
\small
The    normalized symmetric Laplacian is given as
    \begin{align*}
        \bL^s & = \amatr{r}{
     1 &-0.33 & 0 &-0.29 & 0 &-0.33 & 0\\
    -0.33 & 1 &-0.33 &-0.29 & 0 & 0 & 0\\
     0 &-0.33 & 1 &-0.29 & 0 & 0 &-0.33\\
    -0.29 &-0.29 &-0.29 & 1 &-0.29 & 0 & 0\\
     0 & 0 & 0 &-0.29 & 1 &-0.33 &-0.33\\
    -0.33 & 0 & 0 & 0 &-0.33 & 1 &-0.33\\
     0 & 0 &-0.33 & 0 &-0.33 &-0.33 & 1\\
    }
    \end{align*}
    The eigenvalues of $\bL^s$ are as follows:
    $\lambda_1 = 1.7$,
    $\lambda_2 =1.539$,
    $\lambda_3 = 1.405$,
    $\lambda_4 =1.045$,
    $\lambda_5 =0.794$,
    $\lambda_6 =0.517$,
    $\lambda_7 = 0 $
\end{frame}



\begin{frame}{Example Graph: Normalized Asymmetric Laplacian Matrix}
\begin{figure}
    \centerline{
	\scalebox{0.6}{
        \psset{unit=0.75in,dotscale=2,fillcolor=lightgray,dotstyle=Bo}
        \begin{pspicture}(0,-0.25)(0,2.25)
            \psmatrix[mnode=circle]
                & [name=a] 1 & & & [name=f]6\\
                [name=b]2 & & [name=d]4 & [name=e]5 & [mnode=none]\\
                & [name=c]3  & & & [name=g]7
            \endpsmatrix
            \ncline{a}{b}
            \ncline{a}{d}
            \ncline{a}{f}
            \ncline{b}{c}
            \ncline{b}{d}
            \ncline{c}{d}
            \ncline{c}{g}
            \ncline{d}{e}
            \ncline{e}{f}
            \ncline{e}{g}
            \ncline{f}{g}
        \end{pspicture}
		}}
		\vspace{-0.2in}
 \end{figure}
\small
    The normalized asymmetric Laplacian matrix is given as
    \begin{align*}
        \bL^a = \bDelta^{-1}\bL =
        \amatr{r}{
         1   & -0.33&  0   & -0.33&  0   & -0.33&  0\\
        -0.33&  1   & -0.33& -0.33&  0   &  0   &  0\\
         0   & -0.33&  1   & -0.33&  0   &  0   & -0.33\\
        -0.25& -0.25& -0.25&  1   & -0.25&  0   &  0\\
         0   &  0   &  0   & -0.33&  1   & -0.33& -0.33\\
        -0.33&  0   &  0   &  0   & -0.33&  1   & -0.33\\
         0   &  0   & -0.33&  0   & -0.33& -0.33&  1}
    \end{align*}
    The eigenvalues of $\bL^a$ are identical to those for $\bL^s$,
    namely
    $\lambda_1 = 1.7$,
    $\lambda_2 =1.539$,
    $\lambda_3 = 1.405$,
    $\lambda_4 =1.045$,
    $\lambda_5 =0.794$,
    $\lambda_6 =0.517$,
    $\lambda_7 = 0 $
  \end{frame}


\begin{frame}{Clustering as Graph Cuts}
A {\em $k$-way cut} in a graph is a partitioning or clustering of
the vertex set, given as $\cC = \{C_1, \ldots, C_k\}$.
We require $\cC$ to optimize
some objective function that captures the intuition that nodes
within a cluster should have high similarity, and nodes from
different clusters should have low similarity.

\medskip
Given a weighted graph $G$ def\/{i}ned by its similarity matrix $\bA$,
let $S, T \subseteq V$ be any two
subsets of the vertices. We denote by $W(S,T)$ the sum of the
weights on all edges with one vertex in $S$ and the other in $T$,
given as
\begin{align*}
    W(S, T) = \sum_{v_i \in S} \sum_{v_{j} \in T} a_{ij}
\end{align*}

\medskip
Given $S \subseteq V$, we denote by $\ol{S}$ the complementary set
of vertices, that is, $\ol{S} = V - S$. A {\em (vertex) cut} in a
graph is def\/{i}ned as a partitioning of $V$ into $S \subset V$ and
$\ol{S}$. The {\em weight of the cut} or {\em cut weight} is
def\/{i}ned as the sum of all the weights on
edges between vertices in $S$
and $\ol{S}$, given as $W(S, \ol{S})$.
\end{frame}


\begin{frame}{Cuts and Matrix Operations}
  \small
Given a clustering $\cC = \{C_1,\dots,C_k\}$ comprising $k$ clusters.
Let $\bc_i \in \{0,1\}^n$ be the {\em cluster indicator vector}
that records the cluster membership for cluster $C_i$, def\/{i}ned as
\begin{align*}
    c_{ij} =
        \begin{cases}
            1 & \text{if $v_{j} \in C_i$}\\
            0 & \text{if $v_{j} \not\in C_i$}\\
        \end{cases}
\end{align*}
The cluster size can be written as
\begin{align*}
    \card{C_i} = \bc_i^T\bc_i = \norm{\bc_i}^2
\end{align*}
The {\em volume} of a cluster
$C_i$ is def\/{i}ned as the sum of all the weights on edges with one
end in cluster $C_i$:
\begin{align*}
    vol(C_i) = W(C_i, V)  = \sum_{v_r \in C_i} d_r
    = \sum_{v_r \in C_i} c_{ir} d_r c_{ir}
     = \sum_{r=1}^n \sum_{s=1}^n c_{ir} \bDelta_{rs} c_{is}
     = \bc_i^T \bDelta \bc_i
\end{align*}
The sum of weights of all internal edges is:
\begin{align*}
    W(C_i, C_i) & = \sum_{v_r \in C_i} \sum_{v_s \in C_i}
    a_{rs}
     = \sum_{r=1}^n \sum_{s=1}^n c_{ir} a_{rs} c_{is} = \bc_i^T
    \bA \bc_i
\end{align*}
We can get the sum of weights for all the external edges as follows:
\begin{align*}
    W(C_i, \ol{C_i}) & = \sum_{v_r \in C_i} \sum_{v_s \in
    V- C_i} a_{rs} = W(C_i,V) - W(C_i,C_i)
     = \bc_i (\bDelta - \bA) \bc_i = \bc_i^T \bL \bc_i
\end{align*}
\end{frame}



\begin{frame}{Clustering Objective Functions: Ratio Cut}
The clustering objective function can be formulated as an
optimization \hbox{problem} over the $k$-way cut $\cC =
\{C_1,\dots,C_k\}$. 

The {\em ratio cut} objective is def\/{i}ned
over a $k$-way cut as follows:
\begin{align*}
    \min_\cC \; J_{rc}(\cC) & = \sum_{i=1}^k {W(C_i, \ol{C_i}) \over
    \card{C_i}} =  \sum_{i=1}^k {\bc_i^T \bL \bc_i \over
    \bc_i^T\bc_i} = \sum_{i=1}^k {\bc_i^T \bL \bc_i \over
    \norm{\bc_i}^2}
\end{align*}
Ratio cut tries to minimize the sum of the similarities from a
cluster $C_i$ to other points not in the cluster $\ol{C_i}$,
taking into account the size of each cluster. 


Unfortunately, for binary cluster indicator vectors $\bc_i$, the
ratio cut objective is NP-hard. An obvious relaxation is to allow
$\bc_i$ to take on any real value. In this case, we can rewrite
the objective as
\begin{align*}
    \min_\cC \; J_{rc}(\cC) = \sum_{i=1}^k {\bc_i^T \bL \bc_i
    \over \norm{\bc_i}^2} =
    \sum_{i=1}^k \cramped{\lB({\bc_i \over \norm{\bc_i}}\rB)^T}
        \bL
        \cramped{\lB({\bc_i \over \norm{\bc_i}}\rB)}
        =  \sum_{i=1}^k \bu_i^T \bL \bu_i
\end{align*}
The optimal solution comprises the eigenvectors corresponding to the 
$k$ smallest eigenvalues of $\bL$, i.e., the eigenvectors
$\bu_n, \bu_{n-1}, \dots, \bu_{n-k+1}$ represent the relaxed
cluster indicator vectors.
\end{frame}


\begin{frame}{Clustering Objective Functions: Normalized Cut}

{\em Normalized cut} is similar to
ratio cut, except that it divides the cut weight of each cluster
by the volume of a cluster instead of its size. The objective
function is given as
\begin{align*}
    \min_\cC \; J_{nc}(\cC) = \sum_{i=1}^k {W(C_i, \ol{C_i}) \over
    vol(C_i)} = \sum_{i=1}^k {\bc_i^T \bL \bc_i  \over
    \bc_i^T \bDelta \bc_i}
\end{align*}
We can obtain an optimal solution by allowing $\bc_i$ to 
be an arbitrary real vector. 

\medskip
The optimal solution comprise the eigenvectors corresponding to the 
$k$ smallest eigenvalues of either the normalized symmetric or asymmetric
 Laplacian matrices, $\bL^s$ and $\bL^a$.
\end{frame}




\begin{frame}[fragile]{Spectral Clustering Algorithm}

The spectral clustering algorithm takes a dataset $\bD$ as
input and computes the similarity matrix $\bA$.  
For normalized cut we chose either $\bL^s$
or $\bL^a$, whereas for ratio cut we choose $\bL$. 
Next, we
compute the $k$ smallest eigenvalues and corresponding 
eigenvectors of the chosen matrix.

\medskip
The main problem is that the eigenvectors $\bu_i$
are not binary, and thus it is not immediately clear how we can
assign points to clusters. 

\medskip
One solution to this problem is to
treat the $n \times k$ matrix of eigenvectors as a new data matrix:
\begin{align*}
    \bU = \matr{
    | & | &  & | \\
    \bu_n & \bu_{n-1} & \cdots & \bu_{n-k+1}\\
    | & | &  & | \\
    }
	\rightarrow
	\text{ normalize rows }
	\rightarrow
	\matr{
    \mbox{---} & \by_1^T &\mbox{---}\\
    \mbox{---} & \by_2^T &\mbox{---}\\
    &\vdots&\\
    \mbox{---} & \by_n^T &\mbox{---}\\
    }
	= \bY
\end{align*}
We then cluster the new points in $\bY$ into $k$ clusters via
the K-means algorithm or any other fast clustering method to obtain
binary cluster indicator vectors $\bc_i$.
\end{frame}



\newcommand{\spectral}{\textsc{Spectral Clustering}}
\begin{frame}[fragile]{Spectral Clustering Algorithm}
\begin{algorithm}[H]
\SetKwInOut{Algorithm}{\spectral\ ($\bD, k$)} \Algorithm{} Compute
the similarity matrix $\bA \in \setR^{n\times n}$\; \lIf{ratio
cut}{
    \nllabel{alg:clus:spectral:spectral:lineBs}
    $\bB \assign \bL$}
\lElseIf{normalized cut}{$\bB \assign \bL^s \text{ or } \bL^a$}
Solve $\bB \bu_i = \lambda_i \bu_i$ for $i=n,\dots,n-k+1$, where
$\lambda_n \le \lambda_{n-1} \le \dots \le \lambda_{n-k+1}$
\nllabel{alg:clus:spectral:spectral:eig}\;
$\bU \assign \matr{\bu_n & \bu_{n-1}& \cdots & \bu_{n-k+1}}$\;
$\bY \assign$ normalize rows of $\bU$ \; %using Eq.\nosp\eqref{eq:clus:spectral:yi}\;
 $\cC \assign \{C_1,\dots,C_k\}$ via K-means on $\bY$\;
\end{algorithm}
\end{frame}



\begin{frame}{Spectral Clustering on Example Graph}
  \framesubtitle{$k=2$, normalized cut (normalized asymmetric Laplacian)}
\begin{figure}
    \centerline{
	\scalebox{0.6}{
        \psset{unit=0.75in,dotscale=2,fillcolor=lightgray,dotstyle=Bo}
        \begin{pspicture}(0,-0.25)(0,2.25)
            \psmatrix[mnode=circle]
                & [name=a] 1 & & & [name=f]6\\
                [name=b]2 & & [name=d]4 & [name=e]5 & [mnode=none]\\
                & [name=c]3  & & & [name=g]7
            \endpsmatrix
            \ncline{a}{b}
            \ncline{a}{d}
            \ncline{a}{f}
            \ncline{b}{c}
            \ncline{b}{d}
            \ncline{c}{d}
            \ncline{c}{g}
            \ncline{d}{e}
            \ncline{e}{f}
            \ncline{e}{g}
            \ncline{f}{g}
        \end{pspicture}
		}}
		\vspace{0.2in}
    \centerline{
	\scalebox{0.6}{
    \psset{xunit=5in,yunit=1in,
        dotscale=1.5,arrowscale=2,PointName=none}
    \psset{xAxisLabel=$\bu_1$,yAxisLabel= $\bu_2$}
    \psgraph[tickstyle=bottom,Dx=0.1,Ox=-1,Dy=0.5,Oy=-1]{->}%
    (-1,-1)(-1,-1)(-0.5,1){3in}{2in}
    \psset{dotstyle=Bo,fillcolor=lightgray}
    \psdots[](-0.8585673,-0.5127008)(-0.6036752,-0.7972304)%
    (-0.8585673,-0.5127008)(-0.8117383,-0.5840214)%
    (-0.6644345,0.7473465)(-0.6482062,0.7614649)%
    (-0.6482062,0.7614649)
      \uput[90](-0.86,-0.5){\small $1,3$}
        \uput[90](-0.6,-0.8){\small $2$}
        \uput[90](-0.81,-0.584){\small $4$}
        \uput[90](-0.66,0.747){\small $5$}
        \uput[30](-0.65,0.761){\small $6,7$}
    \psset{fillcolor=black}
    \pstGeonode[PointSymbol=none, dotscale=2](-0.654,0.756){A}
    \pstGeonode[PointSymbol=none, dotscale=2](-0.783,-0.602){B}
    \psclip{\psframe[](-1,-1)(-0.5,1)}%
    {
    \psset{linestyle=none, PointSymbol=none}
    \pstMediatorAB{A}{B}{K}{KP}
    \psset{linewidth=1pt,linestyle=dashed}
    \pstGeonode[PointSymbol=none](-1,-1){a}(-1,1){b}(-0.5,1){c}(-0.5,-1){d}
    \pstInterLL[PointSymbol=none]{K}{KP}{b}{c}{ku}
    \pstLineAB{K}{ku}
    \pstInterLL[PointSymbol=none]{K}{KP}{a}{d}{kd}
    \pstLineAB{K}{kd}
    }
    \endpsclip
    \endpsgraph
	}}
\end{figure}
\end{frame}


\begin{frame}{Normalized Cut on Iris Graph}
  \framesubtitle{$k=3$, normalized asymmetric Laplacian}
\begin{figure}
    \centerline{
        \scalebox{0.55}{
            \psset{unit=0.75in,dotscale=2}
            \begin{pspicture}(5,5)
                \pnode(1.108338,3.858886){n0}
\pnode(2.527444,2.445348){n1}
\pnode(2.017666,2.841472){n2}
\pnode(0.6785566,0.8802483){n3}
\pnode(1.257860,2.892203){n4}
\pnode(0.8247274,0.631479){n5}
\pnode(4.581229,3.210751){n6}
\pnode(3.098514,4.866756){n7}
\pnode(4.591683,2.878327){n8}
\pnode(4,2.2){n9}
\pnode(2.888154,1.090116){n10}
\pnode(3.105909,0.7401545){n11}
\pnode(0.3962556,3.984627){n12}
\pnode(2.661068,4.46547){n13}
\pnode(1.439636,1.295117){n14}
\pnode(2.236885,1.274193){n15}
\pnode(2.024484,0.7152085){n16}
\pnode(0.6005647,3.371056){n17}
\pnode(1.814559,0.4981684){n18}
\pnode(4.08826,1.970573){n19}
\pnode(2.577236,4.740179){n20}
\pnode(3.773692,3.552994){n21}
\pnode(3.913548,0.4282107){n22}
\pnode(0.711105,2.899006){n23}
\pnode(1.232538,4.29378){n24}
\pnode(2.521529,1.412281){n25}
\pnode(0.9249148,2.893182){n26}
\pnode(0.5726195,1.294175){n27}
\pnode(3.759983,2.172351){n28}
\pnode(4.7622,2.335772){n29}
\pnode(3.665883,0.6603324){n30}
\pnode(1.422528,0.3128956){n31}
\pnode(2.557950,5){n32}
\pnode(0.1556468,2.998522){n33}
\pnode(1.403207,1.916558){n34}
\pnode(0.3174471,2.600607){n35}
\pnode(0.1307679,3.277233){n36}
\pnode(2.901075,1.575363){n37}
\pnode(2.55,3.97){n38}
\pnode(3.353207,4.403764){n39}
\pnode(0.9114814,4.353879){n40}
\pnode(1.686579,3.819467){n41}
\pnode(4.209552,3.016675){n42}
\pnode(3.891111,1.748183){n43}
\pnode(0.4179513,3.522051){n44}
\pnode(0.8306748,1.037910){n45}
\pnode(4.389942,2.082443){n46}
\pnode(0.1748816,2.281055){n47}
\pnode(3.273419,0.6181322){n48}
\pnode(3.66405,0.2172084){n49}
\pnode(0.5841016,3.811024){n50}
\pnode(2.24852,4.942186){n51}
\pnode(0.2341176,1.037886){n52}
\pnode(4.969442,3.079323){n53}
\pnode(2.864019,0){n54}
\pnode(5,2.809783){n55}
\pnode(3.97574,2.571348){n56}
\pnode(4.371287,1.534671){n57}
\pnode(1.967572,3.868347){n58}
\pnode(0,2.563469){n59}
\pnode(3.764707,4.352432){n60}
\pnode(3.737104,3.981957){n61}
\pnode(2.127025,0.3401537){n62}
\pnode(2.312403,1.030709){n63}
\pnode(0.2217693,3.701035){n64}
\pnode(0.6466549,4.297944){n65}
\pnode(3.364475,4.745668){n66}
\pnode(1.509306,0.9905115){n67}
\pnode(1.631241,0.08123197){n68}
\pnode(2.363231,0.03982867){n69}
\pnode(4.688611,1.662437){n70}
\pnode(4.32925,1.149353){n71}
\pnode(1.004251,3.3599){n72}
\pnode(3.170304,0.08031329){n73}
\pnode(1.092846,0.4743804){n74}
\pnode(0.957797,1.457980){n75}
\pnode(0.4145735,1.587721){n76}
\pnode(2.550006,4.241430){n77}
\pnode(0.8983155,1.243353){n78}
\pnode(1.560778,4.1743){n79}
\pnode(4.791386,3.470398){n80}
\pnode(1.183687,2.360412){n81}
\pnode(2.078577,3.201014){n82}
\pnode(4.135811,4.087185){n83}
\pnode(3.153883,3.140639){n84}
\pnode(0.6058247,2.060042){n85}
\pnode(1.891242,4.717356){n86}
\pnode(4.792091,2.606388){n87}
\pnode(3.635776,1.016917){n88}
\pnode(3.202631,4.133673){n89}
\pnode(1.765180,1.228386){n90}
\pnode(0.4301581,0.7624465){n91}
\pnode(4.292557,3.369169){n92}
\pnode(2.64,3.705){n93}
\pnode(2.869689,4.205452){n94}
\pnode(4.306428,1.798957){n95}
\pnode(0.9419618,2.164323){n96}
\pnode(2.802984,2.471939){n97}
\pnode(0.4990411,2.364772){n98}
\pnode(2.265986,3.781169){n99}
\pnode(2.886530,2.782621){n100}
\pnode(2.074548,4.120629){n101}
\pnode(3.578431,3.339056){n102}
\pnode(4.396008,2.731498){n103}
\pnode(2.161968,2.481101){n104}
\pnode(1.915954,1.686363){n105}
\pnode(1.073082,0.7635667){n106}
\pnode(4.351759,2.410157){n107}
\pnode(0.1632209,1.576404){n108}
\pnode(0.1964432,1.278737){n109}
\pnode(0.5000101,2.853756){n110}
\pnode(2.625767,0.8297721){n111}
\pnode(1.705716,3.114971){n112}
\pnode(2.796314,0.5814426){n113}
\pnode(0.3549037,3.121353){n114}
\pnode(3.949152,0.8549694){n115}
\pnode(4.562201,1.223617){n116}
\pnode(3.053188,3.925351){n117}
\pnode(4.318411,3.852927){n118}
\pnode(2.890681,3.735371){n119}
\pnode(2.880736,4.879838){n120}
\pnode(1.377059,0.7879707){n121}
\pnode(2.617372,2.90596){n122}
\pnode(3.046366,4.518936){n123}
\pnode(2.621703,0.4200536){n124}
\pnode(1.203850,1.499658){n125}
\pnode(0.9849086,4.079775){n126}
\pnode(3.254325,1.04113){n127}
\pnode(2.280932,2.838994){n128}
\pnode(4.117188,3.717776){n129}
\pnode(3.471599,1.285363){n130}
\pnode(4.500053,3.637197){n131}
\pnode(4.069486,1.411759){n132}
\pnode(4.11247,0.5910543){n133}
\pnode(1.252724,1.071603){n134}
\pnode(3.385955,0.3694225){n135}
\pnode(2.407195,0.4605674){n136}
\pnode(1.689756,4.453583){n137}
\pnode(3.989081,3.406712){n138}
\pnode(0.710692,2.532491){n139}
\pnode(4.873979,1.934726){n140}
\pnode(3.645946,2.645275){n141}
\pnode(1.606368,3.440222){n142}
\pnode(2.445606,3.296266){n143}
\pnode(2.274105,4.535248){n144}
\pnode(2.648264,0.1809270){n145}
\pnode(0.7976457,3.6483){n146}
\pnode(3.950723,2.918283){n147}
\pnode(4.776492,1.394465){n148}
\pnode(2.296867,0.7074797){n149}
\psset{linewidth=0.5pt,dotsep=2pt}
\ncline[linecolor=lightgray]{n0}{n44}
\ncline[linecolor=lightgray]{n0}{n137}
\ncline[linecolor=lightgray]{n1}{n2}
\ncline[linecolor=lightgray]{n1}{n104}
\ncline[linecolor=lightgray]{n1}{n128}
\ncline[linecolor=lightgray]{n2}{n142}
\ncline[linecolor=lightgray]{n1}{n2}
\ncline[linecolor=lightgray]{n2}{n41}
\ncline[linecolor=lightgray]{n2}{n97}
\ncline[linecolor=lightgray]{n2}{n58}
\ncline[linecolor=lightgray]{n2}{n84}
\ncline[linecolor=lightgray]{n2}{n99}
\ncline[linecolor=lightgray]{n4}{n33}
\ncline[linecolor=lightgray]{n4}{n35}
\ncline[linecolor=lightgray]{n4}{n36}
\ncline[linecolor=lightgray]{n4}{n139}
\ncline[linecolor=lightgray]{n4}{n26}
\ncline[linecolor=lightgray]{n4}{n38}
\ncline[linecolor=lightgray]{n4}{n98}
\ncline[linecolor=lightgray]{n4}{n47}
\ncline[linecolor=lightgray]{n4}{n93}
\ncline[linecolor=lightgray]{n6}{n107}
\ncline[linecolor=lightgray]{n6}{n87}
\ncline[linecolor=lightgray]{n6}{n138}
\ncline[linecolor=lightgray]{n6}{n83}
\ncline[linecolor=lightgray]{n6}{n53}
\ncline[linecolor=lightgray]{n6}{n118}
\ncline[linecolor=lightgray]{n6}{n55}
\ncline[linecolor=lightgray]{n6}{n92}
\ncline[linecolor=lightgray]{n6}{n56}
\ncline[linecolor=lightgray]{n7}{n13}
\ncline[linecolor=lightgray]{n7}{n144}
\ncline[linecolor=lightgray]{n7}{n66}
\ncline[linecolor=lightgray]{n7}{n77}
\ncline[linecolor=lightgray]{n7}{n89}
\ncline[linecolor=lightgray]{n7}{n39}
\ncline[linecolor=lightgray]{n7}{n86}
\ncline[linecolor=lightgray]{n7}{n94}
\ncline[linecolor=lightgray]{n8}{n131}
\ncline[linecolor=lightgray]{n8}{n55}
\ncline[linecolor=lightgray]{n8}{n129}
\ncline[linecolor=lightgray]{n8}{n141}
\ncline[linecolor=lightgray]{n8}{n46}
\ncline[linecolor=lightgray]{n9}{n129}
\ncline[linecolor=lightgray]{n9}{n83}
\ncline[linecolor=lightgray]{n9}{n131}
\ncline[linecolor=lightgray]{n9}{n46}
\ncline[linecolor=lightgray]{n9}{n28}
\ncline[linecolor=lightgray]{n9}{n60}
\ncline[linecolor=lightgray]{n9}{n95}
\ncline[linecolor=lightgray]{n12}{n146}
\ncline[linecolor=lightgray]{n12}{n72}
\ncline[linecolor=lightgray]{n12}{n65}
\ncline[linecolor=lightgray]{n12}{n142}
\ncline[linecolor=lightgray]{n7}{n13}
\ncline[linecolor=lightgray]{n13}{n120}
\ncline[linecolor=lightgray]{n13}{n41}
\ncline[linecolor=lightgray]{n13}{n79}
\ncline[linestyle=dotted,linecolor=lightgray]{n13}{n102}
\ncline[linestyle=dotted,linecolor=lightgray]{n13}{n21}
\ncline[linecolor=lightgray]{n17}{n40}
\ncline[linecolor=lightgray]{n17}{n59}
\ncline[linecolor=lightgray]{n19}{n95}
\ncline[linecolor=lightgray]{n19}{n29}
\ncline[linecolor=lightgray]{n19}{n107}
\ncline[linecolor=lightgray]{n19}{n57}
\ncline[linecolor=lightgray]{n19}{n116}
\ncline[linecolor=lightgray]{n19}{n140}
\ncline[linecolor=lightgray]{n20}{n77}
\ncline[linecolor=lightgray]{n20}{n89}
\ncline[linecolor=lightgray]{n20}{n117}
\ncline[linecolor=lightgray]{n20}{n32}
\ncline[linecolor=lightgray]{n20}{n66}
\ncline[linecolor=lightgray]{n21}{n129}
\ncline[linecolor=lightgray]{n21}{n56}
\ncline[linecolor=lightgray]{n21}{n83}
\ncline[linecolor=lightgray]{n21}{n131}
\ncline[linestyle=dotted,linecolor=lightgray]{n21}{n89}
\ncline[linecolor=lightgray]{n21}{n141}
\ncline[linestyle=dotted,linecolor=lightgray]{n13}{n21}
\ncline[linecolor=lightgray]{n22}{n133}
\ncline[linecolor=lightgray]{n22}{n49}
\ncline[linecolor=lightgray]{n22}{n130}
\ncline[linecolor=lightgray]{n23}{n36}
\ncline[linecolor=lightgray]{n23}{n50}
\ncline[linecolor=lightgray]{n23}{n64}
\ncline[linecolor=lightgray]{n23}{n59}
\ncline[linecolor=lightgray]{n25}{n54}
\ncline[linecolor=lightgray]{n25}{n37}
\ncline[linecolor=lightgray]{n26}{n146}
\ncline[linecolor=lightgray]{n26}{n110}
\ncline[linecolor=lightgray]{n26}{n114}
\ncline[linecolor=lightgray]{n4}{n26}
\ncline[linecolor=lightgray]{n26}{n47}
\ncline[linecolor=lightgray]{n28}{n95}
\ncline[linecolor=lightgray]{n28}{n46}
\ncline[linecolor=lightgray]{n28}{n43}
\ncline[linecolor=lightgray]{n9}{n28}
\ncline[linecolor=lightgray]{n28}{n57}
\ncline[linecolor=lightgray]{n29}{n42}
\ncline[linecolor=lightgray]{n29}{n53}
\ncline[linecolor=lightgray]{n29}{n56}
\ncline[linecolor=lightgray]{n29}{n55}
\ncline[linecolor=lightgray]{n29}{n140}
\ncline[linecolor=lightgray]{n19}{n29}
\ncline[linecolor=lightgray]{n30}{n133}
\ncline[linecolor=lightgray]{n30}{n49}
\ncline[linecolor=lightgray]{n32}{n66}
\ncline[linecolor=lightgray]{n32}{n123}
\ncline[linecolor=lightgray]{n20}{n32}
\ncline[linecolor=lightgray]{n32}{n39}
\ncline[linecolor=lightgray]{n32}{n94}
\ncline[linecolor=lightgray]{n33}{n64}
\ncline[linecolor=lightgray]{n4}{n33}
\ncline[linecolor=lightgray]{n33}{n146}
\ncline[linecolor=lightgray]{n33}{n98}
\ncline[linecolor=lightgray]{n34}{n85}
\ncline[linecolor=lightgray]{n4}{n35}
\ncline[linecolor=lightgray]{n35}{n146}
\ncline[linecolor=lightgray]{n35}{n110}
\ncline[linecolor=lightgray]{n35}{n59}
\ncline[linecolor=lightgray]{n36}{n64}
\ncline[linecolor=lightgray]{n23}{n36}
\ncline[linecolor=lightgray]{n36}{n146}
\ncline[linecolor=lightgray]{n36}{n126}
\ncline[linecolor=lightgray]{n36}{n139}
\ncline[linecolor=lightgray]{n4}{n36}
\ncline[linecolor=lightgray]{n25}{n37}
\ncline[linecolor=lightgray]{n37}{n105}
\ncline[linecolor=lightgray]{n37}{n130}
\ncline[linecolor=lightgray]{n38}{n119}
\ncline[linecolor=lightgray]{n4}{n38}
\ncline[linecolor=lightgray]{n38}{n112}
\ncline[linecolor=lightgray]{n38}{n58}
\ncline[linecolor=lightgray]{n38}{n94}
\ncline[linecolor=lightgray]{n39}{n123}
\ncline[linecolor=lightgray]{n39}{n117}
\ncline[linecolor=lightgray]{n7}{n39}
\ncline[linecolor=lightgray]{n39}{n120}
\ncline[linecolor=lightgray]{n39}{n94}
\ncline[linecolor=lightgray]{n32}{n39}
\ncline[linecolor=lightgray]{n40}{n44}
\ncline[linecolor=lightgray]{n17}{n40}
\ncline[linecolor=lightgray]{n40}{n146}
\ncline[linecolor=lightgray]{n40}{n41}
\ncline[linecolor=lightgray]{n40}{n86}
\ncline[linecolor=lightgray]{n40}{n137}
\ncline[linecolor=lightgray]{n41}{n126}
\ncline[linecolor=lightgray]{n41}{n137}
\ncline[linecolor=lightgray]{n41}{n77}
\ncline[linecolor=lightgray]{n13}{n41}
\ncline[linecolor=lightgray]{n2}{n41}
\ncline[linecolor=lightgray]{n40}{n41}
\ncline[linecolor=lightgray]{n42}{n131}
\ncline[linecolor=lightgray]{n42}{n141}
\ncline[linecolor=lightgray]{n29}{n42}
\ncline[linecolor=lightgray]{n42}{n46}
\ncline[linecolor=lightgray]{n42}{n118}
\ncline[linecolor=lightgray]{n42}{n92}
\ncline[linecolor=lightgray]{n43}{n116}
\ncline[linecolor=lightgray]{n43}{n46}
\ncline[linecolor=lightgray]{n28}{n43}
\ncline[linecolor=lightgray]{n43}{n132}
\ncline[linecolor=lightgray]{n43}{n57}
\ncline[linecolor=lightgray]{n43}{n71}
\ncline[linecolor=lightgray]{n43}{n140}
\ncline[linecolor=lightgray]{n40}{n44}
\ncline[linecolor=lightgray]{n0}{n44}
\ncline[linecolor=lightgray]{n44}{n65}
\ncline[linecolor=lightgray]{n42}{n46}
\ncline[linecolor=lightgray]{n28}{n46}
\ncline[linecolor=lightgray]{n46}{n103}
\ncline[linecolor=lightgray]{n43}{n46}
\ncline[linecolor=lightgray]{n46}{n56}
\ncline[linecolor=lightgray]{n8}{n46}
\ncline[linecolor=lightgray]{n46}{n57}
\ncline[linecolor=lightgray]{n46}{n87}
\ncline[linecolor=lightgray]{n9}{n46}
\ncline[linecolor=lightgray]{n46}{n140}
\ncline[linecolor=lightgray]{n47}{n114}
\ncline[linecolor=lightgray]{n26}{n47}
\ncline[linecolor=lightgray]{n4}{n47}
\ncline[linecolor=lightgray]{n49}{n145}
\ncline[linecolor=lightgray]{n30}{n49}
\ncline[linecolor=lightgray]{n49}{n73}
\ncline[linecolor=lightgray]{n49}{n115}
\ncline[linecolor=lightgray]{n49}{n133}
\ncline[linecolor=lightgray]{n22}{n49}
\ncline[linecolor=lightgray]{n49}{n130}
\ncline[linecolor=lightgray]{n50}{n79}
\ncline[linecolor=lightgray]{n50}{n65}
\ncline[linecolor=lightgray]{n23}{n50}
\ncline[linecolor=lightgray]{n51}{n66}
\ncline[linecolor=lightgray]{n51}{n144}
\ncline[linecolor=lightgray]{n51}{n86}
\ncline[linecolor=lightgray]{n51}{n65}
\ncline[linecolor=lightgray]{n52}{n74}
\ncline[linecolor=lightgray]{n53}{n87}
\ncline[linecolor=lightgray]{n6}{n53}
\ncline[linecolor=lightgray]{n29}{n53}
\ncline[linecolor=lightgray]{n53}{n80}
\ncline[linecolor=lightgray]{n25}{n54}
\ncline[linecolor=lightgray]{n8}{n55}
\ncline[linecolor=lightgray]{n55}{n103}
\ncline[linecolor=lightgray]{n6}{n55}
\ncline[linecolor=lightgray]{n29}{n55}
\ncline[linecolor=lightgray]{n55}{n131}
\ncline[linecolor=lightgray]{n55}{n107}
\ncline[linecolor=lightgray]{n21}{n56}
\ncline[linecolor=lightgray]{n56}{n138}
\ncline[linecolor=lightgray]{n56}{n102}
\ncline[linecolor=lightgray]{n46}{n56}
\ncline[linecolor=lightgray]{n29}{n56}
\ncline[linecolor=lightgray]{n56}{n92}
\ncline[linecolor=lightgray]{n56}{n87}
\ncline[linecolor=lightgray]{n6}{n56}
\ncline[linecolor=lightgray]{n56}{n118}
\ncline[linecolor=lightgray]{n57}{n95}
\ncline[linecolor=lightgray]{n57}{n116}
\ncline[linecolor=lightgray]{n46}{n57}
\ncline[linecolor=lightgray]{n43}{n57}
\ncline[linecolor=lightgray]{n57}{n132}
\ncline[linecolor=lightgray]{n19}{n57}
\ncline[linecolor=lightgray]{n57}{n140}
\ncline[linecolor=lightgray]{n57}{n71}
\ncline[linecolor=lightgray]{n28}{n57}
\ncline[linecolor=lightgray]{n57}{n70}
\ncline[linecolor=lightgray]{n57}{n148}
\ncline[linecolor=lightgray]{n58}{n119}
\ncline[linecolor=lightgray]{n58}{n143}
\ncline[linecolor=lightgray]{n58}{n112}
\ncline[linecolor=lightgray]{n2}{n58}
\ncline[linecolor=lightgray]{n58}{n117}
\ncline[linecolor=lightgray]{n38}{n58}
\ncline[linecolor=lightgray]{n59}{n64}
\ncline[linecolor=lightgray]{n17}{n59}
\ncline[linecolor=lightgray]{n35}{n59}
\ncline[linecolor=lightgray]{n59}{n110}
\ncline[linecolor=lightgray]{n23}{n59}
\ncline[linecolor=lightgray]{n59}{n139}
\ncline[linecolor=lightgray]{n59}{n98}
\ncline[linecolor=lightgray]{n60}{n83}
\ncline[linestyle=dotted,linecolor=lightgray]{n60}{n123}
\ncline[linestyle=dotted,linecolor=lightgray]{n60}{n117}
\ncline[linestyle=dotted,linecolor=lightgray]{n60}{n94}
\ncline[linecolor=lightgray]{n9}{n60}
\ncline[linestyle=dotted,linecolor=lightgray]{n61}{n117}
\ncline[linestyle=dotted,linecolor=lightgray]{n61}{n123}
\ncline[linestyle=dotted,linecolor=lightgray]{n61}{n84}
\ncline[linecolor=lightgray]{n61}{n102}
\ncline[linecolor=lightgray]{n61}{n147}
\ncline[linecolor=lightgray]{n63}{n105}
\ncline[linecolor=lightgray]{n33}{n64}
\ncline[linecolor=lightgray]{n36}{n64}
\ncline[linecolor=lightgray]{n64}{n65}
\ncline[linecolor=lightgray]{n59}{n64}
\ncline[linecolor=lightgray]{n23}{n64}
\ncline[linecolor=lightgray]{n64}{n146}
\ncline[linecolor=lightgray]{n64}{n110}
\ncline[linecolor=lightgray]{n50}{n65}
\ncline[linecolor=lightgray]{n64}{n65}
\ncline[linecolor=lightgray]{n44}{n65}
\ncline[linecolor=lightgray]{n12}{n65}
\ncline[linecolor=lightgray]{n65}{n146}
\ncline[linecolor=lightgray]{n51}{n65}
\ncline[linecolor=lightgray]{n65}{n110}
\ncline[linecolor=lightgray]{n65}{n137}
\ncline[linecolor=lightgray]{n51}{n66}
\ncline[linecolor=lightgray]{n7}{n66}
\ncline[linecolor=lightgray]{n66}{n120}
\ncline[linecolor=lightgray]{n32}{n66}
\ncline[linecolor=lightgray]{n20}{n66}
\ncline[linecolor=lightgray]{n68}{n91}
\ncline[linecolor=lightgray]{n68}{n109}
\ncline[linecolor=lightgray]{n70}{n116}
\ncline[linecolor=lightgray]{n57}{n70}
\ncline[linecolor=lightgray]{n70}{n140}
\ncline[linecolor=lightgray]{n70}{n132}
\ncline[linecolor=lightgray]{n70}{n71}
\ncline[linecolor=lightgray]{n71}{n116}
\ncline[linecolor=lightgray]{n57}{n71}
\ncline[linecolor=lightgray]{n43}{n71}
\ncline[linecolor=lightgray]{n71}{n95}
\ncline[linecolor=lightgray]{n71}{n148}
\ncline[linecolor=lightgray]{n70}{n71}
\ncline[linecolor=lightgray]{n12}{n72}
\ncline[linecolor=lightgray]{n49}{n73}
\ncline[linecolor=lightgray]{n52}{n74}
\ncline[linecolor=lightgray]{n20}{n77}
\ncline[linecolor=lightgray]{n77}{n143}
\ncline[linecolor=lightgray]{n77}{n79}
\ncline[linecolor=lightgray]{n77}{n86}
\ncline[linecolor=lightgray]{n41}{n77}
\ncline[linecolor=lightgray]{n7}{n77}
\ncline[linecolor=lightgray]{n77}{n120}
\ncline[linecolor=lightgray]{n50}{n79}
\ncline[linecolor=lightgray]{n77}{n79}
\ncline[linecolor=lightgray]{n79}{n112}
\ncline[linecolor=lightgray]{n79}{n82}
\ncline[linecolor=lightgray]{n79}{n137}
\ncline[linecolor=lightgray]{n79}{n143}
\ncline[linecolor=lightgray]{n13}{n79}
\ncline[linecolor=lightgray]{n80}{n102}
\ncline[linecolor=lightgray]{n80}{n118}
\ncline[linecolor=lightgray]{n80}{n87}
\ncline[linecolor=lightgray]{n53}{n80}
\ncline[linestyle=dotted,linecolor=lightgray]{n81}{n98}
\ncline[linecolor=lightgray]{n82}{n101}
\ncline[linecolor=lightgray]{n79}{n82}
\ncline[linecolor=lightgray]{n21}{n83}
\ncline[linecolor=lightgray]{n6}{n83}
\ncline[linecolor=lightgray]{n60}{n83}
\ncline[linecolor=lightgray]{n9}{n83}
\ncline[linecolor=lightgray]{n84}{n117}
\ncline[linecolor=lightgray]{n84}{n89}
\ncline[linestyle=dotted,linecolor=lightgray]{n84}{n138}
\ncline[linestyle=dotted,linecolor=lightgray]{n61}{n84}
\ncline[linecolor=lightgray]{n84}{n97}
\ncline[linecolor=lightgray]{n2}{n84}
\ncline[linecolor=lightgray]{n84}{n99}
\ncline[linestyle=dotted,linecolor=lightgray]{n84}{n118}
\ncline[linestyle=dotted,linecolor=lightgray]{n85}{n98}
\ncline[linecolor=lightgray]{n34}{n85}
\ncline[linecolor=lightgray]{n86}{n99}
\ncline[linecolor=lightgray]{n77}{n86}
\ncline[linecolor=lightgray]{n51}{n86}
\ncline[linecolor=lightgray]{n40}{n86}
\ncline[linecolor=lightgray]{n7}{n86}
\ncline[linecolor=lightgray]{n86}{n120}
\ncline[linecolor=lightgray]{n6}{n87}
\ncline[linecolor=lightgray]{n53}{n87}
\ncline[linecolor=lightgray]{n87}{n92}
\ncline[linecolor=lightgray]{n56}{n87}
\ncline[linecolor=lightgray]{n87}{n147}
\ncline[linecolor=lightgray]{n80}{n87}
\ncline[linecolor=lightgray]{n46}{n87}
\ncline[linecolor=lightgray]{n87}{n141}
\ncline[linecolor=lightgray]{n88}{n115}
\ncline[linecolor=lightgray]{n89}{n143}
\ncline[linestyle=dotted,linecolor=lightgray]{n21}{n89}
\ncline[linecolor=lightgray]{n84}{n89}
\ncline[linecolor=lightgray]{n20}{n89}
\ncline[linecolor=lightgray]{n7}{n89}
\ncline[linecolor=lightgray]{n89}{n120}
\ncline[linecolor=lightgray]{n68}{n91}
\ncline[linecolor=lightgray]{n92}{n131}
\ncline[linecolor=lightgray]{n42}{n92}
\ncline[linecolor=lightgray]{n92}{n129}
\ncline[linecolor=lightgray]{n92}{n103}
\ncline[linecolor=lightgray]{n92}{n118}
\ncline[linecolor=lightgray]{n92}{n141}
\ncline[linecolor=lightgray]{n87}{n92}
\ncline[linecolor=lightgray]{n6}{n92}
\ncline[linecolor=lightgray]{n56}{n92}
\ncline[linecolor=lightgray]{n93}{n119}
\ncline[linecolor=lightgray]{n93}{n94}
\ncline[linecolor=lightgray]{n4}{n93}
\ncline[linecolor=lightgray]{n94}{n123}
\ncline[linecolor=lightgray]{n93}{n94}
\ncline[linecolor=lightgray]{n39}{n94}
\ncline[linecolor=lightgray]{n32}{n94}
\ncline[linecolor=lightgray]{n7}{n94}
\ncline[linecolor=lightgray]{n94}{n120}
\ncline[linecolor=lightgray]{n38}{n94}
\ncline[linestyle=dotted,linecolor=lightgray]{n60}{n94}
\ncline[linecolor=lightgray]{n19}{n95}
\ncline[linecolor=lightgray]{n28}{n95}
\ncline[linecolor=lightgray]{n57}{n95}
\ncline[linecolor=lightgray]{n95}{n116}
\ncline[linecolor=lightgray]{n95}{n132}
\ncline[linecolor=lightgray]{n9}{n95}
\ncline[linecolor=lightgray]{n71}{n95}
\ncline[linecolor=lightgray]{n2}{n97}
\ncline[linecolor=lightgray]{n97}{n128}
\ncline[linecolor=lightgray]{n84}{n97}
\ncline[linecolor=lightgray]{n33}{n98}
\ncline[linecolor=lightgray]{n98}{n114}
\ncline[linecolor=lightgray]{n59}{n98}
\ncline[linecolor=lightgray]{n4}{n98}
\ncline[linecolor=lightgray]{n98}{n110}
\ncline[linestyle=dotted,linecolor=lightgray]{n81}{n98}
\ncline[linestyle=dotted,linecolor=lightgray]{n85}{n98}
\ncline[linecolor=lightgray]{n86}{n99}
\ncline[linecolor=lightgray]{n99}{n144}
\ncline[linecolor=lightgray]{n99}{n101}
\ncline[linecolor=lightgray]{n99}{n128}
\ncline[linecolor=lightgray]{n99}{n104}
\ncline[linecolor=lightgray]{n2}{n99}
\ncline[linecolor=lightgray]{n84}{n99}
\ncline[linecolor=lightgray]{n100}{n128}
\ncline[linecolor=lightgray]{n100}{n104}
\ncline[linecolor=lightgray]{n100}{n119}
\ncline[linecolor=lightgray]{n101}{n126}
\ncline[linecolor=lightgray]{n101}{n117}
\ncline[linecolor=lightgray]{n82}{n101}
\ncline[linecolor=lightgray]{n99}{n101}
\ncline[linecolor=lightgray]{n101}{n123}
\ncline[linecolor=lightgray]{n101}{n119}
\ncline[linecolor=lightgray]{n101}{n112}
\ncline[linecolor=lightgray]{n102}{n118}
\ncline[linecolor=lightgray]{n102}{n138}
\ncline[linecolor=lightgray]{n102}{n141}
\ncline[linecolor=lightgray]{n56}{n102}
\ncline[linestyle=dotted,linecolor=lightgray]{n13}{n102}
\ncline[linecolor=lightgray]{n61}{n102}
\ncline[linecolor=lightgray]{n80}{n102}
\ncline[linecolor=lightgray]{n103}{n118}
\ncline[linecolor=lightgray]{n103}{n138}
\ncline[linecolor=lightgray]{n55}{n103}
\ncline[linecolor=lightgray]{n46}{n103}
\ncline[linecolor=lightgray]{n92}{n103}
\ncline[linecolor=lightgray]{n103}{n129}
\ncline[linecolor=lightgray]{n103}{n131}
\ncline[linecolor=lightgray]{n103}{n147}
\ncline[linecolor=lightgray]{n100}{n104}
\ncline[linecolor=lightgray]{n1}{n104}
\ncline[linecolor=lightgray]{n99}{n104}
\ncline[linecolor=lightgray]{n63}{n105}
\ncline[linecolor=lightgray]{n37}{n105}
\ncline[linecolor=lightgray]{n6}{n107}
\ncline[linecolor=lightgray]{n107}{n147}
\ncline[linecolor=lightgray]{n107}{n141}
\ncline[linecolor=lightgray]{n55}{n107}
\ncline[linecolor=lightgray]{n19}{n107}
\ncline[linecolor=lightgray]{n107}{n140}
\ncline[linecolor=lightgray]{n68}{n109}
\ncline[linecolor=lightgray]{n26}{n110}
\ncline[linecolor=lightgray]{n110}{n139}
\ncline[linecolor=lightgray]{n35}{n110}
\ncline[linecolor=lightgray]{n59}{n110}
\ncline[linecolor=lightgray]{n64}{n110}
\ncline[linecolor=lightgray]{n98}{n110}
\ncline[linecolor=lightgray]{n110}{n146}
\ncline[linecolor=lightgray]{n65}{n110}
\ncline[linecolor=lightgray]{n79}{n112}
\ncline[linecolor=lightgray]{n101}{n112}
\ncline[linecolor=lightgray]{n58}{n112}
\ncline[linecolor=lightgray]{n112}{n122}
\ncline[linecolor=lightgray]{n38}{n112}
\ncline[linecolor=lightgray]{n47}{n114}
\ncline[linecolor=lightgray]{n98}{n114}
\ncline[linecolor=lightgray]{n26}{n114}
\ncline[linecolor=lightgray]{n88}{n115}
\ncline[linecolor=lightgray]{n49}{n115}
\ncline[linecolor=lightgray]{n43}{n116}
\ncline[linecolor=lightgray]{n57}{n116}
\ncline[linecolor=lightgray]{n71}{n116}
\ncline[linecolor=lightgray]{n95}{n116}
\ncline[linecolor=lightgray]{n19}{n116}
\ncline[linecolor=lightgray]{n116}{n148}
\ncline[linecolor=lightgray]{n70}{n116}
\ncline[linecolor=lightgray]{n116}{n140}
\ncline[linecolor=lightgray]{n101}{n117}
\ncline[linestyle=dotted,linecolor=lightgray]{n61}{n117}
\ncline[linecolor=lightgray]{n84}{n117}
\ncline[linecolor=lightgray]{n39}{n117}
\ncline[linecolor=lightgray]{n20}{n117}
\ncline[linecolor=lightgray]{n58}{n117}
\ncline[linecolor=lightgray]{n117}{n144}
\ncline[linestyle=dotted,linecolor=lightgray]{n117}{n138}
\ncline[linestyle=dotted,linecolor=lightgray]{n60}{n117}
\ncline[linecolor=lightgray]{n102}{n118}
\ncline[linecolor=lightgray]{n42}{n118}
\ncline[linecolor=lightgray]{n103}{n118}
\ncline[linecolor=lightgray]{n118}{n131}
\ncline[linecolor=lightgray]{n6}{n118}
\ncline[linecolor=lightgray]{n92}{n118}
\ncline[linecolor=lightgray]{n56}{n118}
\ncline[linestyle=dotted,linecolor=lightgray]{n84}{n118}
\ncline[linecolor=lightgray]{n80}{n118}
\ncline[linecolor=lightgray]{n119}{n143}
\ncline[linecolor=lightgray]{n101}{n119}
\ncline[linecolor=lightgray]{n58}{n119}
\ncline[linecolor=lightgray]{n93}{n119}
\ncline[linecolor=lightgray]{n38}{n119}
\ncline[linecolor=lightgray]{n100}{n119}
\ncline[linecolor=lightgray]{n13}{n120}
\ncline[linecolor=lightgray]{n120}{n144}
\ncline[linecolor=lightgray]{n66}{n120}
\ncline[linecolor=lightgray]{n77}{n120}
\ncline[linecolor=lightgray]{n89}{n120}
\ncline[linecolor=lightgray]{n39}{n120}
\ncline[linecolor=lightgray]{n86}{n120}
\ncline[linecolor=lightgray]{n94}{n120}
\ncline[linecolor=lightgray]{n112}{n122}
\ncline[linecolor=lightgray]{n123}{n144}
\ncline[linecolor=lightgray]{n39}{n123}
\ncline[linecolor=lightgray]{n101}{n123}
\ncline[linecolor=lightgray]{n94}{n123}
\ncline[linestyle=dotted,linecolor=lightgray]{n61}{n123}
\ncline[linecolor=lightgray]{n32}{n123}
\ncline[linestyle=dotted,linecolor=lightgray]{n60}{n123}
\ncline[linecolor=lightgray]{n41}{n126}
\ncline[linecolor=lightgray]{n101}{n126}
\ncline[linecolor=lightgray]{n36}{n126}
\ncline[linecolor=lightgray]{n126}{n142}
\ncline[linecolor=lightgray]{n100}{n128}
\ncline[linecolor=lightgray]{n97}{n128}
\ncline[linecolor=lightgray]{n99}{n128}
\ncline[linecolor=lightgray]{n1}{n128}
\ncline[linecolor=lightgray]{n21}{n129}
\ncline[linecolor=lightgray]{n8}{n129}
\ncline[linecolor=lightgray]{n92}{n129}
\ncline[linecolor=lightgray]{n9}{n129}
\ncline[linecolor=lightgray]{n129}{n147}
\ncline[linecolor=lightgray]{n103}{n129}
\ncline[linecolor=lightgray]{n37}{n130}
\ncline[linecolor=lightgray]{n49}{n130}
\ncline[linecolor=lightgray]{n22}{n130}
\ncline[linecolor=lightgray]{n42}{n131}
\ncline[linecolor=lightgray]{n8}{n131}
\ncline[linecolor=lightgray]{n131}{n138}
\ncline[linecolor=lightgray]{n21}{n131}
\ncline[linecolor=lightgray]{n92}{n131}
\ncline[linecolor=lightgray]{n118}{n131}
\ncline[linecolor=lightgray]{n131}{n147}
\ncline[linecolor=lightgray]{n103}{n131}
\ncline[linecolor=lightgray]{n9}{n131}
\ncline[linecolor=lightgray]{n55}{n131}
\ncline[linecolor=lightgray]{n43}{n132}
\ncline[linecolor=lightgray]{n95}{n132}
\ncline[linecolor=lightgray]{n57}{n132}
\ncline[linecolor=lightgray]{n132}{n148}
\ncline[linecolor=lightgray]{n70}{n132}
\ncline[linecolor=lightgray]{n30}{n133}
\ncline[linecolor=lightgray]{n22}{n133}
\ncline[linecolor=lightgray]{n49}{n133}
\ncline[linecolor=lightgray]{n41}{n137}
\ncline[linecolor=lightgray]{n79}{n137}
\ncline[linecolor=lightgray]{n0}{n137}
\ncline[linecolor=lightgray]{n137}{n144}
\ncline[linecolor=lightgray]{n40}{n137}
\ncline[linecolor=lightgray]{n65}{n137}
\ncline[linecolor=lightgray]{n131}{n138}
\ncline[linecolor=lightgray]{n102}{n138}
\ncline[linecolor=lightgray]{n6}{n138}
\ncline[linecolor=lightgray]{n103}{n138}
\ncline[linecolor=lightgray]{n56}{n138}
\ncline[linestyle=dotted,linecolor=lightgray]{n84}{n138}
\ncline[linestyle=dotted,linecolor=lightgray]{n117}{n138}
\ncline[linecolor=lightgray]{n138}{n147}
\ncline[linecolor=lightgray]{n36}{n139}
\ncline[linecolor=lightgray]{n110}{n139}
\ncline[linecolor=lightgray]{n59}{n139}
\ncline[linecolor=lightgray]{n4}{n139}
\ncline[linecolor=lightgray]{n29}{n140}
\ncline[linecolor=lightgray]{n107}{n140}
\ncline[linecolor=lightgray]{n46}{n140}
\ncline[linecolor=lightgray]{n57}{n140}
\ncline[linecolor=lightgray]{n19}{n140}
\ncline[linecolor=lightgray]{n140}{n148}
\ncline[linecolor=lightgray]{n70}{n140}
\ncline[linecolor=lightgray]{n116}{n140}
\ncline[linecolor=lightgray]{n43}{n140}
\ncline[linecolor=lightgray]{n42}{n141}
\ncline[linecolor=lightgray]{n102}{n141}
\ncline[linecolor=lightgray]{n21}{n141}
\ncline[linecolor=lightgray]{n8}{n141}
\ncline[linecolor=lightgray]{n92}{n141}
\ncline[linecolor=lightgray]{n107}{n141}
\ncline[linecolor=lightgray]{n87}{n141}
\ncline[linecolor=lightgray]{n142}{n143}
\ncline[linecolor=lightgray]{n2}{n142}
\ncline[linecolor=lightgray]{n12}{n142}
\ncline[linecolor=lightgray]{n126}{n142}
\ncline[linecolor=lightgray]{n77}{n143}
\ncline[linecolor=lightgray]{n119}{n143}
\ncline[linecolor=lightgray]{n142}{n143}
\ncline[linecolor=lightgray]{n89}{n143}
\ncline[linecolor=lightgray]{n58}{n143}
\ncline[linecolor=lightgray]{n79}{n143}
\ncline[linecolor=lightgray]{n123}{n144}
\ncline[linecolor=lightgray]{n99}{n144}
\ncline[linecolor=lightgray]{n7}{n144}
\ncline[linecolor=lightgray]{n120}{n144}
\ncline[linecolor=lightgray]{n51}{n144}
\ncline[linecolor=lightgray]{n137}{n144}
\ncline[linecolor=lightgray]{n117}{n144}
\ncline[linecolor=lightgray]{n49}{n145}
\ncline[linecolor=lightgray]{n36}{n146}
\ncline[linecolor=lightgray]{n12}{n146}
\ncline[linecolor=lightgray]{n33}{n146}
\ncline[linecolor=lightgray]{n26}{n146}
\ncline[linecolor=lightgray]{n40}{n146}
\ncline[linecolor=lightgray]{n65}{n146}
\ncline[linecolor=lightgray]{n64}{n146}
\ncline[linecolor=lightgray]{n35}{n146}
\ncline[linecolor=lightgray]{n110}{n146}
\ncline[linecolor=lightgray]{n131}{n147}
\ncline[linecolor=lightgray]{n129}{n147}
\ncline[linecolor=lightgray]{n107}{n147}
\ncline[linecolor=lightgray]{n61}{n147}
\ncline[linecolor=lightgray]{n103}{n147}
\ncline[linecolor=lightgray]{n138}{n147}
\ncline[linecolor=lightgray]{n87}{n147}
\ncline[linecolor=lightgray]{n116}{n148}
\ncline[linecolor=lightgray]{n140}{n148}
\ncline[linecolor=lightgray]{n57}{n148}
\ncline[linecolor=lightgray]{n132}{n148}
\ncline[linecolor=lightgray]{n71}{n148}
\ncline[linecolor=lightgray]{n14}{n34}
\ncline[linecolor=lightgray]{n15}{n34}
\ncline[linecolor=lightgray]{n25}{n34}
\ncline[linecolor=lightgray]{n34}{n37}
\ncline[linecolor=lightgray]{n34}{n54}
\ncline[linecolor=lightgray]{n34}{n63}
\ncline[linecolor=lightgray]{n34}{n67}
\ncline[linecolor=lightgray]{n34}{n75}
\ncline[linecolor=lightgray]{n34}{n78}
\ncline[linecolor=lightgray]{n34}{n105}
\ncline[linecolor=lightgray]{n34}{n121}
\ncline[linecolor=lightgray]{n34}{n125}
\ncline[linecolor=lightgray]{n34}{n134}
\ncline[linecolor=lightgray]{n34}{n145}
\ncline[linecolor=lightgray]{n63}{n96}
\ncline[linecolor=lightgray]{n96}{n105}
\ncline[linecolor=lightgray]{n12}{n24}
\ncline[linecolor=lightgray]{n17}{n24}
\ncline[linecolor=lightgray]{n23}{n24}
\ncline[linecolor=lightgray]{n24}{n26}
\ncline[linecolor=lightgray]{n24}{n32}
\ncline[linecolor=lightgray]{n24}{n35}
\ncline[linecolor=lightgray]{n24}{n40}
\ncline[linecolor=lightgray]{n24}{n44}
\ncline[linecolor=lightgray]{n24}{n50}
\ncline[linecolor=lightgray]{n24}{n51}
\ncline[linecolor=lightgray]{n24}{n64}
\ncline[linecolor=lightgray]{n24}{n65}
\ncline[linecolor=lightgray]{n24}{n110}
\ncline[linecolor=lightgray]{n24}{n126}
\ncline[linecolor=lightgray]{n24}{n139}
\ncline[linecolor=lightgray]{n24}{n146}
\ncline[linestyle=dotted,linecolor=lightgray]{n4}{n108}
\ncline[linestyle=dotted,linecolor=lightgray]{n23}{n108}
\ncline[linestyle=dotted,linecolor=lightgray]{n26}{n108}
\ncline[linestyle=dotted,linecolor=lightgray]{n33}{n108}
\ncline[linecolor=lightgray]{n34}{n108}
\ncline[linestyle=dotted,linecolor=lightgray]{n35}{n108}
\ncline[linestyle=dotted,linecolor=lightgray]{n36}{n108}
\ncline[linestyle=dotted,linecolor=lightgray]{n47}{n108}
\ncline[linestyle=dotted,linecolor=lightgray]{n59}{n108}
\ncline[linecolor=lightgray]{n81}{n108}
\ncline[linecolor=lightgray]{n85}{n108}
\ncline[linecolor=lightgray]{n96}{n108}
\ncline[linestyle=dotted,linecolor=lightgray]{n98}{n108}
\ncline[linestyle=dotted,linecolor=lightgray]{n108}{n110}
\ncline[linestyle=dotted,linecolor=lightgray]{n108}{n114}
\ncline[linestyle=dotted,linecolor=lightgray]{n108}{n139}
\ncline[linestyle=dotted,linecolor=lightgray]{n14}{n24}
\ncline[linestyle=dotted,linecolor=lightgray]{n15}{n24}
\ncline[linestyle=dotted,linecolor=lightgray]{n24}{n25}
\ncline[linestyle=dotted,linecolor=lightgray]{n24}{n27}
\ncline[linestyle=dotted,linecolor=lightgray]{n24}{n37}
\ncline[linestyle=dotted,linecolor=lightgray]{n24}{n54}
\ncline[linestyle=dotted,linecolor=lightgray]{n24}{n63}
\ncline[linestyle=dotted,linecolor=lightgray]{n24}{n67}
\ncline[linestyle=dotted,linecolor=lightgray]{n24}{n75}
\ncline[linestyle=dotted,linecolor=lightgray]{n24}{n78}
\ncline[linestyle=dotted,linecolor=lightgray]{n24}{n88}
\ncline[linestyle=dotted,linecolor=lightgray]{n24}{n90}
\ncline[linestyle=dotted,linecolor=lightgray]{n24}{n105}
\ncline[linestyle=dotted,linecolor=lightgray]{n24}{n121}
\ncline[linestyle=dotted,linecolor=lightgray]{n24}{n125}
\ncline[linestyle=dotted,linecolor=lightgray]{n24}{n134}
\ncline[linestyle=dotted,linecolor=lightgray]{n24}{n145}
\ncline[linecolor=lightgray]{n3}{n108}
\ncline[linecolor=lightgray]{n5}{n108}
\ncline[linecolor=lightgray]{n14}{n108}
\ncline[linecolor=lightgray]{n27}{n108}
\ncline[linecolor=lightgray]{n45}{n108}
\ncline[linecolor=lightgray]{n52}{n108}
\ncline[linecolor=lightgray]{n67}{n108}
\ncline[linecolor=lightgray]{n75}{n108}
\ncline[linecolor=lightgray]{n76}{n108}
\ncline[linecolor=lightgray]{n78}{n108}
\ncline[linecolor=lightgray]{n91}{n108}
\ncline[linecolor=lightgray]{n106}{n108}
\ncline[linecolor=lightgray]{n108}{n109}
\ncline[linecolor=lightgray]{n108}{n121}
\ncline[linecolor=lightgray]{n108}{n125}
\ncline[linecolor=lightgray]{n108}{n134}
\ncline[linestyle=dotted,linecolor=lightgray]{n24}{n108}
\ncline[linecolor=darkgray]{n0}{n79}
\ncline[linecolor=darkgray]{n0}{n12}
\ncline[linecolor=darkgray]{n0}{n17}
\ncline[linecolor=darkgray]{n0}{n64}
\ncline[linecolor=darkgray]{n0}{n142}
\ncline[linecolor=darkgray]{n0}{n72}
\ncline[linecolor=darkgray]{n0}{n40}
\ncline[linecolor=darkgray]{n0}{n41}
\ncline[linecolor=darkgray]{n0}{n126}
\ncline[linecolor=darkgray]{n1}{n100}
\ncline[linecolor=darkgray]{n2}{n112}
\ncline[linecolor=darkgray]{n2}{n82}
\ncline[linecolor=darkgray]{n3}{n90}
\ncline[linecolor=darkgray]{n3}{n14}
\ncline[linecolor=darkgray]{n3}{n67}
\ncline[linecolor=darkgray]{n3}{n134}
\ncline[linecolor=darkgray]{n3}{n76}
\ncline[linecolor=darkgray]{n3}{n106}
\ncline[linecolor=darkgray]{n5}{n125}
\ncline[linecolor=darkgray]{n5}{n52}
\ncline[linecolor=darkgray]{n6}{n8}
\ncline[linecolor=darkgray]{n6}{n103}
\ncline[linecolor=darkgray]{n6}{n29}
\ncline[linecolor=darkgray]{n6}{n42}
\ncline[linecolor=darkgray]{n7}{n20}
\ncline[linecolor=darkgray]{n7}{n123}
\ncline[linecolor=darkgray]{n8}{n42}
\ncline[linecolor=darkgray]{n8}{n29}
\ncline[linecolor=darkgray]{n6}{n8}
\ncline[linecolor=darkgray]{n8}{n103}
\ncline[linecolor=darkgray]{n8}{n56}
\ncline[linecolor=darkgray]{n8}{n147}
\ncline[linecolor=darkgray]{n8}{n92}
\ncline[linecolor=darkgray]{n10}{n15}
\ncline[linecolor=darkgray]{n10}{n149}
\ncline[linecolor=darkgray]{n10}{n135}
\ncline[linecolor=darkgray]{n10}{n37}
\ncline[linecolor=darkgray]{n11}{n124}
\ncline[linecolor=darkgray]{n11}{n113}
\ncline[linecolor=darkgray]{n11}{n135}
\ncline[linecolor=darkgray]{n11}{n62}
\ncline[linecolor=darkgray]{n11}{n130}
\ncline[linecolor=darkgray]{n11}{n30}
\ncline[linecolor=darkgray]{n11}{n88}
\ncline[linecolor=darkgray]{n11}{n37}
\ncline[linecolor=darkgray]{n12}{n44}
\ncline[linecolor=darkgray]{n0}{n12}
\ncline[linecolor=darkgray]{n12}{n126}
\ncline[linecolor=darkgray]{n12}{n36}
\ncline[linecolor=darkgray]{n12}{n40}
\ncline[linecolor=darkgray]{n12}{n114}
\ncline[linecolor=darkgray]{n12}{n33}
\ncline[linecolor=darkgray]{n13}{n89}
\ncline[linecolor=darkgray]{n13}{n123}
\ncline[linecolor=darkgray]{n13}{n117}
\ncline[linecolor=darkgray]{n13}{n99}
\ncline[linecolor=darkgray]{n13}{n101}
\ncline[linecolor=darkgray]{n14}{n16}
\ncline[linecolor=darkgray]{n14}{n31}
\ncline[linecolor=darkgray]{n14}{n90}
\ncline[linecolor=darkgray]{n3}{n14}
\ncline[linecolor=darkgray]{n10}{n15}
\ncline[linecolor=darkgray]{n15}{n62}
\ncline[linecolor=darkgray]{n15}{n69}
\ncline[linecolor=darkgray]{n15}{n111}
\ncline[linecolor=darkgray]{n15}{n105}
\ncline[linecolor=darkgray]{n15}{n125}
\ncline[linecolor=darkgray]{n14}{n16}
\ncline[linecolor=darkgray]{n16}{n67}
\ncline[linecolor=darkgray]{n16}{n134}
\ncline[linecolor=darkgray]{n16}{n31}
\ncline[linecolor=darkgray]{n16}{n75}
\ncline[linecolor=darkgray]{n16}{n125}
\ncline[linecolor=darkgray]{n17}{n36}
\ncline[linecolor=darkgray]{n0}{n17}
\ncline[linecolor=darkgray]{n17}{n23}
\ncline[linecolor=darkgray]{n17}{n126}
\ncline[linecolor=darkgray]{n17}{n72}
\ncline[linecolor=darkgray]{n17}{n35}
\ncline[linecolor=darkgray]{n17}{n146}
\ncline[linecolor=darkgray]{n19}{n28}
\ncline[linecolor=darkgray]{n19}{n46}
\ncline[linecolor=darkgray]{n19}{n43}
\ncline[linecolor=darkgray]{n7}{n20}
\ncline[linecolor=darkgray]{n20}{n120}
\ncline[linecolor=darkgray]{n20}{n86}
\ncline[linecolor=darkgray]{n20}{n123}
\ncline[linecolor=darkgray]{n20}{n51}
\ncline[linecolor=darkgray]{n21}{n61}
\ncline[linecolor=darkgray]{n21}{n138}
\ncline[linecolor=darkgray]{n21}{n147}
\ncline[linecolor=darkgray]{n21}{n92}
\ncline[linecolor=darkgray]{n21}{n118}
\ncline[linestyle=dotted,linecolor=darkgray]{n21}{n84}
\ncline[linecolor=darkgray]{n21}{n42}
\ncline[linecolor=darkgray]{n22}{n115}
\ncline[linecolor=darkgray]{n23}{n33}
\ncline[linecolor=darkgray]{n17}{n23}
\ncline[linecolor=darkgray]{n23}{n110}
\ncline[linecolor=darkgray]{n23}{n114}
\ncline[linecolor=darkgray]{n23}{n146}
\ncline[linecolor=darkgray]{n23}{n98}
\ncline[linecolor=darkgray]{n25}{n136}
\ncline[linecolor=darkgray]{n25}{n62}
\ncline[linecolor=darkgray]{n26}{n33}
\ncline[linecolor=darkgray]{n26}{n98}
\ncline[linecolor=darkgray]{n26}{n44}
\ncline[linecolor=darkgray]{n27}{n74}
\ncline[linecolor=darkgray]{n27}{n106}
\ncline[linecolor=darkgray]{n27}{n90}
\ncline[linecolor=darkgray]{n19}{n28}
\ncline[linecolor=darkgray]{n8}{n29}
\ncline[linecolor=darkgray]{n6}{n29}
\ncline[linecolor=darkgray]{n29}{n46}
\ncline[linecolor=darkgray]{n30}{n135}
\ncline[linecolor=darkgray]{n30}{n48}
\ncline[linecolor=darkgray]{n30}{n115}
\ncline[linecolor=darkgray]{n11}{n30}
\ncline[linecolor=darkgray]{n30}{n88}
\ncline[linecolor=darkgray]{n30}{n73}
\ncline[linecolor=darkgray]{n30}{n130}
\ncline[linecolor=darkgray]{n31}{n62}
\ncline[linecolor=darkgray]{n14}{n31}
\ncline[linecolor=darkgray]{n31}{n67}
\ncline[linecolor=darkgray]{n31}{n69}
\ncline[linecolor=darkgray]{n31}{n134}
\ncline[linecolor=darkgray]{n16}{n31}
\ncline[linecolor=darkgray]{n31}{n45}
\ncline[linecolor=darkgray]{n31}{n78}
\ncline[linecolor=darkgray]{n31}{n124}
\ncline[linecolor=darkgray]{n32}{n51}
\ncline[linecolor=darkgray]{n23}{n33}
\ncline[linecolor=darkgray]{n33}{n72}
\ncline[linecolor=darkgray]{n33}{n50}
\ncline[linecolor=darkgray]{n33}{n139}
\ncline[linecolor=darkgray]{n26}{n33}
\ncline[linecolor=darkgray]{n12}{n33}
\ncline[linecolor=darkgray]{n34}{n96}
\ncline[linecolor=darkgray]{n34}{n81}
\ncline[linecolor=darkgray]{n35}{n36}
\ncline[linecolor=darkgray]{n35}{n44}
\ncline[linecolor=darkgray]{n17}{n35}
\ncline[linecolor=darkgray]{n35}{n47}
\ncline[linecolor=darkgray]{n17}{n36}
\ncline[linecolor=darkgray]{n35}{n36}
\ncline[linecolor=darkgray]{n36}{n44}
\ncline[linecolor=darkgray]{n12}{n36}
\ncline[linecolor=darkgray]{n36}{n50}
\ncline[linecolor=darkgray]{n36}{n72}
\ncline[linecolor=darkgray]{n37}{n127}
\ncline[linecolor=darkgray]{n37}{n88}
\ncline[linecolor=darkgray]{n10}{n37}
\ncline[linecolor=darkgray]{n11}{n37}
\ncline[linestyle=dotted,linecolor=darkgray]{n39}{n61}
\ncline[linecolor=darkgray]{n39}{n77}
\ncline[linecolor=darkgray]{n39}{n119}
\ncline[linecolor=darkgray]{n40}{n50}
\ncline[linecolor=darkgray]{n40}{n79}
\ncline[linecolor=darkgray]{n0}{n40}
\ncline[linecolor=darkgray]{n12}{n40}
\ncline[linecolor=darkgray]{n40}{n64}
\ncline[linecolor=darkgray]{n41}{n142}
\ncline[linecolor=darkgray]{n41}{n82}
\ncline[linecolor=darkgray]{n41}{n99}
\ncline[linecolor=darkgray]{n41}{n112}
\ncline[linecolor=darkgray]{n0}{n41}
\ncline[linecolor=darkgray]{n41}{n144}
\ncline[linecolor=darkgray]{n41}{n143}
\ncline[linecolor=darkgray]{n8}{n42}
\ncline[linecolor=darkgray]{n42}{n107}
\ncline[linecolor=darkgray]{n42}{n147}
\ncline[linecolor=darkgray]{n42}{n103}
\ncline[linecolor=darkgray]{n42}{n138}
\ncline[linecolor=darkgray]{n6}{n42}
\ncline[linecolor=darkgray]{n21}{n42}
\ncline[linecolor=darkgray]{n42}{n129}
\ncline[linecolor=darkgray]{n19}{n43}
\ncline[linecolor=darkgray]{n44}{n126}
\ncline[linecolor=darkgray]{n12}{n44}
\ncline[linecolor=darkgray]{n36}{n44}
\ncline[linecolor=darkgray]{n44}{n114}
\ncline[linecolor=darkgray]{n35}{n44}
\ncline[linecolor=darkgray]{n26}{n44}
\ncline[linecolor=darkgray]{n31}{n45}
\ncline[linecolor=darkgray]{n45}{n52}
\ncline[linecolor=darkgray]{n45}{n76}
\ncline[linecolor=darkgray]{n19}{n46}
\ncline[linecolor=darkgray]{n46}{n107}
\ncline[linecolor=darkgray]{n29}{n46}
\ncline[linecolor=darkgray]{n46}{n95}
\ncline[linecolor=darkgray]{n47}{n98}
\ncline[linecolor=darkgray]{n47}{n139}
\ncline[linecolor=darkgray]{n35}{n47}
\ncline[linecolor=darkgray]{n47}{n59}
\ncline[linecolor=darkgray]{n48}{n111}
\ncline[linecolor=darkgray]{n48}{n69}
\ncline[linecolor=darkgray]{n48}{n73}
\ncline[linecolor=darkgray]{n30}{n48}
\ncline[linecolor=darkgray]{n48}{n88}
\ncline[linecolor=darkgray]{n48}{n149}
\ncline[linecolor=darkgray]{n48}{n145}
\ncline[linecolor=darkgray]{n48}{n54}
\ncline[linecolor=darkgray]{n49}{n88}
\ncline[linecolor=darkgray]{n49}{n135}
\ncline[linecolor=darkgray]{n33}{n50}
\ncline[linecolor=darkgray]{n36}{n50}
\ncline[linecolor=darkgray]{n40}{n50}
\ncline[linecolor=darkgray]{n50}{n114}
\ncline[linecolor=darkgray]{n50}{n146}
\ncline[linecolor=darkgray]{n50}{n72}
\ncline[linecolor=darkgray]{n32}{n51}
\ncline[linecolor=darkgray]{n20}{n51}
\ncline[linecolor=darkgray]{n5}{n52}
\ncline[linecolor=darkgray]{n45}{n52}
\ncline[linecolor=darkgray]{n52}{n78}
\ncline[linecolor=darkgray]{n53}{n55}
\ncline[linecolor=darkgray]{n54}{n136}
\ncline[linecolor=darkgray]{n54}{n135}
\ncline[linecolor=darkgray]{n54}{n149}
\ncline[linecolor=darkgray]{n48}{n54}
\ncline[linecolor=darkgray]{n54}{n63}
\ncline[linecolor=darkgray]{n55}{n92}
\ncline[linecolor=darkgray]{n53}{n55}
\ncline[linecolor=darkgray]{n55}{n87}
\ncline[linecolor=darkgray]{n56}{n147}
\ncline[linecolor=darkgray]{n56}{n141}
\ncline[linecolor=darkgray]{n56}{n107}
\ncline[linecolor=darkgray]{n8}{n56}
\ncline[linecolor=darkgray]{n56}{n103}
\ncline[linecolor=darkgray]{n58}{n79}
\ncline[linecolor=darkgray]{n58}{n142}
\ncline[linecolor=darkgray]{n58}{n126}
\ncline[linecolor=darkgray]{n59}{n114}
\ncline[linecolor=darkgray]{n47}{n59}
\ncline[linecolor=darkgray]{n21}{n61}
\ncline[linecolor=darkgray]{n61}{n129}
\ncline[linestyle=dotted,linecolor=darkgray]{n39}{n61}
\ncline[linecolor=darkgray]{n61}{n83}
\ncline[linecolor=darkgray]{n61}{n138}
\ncline[linestyle=dotted,linecolor=darkgray]{n61}{n89}
\ncline[linecolor=darkgray]{n61}{n131}
\ncline[linecolor=darkgray]{n61}{n118}
\ncline[linecolor=darkgray]{n31}{n62}
\ncline[linecolor=darkgray]{n15}{n62}
\ncline[linecolor=darkgray]{n11}{n62}
\ncline[linecolor=darkgray]{n25}{n62}
\ncline[linecolor=darkgray]{n54}{n63}
\ncline[linecolor=darkgray]{n63}{n145}
\ncline[linecolor=darkgray]{n63}{n121}
\ncline[linecolor=darkgray]{n0}{n64}
\ncline[linecolor=darkgray]{n64}{n114}
\ncline[linecolor=darkgray]{n40}{n64}
\ncline[linecolor=darkgray]{n64}{n126}
\ncline[linecolor=darkgray]{n65}{n126}
\ncline[linecolor=darkgray]{n16}{n67}
\ncline[linecolor=darkgray]{n31}{n67}
\ncline[linecolor=darkgray]{n67}{n90}
\ncline[linecolor=darkgray]{n3}{n67}
\ncline[linecolor=darkgray]{n68}{n74}
\ncline[linecolor=darkgray]{n69}{n73}
\ncline[linecolor=darkgray]{n48}{n69}
\ncline[linecolor=darkgray]{n31}{n69}
\ncline[linecolor=darkgray]{n15}{n69}
\ncline[linecolor=darkgray]{n69}{n145}
\ncline[linecolor=darkgray]{n69}{n135}
\ncline[linecolor=darkgray]{n70}{n148}
\ncline[linecolor=darkgray]{n71}{n132}
\ncline[linecolor=darkgray]{n72}{n142}
\ncline[linecolor=darkgray]{n72}{n114}
\ncline[linecolor=darkgray]{n33}{n72}
\ncline[linecolor=darkgray]{n0}{n72}
\ncline[linecolor=darkgray]{n17}{n72}
\ncline[linecolor=darkgray]{n72}{n112}
\ncline[linecolor=darkgray]{n36}{n72}
\ncline[linecolor=darkgray]{n50}{n72}
\ncline[linecolor=darkgray]{n69}{n73}
\ncline[linecolor=darkgray]{n48}{n73}
\ncline[linecolor=darkgray]{n73}{n145}
\ncline[linecolor=darkgray]{n30}{n73}
\ncline[linecolor=darkgray]{n27}{n74}
\ncline[linecolor=darkgray]{n68}{n74}
\ncline[linecolor=darkgray]{n75}{n106}
\ncline[linecolor=darkgray]{n75}{n121}
\ncline[linecolor=darkgray]{n75}{n105}
\ncline[linecolor=darkgray]{n16}{n75}
\ncline[linecolor=darkgray]{n76}{n109}
\ncline[linecolor=darkgray]{n3}{n76}
\ncline[linecolor=darkgray]{n45}{n76}
\ncline[linecolor=darkgray]{n76}{n78}
\ncline[linecolor=darkgray]{n39}{n77}
\ncline[linecolor=darkgray]{n77}{n123}
\ncline[linecolor=darkgray]{n77}{n117}
\ncline[linecolor=darkgray]{n77}{n101}
\ncline[linecolor=darkgray]{n77}{n119}
\ncline[linecolor=darkgray]{n31}{n78}
\ncline[linecolor=darkgray]{n52}{n78}
\ncline[linecolor=darkgray]{n76}{n78}
\ncline[linecolor=darkgray]{n0}{n79}
\ncline[linecolor=darkgray]{n79}{n142}
\ncline[linecolor=darkgray]{n79}{n126}
\ncline[linecolor=darkgray]{n58}{n79}
\ncline[linecolor=darkgray]{n40}{n79}
\ncline[linecolor=darkgray]{n79}{n144}
\ncline[linecolor=darkgray]{n81}{n85}
\ncline[linecolor=darkgray]{n34}{n81}
\ncline[linecolor=darkgray]{n82}{n104}
\ncline[linecolor=darkgray]{n82}{n122}
\ncline[linecolor=darkgray]{n41}{n82}
\ncline[linecolor=darkgray]{n82}{n112}
\ncline[linecolor=darkgray]{n2}{n82}
\ncline[linecolor=darkgray]{n82}{n143}
\ncline[linecolor=darkgray]{n82}{n137}
\ncline[linecolor=darkgray]{n82}{n142}
\ncline[linecolor=darkgray]{n83}{n129}
\ncline[linecolor=darkgray]{n61}{n83}
\ncline[linecolor=darkgray]{n83}{n131}
\ncline[linecolor=darkgray]{n84}{n122}
\ncline[linestyle=dotted,linecolor=darkgray]{n21}{n84}
\ncline[linecolor=darkgray]{n84}{n100}
\ncline[linestyle=dotted,linecolor=darkgray]{n84}{n102}
\ncline[linecolor=darkgray]{n81}{n85}
\ncline[linecolor=darkgray]{n85}{n96}
\ncline[linecolor=darkgray]{n20}{n86}
\ncline[linecolor=darkgray]{n86}{n137}
\ncline[linecolor=darkgray]{n87}{n103}
\ncline[linecolor=darkgray]{n55}{n87}
\ncline[linecolor=darkgray]{n49}{n88}
\ncline[linecolor=darkgray]{n88}{n135}
\ncline[linecolor=darkgray]{n48}{n88}
\ncline[linecolor=darkgray]{n37}{n88}
\ncline[linecolor=darkgray]{n11}{n88}
\ncline[linecolor=darkgray]{n88}{n127}
\ncline[linecolor=darkgray]{n30}{n88}
\ncline[linecolor=darkgray]{n88}{n130}
\ncline[linecolor=darkgray]{n13}{n89}
\ncline[linecolor=darkgray]{n89}{n119}
\ncline[linecolor=darkgray]{n89}{n117}
\ncline[linecolor=darkgray]{n89}{n123}
\ncline[linestyle=dotted,linecolor=darkgray]{n61}{n89}
\ncline[linecolor=darkgray]{n89}{n144}
\ncline[linecolor=darkgray]{n3}{n90}
\ncline[linecolor=darkgray]{n14}{n90}
\ncline[linecolor=darkgray]{n67}{n90}
\ncline[linecolor=darkgray]{n90}{n134}
\ncline[linecolor=darkgray]{n90}{n111}
\ncline[linecolor=darkgray]{n27}{n90}
\ncline[linecolor=darkgray]{n90}{n124}
\ncline[linecolor=darkgray]{n92}{n102}
\ncline[linecolor=darkgray]{n92}{n147}
\ncline[linecolor=darkgray]{n21}{n92}
\ncline[linecolor=darkgray]{n55}{n92}
\ncline[linecolor=darkgray]{n8}{n92}
\ncline[linecolor=darkgray]{n94}{n119}
\ncline[linecolor=darkgray]{n46}{n95}
\ncline[linecolor=darkgray]{n34}{n96}
\ncline[linecolor=darkgray]{n85}{n96}
\ncline[linecolor=darkgray]{n97}{n122}
\ncline[linecolor=darkgray]{n97}{n104}
\ncline[linecolor=darkgray]{n97}{n100}
\ncline[linecolor=darkgray]{n47}{n98}
\ncline[linecolor=darkgray]{n23}{n98}
\ncline[linecolor=darkgray]{n26}{n98}
\ncline[linecolor=darkgray]{n99}{n137}
\ncline[linecolor=darkgray]{n41}{n99}
\ncline[linecolor=darkgray]{n13}{n99}
\ncline[linecolor=darkgray]{n99}{n122}
\ncline[linecolor=darkgray]{n100}{n122}
\ncline[linecolor=darkgray]{n1}{n100}
\ncline[linecolor=darkgray]{n100}{n143}
\ncline[linecolor=darkgray]{n84}{n100}
\ncline[linecolor=darkgray]{n97}{n100}
\ncline[linecolor=darkgray]{n101}{n142}
\ncline[linecolor=darkgray]{n101}{n144}
\ncline[linecolor=darkgray]{n77}{n101}
\ncline[linecolor=darkgray]{n101}{n143}
\ncline[linecolor=darkgray]{n13}{n101}
\ncline[linecolor=darkgray]{n92}{n102}
\ncline[linestyle=dotted,linecolor=darkgray]{n84}{n102}
\ncline[linecolor=darkgray]{n102}{n147}
\ncline[linecolor=darkgray]{n8}{n103}
\ncline[linecolor=darkgray]{n42}{n103}
\ncline[linecolor=darkgray]{n6}{n103}
\ncline[linecolor=darkgray]{n87}{n103}
\ncline[linecolor=darkgray]{n56}{n103}
\ncline[linecolor=darkgray]{n82}{n104}
\ncline[linecolor=darkgray]{n104}{n112}
\ncline[linecolor=darkgray]{n104}{n143}
\ncline[linecolor=darkgray]{n97}{n104}
\ncline[linecolor=darkgray]{n75}{n105}
\ncline[linecolor=darkgray]{n15}{n105}
\ncline[linecolor=darkgray]{n27}{n106}
\ncline[linecolor=darkgray]{n75}{n106}
\ncline[linecolor=darkgray]{n3}{n106}
\ncline[linecolor=darkgray]{n42}{n107}
\ncline[linecolor=darkgray]{n56}{n107}
\ncline[linecolor=darkgray]{n46}{n107}
\ncline[linecolor=darkgray]{n76}{n109}
\ncline[linecolor=darkgray]{n23}{n110}
\ncline[linecolor=darkgray]{n48}{n111}
\ncline[linecolor=darkgray]{n15}{n111}
\ncline[linecolor=darkgray]{n90}{n111}
\ncline[linecolor=darkgray]{n111}{n135}
\ncline[linecolor=darkgray]{n2}{n112}
\ncline[linecolor=darkgray]{n112}{n143}
\ncline[linecolor=darkgray]{n104}{n112}
\ncline[linecolor=darkgray]{n82}{n112}
\ncline[linecolor=darkgray]{n72}{n112}
\ncline[linecolor=darkgray]{n41}{n112}
\ncline[linecolor=darkgray]{n11}{n113}
\ncline[linecolor=darkgray]{n113}{n127}
\ncline[linecolor=darkgray]{n72}{n114}
\ncline[linecolor=darkgray]{n44}{n114}
\ncline[linecolor=darkgray]{n23}{n114}
\ncline[linecolor=darkgray]{n50}{n114}
\ncline[linecolor=darkgray]{n59}{n114}
\ncline[linecolor=darkgray]{n64}{n114}
\ncline[linecolor=darkgray]{n12}{n114}
\ncline[linecolor=darkgray]{n114}{n139}
\ncline[linecolor=darkgray]{n115}{n130}
\ncline[linecolor=darkgray]{n22}{n115}
\ncline[linecolor=darkgray]{n30}{n115}
\ncline[linecolor=darkgray]{n115}{n133}
\ncline[linecolor=darkgray]{n117}{n123}
\ncline[linecolor=darkgray]{n117}{n119}
\ncline[linecolor=darkgray]{n89}{n117}
\ncline[linecolor=darkgray]{n77}{n117}
\ncline[linecolor=darkgray]{n13}{n117}
\ncline[linecolor=darkgray]{n21}{n118}
\ncline[linecolor=darkgray]{n118}{n129}
\ncline[linecolor=darkgray]{n61}{n118}
\ncline[linecolor=darkgray]{n117}{n119}
\ncline[linecolor=darkgray]{n89}{n119}
\ncline[linecolor=darkgray]{n39}{n119}
\ncline[linecolor=darkgray]{n119}{n123}
\ncline[linecolor=darkgray]{n94}{n119}
\ncline[linecolor=darkgray]{n77}{n119}
\ncline[linecolor=darkgray]{n20}{n120}
\ncline[linecolor=darkgray]{n120}{n123}
\ncline[linecolor=darkgray]{n75}{n121}
\ncline[linecolor=darkgray]{n63}{n121}
\ncline[linecolor=darkgray]{n97}{n122}
\ncline[linecolor=darkgray]{n100}{n122}
\ncline[linecolor=darkgray]{n122}{n128}
\ncline[linecolor=darkgray]{n82}{n122}
\ncline[linecolor=darkgray]{n84}{n122}
\ncline[linecolor=darkgray]{n122}{n143}
\ncline[linecolor=darkgray]{n99}{n122}
\ncline[linecolor=darkgray]{n117}{n123}
\ncline[linecolor=darkgray]{n7}{n123}
\ncline[linecolor=darkgray]{n120}{n123}
\ncline[linecolor=darkgray]{n20}{n123}
\ncline[linecolor=darkgray]{n89}{n123}
\ncline[linecolor=darkgray]{n13}{n123}
\ncline[linecolor=darkgray]{n119}{n123}
\ncline[linecolor=darkgray]{n77}{n123}
\ncline[linecolor=darkgray]{n11}{n124}
\ncline[linecolor=darkgray]{n124}{n127}
\ncline[linecolor=darkgray]{n31}{n124}
\ncline[linecolor=darkgray]{n90}{n124}
\ncline[linecolor=darkgray]{n5}{n125}
\ncline[linecolor=darkgray]{n16}{n125}
\ncline[linecolor=darkgray]{n15}{n125}
\ncline[linecolor=darkgray]{n44}{n126}
\ncline[linecolor=darkgray]{n12}{n126}
\ncline[linecolor=darkgray]{n79}{n126}
\ncline[linecolor=darkgray]{n17}{n126}
\ncline[linecolor=darkgray]{n65}{n126}
\ncline[linecolor=darkgray]{n58}{n126}
\ncline[linecolor=darkgray]{n0}{n126}
\ncline[linecolor=darkgray]{n64}{n126}
\ncline[linecolor=darkgray]{n113}{n127}
\ncline[linecolor=darkgray]{n127}{n135}
\ncline[linecolor=darkgray]{n127}{n130}
\ncline[linecolor=darkgray]{n37}{n127}
\ncline[linecolor=darkgray]{n124}{n127}
\ncline[linecolor=darkgray]{n127}{n136}
\ncline[linecolor=darkgray]{n88}{n127}
\ncline[linecolor=darkgray]{n122}{n128}
\ncline[linecolor=darkgray]{n128}{n142}
\ncline[linecolor=darkgray]{n83}{n129}
\ncline[linecolor=darkgray]{n61}{n129}
\ncline[linecolor=darkgray]{n129}{n138}
\ncline[linecolor=darkgray]{n118}{n129}
\ncline[linecolor=darkgray]{n42}{n129}
\ncline[linecolor=darkgray]{n115}{n130}
\ncline[linecolor=darkgray]{n127}{n130}
\ncline[linecolor=darkgray]{n11}{n130}
\ncline[linecolor=darkgray]{n30}{n130}
\ncline[linecolor=darkgray]{n88}{n130}
\ncline[linecolor=darkgray]{n83}{n131}
\ncline[linecolor=darkgray]{n61}{n131}
\ncline[linecolor=darkgray]{n71}{n132}
\ncline[linecolor=darkgray]{n115}{n133}
\ncline[linecolor=darkgray]{n16}{n134}
\ncline[linecolor=darkgray]{n31}{n134}
\ncline[linecolor=darkgray]{n90}{n134}
\ncline[linecolor=darkgray]{n3}{n134}
\ncline[linecolor=darkgray]{n11}{n135}
\ncline[linecolor=darkgray]{n127}{n135}
\ncline[linecolor=darkgray]{n30}{n135}
\ncline[linecolor=darkgray]{n88}{n135}
\ncline[linecolor=darkgray]{n54}{n135}
\ncline[linecolor=darkgray]{n135}{n145}
\ncline[linecolor=darkgray]{n10}{n135}
\ncline[linecolor=darkgray]{n69}{n135}
\ncline[linecolor=darkgray]{n49}{n135}
\ncline[linecolor=darkgray]{n111}{n135}
\ncline[linecolor=darkgray]{n54}{n136}
\ncline[linecolor=darkgray]{n25}{n136}
\ncline[linecolor=darkgray]{n127}{n136}
\ncline[linecolor=darkgray]{n99}{n137}
\ncline[linecolor=darkgray]{n86}{n137}
\ncline[linecolor=darkgray]{n82}{n137}
\ncline[linecolor=darkgray]{n21}{n138}
\ncline[linecolor=darkgray]{n61}{n138}
\ncline[linecolor=darkgray]{n129}{n138}
\ncline[linecolor=darkgray]{n42}{n138}
\ncline[linecolor=darkgray]{n33}{n139}
\ncline[linecolor=darkgray]{n47}{n139}
\ncline[linecolor=darkgray]{n114}{n139}
\ncline[linecolor=darkgray]{n56}{n141}
\ncline[linecolor=darkgray]{n72}{n142}
\ncline[linecolor=darkgray]{n41}{n142}
\ncline[linecolor=darkgray]{n79}{n142}
\ncline[linecolor=darkgray]{n0}{n142}
\ncline[linecolor=darkgray]{n101}{n142}
\ncline[linecolor=darkgray]{n128}{n142}
\ncline[linecolor=darkgray]{n58}{n142}
\ncline[linecolor=darkgray]{n82}{n142}
\ncline[linecolor=darkgray]{n100}{n143}
\ncline[linecolor=darkgray]{n122}{n143}
\ncline[linecolor=darkgray]{n112}{n143}
\ncline[linecolor=darkgray]{n104}{n143}
\ncline[linecolor=darkgray]{n82}{n143}
\ncline[linecolor=darkgray]{n101}{n143}
\ncline[linecolor=darkgray]{n41}{n143}
\ncline[linecolor=darkgray]{n79}{n144}
\ncline[linecolor=darkgray]{n89}{n144}
\ncline[linecolor=darkgray]{n101}{n144}
\ncline[linecolor=darkgray]{n41}{n144}
\ncline[linecolor=darkgray]{n69}{n145}
\ncline[linecolor=darkgray]{n135}{n145}
\ncline[linecolor=darkgray]{n145}{n149}
\ncline[linecolor=darkgray]{n48}{n145}
\ncline[linecolor=darkgray]{n73}{n145}
\ncline[linecolor=darkgray]{n63}{n145}
\ncline[linecolor=darkgray]{n50}{n146}
\ncline[linecolor=darkgray]{n23}{n146}
\ncline[linecolor=darkgray]{n17}{n146}
\ncline[linecolor=darkgray]{n21}{n147}
\ncline[linecolor=darkgray]{n56}{n147}
\ncline[linecolor=darkgray]{n42}{n147}
\ncline[linecolor=darkgray]{n92}{n147}
\ncline[linecolor=darkgray]{n8}{n147}
\ncline[linecolor=darkgray]{n102}{n147}
\ncline[linecolor=darkgray]{n70}{n148}
\ncline[linecolor=darkgray]{n10}{n149}
\ncline[linecolor=darkgray]{n48}{n149}
\ncline[linecolor=darkgray]{n54}{n149}
\ncline[linecolor=darkgray]{n145}{n149}
\ncline[linecolor=black]{n0}{n0}
\ncline[linecolor=black]{n0}{n50}
\ncline[linecolor=black]{n1}{n1}
\ncline[linecolor=black]{n1}{n97}
\ncline[linecolor=black]{n1}{n122}
\ncline[linecolor=black]{n1}{n84}
\ncline[linecolor=black]{n2}{n2}
\ncline[linecolor=black]{n2}{n128}
\ncline[linecolor=black]{n2}{n143}
\ncline[linecolor=black]{n2}{n104}
\ncline[linecolor=black]{n2}{n100}
\ncline[linecolor=black]{n2}{n122}
\ncline[linecolor=black]{n3}{n3}
\ncline[linecolor=black]{n3}{n5}
\ncline[linecolor=black]{n3}{n27}
\ncline[linecolor=black]{n3}{n91}
\ncline[linecolor=black]{n3}{n74}
\ncline[linecolor=black]{n3}{n75}
\ncline[linecolor=black]{n3}{n52}
\ncline[linecolor=black]{n3}{n45}
\ncline[linecolor=black]{n3}{n78}
\ncline[linecolor=black]{n3}{n125}
\ncline[linecolor=black]{n4}{n4}
\ncline[linecolor=black]{n5}{n5}
\ncline[linecolor=black]{n3}{n5}
\ncline[linecolor=black]{n5}{n27}
\ncline[linecolor=black]{n5}{n45}
\ncline[linecolor=black]{n5}{n74}
\ncline[linecolor=black]{n5}{n78}
\ncline[linecolor=black]{n5}{n91}
\ncline[linecolor=black]{n5}{n75}
\ncline[linecolor=black]{n5}{n106}
\ncline[linecolor=black]{n5}{n14}
\ncline[linecolor=black]{n5}{n31}
\ncline[linecolor=black]{n5}{n67}
\ncline[linecolor=black]{n5}{n134}
\ncline[linecolor=black]{n6}{n6}
\ncline[linecolor=black]{n6}{n131}
\ncline[linecolor=black]{n6}{n129}
\ncline[linecolor=black]{n7}{n7}
\ncline[linecolor=black]{n7}{n120}
\ncline[linecolor=black]{n7}{n32}
\ncline[linecolor=black]{n7}{n51}
\ncline[linecolor=black]{n8}{n8}
\ncline[linecolor=black]{n8}{n87}
\ncline[linecolor=black]{n8}{n107}
\ncline[linecolor=black]{n9}{n9}
\ncline[linecolor=black]{n10}{n10}
\ncline[linecolor=black]{n10}{n11}
\ncline[linecolor=black]{n10}{n127}
\ncline[linecolor=black]{n10}{n136}
\ncline[linecolor=black]{n10}{n111}
\ncline[linecolor=black]{n10}{n48}
\ncline[linecolor=black]{n10}{n113}
\ncline[linecolor=black]{n10}{n124}
\ncline[linecolor=black]{n10}{n25}
\ncline[linecolor=black]{n10}{n62}
\ncline[linecolor=black]{n11}{n11}
\ncline[linecolor=black]{n10}{n11}
\ncline[linecolor=black]{n11}{n127}
\ncline[linecolor=black]{n11}{n48}
\ncline[linecolor=black]{n11}{n111}
\ncline[linecolor=black]{n11}{n136}
\ncline[linecolor=black]{n12}{n12}
\ncline[linecolor=black]{n12}{n50}
\ncline[linecolor=black]{n12}{n64}
\ncline[linecolor=black]{n12}{n17}
\ncline[linecolor=black]{n13}{n13}
\ncline[linecolor=black]{n13}{n144}
\ncline[linecolor=black]{n13}{n77}
\ncline[linecolor=black]{n13}{n20}
\ncline[linecolor=black]{n13}{n86}
\ncline[linecolor=black]{n14}{n14}
\ncline[linecolor=black]{n14}{n67}
\ncline[linecolor=black]{n14}{n134}
\ncline[linecolor=black]{n14}{n45}
\ncline[linecolor=black]{n14}{n106}
\ncline[linecolor=black]{n14}{n125}
\ncline[linecolor=black]{n14}{n121}
\ncline[linecolor=black]{n14}{n18}
\ncline[linecolor=black]{n14}{n75}
\ncline[linecolor=black]{n14}{n78}
\ncline[linecolor=black]{n5}{n14}
\ncline[linecolor=black]{n14}{n27}
\ncline[linecolor=black]{n15}{n15}
\ncline[linecolor=black]{n15}{n63}
\ncline[linecolor=black]{n15}{n16}
\ncline[linecolor=black]{n15}{n25}
\ncline[linecolor=black]{n15}{n149}
\ncline[linecolor=black]{n15}{n136}
\ncline[linecolor=black]{n15}{n90}
\ncline[linecolor=black]{n15}{n18}
\ncline[linecolor=black]{n15}{n124}
\ncline[linecolor=black]{n15}{n113}
\ncline[linecolor=black]{n16}{n16}
\ncline[linecolor=black]{n16}{n149}
\ncline[linecolor=black]{n16}{n18}
\ncline[linecolor=black]{n16}{n124}
\ncline[linecolor=black]{n16}{n136}
\ncline[linecolor=black]{n15}{n16}
\ncline[linecolor=black]{n16}{n69}
\ncline[linecolor=black]{n16}{n90}
\ncline[linecolor=black]{n16}{n111}
\ncline[linecolor=black]{n16}{n113}
\ncline[linecolor=black]{n16}{n62}
\ncline[linecolor=black]{n17}{n17}
\ncline[linecolor=black]{n17}{n50}
\ncline[linecolor=black]{n17}{n44}
\ncline[linecolor=black]{n17}{n64}
\ncline[linecolor=black]{n12}{n17}
\ncline[linecolor=black]{n17}{n33}
\ncline[linecolor=black]{n17}{n114}
\ncline[linecolor=black]{n18}{n18}
\ncline[linecolor=black]{n16}{n18}
\ncline[linecolor=black]{n18}{n149}
\ncline[linecolor=black]{n18}{n31}
\ncline[linecolor=black]{n18}{n69}
\ncline[linecolor=black]{n18}{n124}
\ncline[linecolor=black]{n18}{n62}
\ncline[linecolor=black]{n18}{n136}
\ncline[linecolor=black]{n14}{n18}
\ncline[linecolor=black]{n18}{n67}
\ncline[linecolor=black]{n18}{n134}
\ncline[linecolor=black]{n15}{n18}
\ncline[linecolor=black]{n18}{n90}
\ncline[linecolor=black]{n18}{n111}
\ncline[linecolor=black]{n18}{n113}
\ncline[linecolor=black]{n18}{n106}
\ncline[linecolor=black]{n19}{n19}
\ncline[linecolor=black]{n20}{n20}
\ncline[linecolor=black]{n13}{n20}
\ncline[linecolor=black]{n20}{n144}
\ncline[linecolor=black]{n21}{n21}
\ncline[linecolor=black]{n21}{n102}
\ncline[linecolor=black]{n22}{n22}
\ncline[linecolor=black]{n23}{n23}
\ncline[linecolor=black]{n23}{n26}
\ncline[linecolor=black]{n23}{n35}
\ncline[linecolor=black]{n23}{n44}
\ncline[linecolor=black]{n23}{n139}
\ncline[linecolor=black]{n24}{n24}
\ncline[linecolor=black]{n25}{n25}
\ncline[linecolor=black]{n15}{n25}
\ncline[linecolor=black]{n25}{n63}
\ncline[linecolor=black]{n10}{n25}
\ncline[linecolor=black]{n26}{n26}
\ncline[linecolor=black]{n23}{n26}
\ncline[linecolor=black]{n26}{n139}
\ncline[linecolor=black]{n26}{n35}
\ncline[linecolor=black]{n27}{n27}
\ncline[linecolor=black]{n3}{n27}
\ncline[linecolor=black]{n27}{n75}
\ncline[linecolor=black]{n27}{n125}
\ncline[linecolor=black]{n5}{n27}
\ncline[linecolor=black]{n27}{n45}
\ncline[linecolor=black]{n27}{n78}
\ncline[linecolor=black]{n27}{n52}
\ncline[linecolor=black]{n27}{n91}
\ncline[linecolor=black]{n27}{n76}
\ncline[linecolor=black]{n14}{n27}
\ncline[linecolor=black]{n27}{n67}
\ncline[linecolor=black]{n27}{n134}
\ncline[linecolor=black]{n28}{n28}
\ncline[linecolor=black]{n29}{n29}
\ncline[linecolor=black]{n29}{n107}
\ncline[linecolor=black]{n29}{n103}
\ncline[linecolor=black]{n29}{n87}
\ncline[linecolor=black]{n30}{n30}
\ncline[linecolor=black]{n31}{n31}
\ncline[linecolor=black]{n18}{n31}
\ncline[linecolor=black]{n31}{n106}
\ncline[linecolor=black]{n5}{n31}
\ncline[linecolor=black]{n32}{n32}
\ncline[linecolor=black]{n7}{n32}
\ncline[linecolor=black]{n32}{n120}
\ncline[linecolor=black]{n33}{n33}
\ncline[linecolor=black]{n33}{n114}
\ncline[linecolor=black]{n33}{n36}
\ncline[linecolor=black]{n17}{n33}
\ncline[linecolor=black]{n33}{n35}
\ncline[linecolor=black]{n33}{n44}
\ncline[linecolor=black]{n34}{n34}
\ncline[linecolor=black]{n35}{n35}
\ncline[linecolor=black]{n35}{n139}
\ncline[linecolor=black]{n23}{n35}
\ncline[linecolor=black]{n35}{n98}
\ncline[linecolor=black]{n33}{n35}
\ncline[linecolor=black]{n35}{n114}
\ncline[linecolor=black]{n26}{n35}
\ncline[linecolor=black]{n36}{n36}
\ncline[linecolor=black]{n33}{n36}
\ncline[linecolor=black]{n36}{n114}
\ncline[linecolor=black]{n37}{n37}
\ncline[linecolor=black]{n38}{n38}
\ncline[linecolor=black]{n38}{n93}
\ncline[linecolor=black]{n39}{n39}
\ncline[linecolor=black]{n39}{n89}
\ncline[linecolor=black]{n40}{n40}
\ncline[linecolor=black]{n40}{n65}
\ncline[linecolor=black]{n40}{n126}
\ncline[linecolor=black]{n41}{n41}
\ncline[linecolor=black]{n41}{n101}
\ncline[linecolor=black]{n41}{n79}
\ncline[linecolor=black]{n41}{n58}
\ncline[linecolor=black]{n42}{n42}
\ncline[linecolor=black]{n42}{n56}
\ncline[linecolor=black]{n43}{n43}
\ncline[linecolor=black]{n43}{n95}
\ncline[linecolor=black]{n44}{n44}
\ncline[linecolor=black]{n17}{n44}
\ncline[linecolor=black]{n44}{n50}
\ncline[linecolor=black]{n44}{n146}
\ncline[linecolor=black]{n23}{n44}
\ncline[linecolor=black]{n44}{n64}
\ncline[linecolor=black]{n33}{n44}
\ncline[linecolor=black]{n45}{n45}
\ncline[linecolor=black]{n45}{n106}
\ncline[linecolor=black]{n14}{n45}
\ncline[linecolor=black]{n45}{n67}
\ncline[linecolor=black]{n45}{n134}
\ncline[linecolor=black]{n45}{n78}
\ncline[linecolor=black]{n45}{n125}
\ncline[linecolor=black]{n5}{n45}
\ncline[linecolor=black]{n27}{n45}
\ncline[linecolor=black]{n3}{n45}
\ncline[linecolor=black]{n45}{n121}
\ncline[linecolor=black]{n45}{n75}
\ncline[linecolor=black]{n46}{n46}
\ncline[linecolor=black]{n47}{n47}
\ncline[linecolor=black]{n48}{n48}
\ncline[linecolor=black]{n48}{n135}
\ncline[linecolor=black]{n48}{n113}
\ncline[linecolor=black]{n11}{n48}
\ncline[linecolor=black]{n48}{n127}
\ncline[linecolor=black]{n10}{n48}
\ncline[linecolor=black]{n48}{n124}
\ncline[linecolor=black]{n48}{n136}
\ncline[linecolor=black]{n49}{n49}
\ncline[linecolor=black]{n50}{n50}
\ncline[linecolor=black]{n12}{n50}
\ncline[linecolor=black]{n17}{n50}
\ncline[linecolor=black]{n50}{n64}
\ncline[linecolor=black]{n44}{n50}
\ncline[linecolor=black]{n0}{n50}
\ncline[linecolor=black]{n50}{n126}
\ncline[linecolor=black]{n51}{n51}
\ncline[linecolor=black]{n7}{n51}
\ncline[linecolor=black]{n51}{n120}
\ncline[linecolor=black]{n52}{n52}
\ncline[linecolor=black]{n52}{n76}
\ncline[linecolor=black]{n52}{n91}
\ncline[linecolor=black]{n52}{n109}
\ncline[linecolor=black]{n3}{n52}
\ncline[linecolor=black]{n27}{n52}
\ncline[linecolor=black]{n53}{n53}
\ncline[linecolor=black]{n54}{n54}
\ncline[linecolor=black]{n54}{n145}
\ncline[linecolor=black]{n54}{n69}
\ncline[linecolor=black]{n54}{n113}
\ncline[linecolor=black]{n54}{n73}
\ncline[linecolor=black]{n55}{n55}
\ncline[linecolor=black]{n55}{n80}
\ncline[linecolor=black]{n56}{n56}
\ncline[linecolor=black]{n42}{n56}
\ncline[linecolor=black]{n57}{n57}
\ncline[linecolor=black]{n58}{n58}
\ncline[linecolor=black]{n58}{n101}
\ncline[linecolor=black]{n41}{n58}
\ncline[linecolor=black]{n59}{n59}
\ncline[linecolor=black]{n60}{n60}
\ncline[linecolor=black]{n61}{n61}
\ncline[linecolor=black]{n62}{n62}
\ncline[linecolor=black]{n62}{n136}
\ncline[linecolor=black]{n62}{n124}
\ncline[linecolor=black]{n62}{n111}
\ncline[linecolor=black]{n16}{n62}
\ncline[linecolor=black]{n18}{n62}
\ncline[linecolor=black]{n62}{n69}
\ncline[linecolor=black]{n62}{n149}
\ncline[linecolor=black]{n62}{n113}
\ncline[linecolor=black]{n10}{n62}
\ncline[linecolor=black]{n63}{n63}
\ncline[linecolor=black]{n15}{n63}
\ncline[linecolor=black]{n25}{n63}
\ncline[linecolor=black]{n64}{n64}
\ncline[linecolor=black]{n12}{n64}
\ncline[linecolor=black]{n50}{n64}
\ncline[linecolor=black]{n17}{n64}
\ncline[linecolor=black]{n44}{n64}
\ncline[linecolor=black]{n65}{n65}
\ncline[linecolor=black]{n40}{n65}
\ncline[linecolor=black]{n66}{n66}
\ncline[linecolor=black]{n14}{n67}
\ncline[linecolor=black]{n67}{n67}
\ncline[linecolor=black]{n67}{n134}
\ncline[linecolor=black]{n45}{n67}
\ncline[linecolor=black]{n67}{n106}
\ncline[linecolor=black]{n67}{n125}
\ncline[linecolor=black]{n67}{n121}
\ncline[linecolor=black]{n18}{n67}
\ncline[linecolor=black]{n67}{n75}
\ncline[linecolor=black]{n67}{n78}
\ncline[linecolor=black]{n5}{n67}
\ncline[linecolor=black]{n27}{n67}
\ncline[linecolor=black]{n68}{n68}
\ncline[linecolor=black]{n69}{n69}
\ncline[linecolor=black]{n69}{n113}
\ncline[linecolor=black]{n69}{n124}
\ncline[linecolor=black]{n69}{n149}
\ncline[linecolor=black]{n69}{n136}
\ncline[linecolor=black]{n16}{n69}
\ncline[linecolor=black]{n18}{n69}
\ncline[linecolor=black]{n62}{n69}
\ncline[linecolor=black]{n69}{n111}
\ncline[linecolor=black]{n54}{n69}
\ncline[linecolor=black]{n70}{n70}
\ncline[linecolor=black]{n71}{n71}
\ncline[linecolor=black]{n72}{n72}
\ncline[linecolor=black]{n73}{n73}
\ncline[linecolor=black]{n73}{n135}
\ncline[linecolor=black]{n54}{n73}
\ncline[linecolor=black]{n74}{n74}
\ncline[linecolor=black]{n3}{n74}
\ncline[linecolor=black]{n5}{n74}
\ncline[linecolor=black]{n74}{n90}
\ncline[linecolor=black]{n74}{n91}
\ncline[linecolor=black]{n74}{n75}
\ncline[linecolor=black]{n75}{n75}
\ncline[linecolor=black]{n75}{n125}
\ncline[linecolor=black]{n27}{n75}
\ncline[linecolor=black]{n75}{n90}
\ncline[linecolor=black]{n3}{n75}
\ncline[linecolor=black]{n14}{n75}
\ncline[linecolor=black]{n67}{n75}
\ncline[linecolor=black]{n75}{n134}
\ncline[linecolor=black]{n5}{n75}
\ncline[linecolor=black]{n45}{n75}
\ncline[linecolor=black]{n74}{n75}
\ncline[linecolor=black]{n75}{n78}
\ncline[linecolor=black]{n76}{n76}
\ncline[linecolor=black]{n52}{n76}
\ncline[linecolor=black]{n27}{n76}
\ncline[linecolor=black]{n76}{n91}
\ncline[linecolor=black]{n77}{n77}
\ncline[linecolor=black]{n77}{n89}
\ncline[linecolor=black]{n13}{n77}
\ncline[linecolor=black]{n77}{n144}
\ncline[linecolor=black]{n78}{n78}
\ncline[linecolor=black]{n78}{n106}
\ncline[linecolor=black]{n45}{n78}
\ncline[linecolor=black]{n78}{n125}
\ncline[linecolor=black]{n5}{n78}
\ncline[linecolor=black]{n14}{n78}
\ncline[linecolor=black]{n27}{n78}
\ncline[linecolor=black]{n67}{n78}
\ncline[linecolor=black]{n78}{n134}
\ncline[linecolor=black]{n3}{n78}
\ncline[linecolor=black]{n78}{n121}
\ncline[linecolor=black]{n75}{n78}
\ncline[linecolor=black]{n79}{n79}
\ncline[linecolor=black]{n41}{n79}
\ncline[linecolor=black]{n79}{n101}
\ncline[linecolor=black]{n80}{n80}
\ncline[linecolor=black]{n55}{n80}
\ncline[linecolor=black]{n80}{n92}
\ncline[linecolor=black]{n81}{n81}
\ncline[linecolor=black]{n81}{n96}
\ncline[linecolor=black]{n82}{n82}
\ncline[linecolor=black]{n82}{n99}
\ncline[linecolor=black]{n82}{n128}
\ncline[linecolor=black]{n83}{n83}
\ncline[linecolor=black]{n83}{n118}
\ncline[linecolor=black]{n83}{n138}
\ncline[linecolor=black]{n84}{n84}
\ncline[linecolor=black]{n1}{n84}
\ncline[linecolor=black]{n85}{n85}
\ncline[linecolor=black]{n86}{n86}
\ncline[linecolor=black]{n86}{n144}
\ncline[linecolor=black]{n13}{n86}
\ncline[linecolor=black]{n87}{n87}
\ncline[linecolor=black]{n8}{n87}
\ncline[linecolor=black]{n87}{n107}
\ncline[linecolor=black]{n29}{n87}
\ncline[linecolor=black]{n88}{n88}
\ncline[linecolor=black]{n89}{n89}
\ncline[linecolor=black]{n77}{n89}
\ncline[linecolor=black]{n39}{n89}
\ncline[linecolor=black]{n90}{n90}
\ncline[linecolor=black]{n75}{n90}
\ncline[linecolor=black]{n16}{n90}
\ncline[linecolor=black]{n15}{n90}
\ncline[linecolor=black]{n90}{n105}
\ncline[linecolor=black]{n90}{n125}
\ncline[linecolor=black]{n18}{n90}
\ncline[linecolor=black]{n74}{n90}
\ncline[linecolor=black]{n90}{n149}
\ncline[linecolor=black]{n91}{n91}
\ncline[linecolor=black]{n52}{n91}
\ncline[linecolor=black]{n3}{n91}
\ncline[linecolor=black]{n27}{n91}
\ncline[linecolor=black]{n5}{n91}
\ncline[linecolor=black]{n74}{n91}
\ncline[linecolor=black]{n76}{n91}
\ncline[linecolor=black]{n91}{n109}
\ncline[linecolor=black]{n92}{n92}
\ncline[linecolor=black]{n80}{n92}
\ncline[linecolor=black]{n93}{n93}
\ncline[linecolor=black]{n38}{n93}
\ncline[linecolor=black]{n94}{n94}
\ncline[linecolor=black]{n95}{n95}
\ncline[linecolor=black]{n43}{n95}
\ncline[linecolor=black]{n96}{n96}
\ncline[linecolor=black]{n81}{n96}
\ncline[linecolor=black]{n97}{n97}
\ncline[linecolor=black]{n1}{n97}
\ncline[linecolor=black]{n98}{n98}
\ncline[linecolor=black]{n98}{n139}
\ncline[linecolor=black]{n35}{n98}
\ncline[linecolor=black]{n99}{n99}
\ncline[linecolor=black]{n82}{n99}
\ncline[linecolor=black]{n100}{n100}
\ncline[linecolor=black]{n2}{n100}
\ncline[linecolor=black]{n101}{n101}
\ncline[linecolor=black]{n41}{n101}
\ncline[linecolor=black]{n58}{n101}
\ncline[linecolor=black]{n79}{n101}
\ncline[linecolor=black]{n102}{n102}
\ncline[linecolor=black]{n21}{n102}
\ncline[linecolor=black]{n103}{n103}
\ncline[linecolor=black]{n103}{n107}
\ncline[linecolor=black]{n29}{n103}
\ncline[linecolor=black]{n104}{n104}
\ncline[linecolor=black]{n104}{n128}
\ncline[linecolor=black]{n2}{n104}
\ncline[linecolor=black]{n104}{n122}
\ncline[linecolor=black]{n105}{n105}
\ncline[linecolor=black]{n90}{n105}
\ncline[linecolor=black]{n106}{n106}
\ncline[linecolor=black]{n45}{n106}
\ncline[linecolor=black]{n78}{n106}
\ncline[linecolor=black]{n14}{n106}
\ncline[linecolor=black]{n67}{n106}
\ncline[linecolor=black]{n106}{n134}
\ncline[linecolor=black]{n106}{n121}
\ncline[linecolor=black]{n106}{n125}
\ncline[linecolor=black]{n31}{n106}
\ncline[linecolor=black]{n5}{n106}
\ncline[linecolor=black]{n18}{n106}
\ncline[linecolor=black]{n107}{n107}
\ncline[linecolor=black]{n29}{n107}
\ncline[linecolor=black]{n8}{n107}
\ncline[linecolor=black]{n103}{n107}
\ncline[linecolor=black]{n87}{n107}
\ncline[linecolor=black]{n108}{n108}
\ncline[linecolor=black]{n109}{n109}
\ncline[linecolor=black]{n52}{n109}
\ncline[linecolor=black]{n91}{n109}
\ncline[linecolor=black]{n110}{n110}
\ncline[linecolor=black]{n111}{n111}
\ncline[linecolor=black]{n111}{n124}
\ncline[linecolor=black]{n62}{n111}
\ncline[linecolor=black]{n111}{n136}
\ncline[linecolor=black]{n16}{n111}
\ncline[linecolor=black]{n111}{n113}
\ncline[linecolor=black]{n111}{n149}
\ncline[linecolor=black]{n11}{n111}
\ncline[linecolor=black]{n10}{n111}
\ncline[linecolor=black]{n69}{n111}
\ncline[linecolor=black]{n111}{n127}
\ncline[linecolor=black]{n18}{n111}
\ncline[linecolor=black]{n112}{n112}
\ncline[linecolor=black]{n112}{n142}
\ncline[linecolor=black]{n112}{n128}
\ncline[linecolor=black]{n113}{n113}
\ncline[linecolor=black]{n69}{n113}
\ncline[linecolor=black]{n113}{n124}
\ncline[linecolor=black]{n113}{n149}
\ncline[linecolor=black]{n113}{n136}
\ncline[linecolor=black]{n16}{n113}
\ncline[linecolor=black]{n48}{n113}
\ncline[linecolor=black]{n111}{n113}
\ncline[linecolor=black]{n113}{n135}
\ncline[linecolor=black]{n10}{n113}
\ncline[linecolor=black]{n18}{n113}
\ncline[linecolor=black]{n54}{n113}
\ncline[linecolor=black]{n62}{n113}
\ncline[linecolor=black]{n113}{n145}
\ncline[linecolor=black]{n15}{n113}
\ncline[linecolor=black]{n114}{n114}
\ncline[linecolor=black]{n33}{n114}
\ncline[linecolor=black]{n17}{n114}
\ncline[linecolor=black]{n36}{n114}
\ncline[linecolor=black]{n35}{n114}
\ncline[linecolor=black]{n115}{n115}
\ncline[linecolor=black]{n116}{n116}
\ncline[linecolor=black]{n116}{n132}
\ncline[linecolor=black]{n117}{n117}
\ncline[linecolor=black]{n118}{n118}
\ncline[linecolor=black]{n118}{n138}
\ncline[linecolor=black]{n83}{n118}
\ncline[linecolor=black]{n119}{n119}
\ncline[linecolor=black]{n7}{n120}
\ncline[linecolor=black]{n120}{n120}
\ncline[linecolor=black]{n32}{n120}
\ncline[linecolor=black]{n51}{n120}
\ncline[linecolor=black]{n121}{n121}
\ncline[linecolor=black]{n14}{n121}
\ncline[linecolor=black]{n67}{n121}
\ncline[linecolor=black]{n121}{n134}
\ncline[linecolor=black]{n106}{n121}
\ncline[linecolor=black]{n121}{n125}
\ncline[linecolor=black]{n45}{n121}
\ncline[linecolor=black]{n78}{n121}
\ncline[linecolor=black]{n122}{n122}
\ncline[linecolor=black]{n1}{n122}
\ncline[linecolor=black]{n104}{n122}
\ncline[linecolor=black]{n2}{n122}
\ncline[linecolor=black]{n123}{n123}
\ncline[linecolor=black]{n124}{n124}
\ncline[linecolor=black]{n124}{n136}
\ncline[linecolor=black]{n111}{n124}
\ncline[linecolor=black]{n69}{n124}
\ncline[linecolor=black]{n113}{n124}
\ncline[linecolor=black]{n124}{n149}
\ncline[linecolor=black]{n62}{n124}
\ncline[linecolor=black]{n16}{n124}
\ncline[linecolor=black]{n18}{n124}
\ncline[linecolor=black]{n10}{n124}
\ncline[linecolor=black]{n48}{n124}
\ncline[linecolor=black]{n15}{n124}
\ncline[linecolor=black]{n125}{n125}
\ncline[linecolor=black]{n75}{n125}
\ncline[linecolor=black]{n14}{n125}
\ncline[linecolor=black]{n67}{n125}
\ncline[linecolor=black]{n125}{n134}
\ncline[linecolor=black]{n27}{n125}
\ncline[linecolor=black]{n121}{n125}
\ncline[linecolor=black]{n45}{n125}
\ncline[linecolor=black]{n78}{n125}
\ncline[linecolor=black]{n106}{n125}
\ncline[linecolor=black]{n90}{n125}
\ncline[linecolor=black]{n3}{n125}
\ncline[linecolor=black]{n126}{n126}
\ncline[linecolor=black]{n40}{n126}
\ncline[linecolor=black]{n50}{n126}
\ncline[linecolor=black]{n126}{n146}
\ncline[linecolor=black]{n127}{n127}
\ncline[linecolor=black]{n11}{n127}
\ncline[linecolor=black]{n10}{n127}
\ncline[linecolor=black]{n48}{n127}
\ncline[linecolor=black]{n111}{n127}
\ncline[linecolor=black]{n128}{n128}
\ncline[linecolor=black]{n104}{n128}
\ncline[linecolor=black]{n2}{n128}
\ncline[linecolor=black]{n112}{n128}
\ncline[linecolor=black]{n82}{n128}
\ncline[linecolor=black]{n128}{n143}
\ncline[linecolor=black]{n129}{n129}
\ncline[linecolor=black]{n129}{n131}
\ncline[linecolor=black]{n6}{n129}
\ncline[linecolor=black]{n130}{n130}
\ncline[linecolor=black]{n131}{n131}
\ncline[linecolor=black]{n129}{n131}
\ncline[linecolor=black]{n6}{n131}
\ncline[linecolor=black]{n132}{n132}
\ncline[linecolor=black]{n116}{n132}
\ncline[linecolor=black]{n133}{n133}
\ncline[linecolor=black]{n14}{n134}
\ncline[linecolor=black]{n67}{n134}
\ncline[linecolor=black]{n134}{n134}
\ncline[linecolor=black]{n45}{n134}
\ncline[linecolor=black]{n106}{n134}
\ncline[linecolor=black]{n125}{n134}
\ncline[linecolor=black]{n121}{n134}
\ncline[linecolor=black]{n18}{n134}
\ncline[linecolor=black]{n75}{n134}
\ncline[linecolor=black]{n78}{n134}
\ncline[linecolor=black]{n5}{n134}
\ncline[linecolor=black]{n27}{n134}
\ncline[linecolor=black]{n135}{n135}
\ncline[linecolor=black]{n48}{n135}
\ncline[linecolor=black]{n113}{n135}
\ncline[linecolor=black]{n73}{n135}
\ncline[linecolor=black]{n136}{n136}
\ncline[linecolor=black]{n124}{n136}
\ncline[linecolor=black]{n62}{n136}
\ncline[linecolor=black]{n69}{n136}
\ncline[linecolor=black]{n111}{n136}
\ncline[linecolor=black]{n113}{n136}
\ncline[linecolor=black]{n136}{n149}
\ncline[linecolor=black]{n16}{n136}
\ncline[linecolor=black]{n10}{n136}
\ncline[linecolor=black]{n18}{n136}
\ncline[linecolor=black]{n15}{n136}
\ncline[linecolor=black]{n11}{n136}
\ncline[linecolor=black]{n48}{n136}
\ncline[linecolor=black]{n137}{n137}
\ncline[linecolor=black]{n138}{n138}
\ncline[linecolor=black]{n118}{n138}
\ncline[linecolor=black]{n83}{n138}
\ncline[linecolor=black]{n139}{n139}
\ncline[linecolor=black]{n98}{n139}
\ncline[linecolor=black]{n35}{n139}
\ncline[linecolor=black]{n23}{n139}
\ncline[linecolor=black]{n26}{n139}
\ncline[linecolor=black]{n140}{n140}
\ncline[linecolor=black]{n141}{n141}
\ncline[linecolor=black]{n141}{n147}
\ncline[linecolor=black]{n142}{n142}
\ncline[linecolor=black]{n112}{n142}
\ncline[linecolor=black]{n143}{n143}
\ncline[linecolor=black]{n2}{n143}
\ncline[linecolor=black]{n128}{n143}
\ncline[linecolor=black]{n144}{n144}
\ncline[linecolor=black]{n13}{n144}
\ncline[linecolor=black]{n86}{n144}
\ncline[linecolor=black]{n77}{n144}
\ncline[linecolor=black]{n20}{n144}
\ncline[linecolor=black]{n145}{n145}
\ncline[linecolor=black]{n54}{n145}
\ncline[linecolor=black]{n113}{n145}
\ncline[linecolor=black]{n146}{n146}
\ncline[linecolor=black]{n44}{n146}
\ncline[linecolor=black]{n126}{n146}
\ncline[linecolor=black]{n147}{n147}
\ncline[linecolor=black]{n141}{n147}
\ncline[linecolor=black]{n148}{n148}
\ncline[linecolor=black]{n149}{n149}
\ncline[linecolor=black]{n16}{n149}
\ncline[linecolor=black]{n69}{n149}
\ncline[linecolor=black]{n113}{n149}
\ncline[linecolor=black]{n124}{n149}
\ncline[linecolor=black]{n18}{n149}
\ncline[linecolor=black]{n136}{n149}
\ncline[linecolor=black]{n15}{n149}
\ncline[linecolor=black]{n111}{n149}
\ncline[linecolor=black]{n62}{n149}
\ncline[linecolor=black]{n90}{n149}
\psset{dotstyle=Bo}
\dotnode[](1.108338,3.858886){n0}
\dotnode[](2.527444,2.445348){n1}
\dotnode[](2.017666,2.841472){n2}
\dotnode[](1.257860,2.892203){n4}
\dotnode[](3.098514,4.866756){n7}
\dotnode[](0.3962556,3.984627){n12}
\dotnode[](2.661068,4.46547){n13}
\dotnode[](0.6005647,3.371056){n17}
\dotnode[](2.577236,4.740179){n20}
\dotnode[](0.711105,2.899006){n23}
\dotnode[](1.232538,4.29378){n24}
\dotnode[](0.9249148,2.893182){n26}
\dotnode[](2.557950,5){n32}
\dotnode[](0.1556468,2.998522){n33}
\dotnode[](0.3174471,2.600607){n35}
\dotnode[](0.1307679,3.277233){n36}
\dotnode[](2.55,3.97){n38}
\dotnode[](3.353207,4.403764){n39}
\dotnode[](0.9114814,4.353879){n40}
\dotnode[](1.686579,3.819467){n41}
\dotnode[](0.4179513,3.522051){n44}
\dotnode[](0.1748816,2.281055){n47}
\dotnode[](0.5841016,3.811024){n50}
\dotnode[](2.24852,4.942186){n51}
\dotnode[](1.967572,3.868347){n58}
\dotnode[](0,2.563469){n59}
\dotnode[](0.2217693,3.701035){n64}
\dotnode[](0.6466549,4.297944){n65}
\dotnode[](3.364475,4.745668){n66}
\dotnode[](1.004251,3.3599){n72}
\dotnode[](2.550006,4.241430){n77}
\dotnode[](1.560778,4.1743){n79}
\dotnode[](2.078577,3.201014){n82}
\dotnode[](3.153883,3.140639){n84}
\dotnode[](1.891242,4.717356){n86}
\dotnode[](3.202631,4.133673){n89}
\dotnode[](2.64,3.705){n93}
\dotnode[](2.869689,4.205452){n94}
\dotnode[](2.802984,2.471939){n97}
\dotnode[](0.4990411,2.364772){n98}
\dotnode[](2.265986,3.781169){n99}
\dotnode[](2.886530,2.782621){n100}
\dotnode[](2.074548,4.120629){n101}
\dotnode[](2.161968,2.481101){n104}
\dotnode[](0.5000101,2.853756){n110}
\dotnode[](1.705716,3.114971){n112}
\dotnode[](0.3549037,3.121353){n114}
\dotnode[](3.053188,3.925351){n117}
\dotnode[](2.890681,3.735371){n119}
\dotnode[](2.880736,4.879838){n120}
\dotnode[](2.617372,2.90596){n122}
\dotnode[](3.046366,4.518936){n123}
\dotnode[](0.9849086,4.079775){n126}
\dotnode[](2.280932,2.838994){n128}
\dotnode[](1.689756,4.453583){n137}
\dotnode[](0.710692,2.532491){n139}
\dotnode[](1.606368,3.440222){n142}
\dotnode[](2.445606,3.296266){n143}
\dotnode[](2.274105,4.535248){n144}
\dotnode[](0.7976457,3.6483){n146}
\psset{dotstyle=Bsquare}
\dotnode[](4.581229,3.210751){n6}
\dotnode[](4.591683,2.878327){n8}
\dotnode[](4,2.2){n9}
\dotnode[](4.08826,1.970573){n19}
\dotnode[](3.773692,3.552994){n21}
\dotnode[](3.759983,2.172351){n28}
\dotnode[](4.7622,2.335772){n29}
\dotnode[](4.209552,3.016675){n42}
\dotnode[](3.891111,1.748183){n43}
\dotnode[](4.389942,2.082443){n46}
\dotnode[](4.969442,3.079323){n53}
\dotnode[](5,2.809783){n55}
\dotnode[](3.97574,2.571348){n56}
\dotnode[](4.371287,1.534671){n57}
\dotnode[](3.764707,4.352432){n60}
\dotnode[](3.737104,3.981957){n61}
\dotnode[](4.688611,1.662437){n70}
\dotnode[](4.32925,1.149353){n71}
\dotnode[](4.791386,3.470398){n80}
\dotnode[](4.135811,4.087185){n83}
\dotnode[](4.792091,2.606388){n87}
\dotnode[](4.292557,3.369169){n92}
\dotnode[](4.306428,1.798957){n95}
\dotnode[](3.578431,3.339056){n102}
\dotnode[](4.396008,2.731498){n103}
\dotnode[](4.351759,2.410157){n107}
\dotnode[](4.562201,1.223617){n116}
\dotnode[](4.318411,3.852927){n118}
\dotnode[](4.117188,3.717776){n129}
\dotnode[](4.500053,3.637197){n131}
\dotnode[](4.069486,1.411759){n132}
\dotnode[](3.989081,3.406712){n138}
\dotnode[](4.873979,1.934726){n140}
\dotnode[](3.645946,2.645275){n141}
\dotnode[](3.950723,2.918283){n147}
\dotnode[](4.776492,1.394465){n148}
\psset{dotstyle=Btriangle}
\dotnode[](0.6785566,0.8802483){n3}
\dotnode[](0.8247274,0.631479){n5}
\dotnode[](2.888154,1.090116){n10}
\dotnode[](3.105909,0.7401545){n11}
\dotnode[](1.439636,1.295117){n14}
\dotnode[](2.236885,1.274193){n15}
\dotnode[](2.024484,0.7152085){n16}
\dotnode[](1.814559,0.4981684){n18}
\dotnode[](3.913548,0.4282107){n22}
\dotnode[](2.521529,1.412281){n25}
\dotnode[](0.5726195,1.294175){n27}
\dotnode[](3.665883,0.6603324){n30}
\dotnode[](1.422528,0.3128956){n31}
\dotnode[](1.403207,1.916558){n34}
\dotnode[](2.901075,1.575363){n37}
\dotnode[](0.8306748,1.037910){n45}
\dotnode[](3.273419,0.6181322){n48}
\dotnode[](3.66405,0.2172084){n49}
\dotnode[](0.2341176,1.037886){n52}
\dotnode[](2.864019,0){n54}
\dotnode[](2.127025,0.3401537){n62}
\dotnode[](2.312403,1.030709){n63}
\dotnode[](1.509306,0.9905115){n67}
\dotnode[](1.631241,0.08123197){n68}
\dotnode[](2.363231,0.03982867){n69}
\dotnode[](3.170304,0.08031329){n73}
\dotnode[](1.092846,0.4743804){n74}
\dotnode[](0.957797,1.457980){n75}
\dotnode[](0.4145735,1.587721){n76}
\dotnode[](0.8983155,1.243353){n78}
\dotnode[](1.183687,2.360412){n81}
\dotnode[](0.6058247,2.060042){n85}
\dotnode[](3.635776,1.016917){n88}
\dotnode[](1.765180,1.228386){n90}
\dotnode[](0.4301581,0.7624465){n91}
\dotnode[](0.9419618,2.164323){n96}
\dotnode[](1.915954,1.686363){n105}
\dotnode[](1.073082,0.7635667){n106}
\dotnode[](0.1632209,1.576404){n108}
\dotnode[](0.1964432,1.278737){n109}
\dotnode[](2.625767,0.8297721){n111}
\dotnode[](2.796314,0.5814426){n113}
\dotnode[](3.949152,0.8549694){n115}
\dotnode[](1.377059,0.7879707){n121}
\dotnode[](2.621703,0.4200536){n124}
\dotnode[](1.203850,1.499658){n125}
\dotnode[](3.254325,1.04113){n127}
\dotnode[](3.471599,1.285363){n130}
\dotnode[](4.11247,0.5910543){n133}
\dotnode[](1.252724,1.071603){n134}
\dotnode[](3.385955,0.3694225){n135}
\dotnode[](2.407195,0.4605674){n136}
\dotnode[](2.648264,0.1809270){n145}
\dotnode[](2.296867,0.7074797){n149}

            \end{pspicture}
        }
    }
\end{figure}
\centerline{
\begin{tabular}{|c|c|c|c|}
        \hline
         & {\tt setosa} & {\tt virginica} & {\tt versicolor}\\
         \hline
        $C_1$ (triangle) & 50 & 0 & 4\\
        $C_2$ (square) & 0 & 36 & 0\\
        $C_3$ (circle) & 0 & 14 & 46 \\
        \hline
    \end{tabular}
	}
\end{frame}


\begin{frame}{Maximization Objectives: Average Cut}
 The {\em average weight} objective is
def\/{i}ned as
\begin{align*}
    \max_{\cC}\; J_{aw}(\cC) & =
    \sum_{i=1}^k {W(C_i,C_i) \over |C_i|} =
    \sum_{i=1}^k {\bc^T_i \bA \bc_i
    \over \bc_i^T\bc_i}
	= \sum_{i=1}^k \bu_i^T \bA \bu_i
\end{align*}
where $\bu_i$ is an arbitrary real vector, which is a relaxation of the
binary cluster indicator vectors $\bc_i$.

\medskip
We can maximize
the objective by selecting the $k$ largest eigenvalues of $\bA$,
and the corresponding eigenvectors.

\begin{align*}
    \max_{\cC} \; J_{aw}(\cC) &=
    \bu^T_1 \bA \bu_1 + \dots + \bu^T_k \bA \bu_k\\
    & = \lambda_1 + \dots + \lambda_k
\end{align*}
where $\lambda_1 \ge \lambda_2 \ge \cdots \ge \lambda_n$.
In general, while $\bA$ is symmetric, it may not be 
positive semidef\/{i}nite. This means
that $\bA$ can have negative eigenvalues, and to maximize the objective 
we must consider only the positive eigenvalues and the
corresponding eigenvectors.
\end{frame}

\begin{frame}{Maximization Objectives: Modularity}
Given $\bA$, the weighted adjacency
matrix, the modularity of a clustering is the
difference between the observed and expected fraction of weights
on edges within the clusters. 
The clustering objective is given as
\begin{align*}
    \max_{\cC} \; J_{Q}(\cC) &= \sum_{i=1}^k
    \lB( {\bc_i^T \bA \bc_i \over tr(\bDelta)} -
        {(\bd_i^T \bc_i)^2 \over tr(\bDelta)^2}
    \rB)
    = \sum_{i=1}^k \bc_i^T \bQ \bc_i
\end{align*}
where $\bQ$ is the {\em modularity matrix}:
\begin{align*}
    \bQ = {1 \over tr(\bDelta)} \lB(\bA - {\bd \cdot
    \bd^T \over tr(\bDelta)} \rB)
\end{align*}
The optimal solution comprises the eigenvectors corresponding to the $k$
largest eigenvalues of $\bQ$.
Since $\bQ$ is
symmetric, but not positive semidef\/{i}nite, we use only 
the positive eigenvalues.
\end{frame}


\begin{frame}{Markov Chain Clustering}
A Markov chain is a discrete-time stochastic process
over a set of states, in our case the set of vertices $V$. 

\medskip
The
Markov chain makes a transition from one node to another at
discrete timesteps $t=1,2,\dots$, with the probability of making
a transition from node $i$ to node $j$ given as $m_{ij}$. 

\medskip
Let the
random variable $X_t$ denote the state at time $t$. The Markov
property means that the probability distribution of $X_t$ over the
states at time $t$ depends only on the probability distribution of
$X_{t-1}$, that is,
\begin{align*}
P(X_{t}=i | X_0, X_1, \dots, X_{t-1}) = P(X_{t}=i | X_{t-1})
\end{align*}
Further, we assume that the Markov chain is {\em homogeneous},
that is, the transition probability
\begin{align*}
P(X_t = j | X_{t-1} = i) = m_{ij}
\end{align*}
is independent of the time step $t$.
\end{frame}



\begin{frame}{Markov Chain Clustering: Markov Matrix}
The normalized adjacency matrix
$\bM = \bDelta^{-1} \bA$ can be interpreted as
the $n\times n$ {\em transition matrix} where the entry $m_{ij} =
{a_{ij} \over d_i}$ is the probability of
transitioning or jumping from node $i$ to node $j$ in the graph
$G$. 

\medskip
The matrix $\bM$ is thus the transition matrix for a {\em Markov
chain} 
or a Markov random walk on
graph $G$. That is,
given node $i$ the transition matrix $\bM$ specif\/{i}es the
probabilities of reaching any other node $j$ in one time step.


\medskip
In general, the transition probability matrix
for
 $t$ time steps is given as
\begin{align*}
     \bM^{t-1} \cdot \bM = \bM^t
\end{align*}
\end{frame}

\begin{frame}{Markov Chain Clustering: Random Walk}
A random walk on $G$ thus corresponds to taking successive powers
of the transition matrix $\bM$. 

\medskip
Let $\bpi_0$ specify the initial
state probability vector at time $t=0$.
The state probability vector after 
$t$ steps is
\begin{align*}
    \bpi_t^T & = \bpi_{t-1}^T \bM =
    \bpi_{t-2}^T\bM^2 = \cdots
     = \bpi_0^T \bM^{t}
\end{align*}
Equivalently, taking transpose on both sides, we
get
  $$\bpi_t  = (M^t)^T \bpi_0 = (\bM^T)^t \bpi_0$$
The state probability vector thus converges to the dominant
eigenvector of $\bM^T$.
\end{frame}


\begin{frame}{Markov Clustering Algorithm}
Consider a variation of the random walk, where the
probability of transitioning from node $i$ to $j$ is inflated by
taking each element $m_{ij}$ to the power $r \ge 1$. Given a
transition matrix $\bM$, def\/{i}ne the inflation operator $\Upsilon$
as follows:
\begin{align*}
    \Upsilon(\bM,r) = \lB\{
        \frac{(m_{ij})^r}{\sum_{a=1}^n (m_{ia})^r}
    \rB\}_{i,j=1}^n
\end{align*}
The net
effect of the inflation operator is to increase the higher
probability transitions and decrease the lower probability
transitions.

\medskip
The Markov clustering
algorithm (MCL) is an iterative method that interleaves matrix expansion
and inflation steps. Matrix expansion corresponds to taking
successive powers of the transition matrix, leading to random
walks of longer lengths. On the other hand, matrix inflation makes
the higher probability transitions even more likely and reduces
the lower probability transitions.

\medskip
MCL
takes as input the inflation parameter $r\ge1$. Higher values lead
to more, smaller clusters, whereas smaller values lead to fewer,
but larger clusters.  

\end{frame}


\begin{frame}{Markov Clustering Algorithm: MCL}
The f\/{i}nal clusters are found by enumerating the weakly connected
components in the directed graph induced by the converged
transition matrix $\bM_t$, where the edges are defined as:
\begin{align*}
    E = \bigl\{(i,j) \mid \bM_t(i,j) > 0\bigr\}
\end{align*}
A directed edge $(i,j)$ exists only if node $i$
can transition to node $j$ within $t$ steps of the expansion and
inflation process. 

\medskip
A node $j$ is called an {\em attractor} if
$\bM_t(j,j) > 0$, and we say that node $i$ is attracted to
attractor $j$ if $\bM_t(i,j) > 0$. The MCL process yields a set of
attractor nodes, $V_a \subseteq V$, such that other nodes are
attracted to at least one attractor in $V_a$. 

\medskip
To extract the
clusters from $G_t$, MCL f\/{i}rst f\/{i}nds the strongly
connected components $S_1, S_2, \dots, S_q$ over the set of
attractors $V_a$. Next, for each strongly connected set of
attractors $S_{j}$, MCL f\/{i}nds the weakly connected
 components consisting of all nodes $i \in V_t-V_a$
 attracted to an attractor in $S_{j}$. If a node $i$ is attracted to
 multiple strongly connected components, it is added to each such
 cluster, resulting in possibly overlapping clusters.
\end{frame}




\newcommand{\MCL}{{\textsc{Markov Clustering}}}
\begin{frame}{Algorithm \MCL}
\begin{algorithm}[H]
\SetKwInOut{Algorithm}{\MCL\ ($\bA, r, \epsilon$)} \Algorithm{}
$t\assign 0$\; Add self-edges to $\bA$ if they do not exist\;
$\bM_t \assign  \bDelta^{-1} \bA$\; \Repeat{$\norm{\bM_t -
\bM_{t-1}}_F \le \epsilon$}{
    $t \assign  t+1$\;
    $\bM_t \assign  \bM_{t-1} \cdot \bM_{t-1}$\;
    $\bM_t \assign  \Upsilon(\bM_t, r)$\;
} $G_t \assign  \text{directed graph induced by } \bM_t$\; $\cC
\assign  \{ \text{weakly connected components in } G_t\}$\;
\end{algorithm}
\end{frame}


\begin{frame}{MCL Attractors and Clusters}
  \framesubtitle{$r=2.5$}
\begin{figure}
    \centerline{
	\scalebox{0.6}{
        \psset{unit=0.75in,dotscale=2,fillcolor=lightgray,dotstyle=Bo}
        \begin{pspicture}(0,-0.25)(0,2.25)
            \psmatrix[mnode=circle]
                & [name=a] 1 & & & [name=f]6\\
                [name=b]2 & & [name=d]4 & [name=e]5 & [mnode=none]\\
                & [name=c]3  & & & [name=g]7
            \endpsmatrix
            \ncline{a}{b}
            \ncline{a}{d}
            \ncline{a}{f}
            \ncline{b}{c}
            \ncline{b}{d}
            \ncline{c}{d}
            \ncline{c}{g}
            \ncline{d}{e}
            \ncline{e}{f}
            \ncline{e}{g}
            \ncline{f}{g}
        \end{pspicture}
		}}
		\vspace{0.2in}
\centerline{
\scalebox{0.8}{
\psset{unit=0.75in,fillcolor=lightgray,arrowscale=1.5}
\begin{pspicture}(0,-0.25)(0,2.25)
    \psmatrix[mnode=circle]
        & [name=a] 1 & & & [name=f,fillstyle=solid]6\\
        [name=b]2 & & [name=d,fillstyle=solid]4 & [name=e]5 & [mnode=none]\\
        & [name=c]3  & & & [name=g,fillstyle=solid]7
    \endpsmatrix
    \ncline{->}{a}{d}\naput{1}
    \ncline{->}{b}{d}\naput{1}
    \ncline{->}{c}{d}\naput{1}
    \ncline{->}{e}{f}\naput{0.5}
    \ncline{->}{e}{g}\naput{0.5}
    \ncline{-}{f}{g}\naput{0.5}
    \nccircle[angleA=-90]{->}{d}{0.15in}
    \nccircle[angleA=-90]{->}{f}{0.15in}
    \nccircle[angleA=-90]{->}{g}{0.15in}
\end{pspicture}
} }
\end{figure}
\end{frame}


\begin{frame}[fragile]{MCL on Iris Graph}
\begin{figure}
    \begin{center}
        \subfloat[$r=1.3$]{
        \label{fig:clust:spectral:irisMCLa}
        \scalebox{0.55}{
            \psset{unit=0.75in,dotscale=2}
            \begin{pspicture}(5,5)
                \pnode(1.108338,3.858886){n0}
\pnode(2.527444,2.445348){n1}
\pnode(2.017666,2.841472){n2}
\pnode(0.6785566,0.8802483){n3}
\pnode(1.257860,2.892203){n4}
\pnode(0.8247274,0.631479){n5}
\pnode(4.581229,3.210751){n6}
\pnode(3.098514,4.866756){n7}
\pnode(4.591683,2.878327){n8}
\pnode(4,2.2){n9}
\pnode(2.888154,1.090116){n10}
\pnode(3.105909,0.7401545){n11}
\pnode(0.3962556,3.984627){n12}
\pnode(2.661068,4.46547){n13}
\pnode(1.439636,1.295117){n14}
\pnode(2.236885,1.274193){n15}
\pnode(2.024484,0.7152085){n16}
\pnode(0.6005647,3.371056){n17}
\pnode(1.814559,0.4981684){n18}
\pnode(4.08826,1.970573){n19}
\pnode(2.577236,4.740179){n20}
\pnode(3.773692,3.552994){n21}
\pnode(3.913548,0.4282107){n22}
\pnode(0.711105,2.899006){n23}
\pnode(1.232538,4.29378){n24}
\pnode(2.521529,1.412281){n25}
\pnode(0.9249148,2.893182){n26}
\pnode(0.5726195,1.294175){n27}
\pnode(3.759983,2.172351){n28}
\pnode(4.7622,2.335772){n29}
\pnode(3.665883,0.6603324){n30}
\pnode(1.422528,0.3128956){n31}
\pnode(2.557950,5){n32}
\pnode(0.1556468,2.998522){n33}
\pnode(1.403207,1.916558){n34}
\pnode(0.3174471,2.600607){n35}
\pnode(0.1307679,3.277233){n36}
\pnode(2.901075,1.575363){n37}
\pnode(2.55,3.97){n38}
\pnode(3.353207,4.403764){n39}
\pnode(0.9114814,4.353879){n40}
\pnode(1.686579,3.819467){n41}
\pnode(4.209552,3.016675){n42}
\pnode(3.891111,1.748183){n43}
\pnode(0.4179513,3.522051){n44}
\pnode(0.8306748,1.037910){n45}
\pnode(4.389942,2.082443){n46}
\pnode(0.1748816,2.281055){n47}
\pnode(3.273419,0.6181322){n48}
\pnode(3.66405,0.2172084){n49}
\pnode(0.5841016,3.811024){n50}
\pnode(2.24852,4.942186){n51}
\pnode(0.2341176,1.037886){n52}
\pnode(4.969442,3.079323){n53}
\pnode(2.864019,0){n54}
\pnode(5,2.809783){n55}
\pnode(3.97574,2.571348){n56}
\pnode(4.371287,1.534671){n57}
\pnode(1.967572,3.868347){n58}
\pnode(0,2.563469){n59}
\pnode(3.764707,4.352432){n60}
\pnode(3.737104,3.981957){n61}
\pnode(2.127025,0.3401537){n62}
\pnode(2.312403,1.030709){n63}
\pnode(0.2217693,3.701035){n64}
\pnode(0.6466549,4.297944){n65}
\pnode(3.364475,4.745668){n66}
\pnode(1.509306,0.9905115){n67}
\pnode(1.631241,0.08123197){n68}
\pnode(2.363231,0.03982867){n69}
\pnode(4.688611,1.662437){n70}
\pnode(4.32925,1.149353){n71}
\pnode(1.004251,3.3599){n72}
\pnode(3.170304,0.08031329){n73}
\pnode(1.092846,0.4743804){n74}
\pnode(0.957797,1.457980){n75}
\pnode(0.4145735,1.587721){n76}
\pnode(2.550006,4.241430){n77}
\pnode(0.8983155,1.243353){n78}
\pnode(1.560778,4.1743){n79}
\pnode(4.791386,3.470398){n80}
\pnode(1.183687,2.360412){n81}
\pnode(2.078577,3.201014){n82}
\pnode(4.135811,4.087185){n83}
\pnode(3.153883,3.140639){n84}
\pnode(0.6058247,2.060042){n85}
\pnode(1.891242,4.717356){n86}
\pnode(4.792091,2.606388){n87}
\pnode(3.635776,1.016917){n88}
\pnode(3.202631,4.133673){n89}
\pnode(1.765180,1.228386){n90}
\pnode(0.4301581,0.7624465){n91}
\pnode(4.292557,3.369169){n92}
\pnode(2.64,3.705){n93}
\pnode(2.869689,4.205452){n94}
\pnode(4.306428,1.798957){n95}
\pnode(0.9419618,2.164323){n96}
\pnode(2.802984,2.471939){n97}
\pnode(0.4990411,2.364772){n98}
\pnode(2.265986,3.781169){n99}
\pnode(2.886530,2.782621){n100}
\pnode(2.074548,4.120629){n101}
\pnode(3.578431,3.339056){n102}
\pnode(4.396008,2.731498){n103}
\pnode(2.161968,2.481101){n104}
\pnode(1.915954,1.686363){n105}
\pnode(1.073082,0.7635667){n106}
\pnode(4.351759,2.410157){n107}
\pnode(0.1632209,1.576404){n108}
\pnode(0.1964432,1.278737){n109}
\pnode(0.5000101,2.853756){n110}
\pnode(2.625767,0.8297721){n111}
\pnode(1.705716,3.114971){n112}
\pnode(2.796314,0.5814426){n113}
\pnode(0.3549037,3.121353){n114}
\pnode(3.949152,0.8549694){n115}
\pnode(4.562201,1.223617){n116}
\pnode(3.053188,3.925351){n117}
\pnode(4.318411,3.852927){n118}
\pnode(2.890681,3.735371){n119}
\pnode(2.880736,4.879838){n120}
\pnode(1.377059,0.7879707){n121}
\pnode(2.617372,2.90596){n122}
\pnode(3.046366,4.518936){n123}
\pnode(2.621703,0.4200536){n124}
\pnode(1.203850,1.499658){n125}
\pnode(0.9849086,4.079775){n126}
\pnode(3.254325,1.04113){n127}
\pnode(2.280932,2.838994){n128}
\pnode(4.117188,3.717776){n129}
\pnode(3.471599,1.285363){n130}
\pnode(4.500053,3.637197){n131}
\pnode(4.069486,1.411759){n132}
\pnode(4.11247,0.5910543){n133}
\pnode(1.252724,1.071603){n134}
\pnode(3.385955,0.3694225){n135}
\pnode(2.407195,0.4605674){n136}
\pnode(1.689756,4.453583){n137}
\pnode(3.989081,3.406712){n138}
\pnode(0.710692,2.532491){n139}
\pnode(4.873979,1.934726){n140}
\pnode(3.645946,2.645275){n141}
\pnode(1.606368,3.440222){n142}
\pnode(2.445606,3.296266){n143}
\pnode(2.274105,4.535248){n144}
\pnode(2.648264,0.1809270){n145}
\pnode(0.7976457,3.6483){n146}
\pnode(3.950723,2.918283){n147}
\pnode(4.776492,1.394465){n148}
\pnode(2.296867,0.7074797){n149}
\psset{linewidth=0.5pt,dotsep=2pt}
\ncline[linecolor=lightgray]{n0}{n44}
\ncline[linecolor=lightgray]{n0}{n137}
\ncline[linecolor=lightgray]{n1}{n2}
\ncline[linecolor=lightgray]{n1}{n104}
\ncline[linecolor=lightgray]{n1}{n128}
\ncline[linecolor=lightgray]{n2}{n142}
\ncline[linecolor=lightgray]{n1}{n2}
\ncline[linecolor=lightgray]{n2}{n41}
\ncline[linecolor=lightgray]{n2}{n97}
\ncline[linecolor=lightgray]{n2}{n58}
\ncline[linecolor=lightgray]{n2}{n84}
\ncline[linecolor=lightgray]{n2}{n99}
\ncline[linecolor=lightgray]{n4}{n33}
\ncline[linecolor=lightgray]{n4}{n35}
\ncline[linecolor=lightgray]{n4}{n36}
\ncline[linecolor=lightgray]{n4}{n139}
\ncline[linecolor=lightgray]{n4}{n26}
\ncline[linecolor=lightgray]{n4}{n38}
\ncline[linecolor=lightgray]{n4}{n98}
\ncline[linecolor=lightgray]{n4}{n47}
\ncline[linecolor=lightgray]{n4}{n93}
\ncline[linecolor=lightgray]{n6}{n107}
\ncline[linecolor=lightgray]{n6}{n87}
\ncline[linecolor=lightgray]{n6}{n138}
\ncline[linecolor=lightgray]{n6}{n83}
\ncline[linecolor=lightgray]{n6}{n53}
\ncline[linecolor=lightgray]{n6}{n118}
\ncline[linecolor=lightgray]{n6}{n55}
\ncline[linecolor=lightgray]{n6}{n92}
\ncline[linecolor=lightgray]{n6}{n56}
\ncline[linecolor=lightgray]{n7}{n13}
\ncline[linecolor=lightgray]{n7}{n144}
\ncline[linecolor=lightgray]{n7}{n66}
\ncline[linecolor=lightgray]{n7}{n77}
\ncline[linecolor=lightgray]{n7}{n89}
\ncline[linecolor=lightgray]{n7}{n39}
\ncline[linecolor=lightgray]{n7}{n86}
\ncline[linecolor=lightgray]{n7}{n94}
\ncline[linecolor=lightgray]{n8}{n131}
\ncline[linecolor=lightgray]{n8}{n55}
\ncline[linecolor=lightgray]{n8}{n129}
\ncline[linecolor=lightgray]{n8}{n141}
\ncline[linecolor=lightgray]{n8}{n46}
\ncline[linecolor=lightgray]{n9}{n129}
\ncline[linecolor=lightgray]{n9}{n83}
\ncline[linecolor=lightgray]{n9}{n131}
\ncline[linecolor=lightgray]{n9}{n46}
\ncline[linecolor=lightgray]{n9}{n28}
\ncline[linecolor=lightgray]{n9}{n60}
\ncline[linecolor=lightgray]{n9}{n95}
\ncline[linecolor=lightgray]{n12}{n146}
\ncline[linecolor=lightgray]{n12}{n72}
\ncline[linecolor=lightgray]{n12}{n65}
\ncline[linecolor=lightgray]{n12}{n142}
\ncline[linecolor=lightgray]{n7}{n13}
\ncline[linecolor=lightgray]{n13}{n120}
\ncline[linecolor=lightgray]{n13}{n41}
\ncline[linecolor=lightgray]{n13}{n79}
\ncline[linestyle=dotted,linecolor=lightgray]{n13}{n102}
\ncline[linestyle=dotted,linecolor=lightgray]{n13}{n21}
\ncline[linecolor=lightgray]{n17}{n40}
\ncline[linecolor=lightgray]{n17}{n59}
\ncline[linecolor=lightgray]{n19}{n95}
\ncline[linecolor=lightgray]{n19}{n29}
\ncline[linecolor=lightgray]{n19}{n107}
\ncline[linecolor=lightgray]{n19}{n57}
\ncline[linecolor=lightgray]{n19}{n116}
\ncline[linecolor=lightgray]{n19}{n140}
\ncline[linecolor=lightgray]{n20}{n77}
\ncline[linecolor=lightgray]{n20}{n89}
\ncline[linecolor=lightgray]{n20}{n117}
\ncline[linecolor=lightgray]{n20}{n32}
\ncline[linecolor=lightgray]{n20}{n66}
\ncline[linecolor=lightgray]{n21}{n129}
\ncline[linecolor=lightgray]{n21}{n56}
\ncline[linecolor=lightgray]{n21}{n83}
\ncline[linecolor=lightgray]{n21}{n131}
\ncline[linestyle=dotted,linecolor=lightgray]{n21}{n89}
\ncline[linecolor=lightgray]{n21}{n141}
\ncline[linestyle=dotted,linecolor=lightgray]{n13}{n21}
\ncline[linecolor=lightgray]{n22}{n133}
\ncline[linecolor=lightgray]{n22}{n49}
\ncline[linecolor=lightgray]{n22}{n130}
\ncline[linecolor=lightgray]{n23}{n36}
\ncline[linecolor=lightgray]{n23}{n50}
\ncline[linecolor=lightgray]{n23}{n64}
\ncline[linecolor=lightgray]{n23}{n59}
\ncline[linecolor=lightgray]{n25}{n54}
\ncline[linecolor=lightgray]{n25}{n37}
\ncline[linecolor=lightgray]{n26}{n146}
\ncline[linecolor=lightgray]{n26}{n110}
\ncline[linecolor=lightgray]{n26}{n114}
\ncline[linecolor=lightgray]{n4}{n26}
\ncline[linecolor=lightgray]{n26}{n47}
\ncline[linecolor=lightgray]{n28}{n95}
\ncline[linecolor=lightgray]{n28}{n46}
\ncline[linecolor=lightgray]{n28}{n43}
\ncline[linecolor=lightgray]{n9}{n28}
\ncline[linecolor=lightgray]{n28}{n57}
\ncline[linecolor=lightgray]{n29}{n42}
\ncline[linecolor=lightgray]{n29}{n53}
\ncline[linecolor=lightgray]{n29}{n56}
\ncline[linecolor=lightgray]{n29}{n55}
\ncline[linecolor=lightgray]{n29}{n140}
\ncline[linecolor=lightgray]{n19}{n29}
\ncline[linecolor=lightgray]{n30}{n133}
\ncline[linecolor=lightgray]{n30}{n49}
\ncline[linecolor=lightgray]{n32}{n66}
\ncline[linecolor=lightgray]{n32}{n123}
\ncline[linecolor=lightgray]{n20}{n32}
\ncline[linecolor=lightgray]{n32}{n39}
\ncline[linecolor=lightgray]{n32}{n94}
\ncline[linecolor=lightgray]{n33}{n64}
\ncline[linecolor=lightgray]{n4}{n33}
\ncline[linecolor=lightgray]{n33}{n146}
\ncline[linecolor=lightgray]{n33}{n98}
\ncline[linestyle=dotted,linecolor=lightgray]{n34}{n85}
\ncline[linecolor=lightgray]{n4}{n35}
\ncline[linecolor=lightgray]{n35}{n146}
\ncline[linecolor=lightgray]{n35}{n110}
\ncline[linecolor=lightgray]{n35}{n59}
\ncline[linecolor=lightgray]{n36}{n64}
\ncline[linecolor=lightgray]{n23}{n36}
\ncline[linecolor=lightgray]{n36}{n146}
\ncline[linecolor=lightgray]{n36}{n126}
\ncline[linecolor=lightgray]{n36}{n139}
\ncline[linecolor=lightgray]{n4}{n36}
\ncline[linecolor=lightgray]{n25}{n37}
\ncline[linecolor=lightgray]{n37}{n105}
\ncline[linecolor=lightgray]{n37}{n130}
\ncline[linecolor=lightgray]{n38}{n119}
\ncline[linecolor=lightgray]{n4}{n38}
\ncline[linecolor=lightgray]{n38}{n112}
\ncline[linecolor=lightgray]{n38}{n58}
\ncline[linecolor=lightgray]{n38}{n94}
\ncline[linecolor=lightgray]{n39}{n123}
\ncline[linecolor=lightgray]{n39}{n117}
\ncline[linecolor=lightgray]{n7}{n39}
\ncline[linecolor=lightgray]{n39}{n120}
\ncline[linecolor=lightgray]{n39}{n94}
\ncline[linecolor=lightgray]{n32}{n39}
\ncline[linecolor=lightgray]{n40}{n44}
\ncline[linecolor=lightgray]{n17}{n40}
\ncline[linecolor=lightgray]{n40}{n146}
\ncline[linecolor=lightgray]{n40}{n41}
\ncline[linecolor=lightgray]{n40}{n86}
\ncline[linecolor=lightgray]{n40}{n137}
\ncline[linecolor=lightgray]{n41}{n126}
\ncline[linecolor=lightgray]{n41}{n137}
\ncline[linecolor=lightgray]{n41}{n77}
\ncline[linecolor=lightgray]{n13}{n41}
\ncline[linecolor=lightgray]{n2}{n41}
\ncline[linecolor=lightgray]{n40}{n41}
\ncline[linecolor=lightgray]{n42}{n131}
\ncline[linecolor=lightgray]{n42}{n141}
\ncline[linecolor=lightgray]{n29}{n42}
\ncline[linecolor=lightgray]{n42}{n46}
\ncline[linecolor=lightgray]{n42}{n118}
\ncline[linecolor=lightgray]{n42}{n92}
\ncline[linecolor=lightgray]{n43}{n116}
\ncline[linecolor=lightgray]{n43}{n46}
\ncline[linecolor=lightgray]{n28}{n43}
\ncline[linecolor=lightgray]{n43}{n132}
\ncline[linecolor=lightgray]{n43}{n57}
\ncline[linecolor=lightgray]{n43}{n71}
\ncline[linecolor=lightgray]{n43}{n140}
\ncline[linecolor=lightgray]{n40}{n44}
\ncline[linecolor=lightgray]{n0}{n44}
\ncline[linecolor=lightgray]{n44}{n65}
\ncline[linecolor=lightgray]{n42}{n46}
\ncline[linecolor=lightgray]{n28}{n46}
\ncline[linecolor=lightgray]{n46}{n103}
\ncline[linecolor=lightgray]{n43}{n46}
\ncline[linecolor=lightgray]{n46}{n56}
\ncline[linecolor=lightgray]{n8}{n46}
\ncline[linecolor=lightgray]{n46}{n57}
\ncline[linecolor=lightgray]{n46}{n87}
\ncline[linecolor=lightgray]{n9}{n46}
\ncline[linecolor=lightgray]{n46}{n140}
\ncline[linecolor=lightgray]{n47}{n114}
\ncline[linecolor=lightgray]{n26}{n47}
\ncline[linecolor=lightgray]{n4}{n47}
\ncline[linecolor=lightgray]{n49}{n145}
\ncline[linecolor=lightgray]{n30}{n49}
\ncline[linecolor=lightgray]{n49}{n73}
\ncline[linecolor=lightgray]{n49}{n115}
\ncline[linecolor=lightgray]{n49}{n133}
\ncline[linecolor=lightgray]{n22}{n49}
\ncline[linecolor=lightgray]{n49}{n130}
\ncline[linecolor=lightgray]{n50}{n79}
\ncline[linecolor=lightgray]{n50}{n65}
\ncline[linecolor=lightgray]{n23}{n50}
\ncline[linecolor=lightgray]{n51}{n66}
\ncline[linecolor=lightgray]{n51}{n144}
\ncline[linecolor=lightgray]{n51}{n86}
\ncline[linecolor=lightgray]{n51}{n65}
\ncline[linecolor=lightgray]{n52}{n74}
\ncline[linecolor=lightgray]{n53}{n87}
\ncline[linecolor=lightgray]{n6}{n53}
\ncline[linecolor=lightgray]{n29}{n53}
\ncline[linecolor=lightgray]{n53}{n80}
\ncline[linecolor=lightgray]{n25}{n54}
\ncline[linecolor=lightgray]{n8}{n55}
\ncline[linecolor=lightgray]{n55}{n103}
\ncline[linecolor=lightgray]{n6}{n55}
\ncline[linecolor=lightgray]{n29}{n55}
\ncline[linecolor=lightgray]{n55}{n131}
\ncline[linecolor=lightgray]{n55}{n107}
\ncline[linecolor=lightgray]{n21}{n56}
\ncline[linecolor=lightgray]{n56}{n138}
\ncline[linecolor=lightgray]{n56}{n102}
\ncline[linecolor=lightgray]{n46}{n56}
\ncline[linecolor=lightgray]{n29}{n56}
\ncline[linecolor=lightgray]{n56}{n92}
\ncline[linecolor=lightgray]{n56}{n87}
\ncline[linecolor=lightgray]{n6}{n56}
\ncline[linecolor=lightgray]{n56}{n118}
\ncline[linecolor=lightgray]{n57}{n95}
\ncline[linecolor=lightgray]{n57}{n116}
\ncline[linecolor=lightgray]{n46}{n57}
\ncline[linecolor=lightgray]{n43}{n57}
\ncline[linecolor=lightgray]{n57}{n132}
\ncline[linecolor=lightgray]{n19}{n57}
\ncline[linecolor=lightgray]{n57}{n140}
\ncline[linecolor=lightgray]{n57}{n71}
\ncline[linecolor=lightgray]{n28}{n57}
\ncline[linecolor=lightgray]{n57}{n70}
\ncline[linecolor=lightgray]{n57}{n148}
\ncline[linecolor=lightgray]{n58}{n119}
\ncline[linecolor=lightgray]{n58}{n143}
\ncline[linecolor=lightgray]{n58}{n112}
\ncline[linecolor=lightgray]{n2}{n58}
\ncline[linecolor=lightgray]{n58}{n117}
\ncline[linecolor=lightgray]{n38}{n58}
\ncline[linecolor=lightgray]{n59}{n64}
\ncline[linecolor=lightgray]{n17}{n59}
\ncline[linecolor=lightgray]{n35}{n59}
\ncline[linecolor=lightgray]{n59}{n110}
\ncline[linecolor=lightgray]{n23}{n59}
\ncline[linecolor=lightgray]{n59}{n139}
\ncline[linecolor=lightgray]{n59}{n98}
\ncline[linecolor=lightgray]{n60}{n83}
\ncline[linestyle=dotted,linecolor=lightgray]{n60}{n123}
\ncline[linestyle=dotted,linecolor=lightgray]{n60}{n117}
\ncline[linestyle=dotted,linecolor=lightgray]{n60}{n94}
\ncline[linecolor=lightgray]{n9}{n60}
\ncline[linestyle=dotted,linecolor=lightgray]{n61}{n117}
\ncline[linestyle=dotted,linecolor=lightgray]{n61}{n123}
\ncline[linestyle=dotted,linecolor=lightgray]{n61}{n84}
\ncline[linecolor=lightgray]{n61}{n102}
\ncline[linecolor=lightgray]{n61}{n147}
\ncline[linecolor=lightgray]{n63}{n105}
\ncline[linecolor=lightgray]{n33}{n64}
\ncline[linecolor=lightgray]{n36}{n64}
\ncline[linecolor=lightgray]{n64}{n65}
\ncline[linecolor=lightgray]{n59}{n64}
\ncline[linecolor=lightgray]{n23}{n64}
\ncline[linecolor=lightgray]{n64}{n146}
\ncline[linecolor=lightgray]{n64}{n110}
\ncline[linecolor=lightgray]{n50}{n65}
\ncline[linecolor=lightgray]{n64}{n65}
\ncline[linecolor=lightgray]{n44}{n65}
\ncline[linecolor=lightgray]{n12}{n65}
\ncline[linecolor=lightgray]{n65}{n146}
\ncline[linecolor=lightgray]{n51}{n65}
\ncline[linecolor=lightgray]{n65}{n110}
\ncline[linecolor=lightgray]{n65}{n137}
\ncline[linecolor=lightgray]{n51}{n66}
\ncline[linecolor=lightgray]{n7}{n66}
\ncline[linecolor=lightgray]{n66}{n120}
\ncline[linecolor=lightgray]{n32}{n66}
\ncline[linecolor=lightgray]{n20}{n66}
\ncline[linecolor=lightgray]{n68}{n91}
\ncline[linecolor=lightgray]{n68}{n109}
\ncline[linecolor=lightgray]{n70}{n116}
\ncline[linecolor=lightgray]{n57}{n70}
\ncline[linecolor=lightgray]{n70}{n140}
\ncline[linecolor=lightgray]{n70}{n132}
\ncline[linecolor=lightgray]{n70}{n71}
\ncline[linecolor=lightgray]{n71}{n116}
\ncline[linecolor=lightgray]{n57}{n71}
\ncline[linecolor=lightgray]{n43}{n71}
\ncline[linecolor=lightgray]{n71}{n95}
\ncline[linecolor=lightgray]{n71}{n148}
\ncline[linecolor=lightgray]{n70}{n71}
\ncline[linecolor=lightgray]{n12}{n72}
\ncline[linecolor=lightgray]{n49}{n73}
\ncline[linecolor=lightgray]{n52}{n74}
\ncline[linecolor=lightgray]{n20}{n77}
\ncline[linecolor=lightgray]{n77}{n143}
\ncline[linecolor=lightgray]{n77}{n79}
\ncline[linecolor=lightgray]{n77}{n86}
\ncline[linecolor=lightgray]{n41}{n77}
\ncline[linecolor=lightgray]{n7}{n77}
\ncline[linecolor=lightgray]{n77}{n120}
\ncline[linecolor=lightgray]{n50}{n79}
\ncline[linecolor=lightgray]{n77}{n79}
\ncline[linecolor=lightgray]{n79}{n112}
\ncline[linecolor=lightgray]{n79}{n82}
\ncline[linecolor=lightgray]{n79}{n137}
\ncline[linecolor=lightgray]{n79}{n143}
\ncline[linecolor=lightgray]{n13}{n79}
\ncline[linecolor=lightgray]{n80}{n102}
\ncline[linecolor=lightgray]{n80}{n118}
\ncline[linecolor=lightgray]{n80}{n87}
\ncline[linecolor=lightgray]{n53}{n80}
\ncline[linecolor=lightgray]{n81}{n98}
\ncline[linecolor=lightgray]{n82}{n101}
\ncline[linecolor=lightgray]{n79}{n82}
\ncline[linecolor=lightgray]{n21}{n83}
\ncline[linecolor=lightgray]{n6}{n83}
\ncline[linecolor=lightgray]{n60}{n83}
\ncline[linecolor=lightgray]{n9}{n83}
\ncline[linecolor=lightgray]{n84}{n117}
\ncline[linecolor=lightgray]{n84}{n89}
\ncline[linestyle=dotted,linecolor=lightgray]{n84}{n138}
\ncline[linestyle=dotted,linecolor=lightgray]{n61}{n84}
\ncline[linecolor=lightgray]{n84}{n97}
\ncline[linecolor=lightgray]{n2}{n84}
\ncline[linecolor=lightgray]{n84}{n99}
\ncline[linestyle=dotted,linecolor=lightgray]{n84}{n118}
\ncline[linecolor=lightgray]{n85}{n98}
\ncline[linestyle=dotted,linecolor=lightgray]{n34}{n85}
\ncline[linecolor=lightgray]{n86}{n99}
\ncline[linecolor=lightgray]{n77}{n86}
\ncline[linecolor=lightgray]{n51}{n86}
\ncline[linecolor=lightgray]{n40}{n86}
\ncline[linecolor=lightgray]{n7}{n86}
\ncline[linecolor=lightgray]{n86}{n120}
\ncline[linecolor=lightgray]{n6}{n87}
\ncline[linecolor=lightgray]{n53}{n87}
\ncline[linecolor=lightgray]{n87}{n92}
\ncline[linecolor=lightgray]{n56}{n87}
\ncline[linecolor=lightgray]{n87}{n147}
\ncline[linecolor=lightgray]{n80}{n87}
\ncline[linecolor=lightgray]{n46}{n87}
\ncline[linecolor=lightgray]{n87}{n141}
\ncline[linecolor=lightgray]{n88}{n115}
\ncline[linecolor=lightgray]{n89}{n143}
\ncline[linestyle=dotted,linecolor=lightgray]{n21}{n89}
\ncline[linecolor=lightgray]{n84}{n89}
\ncline[linecolor=lightgray]{n20}{n89}
\ncline[linecolor=lightgray]{n7}{n89}
\ncline[linecolor=lightgray]{n89}{n120}
\ncline[linecolor=lightgray]{n68}{n91}
\ncline[linecolor=lightgray]{n92}{n131}
\ncline[linecolor=lightgray]{n42}{n92}
\ncline[linecolor=lightgray]{n92}{n129}
\ncline[linecolor=lightgray]{n92}{n103}
\ncline[linecolor=lightgray]{n92}{n118}
\ncline[linecolor=lightgray]{n92}{n141}
\ncline[linecolor=lightgray]{n87}{n92}
\ncline[linecolor=lightgray]{n6}{n92}
\ncline[linecolor=lightgray]{n56}{n92}
\ncline[linecolor=lightgray]{n93}{n119}
\ncline[linecolor=lightgray]{n93}{n94}
\ncline[linecolor=lightgray]{n4}{n93}
\ncline[linecolor=lightgray]{n94}{n123}
\ncline[linecolor=lightgray]{n93}{n94}
\ncline[linecolor=lightgray]{n39}{n94}
\ncline[linecolor=lightgray]{n32}{n94}
\ncline[linecolor=lightgray]{n7}{n94}
\ncline[linecolor=lightgray]{n94}{n120}
\ncline[linecolor=lightgray]{n38}{n94}
\ncline[linestyle=dotted,linecolor=lightgray]{n60}{n94}
\ncline[linecolor=lightgray]{n19}{n95}
\ncline[linecolor=lightgray]{n28}{n95}
\ncline[linecolor=lightgray]{n57}{n95}
\ncline[linecolor=lightgray]{n95}{n116}
\ncline[linecolor=lightgray]{n95}{n132}
\ncline[linecolor=lightgray]{n9}{n95}
\ncline[linecolor=lightgray]{n71}{n95}
\ncline[linecolor=lightgray]{n2}{n97}
\ncline[linecolor=lightgray]{n97}{n128}
\ncline[linecolor=lightgray]{n84}{n97}
\ncline[linecolor=lightgray]{n33}{n98}
\ncline[linecolor=lightgray]{n98}{n114}
\ncline[linecolor=lightgray]{n59}{n98}
\ncline[linecolor=lightgray]{n4}{n98}
\ncline[linecolor=lightgray]{n98}{n110}
\ncline[linecolor=lightgray]{n81}{n98}
\ncline[linecolor=lightgray]{n85}{n98}
\ncline[linecolor=lightgray]{n86}{n99}
\ncline[linecolor=lightgray]{n99}{n144}
\ncline[linecolor=lightgray]{n99}{n101}
\ncline[linecolor=lightgray]{n99}{n128}
\ncline[linecolor=lightgray]{n99}{n104}
\ncline[linecolor=lightgray]{n2}{n99}
\ncline[linecolor=lightgray]{n84}{n99}
\ncline[linecolor=lightgray]{n100}{n128}
\ncline[linecolor=lightgray]{n100}{n104}
\ncline[linecolor=lightgray]{n100}{n119}
\ncline[linecolor=lightgray]{n101}{n126}
\ncline[linecolor=lightgray]{n101}{n117}
\ncline[linecolor=lightgray]{n82}{n101}
\ncline[linecolor=lightgray]{n99}{n101}
\ncline[linecolor=lightgray]{n101}{n123}
\ncline[linecolor=lightgray]{n101}{n119}
\ncline[linecolor=lightgray]{n101}{n112}
\ncline[linecolor=lightgray]{n102}{n118}
\ncline[linecolor=lightgray]{n102}{n138}
\ncline[linecolor=lightgray]{n102}{n141}
\ncline[linecolor=lightgray]{n56}{n102}
\ncline[linestyle=dotted,linecolor=lightgray]{n13}{n102}
\ncline[linecolor=lightgray]{n61}{n102}
\ncline[linecolor=lightgray]{n80}{n102}
\ncline[linecolor=lightgray]{n103}{n118}
\ncline[linecolor=lightgray]{n103}{n138}
\ncline[linecolor=lightgray]{n55}{n103}
\ncline[linecolor=lightgray]{n46}{n103}
\ncline[linecolor=lightgray]{n92}{n103}
\ncline[linecolor=lightgray]{n103}{n129}
\ncline[linecolor=lightgray]{n103}{n131}
\ncline[linecolor=lightgray]{n103}{n147}
\ncline[linecolor=lightgray]{n100}{n104}
\ncline[linecolor=lightgray]{n1}{n104}
\ncline[linecolor=lightgray]{n99}{n104}
\ncline[linecolor=lightgray]{n63}{n105}
\ncline[linecolor=lightgray]{n37}{n105}
\ncline[linecolor=lightgray]{n6}{n107}
\ncline[linecolor=lightgray]{n107}{n147}
\ncline[linecolor=lightgray]{n107}{n141}
\ncline[linecolor=lightgray]{n55}{n107}
\ncline[linecolor=lightgray]{n19}{n107}
\ncline[linecolor=lightgray]{n107}{n140}
\ncline[linecolor=lightgray]{n68}{n109}
\ncline[linecolor=lightgray]{n26}{n110}
\ncline[linecolor=lightgray]{n110}{n139}
\ncline[linecolor=lightgray]{n35}{n110}
\ncline[linecolor=lightgray]{n59}{n110}
\ncline[linecolor=lightgray]{n64}{n110}
\ncline[linecolor=lightgray]{n98}{n110}
\ncline[linecolor=lightgray]{n110}{n146}
\ncline[linecolor=lightgray]{n65}{n110}
\ncline[linecolor=lightgray]{n79}{n112}
\ncline[linecolor=lightgray]{n101}{n112}
\ncline[linecolor=lightgray]{n58}{n112}
\ncline[linecolor=lightgray]{n112}{n122}
\ncline[linecolor=lightgray]{n38}{n112}
\ncline[linecolor=lightgray]{n47}{n114}
\ncline[linecolor=lightgray]{n98}{n114}
\ncline[linecolor=lightgray]{n26}{n114}
\ncline[linecolor=lightgray]{n88}{n115}
\ncline[linecolor=lightgray]{n49}{n115}
\ncline[linecolor=lightgray]{n43}{n116}
\ncline[linecolor=lightgray]{n57}{n116}
\ncline[linecolor=lightgray]{n71}{n116}
\ncline[linecolor=lightgray]{n95}{n116}
\ncline[linecolor=lightgray]{n19}{n116}
\ncline[linecolor=lightgray]{n116}{n148}
\ncline[linecolor=lightgray]{n70}{n116}
\ncline[linecolor=lightgray]{n116}{n140}
\ncline[linecolor=lightgray]{n101}{n117}
\ncline[linestyle=dotted,linecolor=lightgray]{n61}{n117}
\ncline[linecolor=lightgray]{n84}{n117}
\ncline[linecolor=lightgray]{n39}{n117}
\ncline[linecolor=lightgray]{n20}{n117}
\ncline[linecolor=lightgray]{n58}{n117}
\ncline[linecolor=lightgray]{n117}{n144}
\ncline[linestyle=dotted,linecolor=lightgray]{n117}{n138}
\ncline[linestyle=dotted,linecolor=lightgray]{n60}{n117}
\ncline[linecolor=lightgray]{n102}{n118}
\ncline[linecolor=lightgray]{n42}{n118}
\ncline[linecolor=lightgray]{n103}{n118}
\ncline[linecolor=lightgray]{n118}{n131}
\ncline[linecolor=lightgray]{n6}{n118}
\ncline[linecolor=lightgray]{n92}{n118}
\ncline[linecolor=lightgray]{n56}{n118}
\ncline[linestyle=dotted,linecolor=lightgray]{n84}{n118}
\ncline[linecolor=lightgray]{n80}{n118}
\ncline[linecolor=lightgray]{n119}{n143}
\ncline[linecolor=lightgray]{n101}{n119}
\ncline[linecolor=lightgray]{n58}{n119}
\ncline[linecolor=lightgray]{n93}{n119}
\ncline[linecolor=lightgray]{n38}{n119}
\ncline[linecolor=lightgray]{n100}{n119}
\ncline[linecolor=lightgray]{n13}{n120}
\ncline[linecolor=lightgray]{n120}{n144}
\ncline[linecolor=lightgray]{n66}{n120}
\ncline[linecolor=lightgray]{n77}{n120}
\ncline[linecolor=lightgray]{n89}{n120}
\ncline[linecolor=lightgray]{n39}{n120}
\ncline[linecolor=lightgray]{n86}{n120}
\ncline[linecolor=lightgray]{n94}{n120}
\ncline[linecolor=lightgray]{n112}{n122}
\ncline[linecolor=lightgray]{n123}{n144}
\ncline[linecolor=lightgray]{n39}{n123}
\ncline[linecolor=lightgray]{n101}{n123}
\ncline[linecolor=lightgray]{n94}{n123}
\ncline[linestyle=dotted,linecolor=lightgray]{n61}{n123}
\ncline[linecolor=lightgray]{n32}{n123}
\ncline[linestyle=dotted,linecolor=lightgray]{n60}{n123}
\ncline[linecolor=lightgray]{n41}{n126}
\ncline[linecolor=lightgray]{n101}{n126}
\ncline[linecolor=lightgray]{n36}{n126}
\ncline[linecolor=lightgray]{n126}{n142}
\ncline[linecolor=lightgray]{n100}{n128}
\ncline[linecolor=lightgray]{n97}{n128}
\ncline[linecolor=lightgray]{n99}{n128}
\ncline[linecolor=lightgray]{n1}{n128}
\ncline[linecolor=lightgray]{n21}{n129}
\ncline[linecolor=lightgray]{n8}{n129}
\ncline[linecolor=lightgray]{n92}{n129}
\ncline[linecolor=lightgray]{n9}{n129}
\ncline[linecolor=lightgray]{n129}{n147}
\ncline[linecolor=lightgray]{n103}{n129}
\ncline[linecolor=lightgray]{n37}{n130}
\ncline[linecolor=lightgray]{n49}{n130}
\ncline[linecolor=lightgray]{n22}{n130}
\ncline[linecolor=lightgray]{n42}{n131}
\ncline[linecolor=lightgray]{n8}{n131}
\ncline[linecolor=lightgray]{n131}{n138}
\ncline[linecolor=lightgray]{n21}{n131}
\ncline[linecolor=lightgray]{n92}{n131}
\ncline[linecolor=lightgray]{n118}{n131}
\ncline[linecolor=lightgray]{n131}{n147}
\ncline[linecolor=lightgray]{n103}{n131}
\ncline[linecolor=lightgray]{n9}{n131}
\ncline[linecolor=lightgray]{n55}{n131}
\ncline[linecolor=lightgray]{n43}{n132}
\ncline[linecolor=lightgray]{n95}{n132}
\ncline[linecolor=lightgray]{n57}{n132}
\ncline[linecolor=lightgray]{n132}{n148}
\ncline[linecolor=lightgray]{n70}{n132}
\ncline[linecolor=lightgray]{n30}{n133}
\ncline[linecolor=lightgray]{n22}{n133}
\ncline[linecolor=lightgray]{n49}{n133}
\ncline[linecolor=lightgray]{n41}{n137}
\ncline[linecolor=lightgray]{n79}{n137}
\ncline[linecolor=lightgray]{n0}{n137}
\ncline[linecolor=lightgray]{n137}{n144}
\ncline[linecolor=lightgray]{n40}{n137}
\ncline[linecolor=lightgray]{n65}{n137}
\ncline[linecolor=lightgray]{n131}{n138}
\ncline[linecolor=lightgray]{n102}{n138}
\ncline[linecolor=lightgray]{n6}{n138}
\ncline[linecolor=lightgray]{n103}{n138}
\ncline[linecolor=lightgray]{n56}{n138}
\ncline[linestyle=dotted,linecolor=lightgray]{n84}{n138}
\ncline[linestyle=dotted,linecolor=lightgray]{n117}{n138}
\ncline[linecolor=lightgray]{n138}{n147}
\ncline[linecolor=lightgray]{n36}{n139}
\ncline[linecolor=lightgray]{n110}{n139}
\ncline[linecolor=lightgray]{n59}{n139}
\ncline[linecolor=lightgray]{n4}{n139}
\ncline[linecolor=lightgray]{n29}{n140}
\ncline[linecolor=lightgray]{n107}{n140}
\ncline[linecolor=lightgray]{n46}{n140}
\ncline[linecolor=lightgray]{n57}{n140}
\ncline[linecolor=lightgray]{n19}{n140}
\ncline[linecolor=lightgray]{n140}{n148}
\ncline[linecolor=lightgray]{n70}{n140}
\ncline[linecolor=lightgray]{n116}{n140}
\ncline[linecolor=lightgray]{n43}{n140}
\ncline[linecolor=lightgray]{n42}{n141}
\ncline[linecolor=lightgray]{n102}{n141}
\ncline[linecolor=lightgray]{n21}{n141}
\ncline[linecolor=lightgray]{n8}{n141}
\ncline[linecolor=lightgray]{n92}{n141}
\ncline[linecolor=lightgray]{n107}{n141}
\ncline[linecolor=lightgray]{n87}{n141}
\ncline[linecolor=lightgray]{n142}{n143}
\ncline[linecolor=lightgray]{n2}{n142}
\ncline[linecolor=lightgray]{n12}{n142}
\ncline[linecolor=lightgray]{n126}{n142}
\ncline[linecolor=lightgray]{n77}{n143}
\ncline[linecolor=lightgray]{n119}{n143}
\ncline[linecolor=lightgray]{n142}{n143}
\ncline[linecolor=lightgray]{n89}{n143}
\ncline[linecolor=lightgray]{n58}{n143}
\ncline[linecolor=lightgray]{n79}{n143}
\ncline[linecolor=lightgray]{n123}{n144}
\ncline[linecolor=lightgray]{n99}{n144}
\ncline[linecolor=lightgray]{n7}{n144}
\ncline[linecolor=lightgray]{n120}{n144}
\ncline[linecolor=lightgray]{n51}{n144}
\ncline[linecolor=lightgray]{n137}{n144}
\ncline[linecolor=lightgray]{n117}{n144}
\ncline[linecolor=lightgray]{n49}{n145}
\ncline[linecolor=lightgray]{n36}{n146}
\ncline[linecolor=lightgray]{n12}{n146}
\ncline[linecolor=lightgray]{n33}{n146}
\ncline[linecolor=lightgray]{n26}{n146}
\ncline[linecolor=lightgray]{n40}{n146}
\ncline[linecolor=lightgray]{n65}{n146}
\ncline[linecolor=lightgray]{n64}{n146}
\ncline[linecolor=lightgray]{n35}{n146}
\ncline[linecolor=lightgray]{n110}{n146}
\ncline[linecolor=lightgray]{n131}{n147}
\ncline[linecolor=lightgray]{n129}{n147}
\ncline[linecolor=lightgray]{n107}{n147}
\ncline[linecolor=lightgray]{n61}{n147}
\ncline[linecolor=lightgray]{n103}{n147}
\ncline[linecolor=lightgray]{n138}{n147}
\ncline[linecolor=lightgray]{n87}{n147}
\ncline[linecolor=lightgray]{n116}{n148}
\ncline[linecolor=lightgray]{n140}{n148}
\ncline[linecolor=lightgray]{n57}{n148}
\ncline[linecolor=lightgray]{n132}{n148}
\ncline[linecolor=lightgray]{n71}{n148}
\ncline[linecolor=lightgray]{n14}{n34}
\ncline[linecolor=lightgray]{n15}{n34}
\ncline[linecolor=lightgray]{n25}{n34}
\ncline[linecolor=lightgray]{n34}{n37}
\ncline[linecolor=lightgray]{n34}{n54}
\ncline[linecolor=lightgray]{n34}{n63}
\ncline[linecolor=lightgray]{n34}{n67}
\ncline[linecolor=lightgray]{n34}{n75}
\ncline[linecolor=lightgray]{n34}{n78}
\ncline[linecolor=lightgray]{n34}{n105}
\ncline[linecolor=lightgray]{n34}{n121}
\ncline[linecolor=lightgray]{n34}{n125}
\ncline[linecolor=lightgray]{n34}{n134}
\ncline[linecolor=lightgray]{n34}{n145}
\ncline[linestyle=dotted,linecolor=lightgray]{n63}{n96}
\ncline[linestyle=dotted,linecolor=lightgray]{n96}{n105}
\ncline[linecolor=lightgray]{n12}{n24}
\ncline[linecolor=lightgray]{n17}{n24}
\ncline[linecolor=lightgray]{n23}{n24}
\ncline[linecolor=lightgray]{n24}{n26}
\ncline[linecolor=lightgray]{n24}{n32}
\ncline[linecolor=lightgray]{n24}{n35}
\ncline[linecolor=lightgray]{n24}{n40}
\ncline[linecolor=lightgray]{n24}{n44}
\ncline[linecolor=lightgray]{n24}{n50}
\ncline[linecolor=lightgray]{n24}{n51}
\ncline[linecolor=lightgray]{n24}{n64}
\ncline[linecolor=lightgray]{n24}{n65}
\ncline[linecolor=lightgray]{n24}{n110}
\ncline[linecolor=lightgray]{n24}{n126}
\ncline[linecolor=lightgray]{n24}{n139}
\ncline[linecolor=lightgray]{n24}{n146}
\ncline[linestyle=dotted,linecolor=lightgray]{n4}{n108}
\ncline[linestyle=dotted,linecolor=lightgray]{n23}{n108}
\ncline[linestyle=dotted,linecolor=lightgray]{n26}{n108}
\ncline[linestyle=dotted,linecolor=lightgray]{n33}{n108}
\ncline[linecolor=lightgray]{n34}{n108}
\ncline[linestyle=dotted,linecolor=lightgray]{n35}{n108}
\ncline[linestyle=dotted,linecolor=lightgray]{n36}{n108}
\ncline[linestyle=dotted,linecolor=lightgray]{n47}{n108}
\ncline[linestyle=dotted,linecolor=lightgray]{n59}{n108}
\ncline[linestyle=dotted,linecolor=lightgray]{n81}{n108}
\ncline[linestyle=dotted,linecolor=lightgray]{n85}{n108}
\ncline[linestyle=dotted,linecolor=lightgray]{n96}{n108}
\ncline[linestyle=dotted,linecolor=lightgray]{n98}{n108}
\ncline[linestyle=dotted,linecolor=lightgray]{n108}{n110}
\ncline[linestyle=dotted,linecolor=lightgray]{n108}{n114}
\ncline[linestyle=dotted,linecolor=lightgray]{n108}{n139}
\ncline[linestyle=dotted,linecolor=lightgray]{n14}{n24}
\ncline[linestyle=dotted,linecolor=lightgray]{n15}{n24}
\ncline[linestyle=dotted,linecolor=lightgray]{n24}{n25}
\ncline[linestyle=dotted,linecolor=lightgray]{n24}{n27}
\ncline[linestyle=dotted,linecolor=lightgray]{n24}{n37}
\ncline[linestyle=dotted,linecolor=lightgray]{n24}{n54}
\ncline[linestyle=dotted,linecolor=lightgray]{n24}{n63}
\ncline[linestyle=dotted,linecolor=lightgray]{n24}{n67}
\ncline[linestyle=dotted,linecolor=lightgray]{n24}{n75}
\ncline[linestyle=dotted,linecolor=lightgray]{n24}{n78}
\ncline[linestyle=dotted,linecolor=lightgray]{n24}{n88}
\ncline[linestyle=dotted,linecolor=lightgray]{n24}{n90}
\ncline[linestyle=dotted,linecolor=lightgray]{n24}{n105}
\ncline[linestyle=dotted,linecolor=lightgray]{n24}{n121}
\ncline[linestyle=dotted,linecolor=lightgray]{n24}{n125}
\ncline[linestyle=dotted,linecolor=lightgray]{n24}{n134}
\ncline[linestyle=dotted,linecolor=lightgray]{n24}{n145}
\ncline[linecolor=lightgray]{n3}{n108}
\ncline[linecolor=lightgray]{n5}{n108}
\ncline[linecolor=lightgray]{n14}{n108}
\ncline[linecolor=lightgray]{n27}{n108}
\ncline[linecolor=lightgray]{n45}{n108}
\ncline[linecolor=lightgray]{n52}{n108}
\ncline[linecolor=lightgray]{n67}{n108}
\ncline[linecolor=lightgray]{n75}{n108}
\ncline[linecolor=lightgray]{n76}{n108}
\ncline[linecolor=lightgray]{n78}{n108}
\ncline[linecolor=lightgray]{n91}{n108}
\ncline[linecolor=lightgray]{n106}{n108}
\ncline[linecolor=lightgray]{n108}{n109}
\ncline[linecolor=lightgray]{n108}{n121}
\ncline[linecolor=lightgray]{n108}{n125}
\ncline[linecolor=lightgray]{n108}{n134}
\ncline[linestyle=dotted,linecolor=lightgray]{n24}{n108}
\ncline[linecolor=gray]{n0}{n79}
\ncline[linecolor=gray]{n0}{n12}
\ncline[linecolor=gray]{n0}{n17}
\ncline[linecolor=gray]{n0}{n64}
\ncline[linecolor=gray]{n0}{n142}
\ncline[linecolor=gray]{n0}{n72}
\ncline[linecolor=gray]{n0}{n40}
\ncline[linecolor=gray]{n0}{n41}
\ncline[linecolor=gray]{n0}{n126}
\ncline[linecolor=gray]{n1}{n100}
\ncline[linecolor=gray]{n2}{n112}
\ncline[linecolor=gray]{n2}{n82}
\ncline[linecolor=gray]{n3}{n90}
\ncline[linecolor=gray]{n3}{n14}
\ncline[linecolor=gray]{n3}{n67}
\ncline[linecolor=gray]{n3}{n134}
\ncline[linecolor=gray]{n3}{n76}
\ncline[linecolor=gray]{n3}{n106}
\ncline[linecolor=gray]{n5}{n125}
\ncline[linecolor=gray]{n5}{n52}
\ncline[linecolor=gray]{n6}{n8}
\ncline[linecolor=gray]{n6}{n103}
\ncline[linecolor=gray]{n6}{n29}
\ncline[linecolor=gray]{n6}{n42}
\ncline[linecolor=gray]{n7}{n20}
\ncline[linecolor=gray]{n7}{n123}
\ncline[linecolor=gray]{n8}{n42}
\ncline[linecolor=gray]{n8}{n29}
\ncline[linecolor=gray]{n6}{n8}
\ncline[linecolor=gray]{n8}{n103}
\ncline[linecolor=gray]{n8}{n56}
\ncline[linecolor=gray]{n8}{n147}
\ncline[linecolor=gray]{n8}{n92}
\ncline[linecolor=gray]{n10}{n15}
\ncline[linecolor=gray]{n10}{n149}
\ncline[linecolor=gray]{n10}{n135}
\ncline[linecolor=gray]{n10}{n37}
\ncline[linecolor=gray]{n11}{n124}
\ncline[linecolor=gray]{n11}{n113}
\ncline[linecolor=gray]{n11}{n135}
\ncline[linecolor=gray]{n11}{n62}
\ncline[linecolor=gray]{n11}{n130}
\ncline[linecolor=gray]{n11}{n30}
\ncline[linecolor=gray]{n11}{n88}
\ncline[linecolor=gray]{n11}{n37}
\ncline[linecolor=gray]{n12}{n44}
\ncline[linecolor=gray]{n0}{n12}
\ncline[linecolor=gray]{n12}{n126}
\ncline[linecolor=gray]{n12}{n36}
\ncline[linecolor=gray]{n12}{n40}
\ncline[linecolor=gray]{n12}{n114}
\ncline[linecolor=gray]{n12}{n33}
\ncline[linecolor=gray]{n13}{n89}
\ncline[linecolor=gray]{n13}{n123}
\ncline[linecolor=gray]{n13}{n117}
\ncline[linecolor=gray]{n13}{n99}
\ncline[linecolor=gray]{n13}{n101}
\ncline[linecolor=gray]{n14}{n16}
\ncline[linecolor=gray]{n14}{n31}
\ncline[linecolor=gray]{n14}{n90}
\ncline[linecolor=gray]{n3}{n14}
\ncline[linecolor=gray]{n10}{n15}
\ncline[linecolor=gray]{n15}{n62}
\ncline[linecolor=gray]{n15}{n69}
\ncline[linecolor=gray]{n15}{n111}
\ncline[linecolor=gray]{n15}{n105}
\ncline[linecolor=gray]{n15}{n125}
\ncline[linecolor=gray]{n14}{n16}
\ncline[linecolor=gray]{n16}{n67}
\ncline[linecolor=gray]{n16}{n134}
\ncline[linecolor=gray]{n16}{n31}
\ncline[linecolor=gray]{n16}{n75}
\ncline[linecolor=gray]{n16}{n125}
\ncline[linecolor=gray]{n17}{n36}
\ncline[linecolor=gray]{n0}{n17}
\ncline[linecolor=gray]{n17}{n23}
\ncline[linecolor=gray]{n17}{n126}
\ncline[linecolor=gray]{n17}{n72}
\ncline[linecolor=gray]{n17}{n35}
\ncline[linecolor=gray]{n17}{n146}
\ncline[linecolor=gray]{n19}{n28}
\ncline[linecolor=gray]{n19}{n46}
\ncline[linecolor=gray]{n19}{n43}
\ncline[linecolor=gray]{n7}{n20}
\ncline[linecolor=gray]{n20}{n120}
\ncline[linecolor=gray]{n20}{n86}
\ncline[linecolor=gray]{n20}{n123}
\ncline[linecolor=gray]{n20}{n51}
\ncline[linecolor=gray]{n21}{n61}
\ncline[linecolor=gray]{n21}{n138}
\ncline[linecolor=gray]{n21}{n147}
\ncline[linecolor=gray]{n21}{n92}
\ncline[linecolor=gray]{n21}{n118}
\ncline[linestyle=dotted,linecolor=gray]{n21}{n84}
\ncline[linecolor=gray]{n21}{n42}
\ncline[linecolor=gray]{n22}{n115}
\ncline[linecolor=gray]{n23}{n33}
\ncline[linecolor=gray]{n17}{n23}
\ncline[linecolor=gray]{n23}{n110}
\ncline[linecolor=gray]{n23}{n114}
\ncline[linecolor=gray]{n23}{n146}
\ncline[linecolor=gray]{n23}{n98}
\ncline[linecolor=gray]{n25}{n136}
\ncline[linecolor=gray]{n25}{n62}
\ncline[linecolor=gray]{n26}{n33}
\ncline[linecolor=gray]{n26}{n98}
\ncline[linecolor=gray]{n26}{n44}
\ncline[linecolor=gray]{n27}{n74}
\ncline[linecolor=gray]{n27}{n106}
\ncline[linecolor=gray]{n27}{n90}
\ncline[linecolor=gray]{n19}{n28}
\ncline[linecolor=gray]{n8}{n29}
\ncline[linecolor=gray]{n6}{n29}
\ncline[linecolor=gray]{n29}{n46}
\ncline[linecolor=gray]{n30}{n135}
\ncline[linecolor=gray]{n30}{n48}
\ncline[linecolor=gray]{n30}{n115}
\ncline[linecolor=gray]{n11}{n30}
\ncline[linecolor=gray]{n30}{n88}
\ncline[linecolor=gray]{n30}{n73}
\ncline[linecolor=gray]{n30}{n130}
\ncline[linecolor=gray]{n31}{n62}
\ncline[linecolor=gray]{n14}{n31}
\ncline[linecolor=gray]{n31}{n67}
\ncline[linecolor=gray]{n31}{n69}
\ncline[linecolor=gray]{n31}{n134}
\ncline[linecolor=gray]{n16}{n31}
\ncline[linecolor=gray]{n31}{n45}
\ncline[linecolor=gray]{n31}{n78}
\ncline[linecolor=gray]{n31}{n124}
\ncline[linecolor=gray]{n32}{n51}
\ncline[linecolor=gray]{n23}{n33}
\ncline[linecolor=gray]{n33}{n72}
\ncline[linecolor=gray]{n33}{n50}
\ncline[linecolor=gray]{n33}{n139}
\ncline[linecolor=gray]{n26}{n33}
\ncline[linecolor=gray]{n12}{n33}
\ncline[linestyle=dotted,linecolor=gray]{n34}{n96}
\ncline[linestyle=dotted,linecolor=gray]{n34}{n81}
\ncline[linecolor=gray]{n35}{n36}
\ncline[linecolor=gray]{n35}{n44}
\ncline[linecolor=gray]{n17}{n35}
\ncline[linecolor=gray]{n35}{n47}
\ncline[linecolor=gray]{n17}{n36}
\ncline[linecolor=gray]{n35}{n36}
\ncline[linecolor=gray]{n36}{n44}
\ncline[linecolor=gray]{n12}{n36}
\ncline[linecolor=gray]{n36}{n50}
\ncline[linecolor=gray]{n36}{n72}
\ncline[linecolor=gray]{n37}{n127}
\ncline[linecolor=gray]{n37}{n88}
\ncline[linecolor=gray]{n10}{n37}
\ncline[linecolor=gray]{n11}{n37}
\ncline[linestyle=dotted,linecolor=gray]{n39}{n61}
\ncline[linecolor=gray]{n39}{n77}
\ncline[linecolor=gray]{n39}{n119}
\ncline[linecolor=gray]{n40}{n50}
\ncline[linecolor=gray]{n40}{n79}
\ncline[linecolor=gray]{n0}{n40}
\ncline[linecolor=gray]{n12}{n40}
\ncline[linecolor=gray]{n40}{n64}
\ncline[linecolor=gray]{n41}{n142}
\ncline[linecolor=gray]{n41}{n82}
\ncline[linecolor=gray]{n41}{n99}
\ncline[linecolor=gray]{n41}{n112}
\ncline[linecolor=gray]{n0}{n41}
\ncline[linecolor=gray]{n41}{n144}
\ncline[linecolor=gray]{n41}{n143}
\ncline[linecolor=gray]{n8}{n42}
\ncline[linecolor=gray]{n42}{n107}
\ncline[linecolor=gray]{n42}{n147}
\ncline[linecolor=gray]{n42}{n103}
\ncline[linecolor=gray]{n42}{n138}
\ncline[linecolor=gray]{n6}{n42}
\ncline[linecolor=gray]{n21}{n42}
\ncline[linecolor=gray]{n42}{n129}
\ncline[linecolor=gray]{n19}{n43}
\ncline[linecolor=gray]{n44}{n126}
\ncline[linecolor=gray]{n12}{n44}
\ncline[linecolor=gray]{n36}{n44}
\ncline[linecolor=gray]{n44}{n114}
\ncline[linecolor=gray]{n35}{n44}
\ncline[linecolor=gray]{n26}{n44}
\ncline[linecolor=gray]{n31}{n45}
\ncline[linecolor=gray]{n45}{n52}
\ncline[linecolor=gray]{n45}{n76}
\ncline[linecolor=gray]{n19}{n46}
\ncline[linecolor=gray]{n46}{n107}
\ncline[linecolor=gray]{n29}{n46}
\ncline[linecolor=gray]{n46}{n95}
\ncline[linecolor=gray]{n47}{n98}
\ncline[linecolor=gray]{n47}{n139}
\ncline[linecolor=gray]{n35}{n47}
\ncline[linecolor=gray]{n47}{n59}
\ncline[linecolor=gray]{n48}{n111}
\ncline[linecolor=gray]{n48}{n69}
\ncline[linecolor=gray]{n48}{n73}
\ncline[linecolor=gray]{n30}{n48}
\ncline[linecolor=gray]{n48}{n88}
\ncline[linecolor=gray]{n48}{n149}
\ncline[linecolor=gray]{n48}{n145}
\ncline[linecolor=gray]{n48}{n54}
\ncline[linecolor=gray]{n49}{n88}
\ncline[linecolor=gray]{n49}{n135}
\ncline[linecolor=gray]{n33}{n50}
\ncline[linecolor=gray]{n36}{n50}
\ncline[linecolor=gray]{n40}{n50}
\ncline[linecolor=gray]{n50}{n114}
\ncline[linecolor=gray]{n50}{n146}
\ncline[linecolor=gray]{n50}{n72}
\ncline[linecolor=gray]{n32}{n51}
\ncline[linecolor=gray]{n20}{n51}
\ncline[linecolor=gray]{n5}{n52}
\ncline[linecolor=gray]{n45}{n52}
\ncline[linecolor=gray]{n52}{n78}
\ncline[linecolor=gray]{n53}{n55}
\ncline[linecolor=gray]{n54}{n136}
\ncline[linecolor=gray]{n54}{n135}
\ncline[linecolor=gray]{n54}{n149}
\ncline[linecolor=gray]{n48}{n54}
\ncline[linecolor=gray]{n54}{n63}
\ncline[linecolor=gray]{n55}{n92}
\ncline[linecolor=gray]{n53}{n55}
\ncline[linecolor=gray]{n55}{n87}
\ncline[linecolor=gray]{n56}{n147}
\ncline[linecolor=gray]{n56}{n141}
\ncline[linecolor=gray]{n56}{n107}
\ncline[linecolor=gray]{n8}{n56}
\ncline[linecolor=gray]{n56}{n103}
\ncline[linecolor=gray]{n58}{n79}
\ncline[linecolor=gray]{n58}{n142}
\ncline[linecolor=gray]{n58}{n126}
\ncline[linecolor=gray]{n59}{n114}
\ncline[linecolor=gray]{n47}{n59}
\ncline[linecolor=gray]{n21}{n61}
\ncline[linecolor=gray]{n61}{n129}
\ncline[linestyle=dotted,linecolor=gray]{n39}{n61}
\ncline[linecolor=gray]{n61}{n83}
\ncline[linecolor=gray]{n61}{n138}
\ncline[linestyle=dotted,linecolor=gray]{n61}{n89}
\ncline[linecolor=gray]{n61}{n131}
\ncline[linecolor=gray]{n61}{n118}
\ncline[linecolor=gray]{n31}{n62}
\ncline[linecolor=gray]{n15}{n62}
\ncline[linecolor=gray]{n11}{n62}
\ncline[linecolor=gray]{n25}{n62}
\ncline[linecolor=gray]{n54}{n63}
\ncline[linecolor=gray]{n63}{n145}
\ncline[linecolor=gray]{n63}{n121}
\ncline[linecolor=gray]{n0}{n64}
\ncline[linecolor=gray]{n64}{n114}
\ncline[linecolor=gray]{n40}{n64}
\ncline[linecolor=gray]{n64}{n126}
\ncline[linecolor=gray]{n65}{n126}
\ncline[linecolor=gray]{n16}{n67}
\ncline[linecolor=gray]{n31}{n67}
\ncline[linecolor=gray]{n67}{n90}
\ncline[linecolor=gray]{n3}{n67}
\ncline[linecolor=gray]{n68}{n74}
\ncline[linecolor=gray]{n69}{n73}
\ncline[linecolor=gray]{n48}{n69}
\ncline[linecolor=gray]{n31}{n69}
\ncline[linecolor=gray]{n15}{n69}
\ncline[linecolor=gray]{n69}{n145}
\ncline[linecolor=gray]{n69}{n135}
\ncline[linecolor=gray]{n70}{n148}
\ncline[linecolor=gray]{n71}{n132}
\ncline[linecolor=gray]{n72}{n142}
\ncline[linecolor=gray]{n72}{n114}
\ncline[linecolor=gray]{n33}{n72}
\ncline[linecolor=gray]{n0}{n72}
\ncline[linecolor=gray]{n17}{n72}
\ncline[linecolor=gray]{n72}{n112}
\ncline[linecolor=gray]{n36}{n72}
\ncline[linecolor=gray]{n50}{n72}
\ncline[linecolor=gray]{n69}{n73}
\ncline[linecolor=gray]{n48}{n73}
\ncline[linecolor=gray]{n73}{n145}
\ncline[linecolor=gray]{n30}{n73}
\ncline[linecolor=gray]{n27}{n74}
\ncline[linecolor=gray]{n68}{n74}
\ncline[linecolor=gray]{n75}{n106}
\ncline[linecolor=gray]{n75}{n121}
\ncline[linecolor=gray]{n75}{n105}
\ncline[linecolor=gray]{n16}{n75}
\ncline[linecolor=gray]{n76}{n109}
\ncline[linecolor=gray]{n3}{n76}
\ncline[linecolor=gray]{n45}{n76}
\ncline[linecolor=gray]{n76}{n78}
\ncline[linecolor=gray]{n39}{n77}
\ncline[linecolor=gray]{n77}{n123}
\ncline[linecolor=gray]{n77}{n117}
\ncline[linecolor=gray]{n77}{n101}
\ncline[linecolor=gray]{n77}{n119}
\ncline[linecolor=gray]{n31}{n78}
\ncline[linecolor=gray]{n52}{n78}
\ncline[linecolor=gray]{n76}{n78}
\ncline[linecolor=gray]{n0}{n79}
\ncline[linecolor=gray]{n79}{n142}
\ncline[linecolor=gray]{n79}{n126}
\ncline[linecolor=gray]{n58}{n79}
\ncline[linecolor=gray]{n40}{n79}
\ncline[linecolor=gray]{n79}{n144}
\ncline[linecolor=gray]{n81}{n85}
\ncline[linestyle=dotted,linecolor=gray]{n34}{n81}
\ncline[linecolor=gray]{n82}{n104}
\ncline[linecolor=gray]{n82}{n122}
\ncline[linecolor=gray]{n41}{n82}
\ncline[linecolor=gray]{n82}{n112}
\ncline[linecolor=gray]{n2}{n82}
\ncline[linecolor=gray]{n82}{n143}
\ncline[linecolor=gray]{n82}{n137}
\ncline[linecolor=gray]{n82}{n142}
\ncline[linecolor=gray]{n83}{n129}
\ncline[linecolor=gray]{n61}{n83}
\ncline[linecolor=gray]{n83}{n131}
\ncline[linecolor=gray]{n84}{n122}
\ncline[linestyle=dotted,linecolor=gray]{n21}{n84}
\ncline[linecolor=gray]{n84}{n100}
\ncline[linestyle=dotted,linecolor=gray]{n84}{n102}
\ncline[linecolor=gray]{n81}{n85}
\ncline[linecolor=gray]{n85}{n96}
\ncline[linecolor=gray]{n20}{n86}
\ncline[linecolor=gray]{n86}{n137}
\ncline[linecolor=gray]{n87}{n103}
\ncline[linecolor=gray]{n55}{n87}
\ncline[linecolor=gray]{n49}{n88}
\ncline[linecolor=gray]{n88}{n135}
\ncline[linecolor=gray]{n48}{n88}
\ncline[linecolor=gray]{n37}{n88}
\ncline[linecolor=gray]{n11}{n88}
\ncline[linecolor=gray]{n88}{n127}
\ncline[linecolor=gray]{n30}{n88}
\ncline[linecolor=gray]{n88}{n130}
\ncline[linecolor=gray]{n13}{n89}
\ncline[linecolor=gray]{n89}{n119}
\ncline[linecolor=gray]{n89}{n117}
\ncline[linecolor=gray]{n89}{n123}
\ncline[linestyle=dotted,linecolor=gray]{n61}{n89}
\ncline[linecolor=gray]{n89}{n144}
\ncline[linecolor=gray]{n3}{n90}
\ncline[linecolor=gray]{n14}{n90}
\ncline[linecolor=gray]{n67}{n90}
\ncline[linecolor=gray]{n90}{n134}
\ncline[linecolor=gray]{n90}{n111}
\ncline[linecolor=gray]{n27}{n90}
\ncline[linecolor=gray]{n90}{n124}
\ncline[linecolor=gray]{n92}{n102}
\ncline[linecolor=gray]{n92}{n147}
\ncline[linecolor=gray]{n21}{n92}
\ncline[linecolor=gray]{n55}{n92}
\ncline[linecolor=gray]{n8}{n92}
\ncline[linecolor=gray]{n94}{n119}
\ncline[linecolor=gray]{n46}{n95}
\ncline[linestyle=dotted,linecolor=gray]{n34}{n96}
\ncline[linecolor=gray]{n85}{n96}
\ncline[linecolor=gray]{n97}{n122}
\ncline[linecolor=gray]{n97}{n104}
\ncline[linecolor=gray]{n97}{n100}
\ncline[linecolor=gray]{n47}{n98}
\ncline[linecolor=gray]{n23}{n98}
\ncline[linecolor=gray]{n26}{n98}
\ncline[linecolor=gray]{n99}{n137}
\ncline[linecolor=gray]{n41}{n99}
\ncline[linecolor=gray]{n13}{n99}
\ncline[linecolor=gray]{n99}{n122}
\ncline[linecolor=gray]{n100}{n122}
\ncline[linecolor=gray]{n1}{n100}
\ncline[linecolor=gray]{n100}{n143}
\ncline[linecolor=gray]{n84}{n100}
\ncline[linecolor=gray]{n97}{n100}
\ncline[linecolor=gray]{n101}{n142}
\ncline[linecolor=gray]{n101}{n144}
\ncline[linecolor=gray]{n77}{n101}
\ncline[linecolor=gray]{n101}{n143}
\ncline[linecolor=gray]{n13}{n101}
\ncline[linecolor=gray]{n92}{n102}
\ncline[linestyle=dotted,linecolor=gray]{n84}{n102}
\ncline[linecolor=gray]{n102}{n147}
\ncline[linecolor=gray]{n8}{n103}
\ncline[linecolor=gray]{n42}{n103}
\ncline[linecolor=gray]{n6}{n103}
\ncline[linecolor=gray]{n87}{n103}
\ncline[linecolor=gray]{n56}{n103}
\ncline[linecolor=gray]{n82}{n104}
\ncline[linecolor=gray]{n104}{n112}
\ncline[linecolor=gray]{n104}{n143}
\ncline[linecolor=gray]{n97}{n104}
\ncline[linecolor=gray]{n75}{n105}
\ncline[linecolor=gray]{n15}{n105}
\ncline[linecolor=gray]{n27}{n106}
\ncline[linecolor=gray]{n75}{n106}
\ncline[linecolor=gray]{n3}{n106}
\ncline[linecolor=gray]{n42}{n107}
\ncline[linecolor=gray]{n56}{n107}
\ncline[linecolor=gray]{n46}{n107}
\ncline[linecolor=gray]{n76}{n109}
\ncline[linecolor=gray]{n23}{n110}
\ncline[linecolor=gray]{n48}{n111}
\ncline[linecolor=gray]{n15}{n111}
\ncline[linecolor=gray]{n90}{n111}
\ncline[linecolor=gray]{n111}{n135}
\ncline[linecolor=gray]{n2}{n112}
\ncline[linecolor=gray]{n112}{n143}
\ncline[linecolor=gray]{n104}{n112}
\ncline[linecolor=gray]{n82}{n112}
\ncline[linecolor=gray]{n72}{n112}
\ncline[linecolor=gray]{n41}{n112}
\ncline[linecolor=gray]{n11}{n113}
\ncline[linecolor=gray]{n113}{n127}
\ncline[linecolor=gray]{n72}{n114}
\ncline[linecolor=gray]{n44}{n114}
\ncline[linecolor=gray]{n23}{n114}
\ncline[linecolor=gray]{n50}{n114}
\ncline[linecolor=gray]{n59}{n114}
\ncline[linecolor=gray]{n64}{n114}
\ncline[linecolor=gray]{n12}{n114}
\ncline[linecolor=gray]{n114}{n139}
\ncline[linecolor=gray]{n115}{n130}
\ncline[linecolor=gray]{n22}{n115}
\ncline[linecolor=gray]{n30}{n115}
\ncline[linecolor=gray]{n115}{n133}
\ncline[linecolor=gray]{n117}{n123}
\ncline[linecolor=gray]{n117}{n119}
\ncline[linecolor=gray]{n89}{n117}
\ncline[linecolor=gray]{n77}{n117}
\ncline[linecolor=gray]{n13}{n117}
\ncline[linecolor=gray]{n21}{n118}
\ncline[linecolor=gray]{n118}{n129}
\ncline[linecolor=gray]{n61}{n118}
\ncline[linecolor=gray]{n117}{n119}
\ncline[linecolor=gray]{n89}{n119}
\ncline[linecolor=gray]{n39}{n119}
\ncline[linecolor=gray]{n119}{n123}
\ncline[linecolor=gray]{n94}{n119}
\ncline[linecolor=gray]{n77}{n119}
\ncline[linecolor=gray]{n20}{n120}
\ncline[linecolor=gray]{n120}{n123}
\ncline[linecolor=gray]{n75}{n121}
\ncline[linecolor=gray]{n63}{n121}
\ncline[linecolor=gray]{n97}{n122}
\ncline[linecolor=gray]{n100}{n122}
\ncline[linecolor=gray]{n122}{n128}
\ncline[linecolor=gray]{n82}{n122}
\ncline[linecolor=gray]{n84}{n122}
\ncline[linecolor=gray]{n122}{n143}
\ncline[linecolor=gray]{n99}{n122}
\ncline[linecolor=gray]{n117}{n123}
\ncline[linecolor=gray]{n7}{n123}
\ncline[linecolor=gray]{n120}{n123}
\ncline[linecolor=gray]{n20}{n123}
\ncline[linecolor=gray]{n89}{n123}
\ncline[linecolor=gray]{n13}{n123}
\ncline[linecolor=gray]{n119}{n123}
\ncline[linecolor=gray]{n77}{n123}
\ncline[linecolor=gray]{n11}{n124}
\ncline[linecolor=gray]{n124}{n127}
\ncline[linecolor=gray]{n31}{n124}
\ncline[linecolor=gray]{n90}{n124}
\ncline[linecolor=gray]{n5}{n125}
\ncline[linecolor=gray]{n16}{n125}
\ncline[linecolor=gray]{n15}{n125}
\ncline[linecolor=gray]{n44}{n126}
\ncline[linecolor=gray]{n12}{n126}
\ncline[linecolor=gray]{n79}{n126}
\ncline[linecolor=gray]{n17}{n126}
\ncline[linecolor=gray]{n65}{n126}
\ncline[linecolor=gray]{n58}{n126}
\ncline[linecolor=gray]{n0}{n126}
\ncline[linecolor=gray]{n64}{n126}
\ncline[linecolor=gray]{n113}{n127}
\ncline[linecolor=gray]{n127}{n135}
\ncline[linecolor=gray]{n127}{n130}
\ncline[linecolor=gray]{n37}{n127}
\ncline[linecolor=gray]{n124}{n127}
\ncline[linecolor=gray]{n127}{n136}
\ncline[linecolor=gray]{n88}{n127}
\ncline[linecolor=gray]{n122}{n128}
\ncline[linecolor=gray]{n128}{n142}
\ncline[linecolor=gray]{n83}{n129}
\ncline[linecolor=gray]{n61}{n129}
\ncline[linecolor=gray]{n129}{n138}
\ncline[linecolor=gray]{n118}{n129}
\ncline[linecolor=gray]{n42}{n129}
\ncline[linecolor=gray]{n115}{n130}
\ncline[linecolor=gray]{n127}{n130}
\ncline[linecolor=gray]{n11}{n130}
\ncline[linecolor=gray]{n30}{n130}
\ncline[linecolor=gray]{n88}{n130}
\ncline[linecolor=gray]{n83}{n131}
\ncline[linecolor=gray]{n61}{n131}
\ncline[linecolor=gray]{n71}{n132}
\ncline[linecolor=gray]{n115}{n133}
\ncline[linecolor=gray]{n16}{n134}
\ncline[linecolor=gray]{n31}{n134}
\ncline[linecolor=gray]{n90}{n134}
\ncline[linecolor=gray]{n3}{n134}
\ncline[linecolor=gray]{n11}{n135}
\ncline[linecolor=gray]{n127}{n135}
\ncline[linecolor=gray]{n30}{n135}
\ncline[linecolor=gray]{n88}{n135}
\ncline[linecolor=gray]{n54}{n135}
\ncline[linecolor=gray]{n135}{n145}
\ncline[linecolor=gray]{n10}{n135}
\ncline[linecolor=gray]{n69}{n135}
\ncline[linecolor=gray]{n49}{n135}
\ncline[linecolor=gray]{n111}{n135}
\ncline[linecolor=gray]{n54}{n136}
\ncline[linecolor=gray]{n25}{n136}
\ncline[linecolor=gray]{n127}{n136}
\ncline[linecolor=gray]{n99}{n137}
\ncline[linecolor=gray]{n86}{n137}
\ncline[linecolor=gray]{n82}{n137}
\ncline[linecolor=gray]{n21}{n138}
\ncline[linecolor=gray]{n61}{n138}
\ncline[linecolor=gray]{n129}{n138}
\ncline[linecolor=gray]{n42}{n138}
\ncline[linecolor=gray]{n33}{n139}
\ncline[linecolor=gray]{n47}{n139}
\ncline[linecolor=gray]{n114}{n139}
\ncline[linecolor=gray]{n56}{n141}
\ncline[linecolor=gray]{n72}{n142}
\ncline[linecolor=gray]{n41}{n142}
\ncline[linecolor=gray]{n79}{n142}
\ncline[linecolor=gray]{n0}{n142}
\ncline[linecolor=gray]{n101}{n142}
\ncline[linecolor=gray]{n128}{n142}
\ncline[linecolor=gray]{n58}{n142}
\ncline[linecolor=gray]{n82}{n142}
\ncline[linecolor=gray]{n100}{n143}
\ncline[linecolor=gray]{n122}{n143}
\ncline[linecolor=gray]{n112}{n143}
\ncline[linecolor=gray]{n104}{n143}
\ncline[linecolor=gray]{n82}{n143}
\ncline[linecolor=gray]{n101}{n143}
\ncline[linecolor=gray]{n41}{n143}
\ncline[linecolor=gray]{n79}{n144}
\ncline[linecolor=gray]{n89}{n144}
\ncline[linecolor=gray]{n101}{n144}
\ncline[linecolor=gray]{n41}{n144}
\ncline[linecolor=gray]{n69}{n145}
\ncline[linecolor=gray]{n135}{n145}
\ncline[linecolor=gray]{n145}{n149}
\ncline[linecolor=gray]{n48}{n145}
\ncline[linecolor=gray]{n73}{n145}
\ncline[linecolor=gray]{n63}{n145}
\ncline[linecolor=gray]{n50}{n146}
\ncline[linecolor=gray]{n23}{n146}
\ncline[linecolor=gray]{n17}{n146}
\ncline[linecolor=gray]{n21}{n147}
\ncline[linecolor=gray]{n56}{n147}
\ncline[linecolor=gray]{n42}{n147}
\ncline[linecolor=gray]{n92}{n147}
\ncline[linecolor=gray]{n8}{n147}
\ncline[linecolor=gray]{n102}{n147}
\ncline[linecolor=gray]{n70}{n148}
\ncline[linecolor=gray]{n0}{n50}
\ncline[linecolor=gray]{n1}{n97}
\ncline[linecolor=gray]{n1}{n122}
\ncline[linecolor=gray]{n1}{n84}
\ncline[linecolor=gray]{n2}{n128}
\ncline[linecolor=gray]{n2}{n143}
\ncline[linecolor=gray]{n2}{n104}
\ncline[linecolor=gray]{n2}{n100}
\ncline[linecolor=gray]{n2}{n122}
\ncline[linecolor=gray]{n3}{n5}
\ncline[linecolor=gray]{n3}{n27}
\ncline[linecolor=gray]{n3}{n91}
\ncline[linecolor=gray]{n3}{n74}
\ncline[linecolor=gray]{n3}{n75}
\ncline[linecolor=gray]{n3}{n52}
\ncline[linecolor=gray]{n3}{n45}
\ncline[linecolor=gray]{n3}{n78}
\ncline[linecolor=gray]{n3}{n125}
\ncline[linecolor=gray]{n3}{n5}
\ncline[linecolor=gray]{n5}{n27}
\ncline[linecolor=gray]{n5}{n45}
\ncline[linecolor=gray]{n5}{n74}
\ncline[linecolor=gray]{n5}{n78}
\ncline[linecolor=gray]{n5}{n91}
\ncline[linecolor=gray]{n5}{n75}
\ncline[linecolor=gray]{n5}{n106}
\ncline[linecolor=gray]{n5}{n14}
\ncline[linecolor=gray]{n5}{n31}
\ncline[linecolor=gray]{n5}{n67}
\ncline[linecolor=gray]{n5}{n134}
\ncline[linecolor=gray]{n6}{n131}
\ncline[linecolor=gray]{n6}{n129}
\ncline[linecolor=gray]{n7}{n120}
\ncline[linecolor=gray]{n7}{n32}
\ncline[linecolor=gray]{n7}{n51}
\ncline[linecolor=gray]{n8}{n87}
\ncline[linecolor=gray]{n8}{n107}
\ncline[linecolor=gray]{n10}{n11}
\ncline[linecolor=gray]{n10}{n127}
\ncline[linecolor=gray]{n10}{n136}
\ncline[linecolor=gray]{n10}{n111}
\ncline[linecolor=gray]{n10}{n48}
\ncline[linecolor=gray]{n10}{n113}
\ncline[linecolor=gray]{n10}{n124}
\ncline[linecolor=gray]{n10}{n25}
\ncline[linecolor=gray]{n10}{n62}
\ncline[linecolor=gray]{n10}{n11}
\ncline[linecolor=gray]{n11}{n127}
\ncline[linecolor=gray]{n11}{n48}
\ncline[linecolor=gray]{n11}{n111}
\ncline[linecolor=gray]{n11}{n136}
\ncline[linecolor=gray]{n12}{n50}
\ncline[linecolor=gray]{n12}{n64}
\ncline[linecolor=gray]{n12}{n17}
\ncline[linecolor=gray]{n13}{n144}
\ncline[linecolor=gray]{n13}{n77}
\ncline[linecolor=gray]{n13}{n20}
\ncline[linecolor=gray]{n13}{n86}
\ncline[linecolor=gray]{n14}{n67}
\ncline[linecolor=gray]{n14}{n134}
\ncline[linecolor=gray]{n14}{n45}
\ncline[linecolor=gray]{n14}{n106}
\ncline[linecolor=gray]{n14}{n125}
\ncline[linecolor=gray]{n14}{n121}
\ncline[linecolor=gray]{n14}{n18}
\ncline[linecolor=gray]{n14}{n75}
\ncline[linecolor=gray]{n14}{n78}
\ncline[linecolor=gray]{n5}{n14}
\ncline[linecolor=gray]{n14}{n27}
\ncline[linecolor=gray]{n15}{n63}
\ncline[linecolor=gray]{n15}{n16}
\ncline[linecolor=gray]{n15}{n25}
\ncline[linecolor=gray]{n15}{n149}
\ncline[linecolor=gray]{n15}{n136}
\ncline[linecolor=gray]{n15}{n90}
\ncline[linecolor=gray]{n15}{n18}
\ncline[linecolor=gray]{n15}{n124}
\ncline[linecolor=gray]{n15}{n113}
\ncline[linecolor=gray]{n16}{n149}
\ncline[linecolor=gray]{n16}{n18}
\ncline[linecolor=gray]{n16}{n124}
\ncline[linecolor=gray]{n16}{n136}
\ncline[linecolor=gray]{n15}{n16}
\ncline[linecolor=gray]{n16}{n69}
\ncline[linecolor=gray]{n16}{n90}
\ncline[linecolor=gray]{n16}{n111}
\ncline[linecolor=gray]{n16}{n113}
\ncline[linecolor=gray]{n16}{n62}
\ncline[linecolor=gray]{n17}{n50}
\ncline[linecolor=gray]{n17}{n44}
\ncline[linecolor=gray]{n17}{n64}
\ncline[linecolor=gray]{n12}{n17}
\ncline[linecolor=gray]{n17}{n33}
\ncline[linecolor=gray]{n17}{n114}
\ncline[linecolor=gray]{n16}{n18}
\ncline[linecolor=gray]{n18}{n149}
\ncline[linecolor=gray]{n18}{n31}
\ncline[linecolor=gray]{n18}{n69}
\ncline[linecolor=gray]{n18}{n124}
\ncline[linecolor=gray]{n18}{n62}
\ncline[linecolor=gray]{n18}{n136}
\ncline[linecolor=gray]{n14}{n18}
\ncline[linecolor=gray]{n18}{n67}
\ncline[linecolor=gray]{n18}{n134}
\ncline[linecolor=gray]{n15}{n18}
\ncline[linecolor=gray]{n18}{n90}
\ncline[linecolor=gray]{n18}{n111}
\ncline[linecolor=gray]{n18}{n113}
\ncline[linecolor=gray]{n18}{n106}
\ncline[linecolor=gray]{n13}{n20}
\ncline[linecolor=gray]{n20}{n144}
\ncline[linecolor=gray]{n21}{n102}
\ncline[linecolor=gray]{n23}{n26}
\ncline[linecolor=gray]{n23}{n35}
\ncline[linecolor=gray]{n23}{n44}
\ncline[linecolor=gray]{n23}{n139}
\ncline[linecolor=gray]{n15}{n25}
\ncline[linecolor=gray]{n25}{n63}
\ncline[linecolor=gray]{n10}{n25}
\ncline[linecolor=gray]{n23}{n26}
\ncline[linecolor=gray]{n26}{n139}
\ncline[linecolor=gray]{n26}{n35}
\ncline[linecolor=gray]{n3}{n27}
\ncline[linecolor=gray]{n27}{n75}
\ncline[linecolor=gray]{n27}{n125}
\ncline[linecolor=gray]{n5}{n27}
\ncline[linecolor=gray]{n27}{n45}
\ncline[linecolor=gray]{n27}{n78}
\ncline[linecolor=gray]{n27}{n52}
\ncline[linecolor=gray]{n27}{n91}
\ncline[linecolor=gray]{n27}{n76}
\ncline[linecolor=gray]{n14}{n27}
\ncline[linecolor=gray]{n27}{n67}
\ncline[linecolor=gray]{n27}{n134}
\ncline[linecolor=gray]{n29}{n107}
\ncline[linecolor=gray]{n29}{n103}
\ncline[linecolor=gray]{n29}{n87}
\ncline[linecolor=gray]{n18}{n31}
\ncline[linecolor=gray]{n31}{n106}
\ncline[linecolor=gray]{n5}{n31}
\ncline[linecolor=gray]{n7}{n32}
\ncline[linecolor=gray]{n32}{n120}
\ncline[linecolor=gray]{n33}{n114}
\ncline[linecolor=gray]{n33}{n36}
\ncline[linecolor=gray]{n17}{n33}
\ncline[linecolor=gray]{n33}{n35}
\ncline[linecolor=gray]{n33}{n44}
\ncline[linecolor=gray]{n35}{n139}
\ncline[linecolor=gray]{n23}{n35}
\ncline[linecolor=gray]{n35}{n98}
\ncline[linecolor=gray]{n33}{n35}
\ncline[linecolor=gray]{n35}{n114}
\ncline[linecolor=gray]{n26}{n35}
\ncline[linecolor=gray]{n33}{n36}
\ncline[linecolor=gray]{n36}{n114}
\ncline[linecolor=gray]{n38}{n93}
\ncline[linecolor=gray]{n39}{n89}
\ncline[linecolor=gray]{n40}{n65}
\ncline[linecolor=gray]{n40}{n126}
\ncline[linecolor=gray]{n41}{n101}
\ncline[linecolor=gray]{n41}{n79}
\ncline[linecolor=gray]{n41}{n58}
\ncline[linecolor=gray]{n42}{n56}
\ncline[linecolor=gray]{n43}{n95}
\ncline[linecolor=gray]{n17}{n44}
\ncline[linecolor=gray]{n44}{n50}
\ncline[linecolor=gray]{n44}{n146}
\ncline[linecolor=gray]{n23}{n44}
\ncline[linecolor=gray]{n44}{n64}
\ncline[linecolor=gray]{n33}{n44}
\ncline[linecolor=gray]{n45}{n106}
\ncline[linecolor=gray]{n14}{n45}
\ncline[linecolor=gray]{n45}{n67}
\ncline[linecolor=gray]{n45}{n134}
\ncline[linecolor=gray]{n45}{n78}
\ncline[linecolor=gray]{n45}{n125}
\ncline[linecolor=gray]{n5}{n45}
\ncline[linecolor=gray]{n27}{n45}
\ncline[linecolor=gray]{n3}{n45}
\ncline[linecolor=gray]{n45}{n121}
\ncline[linecolor=gray]{n45}{n75}
\ncline[linecolor=gray]{n48}{n135}
\ncline[linecolor=gray]{n48}{n113}
\ncline[linecolor=gray]{n11}{n48}
\ncline[linecolor=gray]{n48}{n127}
\ncline[linecolor=gray]{n10}{n48}
\ncline[linecolor=gray]{n48}{n124}
\ncline[linecolor=gray]{n48}{n136}
\ncline[linecolor=gray]{n12}{n50}
\ncline[linecolor=gray]{n17}{n50}
\ncline[linecolor=gray]{n50}{n64}
\ncline[linecolor=gray]{n44}{n50}
\ncline[linecolor=gray]{n0}{n50}
\ncline[linecolor=gray]{n50}{n126}
\ncline[linecolor=gray]{n7}{n51}
\ncline[linecolor=gray]{n51}{n120}
\ncline[linecolor=gray]{n52}{n76}
\ncline[linecolor=gray]{n52}{n91}
\ncline[linecolor=gray]{n52}{n109}
\ncline[linecolor=gray]{n3}{n52}
\ncline[linecolor=gray]{n27}{n52}
\ncline[linecolor=gray]{n54}{n145}
\ncline[linecolor=gray]{n54}{n69}
\ncline[linecolor=gray]{n54}{n113}
\ncline[linecolor=gray]{n54}{n73}
\ncline[linecolor=gray]{n55}{n80}
\ncline[linecolor=gray]{n42}{n56}
\ncline[linecolor=gray]{n58}{n101}
\ncline[linecolor=gray]{n41}{n58}
\ncline[linecolor=gray]{n62}{n136}
\ncline[linecolor=gray]{n62}{n124}
\ncline[linecolor=gray]{n62}{n111}
\ncline[linecolor=gray]{n16}{n62}
\ncline[linecolor=gray]{n18}{n62}
\ncline[linecolor=gray]{n62}{n69}
\ncline[linecolor=gray]{n62}{n149}
\ncline[linecolor=gray]{n62}{n113}
\ncline[linecolor=gray]{n10}{n62}
\ncline[linecolor=gray]{n15}{n63}
\ncline[linecolor=gray]{n25}{n63}
\ncline[linecolor=gray]{n12}{n64}
\ncline[linecolor=gray]{n50}{n64}
\ncline[linecolor=gray]{n17}{n64}
\ncline[linecolor=gray]{n44}{n64}
\ncline[linecolor=gray]{n40}{n65}
\ncline[linecolor=gray]{n14}{n67}
\ncline[linecolor=gray]{n67}{n134}
\ncline[linecolor=gray]{n45}{n67}
\ncline[linecolor=gray]{n67}{n106}
\ncline[linecolor=gray]{n67}{n125}
\ncline[linecolor=gray]{n67}{n121}
\ncline[linecolor=gray]{n18}{n67}
\ncline[linecolor=gray]{n67}{n75}
\ncline[linecolor=gray]{n67}{n78}
\ncline[linecolor=gray]{n5}{n67}
\ncline[linecolor=gray]{n27}{n67}
\ncline[linecolor=gray]{n69}{n113}
\ncline[linecolor=gray]{n69}{n124}
\ncline[linecolor=gray]{n69}{n149}
\ncline[linecolor=gray]{n69}{n136}
\ncline[linecolor=gray]{n16}{n69}
\ncline[linecolor=gray]{n18}{n69}
\ncline[linecolor=gray]{n62}{n69}
\ncline[linecolor=gray]{n69}{n111}
\ncline[linecolor=gray]{n54}{n69}
\ncline[linecolor=gray]{n73}{n135}
\ncline[linecolor=gray]{n54}{n73}
\ncline[linecolor=gray]{n3}{n74}
\ncline[linecolor=gray]{n5}{n74}
\ncline[linecolor=gray]{n74}{n90}
\ncline[linecolor=gray]{n74}{n91}
\ncline[linecolor=gray]{n74}{n75}
\ncline[linecolor=gray]{n75}{n125}
\ncline[linecolor=gray]{n27}{n75}
\ncline[linecolor=gray]{n75}{n90}
\ncline[linecolor=gray]{n3}{n75}
\ncline[linecolor=gray]{n14}{n75}
\ncline[linecolor=gray]{n67}{n75}
\ncline[linecolor=gray]{n75}{n134}
\ncline[linecolor=gray]{n5}{n75}
\ncline[linecolor=gray]{n45}{n75}
\ncline[linecolor=gray]{n74}{n75}
\ncline[linecolor=gray]{n75}{n78}
\ncline[linecolor=gray]{n52}{n76}
\ncline[linecolor=gray]{n27}{n76}
\ncline[linecolor=gray]{n76}{n91}
\ncline[linecolor=gray]{n77}{n89}
\ncline[linecolor=gray]{n13}{n77}
\ncline[linecolor=gray]{n77}{n144}
\ncline[linecolor=gray]{n78}{n106}
\ncline[linecolor=gray]{n45}{n78}
\ncline[linecolor=gray]{n78}{n125}
\ncline[linecolor=gray]{n5}{n78}
\ncline[linecolor=gray]{n14}{n78}
\ncline[linecolor=gray]{n27}{n78}
\ncline[linecolor=gray]{n67}{n78}
\ncline[linecolor=gray]{n78}{n134}
\ncline[linecolor=gray]{n3}{n78}
\ncline[linecolor=gray]{n78}{n121}
\ncline[linecolor=gray]{n75}{n78}
\ncline[linecolor=gray]{n41}{n79}
\ncline[linecolor=gray]{n79}{n101}
\ncline[linecolor=gray]{n55}{n80}
\ncline[linecolor=gray]{n80}{n92}
\ncline[linecolor=gray]{n81}{n96}
\ncline[linecolor=gray]{n82}{n99}
\ncline[linecolor=gray]{n82}{n128}
\ncline[linecolor=gray]{n83}{n118}
\ncline[linecolor=gray]{n83}{n138}
\ncline[linecolor=gray]{n1}{n84}
\ncline[linecolor=gray]{n86}{n144}
\ncline[linecolor=gray]{n13}{n86}
\ncline[linecolor=gray]{n8}{n87}
\ncline[linecolor=gray]{n87}{n107}
\ncline[linecolor=gray]{n29}{n87}
\ncline[linecolor=gray]{n77}{n89}
\ncline[linecolor=gray]{n39}{n89}
\ncline[linecolor=gray]{n75}{n90}
\ncline[linecolor=gray]{n16}{n90}
\ncline[linecolor=gray]{n15}{n90}
\ncline[linecolor=gray]{n90}{n105}
\ncline[linecolor=gray]{n90}{n125}
\ncline[linecolor=gray]{n18}{n90}
\ncline[linecolor=gray]{n74}{n90}
\ncline[linecolor=gray]{n90}{n149}
\ncline[linecolor=gray]{n52}{n91}
\ncline[linecolor=gray]{n3}{n91}
\ncline[linecolor=gray]{n27}{n91}
\ncline[linecolor=gray]{n5}{n91}
\ncline[linecolor=gray]{n74}{n91}
\ncline[linecolor=gray]{n76}{n91}
\ncline[linecolor=gray]{n91}{n109}
\ncline[linecolor=gray]{n80}{n92}
\ncline[linecolor=gray]{n38}{n93}
\ncline[linecolor=gray]{n43}{n95}
\ncline[linecolor=gray]{n81}{n96}
\ncline[linecolor=gray]{n1}{n97}
\ncline[linecolor=gray]{n98}{n139}
\ncline[linecolor=gray]{n35}{n98}
\ncline[linecolor=gray]{n82}{n99}
\ncline[linecolor=gray]{n2}{n100}
\ncline[linecolor=gray]{n41}{n101}
\ncline[linecolor=gray]{n58}{n101}
\ncline[linecolor=gray]{n79}{n101}
\ncline[linecolor=gray]{n21}{n102}
\ncline[linecolor=gray]{n103}{n107}
\ncline[linecolor=gray]{n29}{n103}
\ncline[linecolor=gray]{n104}{n128}
\ncline[linecolor=gray]{n2}{n104}
\ncline[linecolor=gray]{n104}{n122}
\ncline[linecolor=gray]{n90}{n105}
\ncline[linecolor=gray]{n45}{n106}
\ncline[linecolor=gray]{n78}{n106}
\ncline[linecolor=gray]{n14}{n106}
\ncline[linecolor=gray]{n67}{n106}
\ncline[linecolor=gray]{n106}{n134}
\ncline[linecolor=gray]{n106}{n121}
\ncline[linecolor=gray]{n106}{n125}
\ncline[linecolor=gray]{n31}{n106}
\ncline[linecolor=gray]{n5}{n106}
\ncline[linecolor=gray]{n18}{n106}
\ncline[linecolor=gray]{n29}{n107}
\ncline[linecolor=gray]{n8}{n107}
\ncline[linecolor=gray]{n103}{n107}
\ncline[linecolor=gray]{n87}{n107}
\ncline[linecolor=gray]{n52}{n109}
\ncline[linecolor=gray]{n91}{n109}
\ncline[linecolor=gray]{n111}{n124}
\ncline[linecolor=gray]{n62}{n111}
\ncline[linecolor=gray]{n111}{n136}
\ncline[linecolor=gray]{n16}{n111}
\ncline[linecolor=gray]{n111}{n113}
\ncline[linecolor=gray]{n111}{n149}
\ncline[linecolor=gray]{n11}{n111}
\ncline[linecolor=gray]{n10}{n111}
\ncline[linecolor=gray]{n69}{n111}
\ncline[linecolor=gray]{n111}{n127}
\ncline[linecolor=gray]{n18}{n111}
\ncline[linecolor=gray]{n112}{n142}
\ncline[linecolor=gray]{n112}{n128}
\ncline[linecolor=gray]{n69}{n113}
\ncline[linecolor=gray]{n113}{n124}
\ncline[linecolor=gray]{n113}{n149}
\ncline[linecolor=gray]{n113}{n136}
\ncline[linecolor=gray]{n16}{n113}
\ncline[linecolor=gray]{n48}{n113}
\ncline[linecolor=gray]{n111}{n113}
\ncline[linecolor=gray]{n113}{n135}
\ncline[linecolor=gray]{n10}{n113}
\ncline[linecolor=gray]{n18}{n113}
\ncline[linecolor=gray]{n54}{n113}
\ncline[linecolor=gray]{n62}{n113}
\ncline[linecolor=gray]{n113}{n145}
\ncline[linecolor=gray]{n15}{n113}
\ncline[linecolor=gray]{n33}{n114}
\ncline[linecolor=gray]{n17}{n114}
\ncline[linecolor=gray]{n36}{n114}
\ncline[linecolor=gray]{n35}{n114}
\ncline[linecolor=gray]{n116}{n132}
\ncline[linecolor=gray]{n118}{n138}
\ncline[linecolor=gray]{n83}{n118}
\ncline[linecolor=gray]{n7}{n120}
\ncline[linecolor=gray]{n32}{n120}
\ncline[linecolor=gray]{n51}{n120}
\ncline[linecolor=gray]{n14}{n121}
\ncline[linecolor=gray]{n67}{n121}
\ncline[linecolor=gray]{n121}{n134}
\ncline[linecolor=gray]{n106}{n121}
\ncline[linecolor=gray]{n121}{n125}
\ncline[linecolor=gray]{n45}{n121}
\ncline[linecolor=gray]{n78}{n121}
\ncline[linecolor=gray]{n1}{n122}
\ncline[linecolor=gray]{n104}{n122}
\ncline[linecolor=gray]{n2}{n122}
\ncline[linecolor=gray]{n124}{n136}
\ncline[linecolor=gray]{n111}{n124}
\ncline[linecolor=gray]{n69}{n124}
\ncline[linecolor=gray]{n113}{n124}
\ncline[linecolor=gray]{n124}{n149}
\ncline[linecolor=gray]{n62}{n124}
\ncline[linecolor=gray]{n16}{n124}
\ncline[linecolor=gray]{n18}{n124}
\ncline[linecolor=gray]{n10}{n124}
\ncline[linecolor=gray]{n48}{n124}
\ncline[linecolor=gray]{n15}{n124}
\ncline[linecolor=gray]{n75}{n125}
\ncline[linecolor=gray]{n14}{n125}
\ncline[linecolor=gray]{n67}{n125}
\ncline[linecolor=gray]{n125}{n134}
\ncline[linecolor=gray]{n27}{n125}
\ncline[linecolor=gray]{n121}{n125}
\ncline[linecolor=gray]{n45}{n125}
\ncline[linecolor=gray]{n78}{n125}
\ncline[linecolor=gray]{n106}{n125}
\ncline[linecolor=gray]{n90}{n125}
\ncline[linecolor=gray]{n3}{n125}
\ncline[linecolor=gray]{n40}{n126}
\ncline[linecolor=gray]{n50}{n126}
\ncline[linecolor=gray]{n126}{n146}
\ncline[linecolor=gray]{n11}{n127}
\ncline[linecolor=gray]{n10}{n127}
\ncline[linecolor=gray]{n48}{n127}
\ncline[linecolor=gray]{n111}{n127}
\ncline[linecolor=gray]{n104}{n128}
\ncline[linecolor=gray]{n2}{n128}
\ncline[linecolor=gray]{n112}{n128}
\ncline[linecolor=gray]{n82}{n128}
\ncline[linecolor=gray]{n128}{n143}
\ncline[linecolor=gray]{n129}{n131}
\ncline[linecolor=gray]{n6}{n129}
\ncline[linecolor=gray]{n129}{n131}
\ncline[linecolor=gray]{n6}{n131}
\ncline[linecolor=gray]{n116}{n132}
\ncline[linecolor=gray]{n14}{n134}
\ncline[linecolor=gray]{n67}{n134}
\ncline[linecolor=gray]{n45}{n134}
\ncline[linecolor=gray]{n106}{n134}
\ncline[linecolor=gray]{n125}{n134}
\ncline[linecolor=gray]{n121}{n134}
\ncline[linecolor=gray]{n18}{n134}
\ncline[linecolor=gray]{n75}{n134}
\ncline[linecolor=gray]{n78}{n134}
\ncline[linecolor=gray]{n5}{n134}
\ncline[linecolor=gray]{n27}{n134}
\ncline[linecolor=gray]{n48}{n135}
\ncline[linecolor=gray]{n113}{n135}
\ncline[linecolor=gray]{n73}{n135}
\ncline[linecolor=gray]{n124}{n136}
\ncline[linecolor=gray]{n62}{n136}
\ncline[linecolor=gray]{n69}{n136}
\ncline[linecolor=gray]{n111}{n136}
\ncline[linecolor=gray]{n113}{n136}
\ncline[linecolor=gray]{n136}{n149}
\ncline[linecolor=gray]{n16}{n136}
\ncline[linecolor=gray]{n10}{n136}
\ncline[linecolor=gray]{n18}{n136}
\ncline[linecolor=gray]{n15}{n136}
\ncline[linecolor=gray]{n11}{n136}
\ncline[linecolor=gray]{n48}{n136}
\ncline[linecolor=gray]{n118}{n138}
\ncline[linecolor=gray]{n83}{n138}
\ncline[linecolor=gray]{n98}{n139}
\ncline[linecolor=gray]{n35}{n139}
\ncline[linecolor=gray]{n23}{n139}
\ncline[linecolor=gray]{n26}{n139}
\ncline[linecolor=gray]{n141}{n147}
\ncline[linecolor=gray]{n112}{n142}
\ncline[linecolor=gray]{n2}{n143}
\ncline[linecolor=gray]{n128}{n143}
\ncline[linecolor=gray]{n13}{n144}
\ncline[linecolor=gray]{n86}{n144}
\ncline[linecolor=gray]{n77}{n144}
\ncline[linecolor=gray]{n20}{n144}
\ncline[linecolor=gray]{n54}{n145}
\ncline[linecolor=gray]{n113}{n145}
\ncline[linecolor=gray]{n44}{n146}
\ncline[linecolor=gray]{n126}{n146}
\ncline[linecolor=gray]{n141}{n147}
\ncline[linecolor=black]{->}{n0}{n17}
\ncline[linecolor=black]{->}{n1}{n17}
\ncline[linecolor=black]{->}{n2}{n17}
\ncline[linecolor=black]{->}{n3}{n27}
\ncline[linecolor=black]{->}{n4}{n17}
\ncline[linecolor=black]{->}{n5}{n27}
\ncline[linecolor=black]{->}{n7}{n17}
\ncline[linecolor=black]{->}{n8}{n6}
\ncline[linecolor=black]{->}{n9}{n6}
\ncline[linecolor=black]{->}{n10}{n27}
\ncline[linecolor=black]{->}{n11}{n27}
\ncline[linecolor=black]{->}{n12}{n17}
\ncline[linecolor=black]{->}{n13}{n17}
\ncline[linecolor=black]{->}{n14}{n27}
\ncline[linecolor=black]{->}{n15}{n27}
\ncline[linecolor=black]{->}{n16}{n27}
\ncline[linecolor=black]{->}{n18}{n27}
\ncline[linecolor=black]{->}{n19}{n6}
\ncline[linecolor=black]{->}{n20}{n17}
\ncline[linecolor=black]{->}{n21}{n6}
\ncline[linecolor=black]{->}{n22}{n27}
\ncline[linecolor=black]{->}{n23}{n17}
\ncline[linecolor=black]{->}{n24}{n17}
\ncline[linecolor=black]{->}{n25}{n27}
\ncline[linecolor=black]{->}{n26}{n17}
\ncline[linecolor=black]{->}{n28}{n6}
\ncline[linecolor=black]{->}{n29}{n6}
\ncline[linecolor=black]{->}{n30}{n27}
\ncline[linecolor=black]{->}{n31}{n27}
\ncline[linecolor=black]{->}{n32}{n17}
\ncline[linecolor=black]{->}{n33}{n17}
\ncline[linecolor=black]{->}{n34}{n27}
\ncline[linecolor=black]{->}{n35}{n17}
\ncline[linecolor=black]{->}{n36}{n17}
\ncline[linecolor=black]{->}{n37}{n27}
\ncline[linecolor=black]{->}{n38}{n17}
\ncline[linecolor=black]{->}{n39}{n17}
\ncline[linecolor=black]{->}{n40}{n17}
\ncline[linecolor=black]{->}{n41}{n17}
\ncline[linecolor=black]{->}{n42}{n6}
\ncline[linecolor=black]{->}{n43}{n6}
\ncline[linecolor=black]{->}{n44}{n17}
\ncline[linecolor=black]{->}{n45}{n27}
\ncline[linecolor=black]{->}{n46}{n6}
\ncline[linecolor=black]{->}{n47}{n17}
\ncline[linecolor=black]{->}{n48}{n27}
\ncline[linecolor=black]{->}{n49}{n27}
\ncline[linecolor=black]{->}{n50}{n17}
\ncline[linecolor=black]{->}{n51}{n17}
\ncline[linecolor=black]{->}{n52}{n27}
\ncline[linecolor=black]{->}{n53}{n6}
\ncline[linecolor=black]{->}{n54}{n27}
\ncline[linecolor=black]{->}{n55}{n6}
\ncline[linecolor=black]{->}{n56}{n6}
\ncline[linecolor=black]{->}{n57}{n6}
\ncline[linecolor=black]{->}{n58}{n17}
\ncline[linecolor=black]{->}{n59}{n17}
\ncline[linecolor=black]{->}{n60}{n6}
\ncline[linecolor=black]{->}{n61}{n6}
\ncline[linecolor=black]{->}{n62}{n27}
\ncline[linecolor=black]{->}{n63}{n27}
\ncline[linecolor=black]{->}{n64}{n17}
\ncline[linecolor=black]{->}{n65}{n17}
\ncline[linecolor=black]{->}{n66}{n17}
\ncline[linecolor=black]{->}{n67}{n27}
\ncline[linecolor=black]{->}{n68}{n27}
\ncline[linecolor=black]{->}{n69}{n27}
\ncline[linecolor=black]{->}{n70}{n6}
\ncline[linecolor=black]{->}{n71}{n6}
\ncline[linecolor=black]{->}{n72}{n17}
\ncline[linecolor=black]{->}{n73}{n27}
\ncline[linecolor=black]{->}{n74}{n27}
\ncline[linecolor=black]{->}{n75}{n27}
\ncline[linecolor=black]{->}{n76}{n27}
\ncline[linecolor=black]{->}{n77}{n17}
\ncline[linecolor=black]{->}{n78}{n27}
\ncline[linecolor=black]{->}{n79}{n17}
\ncline[linecolor=black]{->}{n80}{n6}
\ncline[linecolor=black]{->}{n81}{n17}
\ncline[linecolor=black]{->}{n82}{n17}
\ncline[linecolor=black]{->}{n83}{n6}
\ncline[linecolor=black]{->}{n84}{n17}
\ncline[linecolor=black]{->}{n85}{n17}
\ncline[linecolor=black]{->}{n86}{n17}
\ncline[linecolor=black]{->}{n87}{n6}
\ncline[linecolor=black]{->}{n88}{n27}
\ncline[linecolor=black]{->}{n89}{n17}
\ncline[linecolor=black]{->}{n90}{n27}
\ncline[linecolor=black]{->}{n91}{n27}
\ncline[linecolor=black]{->}{n92}{n6}
\ncline[linecolor=black]{->}{n93}{n17}
\ncline[linecolor=black]{->}{n94}{n17}
\ncline[linecolor=black]{->}{n95}{n6}
\ncline[linecolor=black]{->}{n96}{n17}
\ncline[linecolor=black]{->}{n97}{n17}
\ncline[linecolor=black]{->}{n98}{n17}
\ncline[linecolor=black]{->}{n99}{n17}
\ncline[linecolor=black]{->}{n100}{n17}
\ncline[linecolor=black]{->}{n101}{n17}
\ncline[linecolor=black]{->}{n102}{n6}
\ncline[linecolor=black]{->}{n103}{n6}
\ncline[linecolor=black]{->}{n104}{n17}
\ncline[linecolor=black]{->}{n105}{n27}
\ncline[linecolor=black]{->}{n106}{n27}
\ncline[linecolor=black]{->}{n107}{n6}
\ncline[linecolor=black]{->}{n108}{n27}
\ncline[linecolor=black]{->}{n109}{n27}
\ncline[linecolor=black]{->}{n110}{n17}
\ncline[linecolor=black]{->}{n111}{n27}
\ncline[linecolor=black]{->}{n112}{n17}
\ncline[linecolor=black]{->}{n113}{n27}
\ncline[linecolor=black]{->}{n114}{n17}
\ncline[linecolor=black]{->}{n115}{n27}
\ncline[linecolor=black]{->}{n116}{n6}
\ncline[linecolor=black]{->}{n117}{n17}
\ncline[linecolor=black]{->}{n118}{n6}
\ncline[linecolor=black]{->}{n119}{n17}
\ncline[linecolor=black]{->}{n120}{n17}
\ncline[linecolor=black]{->}{n121}{n27}
\ncline[linecolor=black]{->}{n122}{n17}
\ncline[linecolor=black]{->}{n123}{n17}
\ncline[linecolor=black]{->}{n124}{n27}
\ncline[linecolor=black]{->}{n125}{n27}
\ncline[linecolor=black]{->}{n126}{n17}
\ncline[linecolor=black]{->}{n127}{n27}
\ncline[linecolor=black]{->}{n128}{n17}
\ncline[linecolor=black]{->}{n129}{n6}
\ncline[linecolor=black]{->}{n130}{n27}
\ncline[linecolor=black]{->}{n131}{n6}
\ncline[linecolor=black]{->}{n132}{n6}
\ncline[linecolor=black]{->}{n133}{n27}
\ncline[linecolor=black]{->}{n134}{n27}
\ncline[linecolor=black]{->}{n135}{n27}
\ncline[linecolor=black]{->}{n136}{n27}
\ncline[linecolor=black]{->}{n137}{n17}
\ncline[linecolor=black]{->}{n138}{n6}
\ncline[linecolor=black]{->}{n139}{n17}
\ncline[linecolor=black]{->}{n140}{n6}
\ncline[linecolor=black]{->}{n141}{n6}
\ncline[linecolor=black]{->}{n142}{n17}
\ncline[linecolor=black]{->}{n143}{n17}
\ncline[linecolor=black]{->}{n144}{n17}
\ncline[linecolor=black]{->}{n145}{n27}
\ncline[linecolor=black]{->}{n146}{n17}
\ncline[linecolor=black]{->}{n147}{n6}
\ncline[linecolor=black]{->}{n148}{n6}
\ncline[linecolor=black]{->}{n149}{n27}
\psset{dotstyle=Bo}
\dotnode[](1.108338,3.858886){n0}
\dotnode[](2.527444,2.445348){n1}
\dotnode[](2.017666,2.841472){n2}
\dotnode[](1.257860,2.892203){n4}
\dotnode[](3.098514,4.866756){n7}
\dotnode[](0.3962556,3.984627){n12}
\dotnode[](2.661068,4.46547){n13}
\dotnode[fillcolor=gray](0.6005647,3.371056){n17}
\dotnode[](2.577236,4.740179){n20}
\dotnode[](0.711105,2.899006){n23}
\dotnode[](1.232538,4.29378){n24}
\dotnode[](0.9249148,2.893182){n26}
\dotnode[](2.557950,5){n32}
\dotnode[](0.1556468,2.998522){n33}
\dotnode[](0.3174471,2.600607){n35}
\dotnode[](0.1307679,3.277233){n36}
\dotnode[](2.55,3.97){n38}
\dotnode[](3.353207,4.403764){n39}
\dotnode[](0.9114814,4.353879){n40}
\dotnode[](1.686579,3.819467){n41}
\dotnode[](0.4179513,3.522051){n44}
\dotnode[](0.1748816,2.281055){n47}
\dotnode[](0.5841016,3.811024){n50}
\dotnode[](2.24852,4.942186){n51}
\dotnode[](1.967572,3.868347){n58}
\dotnode[](0,2.563469){n59}
\dotnode[](0.2217693,3.701035){n64}
\dotnode[](0.6466549,4.297944){n65}
\dotnode[](3.364475,4.745668){n66}
\dotnode[](1.004251,3.3599){n72}
\dotnode[](2.550006,4.241430){n77}
\dotnode[](1.560778,4.1743){n79}
\dotnode[](1.183687,2.360412){n81}
\dotnode[](2.078577,3.201014){n82}
\dotnode[](3.153883,3.140639){n84}
\dotnode[](0.6058247,2.060042){n85}
\dotnode[](1.891242,4.717356){n86}
\dotnode[](3.202631,4.133673){n89}
\dotnode[](2.64,3.705){n93}
\dotnode[](2.869689,4.205452){n94}
\dotnode[](0.9419618,2.164323){n96}
\dotnode[](2.802984,2.471939){n97}
\dotnode[](0.4990411,2.364772){n98}
\dotnode[](2.265986,3.781169){n99}
\dotnode[](2.886530,2.782621){n100}
\dotnode[](2.074548,4.120629){n101}
\dotnode[](2.161968,2.481101){n104}
\dotnode[](0.5000101,2.853756){n110}
\dotnode[](1.705716,3.114971){n112}
\dotnode[](0.3549037,3.121353){n114}
\dotnode[](3.053188,3.925351){n117}
\dotnode[](2.890681,3.735371){n119}
\dotnode[](2.880736,4.879838){n120}
\dotnode[](2.617372,2.90596){n122}
\dotnode[](3.046366,4.518936){n123}
\dotnode[](0.9849086,4.079775){n126}
\dotnode[](2.280932,2.838994){n128}
\dotnode[](1.689756,4.453583){n137}
\dotnode[](0.710692,2.532491){n139}
\dotnode[](1.606368,3.440222){n142}
\dotnode[](2.445606,3.296266){n143}
\dotnode[](2.274105,4.535248){n144}
\dotnode[](0.7976457,3.6483){n146}
\psset{dotstyle=Bsquare}
\dotnode[fillcolor=gray](4.581229,3.210751){n6}
\dotnode[](4.591683,2.878327){n8}
\dotnode[](4,2.2){n9}
\dotnode[](4.08826,1.970573){n19}
\dotnode[](3.773692,3.552994){n21}
\dotnode[](3.759983,2.172351){n28}
\dotnode[](4.7622,2.335772){n29}
\dotnode[](4.209552,3.016675){n42}
\dotnode[](3.891111,1.748183){n43}
\dotnode[](4.389942,2.082443){n46}
\dotnode[](4.969442,3.079323){n53}
\dotnode[](5,2.809783){n55}
\dotnode[](3.97574,2.571348){n56}
\dotnode[](4.371287,1.534671){n57}
\dotnode[](3.764707,4.352432){n60}
\dotnode[](3.737104,3.981957){n61}
\dotnode[](4.688611,1.662437){n70}
\dotnode[](4.32925,1.149353){n71}
\dotnode[](4.791386,3.470398){n80}
\dotnode[](4.135811,4.087185){n83}
\dotnode[](4.792091,2.606388){n87}
\dotnode[](4.292557,3.369169){n92}
\dotnode[](4.306428,1.798957){n95}
\dotnode[](3.578431,3.339056){n102}
\dotnode[](4.396008,2.731498){n103}
\dotnode[](4.351759,2.410157){n107}
\dotnode[](4.562201,1.223617){n116}
\dotnode[](4.318411,3.852927){n118}
\dotnode[](4.117188,3.717776){n129}
\dotnode[](4.500053,3.637197){n131}
\dotnode[](4.069486,1.411759){n132}
\dotnode[](3.989081,3.406712){n138}
\dotnode[](4.873979,1.934726){n140}
\dotnode[](3.645946,2.645275){n141}
\dotnode[](3.950723,2.918283){n147}
\dotnode[](4.776492,1.394465){n148}
\psset{dotstyle=Btriangle}
\dotnode[](0.6785566,0.8802483){n3}
\dotnode[](0.8247274,0.631479){n5}
\dotnode[](2.888154,1.090116){n10}
\dotnode[](3.105909,0.7401545){n11}
\dotnode[](1.439636,1.295117){n14}
\dotnode[](2.236885,1.274193){n15}
\dotnode[](2.024484,0.7152085){n16}
\dotnode[](1.814559,0.4981684){n18}
\dotnode[](3.913548,0.4282107){n22}
\dotnode[](2.521529,1.412281){n25}
\dotnode[fillcolor=gray](0.5726195,1.294175){n27}
\dotnode[](3.665883,0.6603324){n30}
\dotnode[](1.422528,0.3128956){n31}
\dotnode[](1.403207,1.916558){n34}
\dotnode[](2.901075,1.575363){n37}
\dotnode[](0.8306748,1.037910){n45}
\dotnode[](3.273419,0.6181322){n48}
\dotnode[](3.66405,0.2172084){n49}
\dotnode[](0.2341176,1.037886){n52}
\dotnode[](2.864019,0){n54}
\dotnode[](2.127025,0.3401537){n62}
\dotnode[](2.312403,1.030709){n63}
\dotnode[](1.509306,0.9905115){n67}
\dotnode[](1.631241,0.08123197){n68}
\dotnode[](2.363231,0.03982867){n69}
\dotnode[](3.170304,0.08031329){n73}
\dotnode[](1.092846,0.4743804){n74}
\dotnode[](0.957797,1.457980){n75}
\dotnode[](0.4145735,1.587721){n76}
\dotnode[](0.8983155,1.243353){n78}
\dotnode[](3.635776,1.016917){n88}
\dotnode[](1.765180,1.228386){n90}
\dotnode[](0.4301581,0.7624465){n91}
\dotnode[](1.915954,1.686363){n105}
\dotnode[](1.073082,0.7635667){n106}
\dotnode[](0.1632209,1.576404){n108}
\dotnode[](0.1964432,1.278737){n109}
\dotnode[](2.625767,0.8297721){n111}
\dotnode[](2.796314,0.5814426){n113}
\dotnode[](3.949152,0.8549694){n115}
\dotnode[](1.377059,0.7879707){n121}
\dotnode[](2.621703,0.4200536){n124}
\dotnode[](1.203850,1.499658){n125}
\dotnode[](3.254325,1.04113){n127}
\dotnode[](3.471599,1.285363){n130}
\dotnode[](4.11247,0.5910543){n133}
\dotnode[](1.252724,1.071603){n134}
\dotnode[](3.385955,0.3694225){n135}
\dotnode[](2.407195,0.4605674){n136}
\dotnode[](2.648264,0.1809270){n145}
\dotnode[](2.296867,0.7074797){n149}

            \end{pspicture}
        }}
        \hspace{0.1in}
        \subfloat[$r=2$]{
        \label{fig:clust:spectral:irisMCLb}
        \scalebox{0.55}{
            \psset{unit=0.75in,dotscale=2}
            \begin{pspicture}(5,5)
                \pnode(1.108338,3.858886){n0}
\pnode(2.527444,2.445348){n1}
\pnode(2.017666,2.841472){n2}
\pnode(0.6785566,0.8802483){n3}
\pnode(1.257860,2.892203){n4}
\pnode(0.8247274,0.631479){n5}
\pnode(4.581229,3.210751){n6}
\pnode(3.098514,4.866756){n7}
\pnode(4.591683,2.878327){n8}
\pnode(4,2.2){n9}
\pnode(2.888154,1.090116){n10}
\pnode(3.105909,0.7401545){n11}
\pnode(0.3962556,3.984627){n12}
\pnode(2.661068,4.46547){n13}
\pnode(1.439636,1.295117){n14}
\pnode(2.236885,1.274193){n15}
\pnode(2.024484,0.7152085){n16}
\pnode(0.6005647,3.371056){n17}
\pnode(1.814559,0.4981684){n18}
\pnode(4.08826,1.970573){n19}
\pnode(2.577236,4.740179){n20}
\pnode(3.773692,3.552994){n21}
\pnode(3.913548,0.4282107){n22}
\pnode(0.711105,2.899006){n23}
\pnode(1.232538,4.29378){n24}
\pnode(2.521529,1.412281){n25}
\pnode(0.9249148,2.893182){n26}
\pnode(0.5726195,1.294175){n27}
\pnode(3.759983,2.172351){n28}
\pnode(4.7622,2.335772){n29}
\pnode(3.665883,0.6603324){n30}
\pnode(1.422528,0.3128956){n31}
\pnode(2.557950,5){n32}
\pnode(0.1556468,2.998522){n33}
\pnode(1.403207,1.916558){n34}
\pnode(0.3174471,2.600607){n35}
\pnode(0.1307679,3.277233){n36}
\pnode(2.901075,1.575363){n37}
\pnode(2.55,3.97){n38}
\pnode(3.353207,4.403764){n39}
\pnode(0.9114814,4.353879){n40}
\pnode(1.686579,3.819467){n41}
\pnode(4.209552,3.016675){n42}
\pnode(3.891111,1.748183){n43}
\pnode(0.4179513,3.522051){n44}
\pnode(0.8306748,1.037910){n45}
\pnode(4.389942,2.082443){n46}
\pnode(0.1748816,2.281055){n47}
\pnode(3.273419,0.6181322){n48}
\pnode(3.66405,0.2172084){n49}
\pnode(0.5841016,3.811024){n50}
\pnode(2.24852,4.942186){n51}
\pnode(0.2341176,1.037886){n52}
\pnode(4.969442,3.079323){n53}
\pnode(2.864019,0){n54}
\pnode(5,2.809783){n55}
\pnode(3.97574,2.571348){n56}
\pnode(4.371287,1.534671){n57}
\pnode(1.967572,3.868347){n58}
\pnode(0,2.563469){n59}
\pnode(3.764707,4.352432){n60}
\pnode(3.737104,3.981957){n61}
\pnode(2.127025,0.3401537){n62}
\pnode(2.312403,1.030709){n63}
\pnode(0.2217693,3.701035){n64}
\pnode(0.6466549,4.297944){n65}
\pnode(3.364475,4.745668){n66}
\pnode(1.509306,0.9905115){n67}
\pnode(1.631241,0.08123197){n68}
\pnode(2.363231,0.03982867){n69}
\pnode(4.688611,1.662437){n70}
\pnode(4.32925,1.149353){n71}
\pnode(1.004251,3.3599){n72}
\pnode(3.170304,0.08031329){n73}
\pnode(1.092846,0.4743804){n74}
\pnode(0.957797,1.457980){n75}
\pnode(0.4145735,1.587721){n76}
\pnode(2.550006,4.241430){n77}
\pnode(0.8983155,1.243353){n78}
\pnode(1.560778,4.1743){n79}
\pnode(4.791386,3.470398){n80}
\pnode(1.183687,2.360412){n81}
\pnode(2.078577,3.201014){n82}
\pnode(4.135811,4.087185){n83}
\pnode(3.153883,3.140639){n84}
\pnode(0.6058247,2.060042){n85}
\pnode(1.891242,4.717356){n86}
\pnode(4.792091,2.606388){n87}
\pnode(3.635776,1.016917){n88}
\pnode(3.202631,4.133673){n89}
\pnode(1.765180,1.228386){n90}
\pnode(0.4301581,0.7624465){n91}
\pnode(4.292557,3.369169){n92}
\pnode(2.64,3.705){n93}
\pnode(2.869689,4.205452){n94}
\pnode(4.306428,1.798957){n95}
\pnode(0.9419618,2.164323){n96}
\pnode(2.802984,2.471939){n97}
\pnode(0.4990411,2.364772){n98}
\pnode(2.265986,3.781169){n99}
\pnode(2.886530,2.782621){n100}
\pnode(2.074548,4.120629){n101}
\pnode(3.578431,3.339056){n102}
\pnode(4.396008,2.731498){n103}
\pnode(2.161968,2.481101){n104}
\pnode(1.915954,1.686363){n105}
\pnode(1.073082,0.7635667){n106}
\pnode(4.351759,2.410157){n107}
\pnode(0.1632209,1.576404){n108}
\pnode(0.1964432,1.278737){n109}
\pnode(0.5000101,2.853756){n110}
\pnode(2.625767,0.8297721){n111}
\pnode(1.705716,3.114971){n112}
\pnode(2.796314,0.5814426){n113}
\pnode(0.3549037,3.121353){n114}
\pnode(3.949152,0.8549694){n115}
\pnode(4.562201,1.223617){n116}
\pnode(3.053188,3.925351){n117}
\pnode(4.318411,3.852927){n118}
\pnode(2.890681,3.735371){n119}
\pnode(2.880736,4.879838){n120}
\pnode(1.377059,0.7879707){n121}
\pnode(2.617372,2.90596){n122}
\pnode(3.046366,4.518936){n123}
\pnode(2.621703,0.4200536){n124}
\pnode(1.203850,1.499658){n125}
\pnode(0.9849086,4.079775){n126}
\pnode(3.254325,1.04113){n127}
\pnode(2.280932,2.838994){n128}
\pnode(4.117188,3.717776){n129}
\pnode(3.471599,1.285363){n130}
\pnode(4.500053,3.637197){n131}
\pnode(4.069486,1.411759){n132}
\pnode(4.11247,0.5910543){n133}
\pnode(1.252724,1.071603){n134}
\pnode(3.385955,0.3694225){n135}
\pnode(2.407195,0.4605674){n136}
\pnode(1.689756,4.453583){n137}
\pnode(3.989081,3.406712){n138}
\pnode(0.710692,2.532491){n139}
\pnode(4.873979,1.934726){n140}
\pnode(3.645946,2.645275){n141}
\pnode(1.606368,3.440222){n142}
\pnode(2.445606,3.296266){n143}
\pnode(2.274105,4.535248){n144}
\pnode(2.648264,0.1809270){n145}
\pnode(0.7976457,3.6483){n146}
\pnode(3.950723,2.918283){n147}
\pnode(4.776492,1.394465){n148}
\pnode(2.296867,0.7074797){n149}
\psset{linewidth=0.5pt,dotsep=2pt}
\ncline[linecolor=lightgray]{n0}{n44}
\ncline[linestyle=dotted,linecolor=lightgray]{n0}{n137}
\ncline[linecolor=lightgray]{n1}{n2}
\ncline[linecolor=lightgray]{n1}{n104}
\ncline[linecolor=lightgray]{n1}{n128}
\ncline[linecolor=lightgray]{n2}{n142}
\ncline[linecolor=lightgray]{n1}{n2}
\ncline[linecolor=lightgray]{n2}{n41}
\ncline[linecolor=lightgray]{n2}{n97}
\ncline[linecolor=lightgray]{n2}{n58}
\ncline[linecolor=lightgray]{n2}{n84}
\ncline[linecolor=lightgray]{n2}{n99}
\ncline[linecolor=lightgray]{n4}{n33}
\ncline[linecolor=lightgray]{n4}{n35}
\ncline[linecolor=lightgray]{n4}{n36}
\ncline[linecolor=lightgray]{n4}{n139}
\ncline[linecolor=lightgray]{n4}{n26}
\ncline[linestyle=dotted,linecolor=lightgray]{n4}{n38}
\ncline[linecolor=lightgray]{n4}{n98}
\ncline[linecolor=lightgray]{n4}{n47}
\ncline[linestyle=dotted,linecolor=lightgray]{n4}{n93}
\ncline[linecolor=lightgray]{n6}{n107}
\ncline[linecolor=lightgray]{n6}{n87}
\ncline[linecolor=lightgray]{n6}{n138}
\ncline[linecolor=lightgray]{n6}{n83}
\ncline[linecolor=lightgray]{n6}{n53}
\ncline[linecolor=lightgray]{n6}{n118}
\ncline[linecolor=lightgray]{n6}{n55}
\ncline[linecolor=lightgray]{n6}{n92}
\ncline[linecolor=lightgray]{n6}{n56}
\ncline[linecolor=lightgray]{n7}{n13}
\ncline[linecolor=lightgray]{n7}{n144}
\ncline[linecolor=lightgray]{n7}{n66}
\ncline[linecolor=lightgray]{n7}{n77}
\ncline[linecolor=lightgray]{n7}{n89}
\ncline[linecolor=lightgray]{n7}{n39}
\ncline[linecolor=lightgray]{n7}{n86}
\ncline[linecolor=lightgray]{n7}{n94}
\ncline[linecolor=lightgray]{n8}{n131}
\ncline[linecolor=lightgray]{n8}{n55}
\ncline[linecolor=lightgray]{n8}{n129}
\ncline[linecolor=lightgray]{n8}{n141}
\ncline[linestyle=dotted,linecolor=lightgray]{n8}{n46}
\ncline[linestyle=dotted,linecolor=lightgray]{n9}{n129}
\ncline[linestyle=dotted,linecolor=lightgray]{n9}{n83}
\ncline[linestyle=dotted,linecolor=lightgray]{n9}{n131}
\ncline[linecolor=lightgray]{n9}{n46}
\ncline[linecolor=lightgray]{n9}{n28}
\ncline[linestyle=dotted,linecolor=lightgray]{n9}{n60}
\ncline[linecolor=lightgray]{n9}{n95}
\ncline[linecolor=lightgray]{n12}{n146}
\ncline[linecolor=lightgray]{n12}{n72}
\ncline[linecolor=lightgray]{n12}{n65}
\ncline[linestyle=dotted,linecolor=lightgray]{n12}{n142}
\ncline[linecolor=lightgray]{n7}{n13}
\ncline[linecolor=lightgray]{n13}{n120}
\ncline[linestyle=dotted,linecolor=lightgray]{n13}{n41}
\ncline[linestyle=dotted,linecolor=lightgray]{n13}{n79}
\ncline[linestyle=dotted,linecolor=lightgray]{n13}{n102}
\ncline[linestyle=dotted,linecolor=lightgray]{n13}{n21}
\ncline[linecolor=lightgray]{n17}{n40}
\ncline[linecolor=lightgray]{n17}{n59}
\ncline[linecolor=lightgray]{n19}{n95}
\ncline[linestyle=dotted,linecolor=lightgray]{n19}{n29}
\ncline[linestyle=dotted,linecolor=lightgray]{n19}{n107}
\ncline[linecolor=lightgray]{n19}{n57}
\ncline[linecolor=lightgray]{n19}{n116}
\ncline[linecolor=lightgray]{n19}{n140}
\ncline[linecolor=lightgray]{n20}{n77}
\ncline[linecolor=lightgray]{n20}{n89}
\ncline[linecolor=lightgray]{n20}{n117}
\ncline[linecolor=lightgray]{n20}{n32}
\ncline[linecolor=lightgray]{n20}{n66}
\ncline[linecolor=lightgray]{n21}{n129}
\ncline[linecolor=lightgray]{n21}{n56}
\ncline[linecolor=lightgray]{n21}{n83}
\ncline[linecolor=lightgray]{n21}{n131}
\ncline[linestyle=dotted,linecolor=lightgray]{n21}{n89}
\ncline[linecolor=lightgray]{n21}{n141}
\ncline[linestyle=dotted,linecolor=lightgray]{n13}{n21}
\ncline[linecolor=lightgray]{n22}{n133}
\ncline[linecolor=lightgray]{n22}{n49}
\ncline[linecolor=lightgray]{n22}{n130}
\ncline[linecolor=lightgray]{n23}{n36}
\ncline[linecolor=lightgray]{n23}{n50}
\ncline[linecolor=lightgray]{n23}{n64}
\ncline[linecolor=lightgray]{n23}{n59}
\ncline[linecolor=lightgray]{n25}{n54}
\ncline[linecolor=lightgray]{n25}{n37}
\ncline[linecolor=lightgray]{n26}{n146}
\ncline[linecolor=lightgray]{n26}{n110}
\ncline[linecolor=lightgray]{n26}{n114}
\ncline[linecolor=lightgray]{n4}{n26}
\ncline[linecolor=lightgray]{n26}{n47}
\ncline[linecolor=lightgray]{n28}{n95}
\ncline[linecolor=lightgray]{n28}{n46}
\ncline[linecolor=lightgray]{n28}{n43}
\ncline[linecolor=lightgray]{n9}{n28}
\ncline[linecolor=lightgray]{n28}{n57}
\ncline[linecolor=lightgray]{n29}{n42}
\ncline[linecolor=lightgray]{n29}{n53}
\ncline[linecolor=lightgray]{n29}{n56}
\ncline[linecolor=lightgray]{n29}{n55}
\ncline[linestyle=dotted,linecolor=lightgray]{n29}{n140}
\ncline[linestyle=dotted,linecolor=lightgray]{n19}{n29}
\ncline[linecolor=lightgray]{n30}{n133}
\ncline[linecolor=lightgray]{n30}{n49}
\ncline[linecolor=lightgray]{n32}{n66}
\ncline[linecolor=lightgray]{n32}{n123}
\ncline[linecolor=lightgray]{n20}{n32}
\ncline[linecolor=lightgray]{n32}{n39}
\ncline[linecolor=lightgray]{n32}{n94}
\ncline[linecolor=lightgray]{n33}{n64}
\ncline[linecolor=lightgray]{n4}{n33}
\ncline[linecolor=lightgray]{n33}{n146}
\ncline[linecolor=lightgray]{n33}{n98}
\ncline[linecolor=lightgray]{n34}{n85}
\ncline[linecolor=lightgray]{n4}{n35}
\ncline[linecolor=lightgray]{n35}{n146}
\ncline[linecolor=lightgray]{n35}{n110}
\ncline[linecolor=lightgray]{n35}{n59}
\ncline[linecolor=lightgray]{n36}{n64}
\ncline[linecolor=lightgray]{n23}{n36}
\ncline[linecolor=lightgray]{n36}{n146}
\ncline[linecolor=lightgray]{n36}{n126}
\ncline[linecolor=lightgray]{n36}{n139}
\ncline[linecolor=lightgray]{n4}{n36}
\ncline[linecolor=lightgray]{n25}{n37}
\ncline[linecolor=lightgray]{n37}{n105}
\ncline[linestyle=dotted,linecolor=lightgray]{n37}{n130}
\ncline[linecolor=lightgray]{n38}{n119}
\ncline[linestyle=dotted,linecolor=lightgray]{n4}{n38}
\ncline[linestyle=dotted,linecolor=lightgray]{n38}{n112}
\ncline[linestyle=dotted,linecolor=lightgray]{n38}{n58}
\ncline[linecolor=lightgray]{n38}{n94}
\ncline[linecolor=lightgray]{n39}{n123}
\ncline[linecolor=lightgray]{n39}{n117}
\ncline[linecolor=lightgray]{n7}{n39}
\ncline[linecolor=lightgray]{n39}{n120}
\ncline[linecolor=lightgray]{n39}{n94}
\ncline[linecolor=lightgray]{n32}{n39}
\ncline[linecolor=lightgray]{n40}{n44}
\ncline[linecolor=lightgray]{n17}{n40}
\ncline[linecolor=lightgray]{n40}{n146}
\ncline[linestyle=dotted,linecolor=lightgray]{n40}{n41}
\ncline[linestyle=dotted,linecolor=lightgray]{n40}{n86}
\ncline[linestyle=dotted,linecolor=lightgray]{n40}{n137}
\ncline[linestyle=dotted,linecolor=lightgray]{n41}{n126}
\ncline[linecolor=lightgray]{n41}{n137}
\ncline[linestyle=dotted,linecolor=lightgray]{n41}{n77}
\ncline[linestyle=dotted,linecolor=lightgray]{n13}{n41}
\ncline[linecolor=lightgray]{n2}{n41}
\ncline[linestyle=dotted,linecolor=lightgray]{n40}{n41}
\ncline[linecolor=lightgray]{n42}{n131}
\ncline[linecolor=lightgray]{n42}{n141}
\ncline[linecolor=lightgray]{n29}{n42}
\ncline[linestyle=dotted,linecolor=lightgray]{n42}{n46}
\ncline[linecolor=lightgray]{n42}{n118}
\ncline[linecolor=lightgray]{n42}{n92}
\ncline[linecolor=lightgray]{n43}{n116}
\ncline[linecolor=lightgray]{n43}{n46}
\ncline[linecolor=lightgray]{n28}{n43}
\ncline[linecolor=lightgray]{n43}{n132}
\ncline[linecolor=lightgray]{n43}{n57}
\ncline[linecolor=lightgray]{n43}{n71}
\ncline[linecolor=lightgray]{n43}{n140}
\ncline[linecolor=lightgray]{n40}{n44}
\ncline[linecolor=lightgray]{n0}{n44}
\ncline[linecolor=lightgray]{n44}{n65}
\ncline[linestyle=dotted,linecolor=lightgray]{n42}{n46}
\ncline[linecolor=lightgray]{n28}{n46}
\ncline[linestyle=dotted,linecolor=lightgray]{n46}{n103}
\ncline[linecolor=lightgray]{n43}{n46}
\ncline[linestyle=dotted,linecolor=lightgray]{n46}{n56}
\ncline[linestyle=dotted,linecolor=lightgray]{n8}{n46}
\ncline[linecolor=lightgray]{n46}{n57}
\ncline[linestyle=dotted,linecolor=lightgray]{n46}{n87}
\ncline[linecolor=lightgray]{n9}{n46}
\ncline[linecolor=lightgray]{n46}{n140}
\ncline[linecolor=lightgray]{n47}{n114}
\ncline[linecolor=lightgray]{n26}{n47}
\ncline[linecolor=lightgray]{n4}{n47}
\ncline[linestyle=dotted,linecolor=lightgray]{n49}{n145}
\ncline[linecolor=lightgray]{n30}{n49}
\ncline[linestyle=dotted,linecolor=lightgray]{n49}{n73}
\ncline[linecolor=lightgray]{n49}{n115}
\ncline[linecolor=lightgray]{n49}{n133}
\ncline[linecolor=lightgray]{n22}{n49}
\ncline[linecolor=lightgray]{n49}{n130}
\ncline[linestyle=dotted,linecolor=lightgray]{n50}{n79}
\ncline[linecolor=lightgray]{n50}{n65}
\ncline[linecolor=lightgray]{n23}{n50}
\ncline[linecolor=lightgray]{n51}{n66}
\ncline[linecolor=lightgray]{n51}{n144}
\ncline[linecolor=lightgray]{n51}{n86}
\ncline[linestyle=dotted,linecolor=lightgray]{n51}{n65}
\ncline[linecolor=lightgray]{n52}{n74}
\ncline[linecolor=lightgray]{n53}{n87}
\ncline[linecolor=lightgray]{n6}{n53}
\ncline[linecolor=lightgray]{n29}{n53}
\ncline[linecolor=lightgray]{n53}{n80}
\ncline[linecolor=lightgray]{n25}{n54}
\ncline[linecolor=lightgray]{n8}{n55}
\ncline[linecolor=lightgray]{n55}{n103}
\ncline[linecolor=lightgray]{n6}{n55}
\ncline[linecolor=lightgray]{n29}{n55}
\ncline[linecolor=lightgray]{n55}{n131}
\ncline[linecolor=lightgray]{n55}{n107}
\ncline[linecolor=lightgray]{n21}{n56}
\ncline[linecolor=lightgray]{n56}{n138}
\ncline[linecolor=lightgray]{n56}{n102}
\ncline[linestyle=dotted,linecolor=lightgray]{n46}{n56}
\ncline[linecolor=lightgray]{n29}{n56}
\ncline[linecolor=lightgray]{n56}{n92}
\ncline[linecolor=lightgray]{n56}{n87}
\ncline[linecolor=lightgray]{n6}{n56}
\ncline[linecolor=lightgray]{n56}{n118}
\ncline[linecolor=lightgray]{n57}{n95}
\ncline[linecolor=lightgray]{n57}{n116}
\ncline[linecolor=lightgray]{n46}{n57}
\ncline[linecolor=lightgray]{n43}{n57}
\ncline[linecolor=lightgray]{n57}{n132}
\ncline[linecolor=lightgray]{n19}{n57}
\ncline[linecolor=lightgray]{n57}{n140}
\ncline[linecolor=lightgray]{n57}{n71}
\ncline[linecolor=lightgray]{n28}{n57}
\ncline[linecolor=lightgray]{n57}{n70}
\ncline[linecolor=lightgray]{n57}{n148}
\ncline[linestyle=dotted,linecolor=lightgray]{n58}{n119}
\ncline[linecolor=lightgray]{n58}{n143}
\ncline[linecolor=lightgray]{n58}{n112}
\ncline[linecolor=lightgray]{n2}{n58}
\ncline[linestyle=dotted,linecolor=lightgray]{n58}{n117}
\ncline[linestyle=dotted,linecolor=lightgray]{n38}{n58}
\ncline[linecolor=lightgray]{n59}{n64}
\ncline[linecolor=lightgray]{n17}{n59}
\ncline[linecolor=lightgray]{n35}{n59}
\ncline[linecolor=lightgray]{n59}{n110}
\ncline[linecolor=lightgray]{n23}{n59}
\ncline[linecolor=lightgray]{n59}{n139}
\ncline[linecolor=lightgray]{n59}{n98}
\ncline[linestyle=dotted,linecolor=lightgray]{n60}{n83}
\ncline[linecolor=lightgray]{n60}{n123}
\ncline[linecolor=lightgray]{n60}{n117}
\ncline[linecolor=lightgray]{n60}{n94}
\ncline[linestyle=dotted,linecolor=lightgray]{n9}{n60}
\ncline[linestyle=dotted,linecolor=lightgray]{n61}{n117}
\ncline[linestyle=dotted,linecolor=lightgray]{n61}{n123}
\ncline[linestyle=dotted,linecolor=lightgray]{n61}{n84}
\ncline[linecolor=lightgray]{n61}{n102}
\ncline[linecolor=lightgray]{n61}{n147}
\ncline[linecolor=lightgray]{n63}{n105}
\ncline[linecolor=lightgray]{n33}{n64}
\ncline[linecolor=lightgray]{n36}{n64}
\ncline[linecolor=lightgray]{n64}{n65}
\ncline[linecolor=lightgray]{n59}{n64}
\ncline[linecolor=lightgray]{n23}{n64}
\ncline[linecolor=lightgray]{n64}{n146}
\ncline[linecolor=lightgray]{n64}{n110}
\ncline[linecolor=lightgray]{n50}{n65}
\ncline[linecolor=lightgray]{n64}{n65}
\ncline[linecolor=lightgray]{n44}{n65}
\ncline[linecolor=lightgray]{n12}{n65}
\ncline[linecolor=lightgray]{n65}{n146}
\ncline[linestyle=dotted,linecolor=lightgray]{n51}{n65}
\ncline[linecolor=lightgray]{n65}{n110}
\ncline[linestyle=dotted,linecolor=lightgray]{n65}{n137}
\ncline[linecolor=lightgray]{n51}{n66}
\ncline[linecolor=lightgray]{n7}{n66}
\ncline[linecolor=lightgray]{n66}{n120}
\ncline[linecolor=lightgray]{n32}{n66}
\ncline[linecolor=lightgray]{n20}{n66}
\ncline[linecolor=lightgray]{n68}{n91}
\ncline[linecolor=lightgray]{n68}{n109}
\ncline[linecolor=lightgray]{n70}{n116}
\ncline[linecolor=lightgray]{n57}{n70}
\ncline[linecolor=lightgray]{n70}{n140}
\ncline[linecolor=lightgray]{n70}{n132}
\ncline[linecolor=lightgray]{n70}{n71}
\ncline[linecolor=lightgray]{n71}{n116}
\ncline[linecolor=lightgray]{n57}{n71}
\ncline[linecolor=lightgray]{n43}{n71}
\ncline[linecolor=lightgray]{n71}{n95}
\ncline[linecolor=lightgray]{n71}{n148}
\ncline[linecolor=lightgray]{n70}{n71}
\ncline[linecolor=lightgray]{n12}{n72}
\ncline[linestyle=dotted,linecolor=lightgray]{n49}{n73}
\ncline[linecolor=lightgray]{n52}{n74}
\ncline[linecolor=lightgray]{n20}{n77}
\ncline[linestyle=dotted,linecolor=lightgray]{n77}{n143}
\ncline[linestyle=dotted,linecolor=lightgray]{n77}{n79}
\ncline[linecolor=lightgray]{n77}{n86}
\ncline[linestyle=dotted,linecolor=lightgray]{n41}{n77}
\ncline[linecolor=lightgray]{n7}{n77}
\ncline[linecolor=lightgray]{n77}{n120}
\ncline[linestyle=dotted,linecolor=lightgray]{n50}{n79}
\ncline[linestyle=dotted,linecolor=lightgray]{n77}{n79}
\ncline[linecolor=lightgray]{n79}{n112}
\ncline[linecolor=lightgray]{n79}{n82}
\ncline[linecolor=lightgray]{n79}{n137}
\ncline[linecolor=lightgray]{n79}{n143}
\ncline[linestyle=dotted,linecolor=lightgray]{n13}{n79}
\ncline[linecolor=lightgray]{n80}{n102}
\ncline[linecolor=lightgray]{n80}{n118}
\ncline[linecolor=lightgray]{n80}{n87}
\ncline[linecolor=lightgray]{n53}{n80}
\ncline[linestyle=dotted,linecolor=lightgray]{n81}{n98}
\ncline[linecolor=lightgray]{n82}{n101}
\ncline[linecolor=lightgray]{n79}{n82}
\ncline[linecolor=lightgray]{n21}{n83}
\ncline[linecolor=lightgray]{n6}{n83}
\ncline[linestyle=dotted,linecolor=lightgray]{n60}{n83}
\ncline[linestyle=dotted,linecolor=lightgray]{n9}{n83}
\ncline[linestyle=dotted,linecolor=lightgray]{n84}{n117}
\ncline[linestyle=dotted,linecolor=lightgray]{n84}{n89}
\ncline[linestyle=dotted,linecolor=lightgray]{n84}{n138}
\ncline[linestyle=dotted,linecolor=lightgray]{n61}{n84}
\ncline[linecolor=lightgray]{n84}{n97}
\ncline[linecolor=lightgray]{n2}{n84}
\ncline[linecolor=lightgray]{n84}{n99}
\ncline[linestyle=dotted,linecolor=lightgray]{n84}{n118}
\ncline[linestyle=dotted,linecolor=lightgray]{n85}{n98}
\ncline[linecolor=lightgray]{n34}{n85}
\ncline[linestyle=dotted,linecolor=lightgray]{n86}{n99}
\ncline[linecolor=lightgray]{n77}{n86}
\ncline[linecolor=lightgray]{n51}{n86}
\ncline[linestyle=dotted,linecolor=lightgray]{n40}{n86}
\ncline[linecolor=lightgray]{n7}{n86}
\ncline[linecolor=lightgray]{n86}{n120}
\ncline[linecolor=lightgray]{n6}{n87}
\ncline[linecolor=lightgray]{n53}{n87}
\ncline[linecolor=lightgray]{n87}{n92}
\ncline[linecolor=lightgray]{n56}{n87}
\ncline[linecolor=lightgray]{n87}{n147}
\ncline[linecolor=lightgray]{n80}{n87}
\ncline[linestyle=dotted,linecolor=lightgray]{n46}{n87}
\ncline[linecolor=lightgray]{n87}{n141}
\ncline[linecolor=lightgray]{n88}{n115}
\ncline[linestyle=dotted,linecolor=lightgray]{n89}{n143}
\ncline[linestyle=dotted,linecolor=lightgray]{n21}{n89}
\ncline[linestyle=dotted,linecolor=lightgray]{n84}{n89}
\ncline[linecolor=lightgray]{n20}{n89}
\ncline[linecolor=lightgray]{n7}{n89}
\ncline[linecolor=lightgray]{n89}{n120}
\ncline[linecolor=lightgray]{n68}{n91}
\ncline[linecolor=lightgray]{n92}{n131}
\ncline[linecolor=lightgray]{n42}{n92}
\ncline[linecolor=lightgray]{n92}{n129}
\ncline[linecolor=lightgray]{n92}{n103}
\ncline[linecolor=lightgray]{n92}{n118}
\ncline[linecolor=lightgray]{n92}{n141}
\ncline[linecolor=lightgray]{n87}{n92}
\ncline[linecolor=lightgray]{n6}{n92}
\ncline[linecolor=lightgray]{n56}{n92}
\ncline[linecolor=lightgray]{n93}{n119}
\ncline[linecolor=lightgray]{n93}{n94}
\ncline[linestyle=dotted,linecolor=lightgray]{n4}{n93}
\ncline[linecolor=lightgray]{n94}{n123}
\ncline[linecolor=lightgray]{n93}{n94}
\ncline[linecolor=lightgray]{n39}{n94}
\ncline[linecolor=lightgray]{n32}{n94}
\ncline[linecolor=lightgray]{n7}{n94}
\ncline[linecolor=lightgray]{n94}{n120}
\ncline[linecolor=lightgray]{n38}{n94}
\ncline[linecolor=lightgray]{n60}{n94}
\ncline[linecolor=lightgray]{n19}{n95}
\ncline[linecolor=lightgray]{n28}{n95}
\ncline[linecolor=lightgray]{n57}{n95}
\ncline[linecolor=lightgray]{n95}{n116}
\ncline[linecolor=lightgray]{n95}{n132}
\ncline[linecolor=lightgray]{n9}{n95}
\ncline[linecolor=lightgray]{n71}{n95}
\ncline[linecolor=lightgray]{n2}{n97}
\ncline[linecolor=lightgray]{n97}{n128}
\ncline[linecolor=lightgray]{n84}{n97}
\ncline[linecolor=lightgray]{n33}{n98}
\ncline[linecolor=lightgray]{n98}{n114}
\ncline[linecolor=lightgray]{n59}{n98}
\ncline[linecolor=lightgray]{n4}{n98}
\ncline[linecolor=lightgray]{n98}{n110}
\ncline[linestyle=dotted,linecolor=lightgray]{n81}{n98}
\ncline[linestyle=dotted,linecolor=lightgray]{n85}{n98}
\ncline[linestyle=dotted,linecolor=lightgray]{n86}{n99}
\ncline[linestyle=dotted,linecolor=lightgray]{n99}{n144}
\ncline[linecolor=lightgray]{n99}{n101}
\ncline[linecolor=lightgray]{n99}{n128}
\ncline[linecolor=lightgray]{n99}{n104}
\ncline[linecolor=lightgray]{n2}{n99}
\ncline[linecolor=lightgray]{n84}{n99}
\ncline[linecolor=lightgray]{n100}{n128}
\ncline[linecolor=lightgray]{n100}{n104}
\ncline[linestyle=dotted,linecolor=lightgray]{n100}{n119}
\ncline[linestyle=dotted,linecolor=lightgray]{n101}{n126}
\ncline[linestyle=dotted,linecolor=lightgray]{n101}{n117}
\ncline[linecolor=lightgray]{n82}{n101}
\ncline[linecolor=lightgray]{n99}{n101}
\ncline[linestyle=dotted,linecolor=lightgray]{n101}{n123}
\ncline[linestyle=dotted,linecolor=lightgray]{n101}{n119}
\ncline[linecolor=lightgray]{n101}{n112}
\ncline[linecolor=lightgray]{n102}{n118}
\ncline[linecolor=lightgray]{n102}{n138}
\ncline[linecolor=lightgray]{n102}{n141}
\ncline[linecolor=lightgray]{n56}{n102}
\ncline[linestyle=dotted,linecolor=lightgray]{n13}{n102}
\ncline[linecolor=lightgray]{n61}{n102}
\ncline[linecolor=lightgray]{n80}{n102}
\ncline[linecolor=lightgray]{n103}{n118}
\ncline[linecolor=lightgray]{n103}{n138}
\ncline[linecolor=lightgray]{n55}{n103}
\ncline[linestyle=dotted,linecolor=lightgray]{n46}{n103}
\ncline[linecolor=lightgray]{n92}{n103}
\ncline[linecolor=lightgray]{n103}{n129}
\ncline[linecolor=lightgray]{n103}{n131}
\ncline[linecolor=lightgray]{n103}{n147}
\ncline[linecolor=lightgray]{n100}{n104}
\ncline[linecolor=lightgray]{n1}{n104}
\ncline[linecolor=lightgray]{n99}{n104}
\ncline[linecolor=lightgray]{n63}{n105}
\ncline[linecolor=lightgray]{n37}{n105}
\ncline[linecolor=lightgray]{n6}{n107}
\ncline[linecolor=lightgray]{n107}{n147}
\ncline[linecolor=lightgray]{n107}{n141}
\ncline[linecolor=lightgray]{n55}{n107}
\ncline[linestyle=dotted,linecolor=lightgray]{n19}{n107}
\ncline[linestyle=dotted,linecolor=lightgray]{n107}{n140}
\ncline[linecolor=lightgray]{n68}{n109}
\ncline[linecolor=lightgray]{n26}{n110}
\ncline[linecolor=lightgray]{n110}{n139}
\ncline[linecolor=lightgray]{n35}{n110}
\ncline[linecolor=lightgray]{n59}{n110}
\ncline[linecolor=lightgray]{n64}{n110}
\ncline[linecolor=lightgray]{n98}{n110}
\ncline[linecolor=lightgray]{n110}{n146}
\ncline[linecolor=lightgray]{n65}{n110}
\ncline[linecolor=lightgray]{n79}{n112}
\ncline[linecolor=lightgray]{n101}{n112}
\ncline[linecolor=lightgray]{n58}{n112}
\ncline[linecolor=lightgray]{n112}{n122}
\ncline[linestyle=dotted,linecolor=lightgray]{n38}{n112}
\ncline[linecolor=lightgray]{n47}{n114}
\ncline[linecolor=lightgray]{n98}{n114}
\ncline[linecolor=lightgray]{n26}{n114}
\ncline[linecolor=lightgray]{n88}{n115}
\ncline[linecolor=lightgray]{n49}{n115}
\ncline[linecolor=lightgray]{n43}{n116}
\ncline[linecolor=lightgray]{n57}{n116}
\ncline[linecolor=lightgray]{n71}{n116}
\ncline[linecolor=lightgray]{n95}{n116}
\ncline[linecolor=lightgray]{n19}{n116}
\ncline[linecolor=lightgray]{n116}{n148}
\ncline[linecolor=lightgray]{n70}{n116}
\ncline[linecolor=lightgray]{n116}{n140}
\ncline[linestyle=dotted,linecolor=lightgray]{n101}{n117}
\ncline[linestyle=dotted,linecolor=lightgray]{n61}{n117}
\ncline[linestyle=dotted,linecolor=lightgray]{n84}{n117}
\ncline[linecolor=lightgray]{n39}{n117}
\ncline[linecolor=lightgray]{n20}{n117}
\ncline[linestyle=dotted,linecolor=lightgray]{n58}{n117}
\ncline[linecolor=lightgray]{n117}{n144}
\ncline[linestyle=dotted,linecolor=lightgray]{n117}{n138}
\ncline[linecolor=lightgray]{n60}{n117}
\ncline[linecolor=lightgray]{n102}{n118}
\ncline[linecolor=lightgray]{n42}{n118}
\ncline[linecolor=lightgray]{n103}{n118}
\ncline[linecolor=lightgray]{n118}{n131}
\ncline[linecolor=lightgray]{n6}{n118}
\ncline[linecolor=lightgray]{n92}{n118}
\ncline[linecolor=lightgray]{n56}{n118}
\ncline[linestyle=dotted,linecolor=lightgray]{n84}{n118}
\ncline[linecolor=lightgray]{n80}{n118}
\ncline[linestyle=dotted,linecolor=lightgray]{n119}{n143}
\ncline[linestyle=dotted,linecolor=lightgray]{n101}{n119}
\ncline[linestyle=dotted,linecolor=lightgray]{n58}{n119}
\ncline[linecolor=lightgray]{n93}{n119}
\ncline[linecolor=lightgray]{n38}{n119}
\ncline[linestyle=dotted,linecolor=lightgray]{n100}{n119}
\ncline[linecolor=lightgray]{n13}{n120}
\ncline[linecolor=lightgray]{n120}{n144}
\ncline[linecolor=lightgray]{n66}{n120}
\ncline[linecolor=lightgray]{n77}{n120}
\ncline[linecolor=lightgray]{n89}{n120}
\ncline[linecolor=lightgray]{n39}{n120}
\ncline[linecolor=lightgray]{n86}{n120}
\ncline[linecolor=lightgray]{n94}{n120}
\ncline[linecolor=lightgray]{n112}{n122}
\ncline[linecolor=lightgray]{n123}{n144}
\ncline[linecolor=lightgray]{n39}{n123}
\ncline[linestyle=dotted,linecolor=lightgray]{n101}{n123}
\ncline[linecolor=lightgray]{n94}{n123}
\ncline[linestyle=dotted,linecolor=lightgray]{n61}{n123}
\ncline[linecolor=lightgray]{n32}{n123}
\ncline[linecolor=lightgray]{n60}{n123}
\ncline[linestyle=dotted,linecolor=lightgray]{n41}{n126}
\ncline[linestyle=dotted,linecolor=lightgray]{n101}{n126}
\ncline[linecolor=lightgray]{n36}{n126}
\ncline[linestyle=dotted,linecolor=lightgray]{n126}{n142}
\ncline[linecolor=lightgray]{n100}{n128}
\ncline[linecolor=lightgray]{n97}{n128}
\ncline[linecolor=lightgray]{n99}{n128}
\ncline[linecolor=lightgray]{n1}{n128}
\ncline[linecolor=lightgray]{n21}{n129}
\ncline[linecolor=lightgray]{n8}{n129}
\ncline[linecolor=lightgray]{n92}{n129}
\ncline[linestyle=dotted,linecolor=lightgray]{n9}{n129}
\ncline[linecolor=lightgray]{n129}{n147}
\ncline[linecolor=lightgray]{n103}{n129}
\ncline[linestyle=dotted,linecolor=lightgray]{n37}{n130}
\ncline[linecolor=lightgray]{n49}{n130}
\ncline[linecolor=lightgray]{n22}{n130}
\ncline[linecolor=lightgray]{n42}{n131}
\ncline[linecolor=lightgray]{n8}{n131}
\ncline[linecolor=lightgray]{n131}{n138}
\ncline[linecolor=lightgray]{n21}{n131}
\ncline[linecolor=lightgray]{n92}{n131}
\ncline[linecolor=lightgray]{n118}{n131}
\ncline[linecolor=lightgray]{n131}{n147}
\ncline[linecolor=lightgray]{n103}{n131}
\ncline[linestyle=dotted,linecolor=lightgray]{n9}{n131}
\ncline[linecolor=lightgray]{n55}{n131}
\ncline[linecolor=lightgray]{n43}{n132}
\ncline[linecolor=lightgray]{n95}{n132}
\ncline[linecolor=lightgray]{n57}{n132}
\ncline[linecolor=lightgray]{n132}{n148}
\ncline[linecolor=lightgray]{n70}{n132}
\ncline[linecolor=lightgray]{n30}{n133}
\ncline[linecolor=lightgray]{n22}{n133}
\ncline[linecolor=lightgray]{n49}{n133}
\ncline[linecolor=lightgray]{n41}{n137}
\ncline[linecolor=lightgray]{n79}{n137}
\ncline[linestyle=dotted,linecolor=lightgray]{n0}{n137}
\ncline[linestyle=dotted,linecolor=lightgray]{n137}{n144}
\ncline[linestyle=dotted,linecolor=lightgray]{n40}{n137}
\ncline[linestyle=dotted,linecolor=lightgray]{n65}{n137}
\ncline[linecolor=lightgray]{n131}{n138}
\ncline[linecolor=lightgray]{n102}{n138}
\ncline[linecolor=lightgray]{n6}{n138}
\ncline[linecolor=lightgray]{n103}{n138}
\ncline[linecolor=lightgray]{n56}{n138}
\ncline[linestyle=dotted,linecolor=lightgray]{n84}{n138}
\ncline[linestyle=dotted,linecolor=lightgray]{n117}{n138}
\ncline[linecolor=lightgray]{n138}{n147}
\ncline[linecolor=lightgray]{n36}{n139}
\ncline[linecolor=lightgray]{n110}{n139}
\ncline[linecolor=lightgray]{n59}{n139}
\ncline[linecolor=lightgray]{n4}{n139}
\ncline[linestyle=dotted,linecolor=lightgray]{n29}{n140}
\ncline[linestyle=dotted,linecolor=lightgray]{n107}{n140}
\ncline[linecolor=lightgray]{n46}{n140}
\ncline[linecolor=lightgray]{n57}{n140}
\ncline[linecolor=lightgray]{n19}{n140}
\ncline[linecolor=lightgray]{n140}{n148}
\ncline[linecolor=lightgray]{n70}{n140}
\ncline[linecolor=lightgray]{n116}{n140}
\ncline[linecolor=lightgray]{n43}{n140}
\ncline[linecolor=lightgray]{n42}{n141}
\ncline[linecolor=lightgray]{n102}{n141}
\ncline[linecolor=lightgray]{n21}{n141}
\ncline[linecolor=lightgray]{n8}{n141}
\ncline[linecolor=lightgray]{n92}{n141}
\ncline[linecolor=lightgray]{n107}{n141}
\ncline[linecolor=lightgray]{n87}{n141}
\ncline[linecolor=lightgray]{n142}{n143}
\ncline[linecolor=lightgray]{n2}{n142}
\ncline[linestyle=dotted,linecolor=lightgray]{n12}{n142}
\ncline[linestyle=dotted,linecolor=lightgray]{n126}{n142}
\ncline[linestyle=dotted,linecolor=lightgray]{n77}{n143}
\ncline[linestyle=dotted,linecolor=lightgray]{n119}{n143}
\ncline[linecolor=lightgray]{n142}{n143}
\ncline[linestyle=dotted,linecolor=lightgray]{n89}{n143}
\ncline[linecolor=lightgray]{n58}{n143}
\ncline[linecolor=lightgray]{n79}{n143}
\ncline[linecolor=lightgray]{n123}{n144}
\ncline[linestyle=dotted,linecolor=lightgray]{n99}{n144}
\ncline[linecolor=lightgray]{n7}{n144}
\ncline[linecolor=lightgray]{n120}{n144}
\ncline[linecolor=lightgray]{n51}{n144}
\ncline[linestyle=dotted,linecolor=lightgray]{n137}{n144}
\ncline[linecolor=lightgray]{n117}{n144}
\ncline[linestyle=dotted,linecolor=lightgray]{n49}{n145}
\ncline[linecolor=lightgray]{n36}{n146}
\ncline[linecolor=lightgray]{n12}{n146}
\ncline[linecolor=lightgray]{n33}{n146}
\ncline[linecolor=lightgray]{n26}{n146}
\ncline[linecolor=lightgray]{n40}{n146}
\ncline[linecolor=lightgray]{n65}{n146}
\ncline[linecolor=lightgray]{n64}{n146}
\ncline[linecolor=lightgray]{n35}{n146}
\ncline[linecolor=lightgray]{n110}{n146}
\ncline[linecolor=lightgray]{n131}{n147}
\ncline[linecolor=lightgray]{n129}{n147}
\ncline[linecolor=lightgray]{n107}{n147}
\ncline[linecolor=lightgray]{n61}{n147}
\ncline[linecolor=lightgray]{n103}{n147}
\ncline[linecolor=lightgray]{n138}{n147}
\ncline[linecolor=lightgray]{n87}{n147}
\ncline[linecolor=lightgray]{n116}{n148}
\ncline[linecolor=lightgray]{n140}{n148}
\ncline[linecolor=lightgray]{n57}{n148}
\ncline[linecolor=lightgray]{n132}{n148}
\ncline[linecolor=lightgray]{n71}{n148}
\ncline[linestyle=dotted,linecolor=lightgray]{n14}{n34}
\ncline[linestyle=dotted,linecolor=lightgray]{n15}{n34}
\ncline[linestyle=dotted,linecolor=lightgray]{n25}{n34}
\ncline[linestyle=dotted,linecolor=lightgray]{n34}{n37}
\ncline[linestyle=dotted,linecolor=lightgray]{n34}{n54}
\ncline[linestyle=dotted,linecolor=lightgray]{n34}{n63}
\ncline[linestyle=dotted,linecolor=lightgray]{n34}{n67}
\ncline[linestyle=dotted,linecolor=lightgray]{n34}{n75}
\ncline[linestyle=dotted,linecolor=lightgray]{n34}{n78}
\ncline[linestyle=dotted,linecolor=lightgray]{n34}{n105}
\ncline[linestyle=dotted,linecolor=lightgray]{n34}{n121}
\ncline[linestyle=dotted,linecolor=lightgray]{n34}{n125}
\ncline[linestyle=dotted,linecolor=lightgray]{n34}{n134}
\ncline[linestyle=dotted,linecolor=lightgray]{n34}{n145}
\ncline[linestyle=dotted,linecolor=lightgray]{n63}{n96}
\ncline[linestyle=dotted,linecolor=lightgray]{n96}{n105}
\ncline[linecolor=lightgray]{n12}{n24}
\ncline[linecolor=lightgray]{n17}{n24}
\ncline[linecolor=lightgray]{n23}{n24}
\ncline[linecolor=lightgray]{n24}{n26}
\ncline[linestyle=dotted,linecolor=lightgray]{n24}{n32}
\ncline[linecolor=lightgray]{n24}{n35}
\ncline[linecolor=lightgray]{n24}{n40}
\ncline[linecolor=lightgray]{n24}{n44}
\ncline[linecolor=lightgray]{n24}{n50}
\ncline[linestyle=dotted,linecolor=lightgray]{n24}{n51}
\ncline[linecolor=lightgray]{n24}{n64}
\ncline[linecolor=lightgray]{n24}{n65}
\ncline[linecolor=lightgray]{n24}{n110}
\ncline[linecolor=lightgray]{n24}{n126}
\ncline[linecolor=lightgray]{n24}{n139}
\ncline[linecolor=lightgray]{n24}{n146}
\ncline[linestyle=dotted,linecolor=lightgray]{n4}{n108}
\ncline[linestyle=dotted,linecolor=lightgray]{n23}{n108}
\ncline[linestyle=dotted,linecolor=lightgray]{n26}{n108}
\ncline[linestyle=dotted,linecolor=lightgray]{n33}{n108}
\ncline[linestyle=dotted,linecolor=lightgray]{n34}{n108}
\ncline[linestyle=dotted,linecolor=lightgray]{n35}{n108}
\ncline[linestyle=dotted,linecolor=lightgray]{n36}{n108}
\ncline[linestyle=dotted,linecolor=lightgray]{n47}{n108}
\ncline[linestyle=dotted,linecolor=lightgray]{n59}{n108}
\ncline[linestyle=dotted,linecolor=lightgray]{n81}{n108}
\ncline[linestyle=dotted,linecolor=lightgray]{n85}{n108}
\ncline[linestyle=dotted,linecolor=lightgray]{n96}{n108}
\ncline[linestyle=dotted,linecolor=lightgray]{n98}{n108}
\ncline[linestyle=dotted,linecolor=lightgray]{n108}{n110}
\ncline[linestyle=dotted,linecolor=lightgray]{n108}{n114}
\ncline[linestyle=dotted,linecolor=lightgray]{n108}{n139}
\ncline[linestyle=dotted,linecolor=lightgray]{n14}{n24}
\ncline[linestyle=dotted,linecolor=lightgray]{n15}{n24}
\ncline[linestyle=dotted,linecolor=lightgray]{n24}{n25}
\ncline[linestyle=dotted,linecolor=lightgray]{n24}{n27}
\ncline[linestyle=dotted,linecolor=lightgray]{n24}{n37}
\ncline[linestyle=dotted,linecolor=lightgray]{n24}{n54}
\ncline[linestyle=dotted,linecolor=lightgray]{n24}{n63}
\ncline[linestyle=dotted,linecolor=lightgray]{n24}{n67}
\ncline[linestyle=dotted,linecolor=lightgray]{n24}{n75}
\ncline[linestyle=dotted,linecolor=lightgray]{n24}{n78}
\ncline[linestyle=dotted,linecolor=lightgray]{n24}{n88}
\ncline[linestyle=dotted,linecolor=lightgray]{n24}{n90}
\ncline[linestyle=dotted,linecolor=lightgray]{n24}{n105}
\ncline[linestyle=dotted,linecolor=lightgray]{n24}{n121}
\ncline[linestyle=dotted,linecolor=lightgray]{n24}{n125}
\ncline[linestyle=dotted,linecolor=lightgray]{n24}{n134}
\ncline[linestyle=dotted,linecolor=lightgray]{n24}{n145}
\ncline[linecolor=lightgray]{n3}{n108}
\ncline[linecolor=lightgray]{n5}{n108}
\ncline[linecolor=lightgray]{n14}{n108}
\ncline[linecolor=lightgray]{n27}{n108}
\ncline[linecolor=lightgray]{n45}{n108}
\ncline[linecolor=lightgray]{n52}{n108}
\ncline[linecolor=lightgray]{n67}{n108}
\ncline[linecolor=lightgray]{n75}{n108}
\ncline[linecolor=lightgray]{n76}{n108}
\ncline[linecolor=lightgray]{n78}{n108}
\ncline[linecolor=lightgray]{n91}{n108}
\ncline[linecolor=lightgray]{n106}{n108}
\ncline[linecolor=lightgray]{n108}{n109}
\ncline[linecolor=lightgray]{n108}{n121}
\ncline[linecolor=lightgray]{n108}{n125}
\ncline[linecolor=lightgray]{n108}{n134}
\ncline[linestyle=dotted,linecolor=lightgray]{n24}{n108}
\ncline[linestyle=dotted,linecolor=gray]{n0}{n79}
\ncline[linecolor=gray]{n0}{n12}
\ncline[linecolor=gray]{n0}{n17}
\ncline[linecolor=gray]{n0}{n64}
\ncline[linestyle=dotted,linecolor=gray]{n0}{n142}
\ncline[linecolor=gray]{n0}{n72}
\ncline[linecolor=gray]{n0}{n40}
\ncline[linestyle=dotted,linecolor=gray]{n0}{n41}
\ncline[linecolor=gray]{n0}{n126}
\ncline[linecolor=gray]{n1}{n100}
\ncline[linecolor=gray]{n2}{n112}
\ncline[linecolor=gray]{n2}{n82}
\ncline[linecolor=gray]{n3}{n90}
\ncline[linecolor=gray]{n3}{n14}
\ncline[linecolor=gray]{n3}{n67}
\ncline[linecolor=gray]{n3}{n134}
\ncline[linecolor=gray]{n3}{n76}
\ncline[linecolor=gray]{n3}{n106}
\ncline[linecolor=gray]{n5}{n125}
\ncline[linecolor=gray]{n5}{n52}
\ncline[linecolor=gray]{n6}{n8}
\ncline[linecolor=gray]{n6}{n103}
\ncline[linecolor=gray]{n6}{n29}
\ncline[linecolor=gray]{n6}{n42}
\ncline[linecolor=gray]{n7}{n20}
\ncline[linecolor=gray]{n7}{n123}
\ncline[linecolor=gray]{n8}{n42}
\ncline[linecolor=gray]{n8}{n29}
\ncline[linecolor=gray]{n6}{n8}
\ncline[linecolor=gray]{n8}{n103}
\ncline[linecolor=gray]{n8}{n56}
\ncline[linecolor=gray]{n8}{n147}
\ncline[linecolor=gray]{n8}{n92}
\ncline[linecolor=gray]{n10}{n15}
\ncline[linecolor=gray]{n10}{n149}
\ncline[linecolor=gray]{n10}{n135}
\ncline[linecolor=gray]{n10}{n37}
\ncline[linecolor=gray]{n11}{n124}
\ncline[linecolor=gray]{n11}{n113}
\ncline[linecolor=gray]{n11}{n135}
\ncline[linecolor=gray]{n11}{n62}
\ncline[linestyle=dotted,linecolor=gray]{n11}{n130}
\ncline[linestyle=dotted,linecolor=gray]{n11}{n30}
\ncline[linestyle=dotted,linecolor=gray]{n11}{n88}
\ncline[linecolor=gray]{n11}{n37}
\ncline[linecolor=gray]{n12}{n44}
\ncline[linecolor=gray]{n0}{n12}
\ncline[linecolor=gray]{n12}{n126}
\ncline[linecolor=gray]{n12}{n36}
\ncline[linecolor=gray]{n12}{n40}
\ncline[linecolor=gray]{n12}{n114}
\ncline[linecolor=gray]{n12}{n33}
\ncline[linecolor=gray]{n13}{n89}
\ncline[linecolor=gray]{n13}{n123}
\ncline[linecolor=gray]{n13}{n117}
\ncline[linestyle=dotted,linecolor=gray]{n13}{n99}
\ncline[linestyle=dotted,linecolor=gray]{n13}{n101}
\ncline[linestyle=dotted,linecolor=gray]{n14}{n16}
\ncline[linecolor=gray]{n14}{n31}
\ncline[linecolor=gray]{n14}{n90}
\ncline[linecolor=gray]{n3}{n14}
\ncline[linecolor=gray]{n10}{n15}
\ncline[linecolor=gray]{n15}{n62}
\ncline[linecolor=gray]{n15}{n69}
\ncline[linecolor=gray]{n15}{n111}
\ncline[linecolor=gray]{n15}{n105}
\ncline[linestyle=dotted,linecolor=gray]{n15}{n125}
\ncline[linestyle=dotted,linecolor=gray]{n14}{n16}
\ncline[linestyle=dotted,linecolor=gray]{n16}{n67}
\ncline[linestyle=dotted,linecolor=gray]{n16}{n134}
\ncline[linestyle=dotted,linecolor=gray]{n16}{n31}
\ncline[linestyle=dotted,linecolor=gray]{n16}{n75}
\ncline[linestyle=dotted,linecolor=gray]{n16}{n125}
\ncline[linecolor=gray]{n17}{n36}
\ncline[linecolor=gray]{n0}{n17}
\ncline[linecolor=gray]{n17}{n23}
\ncline[linecolor=gray]{n17}{n126}
\ncline[linecolor=gray]{n17}{n72}
\ncline[linecolor=gray]{n17}{n35}
\ncline[linecolor=gray]{n17}{n146}
\ncline[linecolor=gray]{n19}{n28}
\ncline[linecolor=gray]{n19}{n46}
\ncline[linecolor=gray]{n19}{n43}
\ncline[linecolor=gray]{n7}{n20}
\ncline[linecolor=gray]{n20}{n120}
\ncline[linecolor=gray]{n20}{n86}
\ncline[linecolor=gray]{n20}{n123}
\ncline[linecolor=gray]{n20}{n51}
\ncline[linecolor=gray]{n21}{n61}
\ncline[linecolor=gray]{n21}{n138}
\ncline[linecolor=gray]{n21}{n147}
\ncline[linecolor=gray]{n21}{n92}
\ncline[linecolor=gray]{n21}{n118}
\ncline[linestyle=dotted,linecolor=gray]{n21}{n84}
\ncline[linecolor=gray]{n21}{n42}
\ncline[linecolor=gray]{n22}{n115}
\ncline[linecolor=gray]{n23}{n33}
\ncline[linecolor=gray]{n17}{n23}
\ncline[linecolor=gray]{n23}{n110}
\ncline[linecolor=gray]{n23}{n114}
\ncline[linecolor=gray]{n23}{n146}
\ncline[linecolor=gray]{n23}{n98}
\ncline[linecolor=gray]{n25}{n136}
\ncline[linecolor=gray]{n25}{n62}
\ncline[linecolor=gray]{n26}{n33}
\ncline[linecolor=gray]{n26}{n98}
\ncline[linecolor=gray]{n26}{n44}
\ncline[linecolor=gray]{n27}{n74}
\ncline[linecolor=gray]{n27}{n106}
\ncline[linecolor=gray]{n27}{n90}
\ncline[linecolor=gray]{n19}{n28}
\ncline[linecolor=gray]{n8}{n29}
\ncline[linecolor=gray]{n6}{n29}
\ncline[linestyle=dotted,linecolor=gray]{n29}{n46}
\ncline[linestyle=dotted,linecolor=gray]{n30}{n135}
\ncline[linestyle=dotted,linecolor=gray]{n30}{n48}
\ncline[linecolor=gray]{n30}{n115}
\ncline[linestyle=dotted,linecolor=gray]{n11}{n30}
\ncline[linecolor=gray]{n30}{n88}
\ncline[linestyle=dotted,linecolor=gray]{n30}{n73}
\ncline[linecolor=gray]{n30}{n130}
\ncline[linestyle=dotted,linecolor=gray]{n31}{n62}
\ncline[linecolor=gray]{n14}{n31}
\ncline[linecolor=gray]{n31}{n67}
\ncline[linestyle=dotted,linecolor=gray]{n31}{n69}
\ncline[linecolor=gray]{n31}{n134}
\ncline[linestyle=dotted,linecolor=gray]{n16}{n31}
\ncline[linecolor=gray]{n31}{n45}
\ncline[linecolor=gray]{n31}{n78}
\ncline[linestyle=dotted,linecolor=gray]{n31}{n124}
\ncline[linecolor=gray]{n32}{n51}
\ncline[linecolor=gray]{n23}{n33}
\ncline[linecolor=gray]{n33}{n72}
\ncline[linecolor=gray]{n33}{n50}
\ncline[linecolor=gray]{n33}{n139}
\ncline[linecolor=gray]{n26}{n33}
\ncline[linecolor=gray]{n12}{n33}
\ncline[linecolor=gray]{n34}{n96}
\ncline[linecolor=gray]{n34}{n81}
\ncline[linecolor=gray]{n35}{n36}
\ncline[linecolor=gray]{n35}{n44}
\ncline[linecolor=gray]{n17}{n35}
\ncline[linecolor=gray]{n35}{n47}
\ncline[linecolor=gray]{n17}{n36}
\ncline[linecolor=gray]{n35}{n36}
\ncline[linecolor=gray]{n36}{n44}
\ncline[linecolor=gray]{n12}{n36}
\ncline[linecolor=gray]{n36}{n50}
\ncline[linecolor=gray]{n36}{n72}
\ncline[linecolor=gray]{n37}{n127}
\ncline[linestyle=dotted,linecolor=gray]{n37}{n88}
\ncline[linecolor=gray]{n10}{n37}
\ncline[linecolor=gray]{n11}{n37}
\ncline[linestyle=dotted,linecolor=gray]{n39}{n61}
\ncline[linecolor=gray]{n39}{n77}
\ncline[linecolor=gray]{n39}{n119}
\ncline[linecolor=gray]{n40}{n50}
\ncline[linestyle=dotted,linecolor=gray]{n40}{n79}
\ncline[linecolor=gray]{n0}{n40}
\ncline[linecolor=gray]{n12}{n40}
\ncline[linecolor=gray]{n40}{n64}
\ncline[linecolor=gray]{n41}{n142}
\ncline[linecolor=gray]{n41}{n82}
\ncline[linecolor=gray]{n41}{n99}
\ncline[linecolor=gray]{n41}{n112}
\ncline[linestyle=dotted,linecolor=gray]{n0}{n41}
\ncline[linestyle=dotted,linecolor=gray]{n41}{n144}
\ncline[linecolor=gray]{n41}{n143}
\ncline[linecolor=gray]{n8}{n42}
\ncline[linecolor=gray]{n42}{n107}
\ncline[linecolor=gray]{n42}{n147}
\ncline[linecolor=gray]{n42}{n103}
\ncline[linecolor=gray]{n42}{n138}
\ncline[linecolor=gray]{n6}{n42}
\ncline[linecolor=gray]{n21}{n42}
\ncline[linecolor=gray]{n42}{n129}
\ncline[linecolor=gray]{n19}{n43}
\ncline[linecolor=gray]{n44}{n126}
\ncline[linecolor=gray]{n12}{n44}
\ncline[linecolor=gray]{n36}{n44}
\ncline[linecolor=gray]{n44}{n114}
\ncline[linecolor=gray]{n35}{n44}
\ncline[linecolor=gray]{n26}{n44}
\ncline[linecolor=gray]{n31}{n45}
\ncline[linecolor=gray]{n45}{n52}
\ncline[linecolor=gray]{n45}{n76}
\ncline[linecolor=gray]{n19}{n46}
\ncline[linestyle=dotted,linecolor=gray]{n46}{n107}
\ncline[linestyle=dotted,linecolor=gray]{n29}{n46}
\ncline[linecolor=gray]{n46}{n95}
\ncline[linecolor=gray]{n47}{n98}
\ncline[linecolor=gray]{n47}{n139}
\ncline[linecolor=gray]{n35}{n47}
\ncline[linecolor=gray]{n47}{n59}
\ncline[linecolor=gray]{n48}{n111}
\ncline[linecolor=gray]{n48}{n69}
\ncline[linecolor=gray]{n48}{n73}
\ncline[linestyle=dotted,linecolor=gray]{n30}{n48}
\ncline[linestyle=dotted,linecolor=gray]{n48}{n88}
\ncline[linecolor=gray]{n48}{n149}
\ncline[linecolor=gray]{n48}{n145}
\ncline[linecolor=gray]{n48}{n54}
\ncline[linecolor=gray]{n49}{n88}
\ncline[linestyle=dotted,linecolor=gray]{n49}{n135}
\ncline[linecolor=gray]{n33}{n50}
\ncline[linecolor=gray]{n36}{n50}
\ncline[linecolor=gray]{n40}{n50}
\ncline[linecolor=gray]{n50}{n114}
\ncline[linecolor=gray]{n50}{n146}
\ncline[linecolor=gray]{n50}{n72}
\ncline[linecolor=gray]{n32}{n51}
\ncline[linecolor=gray]{n20}{n51}
\ncline[linecolor=gray]{n5}{n52}
\ncline[linecolor=gray]{n45}{n52}
\ncline[linecolor=gray]{n52}{n78}
\ncline[linecolor=gray]{n53}{n55}
\ncline[linecolor=gray]{n54}{n136}
\ncline[linecolor=gray]{n54}{n135}
\ncline[linecolor=gray]{n54}{n149}
\ncline[linecolor=gray]{n48}{n54}
\ncline[linecolor=gray]{n54}{n63}
\ncline[linecolor=gray]{n55}{n92}
\ncline[linecolor=gray]{n53}{n55}
\ncline[linecolor=gray]{n55}{n87}
\ncline[linecolor=gray]{n56}{n147}
\ncline[linecolor=gray]{n56}{n141}
\ncline[linecolor=gray]{n56}{n107}
\ncline[linecolor=gray]{n8}{n56}
\ncline[linecolor=gray]{n56}{n103}
\ncline[linecolor=gray]{n58}{n79}
\ncline[linecolor=gray]{n58}{n142}
\ncline[linestyle=dotted,linecolor=gray]{n58}{n126}
\ncline[linecolor=gray]{n59}{n114}
\ncline[linecolor=gray]{n47}{n59}
\ncline[linecolor=gray]{n21}{n61}
\ncline[linecolor=gray]{n61}{n129}
\ncline[linestyle=dotted,linecolor=gray]{n39}{n61}
\ncline[linecolor=gray]{n61}{n83}
\ncline[linecolor=gray]{n61}{n138}
\ncline[linestyle=dotted,linecolor=gray]{n61}{n89}
\ncline[linecolor=gray]{n61}{n131}
\ncline[linecolor=gray]{n61}{n118}
\ncline[linestyle=dotted,linecolor=gray]{n31}{n62}
\ncline[linecolor=gray]{n15}{n62}
\ncline[linecolor=gray]{n11}{n62}
\ncline[linecolor=gray]{n25}{n62}
\ncline[linecolor=gray]{n54}{n63}
\ncline[linecolor=gray]{n63}{n145}
\ncline[linestyle=dotted,linecolor=gray]{n63}{n121}
\ncline[linecolor=gray]{n0}{n64}
\ncline[linecolor=gray]{n64}{n114}
\ncline[linecolor=gray]{n40}{n64}
\ncline[linecolor=gray]{n64}{n126}
\ncline[linecolor=gray]{n65}{n126}
\ncline[linestyle=dotted,linecolor=gray]{n16}{n67}
\ncline[linecolor=gray]{n31}{n67}
\ncline[linecolor=gray]{n67}{n90}
\ncline[linecolor=gray]{n3}{n67}
\ncline[linecolor=gray]{n68}{n74}
\ncline[linecolor=gray]{n69}{n73}
\ncline[linecolor=gray]{n48}{n69}
\ncline[linestyle=dotted,linecolor=gray]{n31}{n69}
\ncline[linecolor=gray]{n15}{n69}
\ncline[linecolor=gray]{n69}{n145}
\ncline[linecolor=gray]{n69}{n135}
\ncline[linecolor=gray]{n70}{n148}
\ncline[linecolor=gray]{n71}{n132}
\ncline[linestyle=dotted,linecolor=gray]{n72}{n142}
\ncline[linecolor=gray]{n72}{n114}
\ncline[linecolor=gray]{n33}{n72}
\ncline[linecolor=gray]{n0}{n72}
\ncline[linecolor=gray]{n17}{n72}
\ncline[linestyle=dotted,linecolor=gray]{n72}{n112}
\ncline[linecolor=gray]{n36}{n72}
\ncline[linecolor=gray]{n50}{n72}
\ncline[linecolor=gray]{n69}{n73}
\ncline[linecolor=gray]{n48}{n73}
\ncline[linecolor=gray]{n73}{n145}
\ncline[linestyle=dotted,linecolor=gray]{n30}{n73}
\ncline[linecolor=gray]{n27}{n74}
\ncline[linecolor=gray]{n68}{n74}
\ncline[linecolor=gray]{n75}{n106}
\ncline[linecolor=gray]{n75}{n121}
\ncline[linestyle=dotted,linecolor=gray]{n75}{n105}
\ncline[linestyle=dotted,linecolor=gray]{n16}{n75}
\ncline[linecolor=gray]{n76}{n109}
\ncline[linecolor=gray]{n3}{n76}
\ncline[linecolor=gray]{n45}{n76}
\ncline[linecolor=gray]{n76}{n78}
\ncline[linecolor=gray]{n39}{n77}
\ncline[linecolor=gray]{n77}{n123}
\ncline[linecolor=gray]{n77}{n117}
\ncline[linestyle=dotted,linecolor=gray]{n77}{n101}
\ncline[linecolor=gray]{n77}{n119}
\ncline[linecolor=gray]{n31}{n78}
\ncline[linecolor=gray]{n52}{n78}
\ncline[linecolor=gray]{n76}{n78}
\ncline[linestyle=dotted,linecolor=gray]{n0}{n79}
\ncline[linecolor=gray]{n79}{n142}
\ncline[linestyle=dotted,linecolor=gray]{n79}{n126}
\ncline[linecolor=gray]{n58}{n79}
\ncline[linestyle=dotted,linecolor=gray]{n40}{n79}
\ncline[linestyle=dotted,linecolor=gray]{n79}{n144}
\ncline[linecolor=gray]{n81}{n85}
\ncline[linecolor=gray]{n34}{n81}
\ncline[linecolor=gray]{n82}{n104}
\ncline[linecolor=gray]{n82}{n122}
\ncline[linecolor=gray]{n41}{n82}
\ncline[linecolor=gray]{n82}{n112}
\ncline[linecolor=gray]{n2}{n82}
\ncline[linecolor=gray]{n82}{n143}
\ncline[linecolor=gray]{n82}{n137}
\ncline[linecolor=gray]{n82}{n142}
\ncline[linecolor=gray]{n83}{n129}
\ncline[linecolor=gray]{n61}{n83}
\ncline[linecolor=gray]{n83}{n131}
\ncline[linecolor=gray]{n84}{n122}
\ncline[linestyle=dotted,linecolor=gray]{n21}{n84}
\ncline[linecolor=gray]{n84}{n100}
\ncline[linestyle=dotted,linecolor=gray]{n84}{n102}
\ncline[linecolor=gray]{n81}{n85}
\ncline[linecolor=gray]{n85}{n96}
\ncline[linecolor=gray]{n20}{n86}
\ncline[linestyle=dotted,linecolor=gray]{n86}{n137}
\ncline[linecolor=gray]{n87}{n103}
\ncline[linecolor=gray]{n55}{n87}
\ncline[linecolor=gray]{n49}{n88}
\ncline[linestyle=dotted,linecolor=gray]{n88}{n135}
\ncline[linestyle=dotted,linecolor=gray]{n48}{n88}
\ncline[linestyle=dotted,linecolor=gray]{n37}{n88}
\ncline[linestyle=dotted,linecolor=gray]{n11}{n88}
\ncline[linestyle=dotted,linecolor=gray]{n88}{n127}
\ncline[linecolor=gray]{n30}{n88}
\ncline[linecolor=gray]{n88}{n130}
\ncline[linecolor=gray]{n13}{n89}
\ncline[linecolor=gray]{n89}{n119}
\ncline[linecolor=gray]{n89}{n117}
\ncline[linecolor=gray]{n89}{n123}
\ncline[linestyle=dotted,linecolor=gray]{n61}{n89}
\ncline[linecolor=gray]{n89}{n144}
\ncline[linecolor=gray]{n3}{n90}
\ncline[linecolor=gray]{n14}{n90}
\ncline[linecolor=gray]{n67}{n90}
\ncline[linecolor=gray]{n90}{n134}
\ncline[linestyle=dotted,linecolor=gray]{n90}{n111}
\ncline[linecolor=gray]{n27}{n90}
\ncline[linestyle=dotted,linecolor=gray]{n90}{n124}
\ncline[linecolor=gray]{n92}{n102}
\ncline[linecolor=gray]{n92}{n147}
\ncline[linecolor=gray]{n21}{n92}
\ncline[linecolor=gray]{n55}{n92}
\ncline[linecolor=gray]{n8}{n92}
\ncline[linecolor=gray]{n94}{n119}
\ncline[linecolor=gray]{n46}{n95}
\ncline[linecolor=gray]{n34}{n96}
\ncline[linecolor=gray]{n85}{n96}
\ncline[linecolor=gray]{n97}{n122}
\ncline[linecolor=gray]{n97}{n104}
\ncline[linecolor=gray]{n97}{n100}
\ncline[linecolor=gray]{n47}{n98}
\ncline[linecolor=gray]{n23}{n98}
\ncline[linecolor=gray]{n26}{n98}
\ncline[linecolor=gray]{n99}{n137}
\ncline[linecolor=gray]{n41}{n99}
\ncline[linestyle=dotted,linecolor=gray]{n13}{n99}
\ncline[linecolor=gray]{n99}{n122}
\ncline[linecolor=gray]{n100}{n122}
\ncline[linecolor=gray]{n1}{n100}
\ncline[linecolor=gray]{n100}{n143}
\ncline[linecolor=gray]{n84}{n100}
\ncline[linecolor=gray]{n97}{n100}
\ncline[linecolor=gray]{n101}{n142}
\ncline[linestyle=dotted,linecolor=gray]{n101}{n144}
\ncline[linestyle=dotted,linecolor=gray]{n77}{n101}
\ncline[linecolor=gray]{n101}{n143}
\ncline[linestyle=dotted,linecolor=gray]{n13}{n101}
\ncline[linecolor=gray]{n92}{n102}
\ncline[linestyle=dotted,linecolor=gray]{n84}{n102}
\ncline[linecolor=gray]{n102}{n147}
\ncline[linecolor=gray]{n8}{n103}
\ncline[linecolor=gray]{n42}{n103}
\ncline[linecolor=gray]{n6}{n103}
\ncline[linecolor=gray]{n87}{n103}
\ncline[linecolor=gray]{n56}{n103}
\ncline[linecolor=gray]{n82}{n104}
\ncline[linecolor=gray]{n104}{n112}
\ncline[linecolor=gray]{n104}{n143}
\ncline[linecolor=gray]{n97}{n104}
\ncline[linestyle=dotted,linecolor=gray]{n75}{n105}
\ncline[linecolor=gray]{n15}{n105}
\ncline[linecolor=gray]{n27}{n106}
\ncline[linecolor=gray]{n75}{n106}
\ncline[linecolor=gray]{n3}{n106}
\ncline[linecolor=gray]{n42}{n107}
\ncline[linecolor=gray]{n56}{n107}
\ncline[linestyle=dotted,linecolor=gray]{n46}{n107}
\ncline[linecolor=gray]{n76}{n109}
\ncline[linecolor=gray]{n23}{n110}
\ncline[linecolor=gray]{n48}{n111}
\ncline[linecolor=gray]{n15}{n111}
\ncline[linestyle=dotted,linecolor=gray]{n90}{n111}
\ncline[linecolor=gray]{n111}{n135}
\ncline[linecolor=gray]{n2}{n112}
\ncline[linecolor=gray]{n112}{n143}
\ncline[linecolor=gray]{n104}{n112}
\ncline[linecolor=gray]{n82}{n112}
\ncline[linestyle=dotted,linecolor=gray]{n72}{n112}
\ncline[linecolor=gray]{n41}{n112}
\ncline[linecolor=gray]{n11}{n113}
\ncline[linecolor=gray]{n113}{n127}
\ncline[linecolor=gray]{n72}{n114}
\ncline[linecolor=gray]{n44}{n114}
\ncline[linecolor=gray]{n23}{n114}
\ncline[linecolor=gray]{n50}{n114}
\ncline[linecolor=gray]{n59}{n114}
\ncline[linecolor=gray]{n64}{n114}
\ncline[linecolor=gray]{n12}{n114}
\ncline[linecolor=gray]{n114}{n139}
\ncline[linecolor=gray]{n115}{n130}
\ncline[linecolor=gray]{n22}{n115}
\ncline[linecolor=gray]{n30}{n115}
\ncline[linecolor=gray]{n115}{n133}
\ncline[linecolor=gray]{n117}{n123}
\ncline[linecolor=gray]{n117}{n119}
\ncline[linecolor=gray]{n89}{n117}
\ncline[linecolor=gray]{n77}{n117}
\ncline[linecolor=gray]{n13}{n117}
\ncline[linecolor=gray]{n21}{n118}
\ncline[linecolor=gray]{n118}{n129}
\ncline[linecolor=gray]{n61}{n118}
\ncline[linecolor=gray]{n117}{n119}
\ncline[linecolor=gray]{n89}{n119}
\ncline[linecolor=gray]{n39}{n119}
\ncline[linecolor=gray]{n119}{n123}
\ncline[linecolor=gray]{n94}{n119}
\ncline[linecolor=gray]{n77}{n119}
\ncline[linecolor=gray]{n20}{n120}
\ncline[linecolor=gray]{n120}{n123}
\ncline[linecolor=gray]{n75}{n121}
\ncline[linestyle=dotted,linecolor=gray]{n63}{n121}
\ncline[linecolor=gray]{n97}{n122}
\ncline[linecolor=gray]{n100}{n122}
\ncline[linecolor=gray]{n122}{n128}
\ncline[linecolor=gray]{n82}{n122}
\ncline[linecolor=gray]{n84}{n122}
\ncline[linecolor=gray]{n122}{n143}
\ncline[linecolor=gray]{n99}{n122}
\ncline[linecolor=gray]{n117}{n123}
\ncline[linecolor=gray]{n7}{n123}
\ncline[linecolor=gray]{n120}{n123}
\ncline[linecolor=gray]{n20}{n123}
\ncline[linecolor=gray]{n89}{n123}
\ncline[linecolor=gray]{n13}{n123}
\ncline[linecolor=gray]{n119}{n123}
\ncline[linecolor=gray]{n77}{n123}
\ncline[linecolor=gray]{n11}{n124}
\ncline[linecolor=gray]{n124}{n127}
\ncline[linestyle=dotted,linecolor=gray]{n31}{n124}
\ncline[linestyle=dotted,linecolor=gray]{n90}{n124}
\ncline[linecolor=gray]{n5}{n125}
\ncline[linestyle=dotted,linecolor=gray]{n16}{n125}
\ncline[linestyle=dotted,linecolor=gray]{n15}{n125}
\ncline[linecolor=gray]{n44}{n126}
\ncline[linecolor=gray]{n12}{n126}
\ncline[linestyle=dotted,linecolor=gray]{n79}{n126}
\ncline[linecolor=gray]{n17}{n126}
\ncline[linecolor=gray]{n65}{n126}
\ncline[linestyle=dotted,linecolor=gray]{n58}{n126}
\ncline[linecolor=gray]{n0}{n126}
\ncline[linecolor=gray]{n64}{n126}
\ncline[linecolor=gray]{n113}{n127}
\ncline[linecolor=gray]{n127}{n135}
\ncline[linestyle=dotted,linecolor=gray]{n127}{n130}
\ncline[linecolor=gray]{n37}{n127}
\ncline[linecolor=gray]{n124}{n127}
\ncline[linecolor=gray]{n127}{n136}
\ncline[linestyle=dotted,linecolor=gray]{n88}{n127}
\ncline[linecolor=gray]{n122}{n128}
\ncline[linecolor=gray]{n128}{n142}
\ncline[linecolor=gray]{n83}{n129}
\ncline[linecolor=gray]{n61}{n129}
\ncline[linecolor=gray]{n129}{n138}
\ncline[linecolor=gray]{n118}{n129}
\ncline[linecolor=gray]{n42}{n129}
\ncline[linecolor=gray]{n115}{n130}
\ncline[linestyle=dotted,linecolor=gray]{n127}{n130}
\ncline[linestyle=dotted,linecolor=gray]{n11}{n130}
\ncline[linecolor=gray]{n30}{n130}
\ncline[linecolor=gray]{n88}{n130}
\ncline[linecolor=gray]{n83}{n131}
\ncline[linecolor=gray]{n61}{n131}
\ncline[linecolor=gray]{n71}{n132}
\ncline[linecolor=gray]{n115}{n133}
\ncline[linestyle=dotted,linecolor=gray]{n16}{n134}
\ncline[linecolor=gray]{n31}{n134}
\ncline[linecolor=gray]{n90}{n134}
\ncline[linecolor=gray]{n3}{n134}
\ncline[linecolor=gray]{n11}{n135}
\ncline[linecolor=gray]{n127}{n135}
\ncline[linestyle=dotted,linecolor=gray]{n30}{n135}
\ncline[linestyle=dotted,linecolor=gray]{n88}{n135}
\ncline[linecolor=gray]{n54}{n135}
\ncline[linecolor=gray]{n135}{n145}
\ncline[linecolor=gray]{n10}{n135}
\ncline[linecolor=gray]{n69}{n135}
\ncline[linestyle=dotted,linecolor=gray]{n49}{n135}
\ncline[linecolor=gray]{n111}{n135}
\ncline[linecolor=gray]{n54}{n136}
\ncline[linecolor=gray]{n25}{n136}
\ncline[linecolor=gray]{n127}{n136}
\ncline[linecolor=gray]{n99}{n137}
\ncline[linestyle=dotted,linecolor=gray]{n86}{n137}
\ncline[linecolor=gray]{n82}{n137}
\ncline[linecolor=gray]{n21}{n138}
\ncline[linecolor=gray]{n61}{n138}
\ncline[linecolor=gray]{n129}{n138}
\ncline[linecolor=gray]{n42}{n138}
\ncline[linecolor=gray]{n33}{n139}
\ncline[linecolor=gray]{n47}{n139}
\ncline[linecolor=gray]{n114}{n139}
\ncline[linecolor=gray]{n56}{n141}
\ncline[linestyle=dotted,linecolor=gray]{n72}{n142}
\ncline[linecolor=gray]{n41}{n142}
\ncline[linecolor=gray]{n79}{n142}
\ncline[linestyle=dotted,linecolor=gray]{n0}{n142}
\ncline[linecolor=gray]{n101}{n142}
\ncline[linecolor=gray]{n128}{n142}
\ncline[linecolor=gray]{n58}{n142}
\ncline[linecolor=gray]{n82}{n142}
\ncline[linecolor=gray]{n100}{n143}
\ncline[linecolor=gray]{n122}{n143}
\ncline[linecolor=gray]{n112}{n143}
\ncline[linecolor=gray]{n104}{n143}
\ncline[linecolor=gray]{n82}{n143}
\ncline[linecolor=gray]{n101}{n143}
\ncline[linecolor=gray]{n41}{n143}
\ncline[linestyle=dotted,linecolor=gray]{n79}{n144}
\ncline[linecolor=gray]{n89}{n144}
\ncline[linestyle=dotted,linecolor=gray]{n101}{n144}
\ncline[linestyle=dotted,linecolor=gray]{n41}{n144}
\ncline[linecolor=gray]{n69}{n145}
\ncline[linecolor=gray]{n135}{n145}
\ncline[linecolor=gray]{n145}{n149}
\ncline[linecolor=gray]{n48}{n145}
\ncline[linecolor=gray]{n73}{n145}
\ncline[linecolor=gray]{n63}{n145}
\ncline[linecolor=gray]{n50}{n146}
\ncline[linecolor=gray]{n23}{n146}
\ncline[linecolor=gray]{n17}{n146}
\ncline[linecolor=gray]{n21}{n147}
\ncline[linecolor=gray]{n56}{n147}
\ncline[linecolor=gray]{n42}{n147}
\ncline[linecolor=gray]{n92}{n147}
\ncline[linecolor=gray]{n8}{n147}
\ncline[linecolor=gray]{n102}{n147}
\ncline[linecolor=gray]{n70}{n148}
\ncline[linecolor=gray]{n0}{n50}
\ncline[linecolor=gray]{n1}{n97}
\ncline[linecolor=gray]{n1}{n122}
\ncline[linecolor=gray]{n1}{n84}
\ncline[linecolor=gray]{n2}{n128}
\ncline[linecolor=gray]{n2}{n143}
\ncline[linecolor=gray]{n2}{n104}
\ncline[linecolor=gray]{n2}{n100}
\ncline[linecolor=gray]{n2}{n122}
\ncline[linecolor=gray]{n3}{n5}
\ncline[linecolor=gray]{n3}{n27}
\ncline[linecolor=gray]{n3}{n91}
\ncline[linecolor=gray]{n3}{n74}
\ncline[linecolor=gray]{n3}{n75}
\ncline[linecolor=gray]{n3}{n52}
\ncline[linecolor=gray]{n3}{n45}
\ncline[linecolor=gray]{n3}{n78}
\ncline[linecolor=gray]{n3}{n125}
\ncline[linecolor=gray]{n3}{n5}
\ncline[linecolor=gray]{n5}{n27}
\ncline[linecolor=gray]{n5}{n45}
\ncline[linecolor=gray]{n5}{n74}
\ncline[linecolor=gray]{n5}{n78}
\ncline[linecolor=gray]{n5}{n91}
\ncline[linecolor=gray]{n5}{n75}
\ncline[linecolor=gray]{n5}{n106}
\ncline[linecolor=gray]{n5}{n14}
\ncline[linecolor=gray]{n5}{n31}
\ncline[linecolor=gray]{n5}{n67}
\ncline[linecolor=gray]{n5}{n134}
\ncline[linecolor=gray]{n6}{n131}
\ncline[linecolor=gray]{n6}{n129}
\ncline[linecolor=gray]{n7}{n120}
\ncline[linecolor=gray]{n7}{n32}
\ncline[linecolor=gray]{n7}{n51}
\ncline[linecolor=gray]{n8}{n87}
\ncline[linecolor=gray]{n8}{n107}
\ncline[linecolor=gray]{n10}{n11}
\ncline[linecolor=gray]{n10}{n127}
\ncline[linecolor=gray]{n10}{n136}
\ncline[linecolor=gray]{n10}{n111}
\ncline[linecolor=gray]{n10}{n48}
\ncline[linecolor=gray]{n10}{n113}
\ncline[linecolor=gray]{n10}{n124}
\ncline[linecolor=gray]{n10}{n25}
\ncline[linecolor=gray]{n10}{n62}
\ncline[linecolor=gray]{n10}{n11}
\ncline[linecolor=gray]{n11}{n127}
\ncline[linecolor=gray]{n11}{n48}
\ncline[linecolor=gray]{n11}{n111}
\ncline[linecolor=gray]{n11}{n136}
\ncline[linecolor=gray]{n12}{n50}
\ncline[linecolor=gray]{n12}{n64}
\ncline[linecolor=gray]{n12}{n17}
\ncline[linecolor=gray]{n13}{n144}
\ncline[linecolor=gray]{n13}{n77}
\ncline[linecolor=gray]{n13}{n20}
\ncline[linecolor=gray]{n13}{n86}
\ncline[linecolor=gray]{n14}{n67}
\ncline[linecolor=gray]{n14}{n134}
\ncline[linecolor=gray]{n14}{n45}
\ncline[linecolor=gray]{n14}{n106}
\ncline[linecolor=gray]{n14}{n125}
\ncline[linecolor=gray]{n14}{n121}
\ncline[linestyle=dotted,linecolor=gray]{n14}{n18}
\ncline[linecolor=gray]{n14}{n75}
\ncline[linecolor=gray]{n14}{n78}
\ncline[linecolor=gray]{n5}{n14}
\ncline[linecolor=gray]{n14}{n27}
\ncline[linecolor=gray]{n15}{n63}
\ncline[linecolor=gray]{n15}{n16}
\ncline[linecolor=gray]{n15}{n25}
\ncline[linecolor=gray]{n15}{n149}
\ncline[linecolor=gray]{n15}{n136}
\ncline[linestyle=dotted,linecolor=gray]{n15}{n90}
\ncline[linecolor=gray]{n15}{n18}
\ncline[linecolor=gray]{n15}{n124}
\ncline[linecolor=gray]{n15}{n113}
\ncline[linecolor=gray]{n16}{n149}
\ncline[linecolor=gray]{n16}{n18}
\ncline[linecolor=gray]{n16}{n124}
\ncline[linecolor=gray]{n16}{n136}
\ncline[linecolor=gray]{n15}{n16}
\ncline[linecolor=gray]{n16}{n69}
\ncline[linestyle=dotted,linecolor=gray]{n16}{n90}
\ncline[linecolor=gray]{n16}{n111}
\ncline[linecolor=gray]{n16}{n113}
\ncline[linecolor=gray]{n16}{n62}
\ncline[linecolor=gray]{n17}{n50}
\ncline[linecolor=gray]{n17}{n44}
\ncline[linecolor=gray]{n17}{n64}
\ncline[linecolor=gray]{n12}{n17}
\ncline[linecolor=gray]{n17}{n33}
\ncline[linecolor=gray]{n17}{n114}
\ncline[linecolor=gray]{n16}{n18}
\ncline[linecolor=gray]{n18}{n149}
\ncline[linestyle=dotted,linecolor=gray]{n18}{n31}
\ncline[linecolor=gray]{n18}{n69}
\ncline[linecolor=gray]{n18}{n124}
\ncline[linecolor=gray]{n18}{n62}
\ncline[linecolor=gray]{n18}{n136}
\ncline[linestyle=dotted,linecolor=gray]{n14}{n18}
\ncline[linestyle=dotted,linecolor=gray]{n18}{n67}
\ncline[linestyle=dotted,linecolor=gray]{n18}{n134}
\ncline[linecolor=gray]{n15}{n18}
\ncline[linestyle=dotted,linecolor=gray]{n18}{n90}
\ncline[linecolor=gray]{n18}{n111}
\ncline[linecolor=gray]{n18}{n113}
\ncline[linestyle=dotted,linecolor=gray]{n18}{n106}
\ncline[linecolor=gray]{n13}{n20}
\ncline[linecolor=gray]{n20}{n144}
\ncline[linecolor=gray]{n21}{n102}
\ncline[linecolor=gray]{n23}{n26}
\ncline[linecolor=gray]{n23}{n35}
\ncline[linecolor=gray]{n23}{n44}
\ncline[linecolor=gray]{n23}{n139}
\ncline[linecolor=gray]{n15}{n25}
\ncline[linecolor=gray]{n25}{n63}
\ncline[linecolor=gray]{n10}{n25}
\ncline[linecolor=gray]{n23}{n26}
\ncline[linecolor=gray]{n26}{n139}
\ncline[linecolor=gray]{n26}{n35}
\ncline[linecolor=gray]{n3}{n27}
\ncline[linecolor=gray]{n27}{n75}
\ncline[linecolor=gray]{n27}{n125}
\ncline[linecolor=gray]{n5}{n27}
\ncline[linecolor=gray]{n27}{n45}
\ncline[linecolor=gray]{n27}{n78}
\ncline[linecolor=gray]{n27}{n52}
\ncline[linecolor=gray]{n27}{n91}
\ncline[linecolor=gray]{n27}{n76}
\ncline[linecolor=gray]{n14}{n27}
\ncline[linecolor=gray]{n27}{n67}
\ncline[linecolor=gray]{n27}{n134}
\ncline[linecolor=gray]{n29}{n107}
\ncline[linecolor=gray]{n29}{n103}
\ncline[linecolor=gray]{n29}{n87}
\ncline[linestyle=dotted,linecolor=gray]{n18}{n31}
\ncline[linecolor=gray]{n31}{n106}
\ncline[linecolor=gray]{n5}{n31}
\ncline[linecolor=gray]{n7}{n32}
\ncline[linecolor=gray]{n32}{n120}
\ncline[linecolor=gray]{n33}{n114}
\ncline[linecolor=gray]{n33}{n36}
\ncline[linecolor=gray]{n17}{n33}
\ncline[linecolor=gray]{n33}{n35}
\ncline[linecolor=gray]{n33}{n44}
\ncline[linecolor=gray]{n35}{n139}
\ncline[linecolor=gray]{n23}{n35}
\ncline[linecolor=gray]{n35}{n98}
\ncline[linecolor=gray]{n33}{n35}
\ncline[linecolor=gray]{n35}{n114}
\ncline[linecolor=gray]{n26}{n35}
\ncline[linecolor=gray]{n33}{n36}
\ncline[linecolor=gray]{n36}{n114}
\ncline[linecolor=gray]{n38}{n93}
\ncline[linecolor=gray]{n39}{n89}
\ncline[linecolor=gray]{n40}{n65}
\ncline[linecolor=gray]{n40}{n126}
\ncline[linecolor=gray]{n41}{n101}
\ncline[linecolor=gray]{n41}{n79}
\ncline[linecolor=gray]{n41}{n58}
\ncline[linecolor=gray]{n42}{n56}
\ncline[linecolor=gray]{n43}{n95}
\ncline[linecolor=gray]{n17}{n44}
\ncline[linecolor=gray]{n44}{n50}
\ncline[linecolor=gray]{n44}{n146}
\ncline[linecolor=gray]{n23}{n44}
\ncline[linecolor=gray]{n44}{n64}
\ncline[linecolor=gray]{n33}{n44}
\ncline[linecolor=gray]{n45}{n106}
\ncline[linecolor=gray]{n14}{n45}
\ncline[linecolor=gray]{n45}{n67}
\ncline[linecolor=gray]{n45}{n134}
\ncline[linecolor=gray]{n45}{n78}
\ncline[linecolor=gray]{n45}{n125}
\ncline[linecolor=gray]{n5}{n45}
\ncline[linecolor=gray]{n27}{n45}
\ncline[linecolor=gray]{n3}{n45}
\ncline[linecolor=gray]{n45}{n121}
\ncline[linecolor=gray]{n45}{n75}
\ncline[linecolor=gray]{n48}{n135}
\ncline[linecolor=gray]{n48}{n113}
\ncline[linecolor=gray]{n11}{n48}
\ncline[linecolor=gray]{n48}{n127}
\ncline[linecolor=gray]{n10}{n48}
\ncline[linecolor=gray]{n48}{n124}
\ncline[linecolor=gray]{n48}{n136}
\ncline[linecolor=gray]{n12}{n50}
\ncline[linecolor=gray]{n17}{n50}
\ncline[linecolor=gray]{n50}{n64}
\ncline[linecolor=gray]{n44}{n50}
\ncline[linecolor=gray]{n0}{n50}
\ncline[linecolor=gray]{n50}{n126}
\ncline[linecolor=gray]{n7}{n51}
\ncline[linecolor=gray]{n51}{n120}
\ncline[linecolor=gray]{n52}{n76}
\ncline[linecolor=gray]{n52}{n91}
\ncline[linecolor=gray]{n52}{n109}
\ncline[linecolor=gray]{n3}{n52}
\ncline[linecolor=gray]{n27}{n52}
\ncline[linecolor=gray]{n54}{n145}
\ncline[linecolor=gray]{n54}{n69}
\ncline[linecolor=gray]{n54}{n113}
\ncline[linecolor=gray]{n54}{n73}
\ncline[linecolor=gray]{n55}{n80}
\ncline[linecolor=gray]{n42}{n56}
\ncline[linecolor=gray]{n58}{n101}
\ncline[linecolor=gray]{n41}{n58}
\ncline[linecolor=gray]{n62}{n136}
\ncline[linecolor=gray]{n62}{n124}
\ncline[linecolor=gray]{n62}{n111}
\ncline[linecolor=gray]{n16}{n62}
\ncline[linecolor=gray]{n18}{n62}
\ncline[linecolor=gray]{n62}{n69}
\ncline[linecolor=gray]{n62}{n149}
\ncline[linecolor=gray]{n62}{n113}
\ncline[linecolor=gray]{n10}{n62}
\ncline[linecolor=gray]{n15}{n63}
\ncline[linecolor=gray]{n25}{n63}
\ncline[linecolor=gray]{n12}{n64}
\ncline[linecolor=gray]{n50}{n64}
\ncline[linecolor=gray]{n17}{n64}
\ncline[linecolor=gray]{n44}{n64}
\ncline[linecolor=gray]{n40}{n65}
\ncline[linecolor=gray]{n14}{n67}
\ncline[linecolor=gray]{n67}{n134}
\ncline[linecolor=gray]{n45}{n67}
\ncline[linecolor=gray]{n67}{n106}
\ncline[linecolor=gray]{n67}{n125}
\ncline[linecolor=gray]{n67}{n121}
\ncline[linestyle=dotted,linecolor=gray]{n18}{n67}
\ncline[linecolor=gray]{n67}{n75}
\ncline[linecolor=gray]{n67}{n78}
\ncline[linecolor=gray]{n5}{n67}
\ncline[linecolor=gray]{n27}{n67}
\ncline[linecolor=gray]{n69}{n113}
\ncline[linecolor=gray]{n69}{n124}
\ncline[linecolor=gray]{n69}{n149}
\ncline[linecolor=gray]{n69}{n136}
\ncline[linecolor=gray]{n16}{n69}
\ncline[linecolor=gray]{n18}{n69}
\ncline[linecolor=gray]{n62}{n69}
\ncline[linecolor=gray]{n69}{n111}
\ncline[linecolor=gray]{n54}{n69}
\ncline[linecolor=gray]{n73}{n135}
\ncline[linecolor=gray]{n54}{n73}
\ncline[linecolor=gray]{n3}{n74}
\ncline[linecolor=gray]{n5}{n74}
\ncline[linecolor=gray]{n74}{n90}
\ncline[linecolor=gray]{n74}{n91}
\ncline[linecolor=gray]{n74}{n75}
\ncline[linecolor=gray]{n75}{n125}
\ncline[linecolor=gray]{n27}{n75}
\ncline[linecolor=gray]{n75}{n90}
\ncline[linecolor=gray]{n3}{n75}
\ncline[linecolor=gray]{n14}{n75}
\ncline[linecolor=gray]{n67}{n75}
\ncline[linecolor=gray]{n75}{n134}
\ncline[linecolor=gray]{n5}{n75}
\ncline[linecolor=gray]{n45}{n75}
\ncline[linecolor=gray]{n74}{n75}
\ncline[linecolor=gray]{n75}{n78}
\ncline[linecolor=gray]{n52}{n76}
\ncline[linecolor=gray]{n27}{n76}
\ncline[linecolor=gray]{n76}{n91}
\ncline[linecolor=gray]{n77}{n89}
\ncline[linecolor=gray]{n13}{n77}
\ncline[linecolor=gray]{n77}{n144}
\ncline[linecolor=gray]{n78}{n106}
\ncline[linecolor=gray]{n45}{n78}
\ncline[linecolor=gray]{n78}{n125}
\ncline[linecolor=gray]{n5}{n78}
\ncline[linecolor=gray]{n14}{n78}
\ncline[linecolor=gray]{n27}{n78}
\ncline[linecolor=gray]{n67}{n78}
\ncline[linecolor=gray]{n78}{n134}
\ncline[linecolor=gray]{n3}{n78}
\ncline[linecolor=gray]{n78}{n121}
\ncline[linecolor=gray]{n75}{n78}
\ncline[linecolor=gray]{n41}{n79}
\ncline[linecolor=gray]{n79}{n101}
\ncline[linecolor=gray]{n55}{n80}
\ncline[linecolor=gray]{n80}{n92}
\ncline[linecolor=gray]{n81}{n96}
\ncline[linecolor=gray]{n82}{n99}
\ncline[linecolor=gray]{n82}{n128}
\ncline[linecolor=gray]{n83}{n118}
\ncline[linecolor=gray]{n83}{n138}
\ncline[linecolor=gray]{n1}{n84}
\ncline[linecolor=gray]{n86}{n144}
\ncline[linecolor=gray]{n13}{n86}
\ncline[linecolor=gray]{n8}{n87}
\ncline[linecolor=gray]{n87}{n107}
\ncline[linecolor=gray]{n29}{n87}
\ncline[linecolor=gray]{n77}{n89}
\ncline[linecolor=gray]{n39}{n89}
\ncline[linecolor=gray]{n75}{n90}
\ncline[linestyle=dotted,linecolor=gray]{n16}{n90}
\ncline[linestyle=dotted,linecolor=gray]{n15}{n90}
\ncline[linestyle=dotted,linecolor=gray]{n90}{n105}
\ncline[linecolor=gray]{n90}{n125}
\ncline[linestyle=dotted,linecolor=gray]{n18}{n90}
\ncline[linecolor=gray]{n74}{n90}
\ncline[linestyle=dotted,linecolor=gray]{n90}{n149}
\ncline[linecolor=gray]{n52}{n91}
\ncline[linecolor=gray]{n3}{n91}
\ncline[linecolor=gray]{n27}{n91}
\ncline[linecolor=gray]{n5}{n91}
\ncline[linecolor=gray]{n74}{n91}
\ncline[linecolor=gray]{n76}{n91}
\ncline[linecolor=gray]{n91}{n109}
\ncline[linecolor=gray]{n80}{n92}
\ncline[linecolor=gray]{n38}{n93}
\ncline[linecolor=gray]{n43}{n95}
\ncline[linecolor=gray]{n81}{n96}
\ncline[linecolor=gray]{n1}{n97}
\ncline[linecolor=gray]{n98}{n139}
\ncline[linecolor=gray]{n35}{n98}
\ncline[linecolor=gray]{n82}{n99}
\ncline[linecolor=gray]{n2}{n100}
\ncline[linecolor=gray]{n41}{n101}
\ncline[linecolor=gray]{n58}{n101}
\ncline[linecolor=gray]{n79}{n101}
\ncline[linecolor=gray]{n21}{n102}
\ncline[linecolor=gray]{n103}{n107}
\ncline[linecolor=gray]{n29}{n103}
\ncline[linecolor=gray]{n104}{n128}
\ncline[linecolor=gray]{n2}{n104}
\ncline[linecolor=gray]{n104}{n122}
\ncline[linestyle=dotted,linecolor=gray]{n90}{n105}
\ncline[linecolor=gray]{n45}{n106}
\ncline[linecolor=gray]{n78}{n106}
\ncline[linecolor=gray]{n14}{n106}
\ncline[linecolor=gray]{n67}{n106}
\ncline[linecolor=gray]{n106}{n134}
\ncline[linecolor=gray]{n106}{n121}
\ncline[linecolor=gray]{n106}{n125}
\ncline[linecolor=gray]{n31}{n106}
\ncline[linecolor=gray]{n5}{n106}
\ncline[linestyle=dotted,linecolor=gray]{n18}{n106}
\ncline[linecolor=gray]{n29}{n107}
\ncline[linecolor=gray]{n8}{n107}
\ncline[linecolor=gray]{n103}{n107}
\ncline[linecolor=gray]{n87}{n107}
\ncline[linecolor=gray]{n52}{n109}
\ncline[linecolor=gray]{n91}{n109}
\ncline[linecolor=gray]{n111}{n124}
\ncline[linecolor=gray]{n62}{n111}
\ncline[linecolor=gray]{n111}{n136}
\ncline[linecolor=gray]{n16}{n111}
\ncline[linecolor=gray]{n111}{n113}
\ncline[linecolor=gray]{n111}{n149}
\ncline[linecolor=gray]{n11}{n111}
\ncline[linecolor=gray]{n10}{n111}
\ncline[linecolor=gray]{n69}{n111}
\ncline[linecolor=gray]{n111}{n127}
\ncline[linecolor=gray]{n18}{n111}
\ncline[linecolor=gray]{n112}{n142}
\ncline[linecolor=gray]{n112}{n128}
\ncline[linecolor=gray]{n69}{n113}
\ncline[linecolor=gray]{n113}{n124}
\ncline[linecolor=gray]{n113}{n149}
\ncline[linecolor=gray]{n113}{n136}
\ncline[linecolor=gray]{n16}{n113}
\ncline[linecolor=gray]{n48}{n113}
\ncline[linecolor=gray]{n111}{n113}
\ncline[linecolor=gray]{n113}{n135}
\ncline[linecolor=gray]{n10}{n113}
\ncline[linecolor=gray]{n18}{n113}
\ncline[linecolor=gray]{n54}{n113}
\ncline[linecolor=gray]{n62}{n113}
\ncline[linecolor=gray]{n113}{n145}
\ncline[linecolor=gray]{n15}{n113}
\ncline[linecolor=gray]{n33}{n114}
\ncline[linecolor=gray]{n17}{n114}
\ncline[linecolor=gray]{n36}{n114}
\ncline[linecolor=gray]{n35}{n114}
\ncline[linecolor=gray]{n116}{n132}
\ncline[linecolor=gray]{n118}{n138}
\ncline[linecolor=gray]{n83}{n118}
\ncline[linecolor=gray]{n7}{n120}
\ncline[linecolor=gray]{n32}{n120}
\ncline[linecolor=gray]{n51}{n120}
\ncline[linecolor=gray]{n14}{n121}
\ncline[linecolor=gray]{n67}{n121}
\ncline[linecolor=gray]{n121}{n134}
\ncline[linecolor=gray]{n106}{n121}
\ncline[linecolor=gray]{n121}{n125}
\ncline[linecolor=gray]{n45}{n121}
\ncline[linecolor=gray]{n78}{n121}
\ncline[linecolor=gray]{n1}{n122}
\ncline[linecolor=gray]{n104}{n122}
\ncline[linecolor=gray]{n2}{n122}
\ncline[linecolor=gray]{n124}{n136}
\ncline[linecolor=gray]{n111}{n124}
\ncline[linecolor=gray]{n69}{n124}
\ncline[linecolor=gray]{n113}{n124}
\ncline[linecolor=gray]{n124}{n149}
\ncline[linecolor=gray]{n62}{n124}
\ncline[linecolor=gray]{n16}{n124}
\ncline[linecolor=gray]{n18}{n124}
\ncline[linecolor=gray]{n10}{n124}
\ncline[linecolor=gray]{n48}{n124}
\ncline[linecolor=gray]{n15}{n124}
\ncline[linecolor=gray]{n75}{n125}
\ncline[linecolor=gray]{n14}{n125}
\ncline[linecolor=gray]{n67}{n125}
\ncline[linecolor=gray]{n125}{n134}
\ncline[linecolor=gray]{n27}{n125}
\ncline[linecolor=gray]{n121}{n125}
\ncline[linecolor=gray]{n45}{n125}
\ncline[linecolor=gray]{n78}{n125}
\ncline[linecolor=gray]{n106}{n125}
\ncline[linecolor=gray]{n90}{n125}
\ncline[linecolor=gray]{n3}{n125}
\ncline[linecolor=gray]{n40}{n126}
\ncline[linecolor=gray]{n50}{n126}
\ncline[linecolor=gray]{n126}{n146}
\ncline[linecolor=gray]{n11}{n127}
\ncline[linecolor=gray]{n10}{n127}
\ncline[linecolor=gray]{n48}{n127}
\ncline[linecolor=gray]{n111}{n127}
\ncline[linecolor=gray]{n104}{n128}
\ncline[linecolor=gray]{n2}{n128}
\ncline[linecolor=gray]{n112}{n128}
\ncline[linecolor=gray]{n82}{n128}
\ncline[linecolor=gray]{n128}{n143}
\ncline[linecolor=gray]{n129}{n131}
\ncline[linecolor=gray]{n6}{n129}
\ncline[linecolor=gray]{n129}{n131}
\ncline[linecolor=gray]{n6}{n131}
\ncline[linecolor=gray]{n116}{n132}
\ncline[linecolor=gray]{n14}{n134}
\ncline[linecolor=gray]{n67}{n134}
\ncline[linecolor=gray]{n45}{n134}
\ncline[linecolor=gray]{n106}{n134}
\ncline[linecolor=gray]{n125}{n134}
\ncline[linecolor=gray]{n121}{n134}
\ncline[linestyle=dotted,linecolor=gray]{n18}{n134}
\ncline[linecolor=gray]{n75}{n134}
\ncline[linecolor=gray]{n78}{n134}
\ncline[linecolor=gray]{n5}{n134}
\ncline[linecolor=gray]{n27}{n134}
\ncline[linecolor=gray]{n48}{n135}
\ncline[linecolor=gray]{n113}{n135}
\ncline[linecolor=gray]{n73}{n135}
\ncline[linecolor=gray]{n124}{n136}
\ncline[linecolor=gray]{n62}{n136}
\ncline[linecolor=gray]{n69}{n136}
\ncline[linecolor=gray]{n111}{n136}
\ncline[linecolor=gray]{n113}{n136}
\ncline[linecolor=gray]{n136}{n149}
\ncline[linecolor=gray]{n16}{n136}
\ncline[linecolor=gray]{n10}{n136}
\ncline[linecolor=gray]{n18}{n136}
\ncline[linecolor=gray]{n15}{n136}
\ncline[linecolor=gray]{n11}{n136}
\ncline[linecolor=gray]{n48}{n136}
\ncline[linecolor=gray]{n118}{n138}
\ncline[linecolor=gray]{n83}{n138}
\ncline[linecolor=gray]{n98}{n139}
\ncline[linecolor=gray]{n35}{n139}
\ncline[linecolor=gray]{n23}{n139}
\ncline[linecolor=gray]{n26}{n139}
\ncline[linecolor=gray]{n141}{n147}
\ncline[linecolor=gray]{n112}{n142}
\ncline[linecolor=gray]{n2}{n143}
\ncline[linecolor=gray]{n128}{n143}
\ncline[linecolor=gray]{n13}{n144}
\ncline[linecolor=gray]{n86}{n144}
\ncline[linecolor=gray]{n77}{n144}
\ncline[linecolor=gray]{n20}{n144}
\ncline[linecolor=gray]{n54}{n145}
\ncline[linecolor=gray]{n113}{n145}
\ncline[linecolor=gray]{n44}{n146}
\ncline[linecolor=gray]{n126}{n146}
\ncline[linecolor=gray]{n141}{n147}
\ncline[linecolor=black]{->}{n0}{n35}
\ncline[linecolor=black]{->}{n1}{n2}
\ncline[linecolor=black]{->}{n3}{n27}
\ncline[linecolor=black]{->}{n4}{n35}
\ncline[linecolor=black]{->}{n5}{n27}
\ncline[linecolor=black]{->}{n7}{n120}
\ncline[linecolor=black]{->}{n8}{n6}
\ncline[linecolor=black]{->}{n9}{n57}
\ncline[linecolor=black]{->}{n10}{n113}
\ncline[linecolor=black]{->}{n11}{n113}
\ncline[linecolor=black]{->}{n12}{n35}
\ncline[linecolor=black]{->}{n13}{n7}
\ncline[linecolor=black]{->}{n13}{n120}
\ncline[linecolor=black]{->}{n14}{n27}
\ncline[linecolor=black]{->}{n15}{n113}
\ncline[linecolor=black]{->}{n16}{n113}
\ncline[linecolor=black]{->}{n17}{n35}
\ncline[linecolor=black]{->}{n18}{n113}
\ncline[linecolor=black]{->}{n19}{n57}
\ncline[linecolor=black]{->}{n20}{n7}
\ncline[linecolor=black]{->}{n20}{n120}
\ncline[linecolor=black]{->}{n21}{n6}
\ncline[linecolor=black]{->}{n22}{n49}
\ncline[linecolor=black]{->}{n23}{n35}
\ncline[linecolor=black]{->}{n24}{n35}
\ncline[linecolor=black]{->}{n25}{n113}
\ncline[linecolor=black]{->}{n26}{n35}
\ncline[linecolor=black]{->}{n28}{n57}
\ncline[linecolor=black]{->}{n29}{n6}
\ncline[linecolor=black]{->}{n30}{n49}
\ncline[linecolor=black]{->}{n31}{n27}
\ncline[linecolor=black]{->}{n32}{n7}
\ncline[linecolor=black]{->}{n32}{n120}
\ncline[linecolor=black]{->}{n33}{n35}
\ncline[linecolor=black]{->}{n34}{n81}
\ncline[linecolor=black]{->}{n36}{n35}
\ncline[linecolor=black]{->}{n37}{n113}
\ncline[linecolor=black]{->}{n38}{n7}
\ncline[linecolor=black]{->}{n38}{n120}
\ncline[linecolor=black]{->}{n39}{n7}
\ncline[linecolor=black]{->}{n39}{n120}
\ncline[linecolor=black]{->}{n40}{n35}
\ncline[linecolor=black]{->}{n41}{n2}
\ncline[linecolor=black]{->}{n42}{n6}
\ncline[linecolor=black]{->}{n43}{n57}
\ncline[linecolor=black]{->}{n44}{n35}
\ncline[linecolor=black]{->}{n45}{n27}
\ncline[linecolor=black]{->}{n46}{n57}
\ncline[linecolor=black]{->}{n47}{n35}
\ncline[linecolor=black]{->}{n48}{n113}
\ncline[linecolor=black]{->}{n50}{n35}
\ncline[linecolor=black]{->}{n51}{n7}
\ncline[linecolor=black]{->}{n51}{n120}
\ncline[linecolor=black]{->}{n52}{n27}
\ncline[linecolor=black]{->}{n53}{n6}
\ncline[linecolor=black]{->}{n54}{n113}
\ncline[linecolor=black]{->}{n55}{n6}
\ncline[linecolor=black]{->}{n56}{n6}
\ncline[linecolor=black]{->}{n58}{n2}
\ncline[linecolor=black]{->}{n59}{n35}
\ncline[linecolor=black]{->}{n60}{n7}
\ncline[linecolor=black]{->}{n60}{n120}
\ncline[linecolor=black]{->}{n61}{n6}
\ncline[linecolor=black]{->}{n62}{n113}
\ncline[linecolor=black]{->}{n63}{n113}
\ncline[linecolor=black]{->}{n64}{n35}
\ncline[linecolor=black]{->}{n65}{n35}
\ncline[linecolor=black]{->}{n66}{n7}
\ncline[linecolor=black]{->}{n66}{n120}
\ncline[linecolor=black]{->}{n67}{n27}
\ncline[linecolor=black]{->}{n68}{n27}
\ncline[linecolor=black]{->}{n69}{n113}
\ncline[linecolor=black]{->}{n70}{n57}
\ncline[linecolor=black]{->}{n71}{n57}
\ncline[linecolor=black]{->}{n72}{n35}
\ncline[linecolor=black]{->}{n73}{n113}
\ncline[linecolor=black]{->}{n74}{n27}
\ncline[linecolor=black]{->}{n75}{n27}
\ncline[linecolor=black]{->}{n76}{n27}
\ncline[linecolor=black]{->}{n77}{n7}
\ncline[linecolor=black]{->}{n77}{n120}
\ncline[linecolor=black]{->}{n78}{n27}
\ncline[linecolor=black]{->}{n79}{n2}
\ncline[linecolor=black]{->}{n80}{n6}
\ncline[linecolor=black]{->}{n82}{n2}
\ncline[linecolor=black]{->}{n83}{n6}
\ncline[linecolor=black]{->}{n84}{n2}
\ncline[linecolor=black]{->}{n85}{n81}
\ncline[linecolor=black]{->}{n86}{n7}
\ncline[linecolor=black]{->}{n86}{n120}
\ncline[linecolor=black]{->}{n87}{n6}
\ncline[linecolor=black]{->}{n88}{n49}
\ncline[linecolor=black]{->}{n89}{n7}
\ncline[linecolor=black]{->}{n89}{n120}
\ncline[linecolor=black]{->}{n90}{n27}
\ncline[linecolor=black]{->}{n91}{n27}
\ncline[linecolor=black]{->}{n92}{n6}
\ncline[linecolor=black]{->}{n93}{n7}
\ncline[linecolor=black]{->}{n93}{n120}
\ncline[linecolor=black]{->}{n94}{n7}
\ncline[linecolor=black]{->}{n94}{n120}
\ncline[linecolor=black]{->}{n95}{n57}
\ncline[linecolor=black]{->}{n96}{n81}
\ncline[linecolor=black]{->}{n97}{n2}
\ncline[linecolor=black]{->}{n98}{n35}
\ncline[linecolor=black]{->}{n99}{n2}
\ncline[linecolor=black]{->}{n100}{n2}
\ncline[linecolor=black]{->}{n101}{n2}
\ncline[linecolor=black]{->}{n102}{n6}
\ncline[linecolor=black]{->}{n103}{n6}
\ncline[linecolor=black]{->}{n104}{n2}
\ncline[linecolor=black]{->}{n105}{n113}
\ncline[linecolor=black]{->}{n106}{n27}
\ncline[linecolor=black]{->}{n107}{n6}
\ncline[linecolor=black]{->}{n108}{n27}
\ncline[linecolor=black]{->}{n109}{n27}
\ncline[linecolor=black]{->}{n110}{n35}
\ncline[linecolor=black]{->}{n111}{n113}
\ncline[linecolor=black]{->}{n112}{n2}
\ncline[linecolor=black]{->}{n114}{n35}
\ncline[linecolor=black]{->}{n115}{n49}
\ncline[linecolor=black]{->}{n116}{n57}
\ncline[linecolor=black]{->}{n117}{n7}
\ncline[linecolor=black]{->}{n117}{n120}
\ncline[linecolor=black]{->}{n118}{n6}
\ncline[linecolor=black]{->}{n119}{n7}
\ncline[linecolor=black]{->}{n119}{n120}
\ncline[linecolor=black]{->}{n120}{n7}
\ncline[linecolor=black]{->}{n121}{n27}
\ncline[linecolor=black]{->}{n122}{n2}
\ncline[linecolor=black]{->}{n123}{n7}
\ncline[linecolor=black]{->}{n123}{n120}
\ncline[linecolor=black]{->}{n124}{n113}
\ncline[linecolor=black]{->}{n125}{n27}
\ncline[linecolor=black]{->}{n126}{n35}
\ncline[linecolor=black]{->}{n127}{n113}
\ncline[linecolor=black]{->}{n128}{n2}
\ncline[linecolor=black]{->}{n129}{n6}
\ncline[linecolor=black]{->}{n130}{n49}
\ncline[linecolor=black]{->}{n131}{n6}
\ncline[linecolor=black]{->}{n132}{n57}
\ncline[linecolor=black]{->}{n133}{n49}
\ncline[linecolor=black]{->}{n134}{n27}
\ncline[linecolor=black]{->}{n135}{n113}
\ncline[linecolor=black]{->}{n136}{n113}
\ncline[linecolor=black]{->}{n137}{n2}
\ncline[linecolor=black]{->}{n138}{n6}
\ncline[linecolor=black]{->}{n139}{n35}
\ncline[linecolor=black]{->}{n140}{n57}
\ncline[linecolor=black]{->}{n141}{n6}
\ncline[linecolor=black]{->}{n142}{n2}
\ncline[linecolor=black]{->}{n143}{n2}
\ncline[linecolor=black]{->}{n144}{n7}
\ncline[linecolor=black]{->}{n144}{n120}
\ncline[linecolor=black]{->}{n145}{n113}
\ncline[linecolor=black]{->}{n146}{n35}
\ncline[linecolor=black]{->}{n147}{n6}
\ncline[linecolor=black]{->}{n148}{n57}
\ncline[linecolor=black]{->}{n149}{n113}
\psset{dotstyle=Bo}
\dotnode[](1.108338,3.858886){n0}
\dotnode[](1.257860,2.892203){n4}
\dotnode[](0.3962556,3.984627){n12}
\dotnode[](0.6005647,3.371056){n17}
\dotnode[](0.711105,2.899006){n23}
\dotnode[](1.232538,4.29378){n24}
\dotnode[](0.9249148,2.893182){n26}
\dotnode[](0.1556468,2.998522){n33}
\dotnode[fillcolor=gray](0.3174471,2.600607){n35}
\dotnode[](0.1307679,3.277233){n36}
\dotnode[](0.9114814,4.353879){n40}
\dotnode[](0.4179513,3.522051){n44}
\dotnode[](0.1748816,2.281055){n47}
\dotnode[](0.5841016,3.811024){n50}
\dotnode[](0,2.563469){n59}
\dotnode[](0.2217693,3.701035){n64}
\dotnode[](0.6466549,4.297944){n65}
\dotnode[](1.004251,3.3599){n72}
\dotnode[](0.4990411,2.364772){n98}
\dotnode[](0.5000101,2.853756){n110}
\dotnode[](0.3549037,3.121353){n114}
\dotnode[](0.9849086,4.079775){n126}
\dotnode[](0.710692,2.532491){n139}
\dotnode[](0.7976457,3.6483){n146}
\psset{dotstyle=Bo}
\dotnode[](0.6785566,0.8802483){n3}
\dotnode[](0.8247274,0.631479){n5}
\dotnode[](1.439636,1.295117){n14}
\dotnode[fillcolor=gray](0.5726195,1.294175){n27}
\dotnode[](1.422528,0.3128956){n31}
\dotnode[](0.8306748,1.037910){n45}
\dotnode[](0.2341176,1.037886){n52}
\dotnode[](1.509306,0.9905115){n67}
\dotnode[](1.631241,0.08123197){n68}
\dotnode[](1.092846,0.4743804){n74}
\dotnode[](0.957797,1.457980){n75}
\dotnode[](0.4145735,1.587721){n76}
\dotnode[](0.8983155,1.243353){n78}
\dotnode[](1.765180,1.228386){n90}
\dotnode[](0.4301581,0.7624465){n91}
\dotnode[](1.073082,0.7635667){n106}
\dotnode[](0.1632209,1.576404){n108}
\dotnode[](0.1964432,1.278737){n109}
\dotnode[](1.377059,0.7879707){n121}
\dotnode[](1.203850,1.499658){n125}
\dotnode[](1.252724,1.071603){n134}
\psset{dotstyle=Bo}
\dotnode[](2.527444,2.445348){n1}
\dotnode[fillcolor=gray](2.017666,2.841472){n2}
\dotnode[](1.686579,3.819467){n41}
\dotnode[](1.967572,3.868347){n58}
\dotnode[](1.560778,4.1743){n79}
\dotnode[](2.078577,3.201014){n82}
\dotnode[](3.153883,3.140639){n84}
\dotnode[](2.802984,2.471939){n97}
\dotnode[](2.265986,3.781169){n99}
\dotnode[](2.886530,2.782621){n100}
\dotnode[](2.074548,4.120629){n101}
\dotnode[](2.161968,2.481101){n104}
\dotnode[](1.705716,3.114971){n112}
\dotnode[](2.617372,2.90596){n122}
\dotnode[](2.280932,2.838994){n128}
\dotnode[](1.689756,4.453583){n137}
\dotnode[](1.606368,3.440222){n142}
\dotnode[](2.445606,3.296266){n143}
\psset{dotstyle=Bo}
\dotnode[fillcolor=gray](3.098514,4.866756){n7}
\dotnode[](2.661068,4.46547){n13}
\dotnode[](2.577236,4.740179){n20}
\dotnode[](2.557950,5){n32}
\dotnode[](2.55,3.97){n38}
\dotnode[](3.353207,4.403764){n39}
\dotnode[](2.24852,4.942186){n51}
\dotnode[](3.764707,4.352432){n60}
\dotnode[](3.364475,4.745668){n66}
\dotnode[](2.550006,4.241430){n77}
\dotnode[](1.891242,4.717356){n86}
\dotnode[](3.202631,4.133673){n89}
\dotnode[](2.64,3.705){n93}
\dotnode[](2.869689,4.205452){n94}
\dotnode[](3.053188,3.925351){n117}
\dotnode[](2.890681,3.735371){n119}
\dotnode[fillcolor=gray](2.880736,4.879838){n120}
\dotnode[](3.046366,4.518936){n123}
\dotnode[](2.274105,4.535248){n144}
\psset{dotstyle=Bo}
\dotnode[fillcolor=gray](4.581229,3.210751){n6}
\dotnode[](4.591683,2.878327){n8}
\dotnode[](3.773692,3.552994){n21}
\dotnode[](4.7622,2.335772){n29}
\dotnode[](4.209552,3.016675){n42}
\dotnode[](4.969442,3.079323){n53}
\dotnode[](5,2.809783){n55}
\dotnode[](3.97574,2.571348){n56}
\dotnode[](3.737104,3.981957){n61}
\dotnode[](4.791386,3.470398){n80}
\dotnode[](4.135811,4.087185){n83}
\dotnode[](4.792091,2.606388){n87}
\dotnode[](4.292557,3.369169){n92}
\dotnode[](3.578431,3.339056){n102}
\dotnode[](4.396008,2.731498){n103}
\dotnode[](4.351759,2.410157){n107}
\dotnode[](4.318411,3.852927){n118}
\dotnode[](4.117188,3.717776){n129}
\dotnode[](4.500053,3.637197){n131}
\dotnode[](3.989081,3.406712){n138}
\dotnode[](3.645946,2.645275){n141}
\dotnode[](3.950723,2.918283){n147}
\psset{dotstyle=Bo}
\dotnode[](2.888154,1.090116){n10}
\dotnode[](3.105909,0.7401545){n11}
\dotnode[](2.236885,1.274193){n15}
\dotnode[](2.024484,0.7152085){n16}
\dotnode[](1.814559,0.4981684){n18}
\dotnode[](2.521529,1.412281){n25}
\dotnode[](2.901075,1.575363){n37}
\dotnode[](3.273419,0.6181322){n48}
\dotnode[](2.864019,0){n54}
\dotnode[](2.127025,0.3401537){n62}
\dotnode[](2.312403,1.030709){n63}
\dotnode[](2.363231,0.03982867){n69}
\dotnode[](3.170304,0.08031329){n73}
\dotnode[](1.915954,1.686363){n105}
\dotnode[](2.625767,0.8297721){n111}
\dotnode[fillcolor=gray](2.796314,0.5814426){n113}
\dotnode[](2.621703,0.4200536){n124}
\dotnode[](3.254325,1.04113){n127}
\dotnode[](3.385955,0.3694225){n135}
\dotnode[](2.407195,0.4605674){n136}
\dotnode[](2.648264,0.1809270){n145}
\dotnode[](2.296867,0.7074797){n149}
\psset{dotstyle=Bo}
\dotnode[](4,2.2){n9}
\dotnode[](4.08826,1.970573){n19}
\dotnode[](3.759983,2.172351){n28}
\dotnode[](3.891111,1.748183){n43}
\dotnode[](4.389942,2.082443){n46}
\dotnode[fillcolor=gray](4.371287,1.534671){n57}
\dotnode[](4.688611,1.662437){n70}
\dotnode[](4.32925,1.149353){n71}
\dotnode[](4.306428,1.798957){n95}
\dotnode[](4.562201,1.223617){n116}
\dotnode[](4.069486,1.411759){n132}
\dotnode[](4.873979,1.934726){n140}
\dotnode[](4.776492,1.394465){n148}
\psset{dotstyle=Bo}
\dotnode[](1.403207,1.916558){n34}
\dotnode[fillcolor=gray](1.183687,2.360412){n81}
\dotnode[](0.6058247,2.060042){n85}
\dotnode[](0.9419618,2.164323){n96}
\psset{dotstyle=Bo}
\dotnode[](3.913548,0.4282107){n22}
\dotnode[](3.665883,0.6603324){n30}
\dotnode[fillcolor=gray](3.66405,0.2172084){n49}
\dotnode[](3.635776,1.016917){n88}
\dotnode[](3.949152,0.8549694){n115}
\dotnode[](3.471599,1.285363){n130}
\dotnode[](4.11247,0.5910543){n133}

            \end{pspicture}
        }}
    \end{center}
\end{figure}
\end{frame}


\begin{frame}{Contingency Table: MCL Clusters versus Iris Types}
  \framesubtitle{$r=1.3$}
\renewcommand{\arraystretch}{1.1}
\begin{center}
\begin{tabular}{|c|c|c|c|}
        \hline
         & {\tt iris-setosa} & {\tt iris-virginica} & {\tt
         iris-versicolor}\\
         \hline
        $C_1$ (triangle) & 50 & 0 & 1\\
        $C_2$ (square) & 0 & 36 & 0\\
        $C_3$ (circle) & 0 & 14 & 49 \\
        \hline
    \end{tabular}
\end{center}
\end{frame}
