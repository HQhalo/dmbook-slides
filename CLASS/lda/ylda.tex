\lecture{lda}{lda}

\date{Chap.\ 20: Linear Discriminant Analysis}

\begin{frame}
\titlepage
\end{frame}

\begin{frame}{Linear Discriminant Analysis}
Given
labeled data consisting of $d$-dimensional points $\bx_i$ along
with their classes $y_i$, the goal of linear discriminant analysis
(LDA) is to f\/{i}nd a vector $\bw$ that maximizes the separation
between the classes after projection onto $\bw$. 

\medskip
The key difference between principal component analysis and LDA is
that the former deals with unlabeled data and tries to maximize
variance, whereas the latter deals with labeled data and tries to
maximize the discrimination between the classes.
\end{frame}


\begin{frame}{Projection onto a Line}
 Let $\bD_i$ denote the subset of points
labeled with class $c_i$, i.e., $\bD_i = \{\bx_{j} | y_{j} = c_i\}$,
and let $\card{\bD_i} = n_i$ denote the number of points with
class $c_i$. We assume that there are only $k=2$ classes. 


\medskip
The projection  of any $d$-dimensional
point $\bx_i$ onto a unit vector $\bw$ is given as
\begin{align*}
  \bx'_i = \lB( \frac{\bw^T \bx_i}{\bw^T\bw} \rB) \bw
    = \lB( \bw^T \bx_i \rB) \bw = a_i \bw
\end{align*}
where $a_i$ specif\/{i}es the offset or coordinate of $\bx'_i$ along
the line $\bw$:
\begin{align*}
  a_i  = \bw^T\bx_i
\end{align*}

\medskip
The set of $n$ scalars $\{a_1, a_2, \ldots, a_n\}$ represents the mapping from $\setR^d$ to $\setR$, that is, from the original
$d$-dimensional space to a 1-dimensional space (along $\bw$).
\end{frame}


\begin{frame}{Projection onto $\bw$: Iris 2D Data}
  \framesubtitle{{\tt iris-setosa} as class $c_1$
(circles), and the other two Iris types as class $c_2$
(triangles)}
\begin{figure}[!t]
    \centering
    \scalebox{0.75}{%
    \psset{dotscale=1.5,arrowscale=2,PointName=none,
    dotsep=0.05, unit=1.0in}
    \begin{pspicture}(4,1.5)(8.5,5)
    \psaxes[Dx=0.5,Dy=0.5,Ox=4.0,Oy=1.5]{->}(4.0,1.5)(8.5,5)
    \pstGeonode[PointSymbol=none](8.04,4.15){na}
    \pstGeonode[PointSymbol=none](3.94,2.1){nb}
    \pstLineAB[linewidth=2pt,arrows=<-]{na}{nb}
    \nput{90}{na}{$\bw$}
    \pstGeonode[PointSymbol=Btriangle,fillcolor=lightgray](6.50,3.00){n1}
\pstProjection[PointSymbol=triangle,dotscale=1.25]{na}{nb}{n1}[p1]
\pstLineAB[linestyle=dotted,linecolor=gray]{n1}{p1}
\pstGeonode[PointSymbol=Btriangle,fillcolor=lightgray](5.80,2.70){n2}
\pstProjection[PointSymbol=triangle,dotscale=1.25]{na}{nb}{n2}[p2]
\pstLineAB[linestyle=dotted,linecolor=gray]{n2}{p2}
\pstGeonode[PointSymbol=Btriangle,fillcolor=lightgray](6.70,3.10){n3}
\pstProjection[PointSymbol=triangle,dotscale=1.25]{na}{nb}{n3}[p3]
\pstLineAB[linestyle=dotted,linecolor=gray]{n3}{p3}
\pstGeonode[PointSymbol=Btriangle,fillcolor=lightgray](6.70,2.50){n4}
\pstProjection[PointSymbol=triangle,dotscale=1.25]{na}{nb}{n4}[p4]
\pstLineAB[linestyle=dotted,linecolor=gray]{n4}{p4}
\pstGeonode[PointSymbol=Btriangle,fillcolor=lightgray](6.10,3.00){n5}
\pstProjection[PointSymbol=triangle,dotscale=1.25]{na}{nb}{n5}[p5]
\pstLineAB[linestyle=dotted,linecolor=gray]{n5}{p5}
\pstGeonode[PointSymbol=Btriangle,fillcolor=lightgray](7.20,3.20){n6}
\pstProjection[PointSymbol=triangle,dotscale=1.25]{na}{nb}{n6}[p6]
\pstLineAB[linestyle=dotted,linecolor=gray]{n6}{p6}
\pstGeonode[PointSymbol=Btriangle,fillcolor=lightgray](5.90,3.00){n7}
\pstProjection[PointSymbol=triangle,dotscale=1.25]{na}{nb}{n7}[p7]
\pstLineAB[linestyle=dotted,linecolor=gray]{n7}{p7}
\pstGeonode[PointSymbol=Btriangle,fillcolor=lightgray](6.50,3.00){n8}
\pstProjection[PointSymbol=triangle,dotscale=1.25]{na}{nb}{n8}[p8]
\pstLineAB[linestyle=dotted,linecolor=gray]{n8}{p8}
\pstGeonode[PointSymbol=Btriangle,fillcolor=lightgray](4.90,2.50){n9}
\pstProjection[PointSymbol=triangle,dotscale=1.25]{na}{nb}{n9}[p9]
\pstLineAB[linestyle=dotted,linecolor=gray]{n9}{p9}
\pstGeonode[PointSymbol=Btriangle,fillcolor=lightgray](7.20,3.00){n10}
\pstProjection[PointSymbol=triangle,dotscale=1.25]{na}{nb}{n10}[p10]
\pstLineAB[linestyle=dotted,linecolor=gray]{n10}{p10}
\pstGeonode[PointSymbol=Btriangle,fillcolor=lightgray](6.80,3.20){n11}
\pstProjection[PointSymbol=triangle,dotscale=1.25]{na}{nb}{n11}[p11]
\pstLineAB[linestyle=dotted,linecolor=gray]{n11}{p11}
\pstGeonode[PointSymbol=Btriangle,fillcolor=lightgray](5.70,2.50){n12}
\pstProjection[PointSymbol=triangle,dotscale=1.25]{na}{nb}{n12}[p12]
\pstLineAB[linestyle=dotted,linecolor=gray]{n12}{p12}
\pstGeonode[PointSymbol=Btriangle,fillcolor=lightgray](6.30,2.50){n13}
\pstProjection[PointSymbol=triangle,dotscale=1.25]{na}{nb}{n13}[p13]
\pstLineAB[linestyle=dotted,linecolor=gray]{n13}{p13}
\pstGeonode[PointSymbol=Btriangle,fillcolor=lightgray](6.80,3.00){n14}
\pstProjection[PointSymbol=triangle,dotscale=1.25]{na}{nb}{n14}[p14]
\pstLineAB[linestyle=dotted,linecolor=gray]{n14}{p14}
\pstGeonode[PointSymbol=Btriangle,fillcolor=lightgray](7.30,2.90){n15}
\pstProjection[PointSymbol=triangle,dotscale=1.25]{na}{nb}{n15}[p15]
\pstLineAB[linestyle=dotted,linecolor=gray]{n15}{p15}
\pstGeonode[PointSymbol=Btriangle,fillcolor=lightgray](7.10,3.00){n16}
\pstProjection[PointSymbol=triangle,dotscale=1.25]{na}{nb}{n16}[p16]
\pstLineAB[linestyle=dotted,linecolor=gray]{n16}{p16}
\pstGeonode[PointSymbol=Btriangle,fillcolor=lightgray](5.60,2.80){n17}
\pstProjection[PointSymbol=triangle,dotscale=1.25]{na}{nb}{n17}[p17]
\pstLineAB[linestyle=dotted,linecolor=gray]{n17}{p17}
\pstGeonode[PointSymbol=Btriangle,fillcolor=lightgray](6.30,3.30){n18}
\pstProjection[PointSymbol=triangle,dotscale=1.25]{na}{nb}{n18}[p18]
\pstLineAB[linestyle=dotted,linecolor=gray]{n18}{p18}
\pstGeonode[PointSymbol=Btriangle,fillcolor=lightgray](6.30,3.40){n19}
\pstProjection[PointSymbol=triangle,dotscale=1.25]{na}{nb}{n19}[p19]
\pstLineAB[linestyle=dotted,linecolor=gray]{n19}{p19}
\pstGeonode[PointSymbol=Btriangle,fillcolor=lightgray](6.90,3.10){n20}
\pstProjection[PointSymbol=triangle,dotscale=1.25]{na}{nb}{n20}[p20]
\pstLineAB[linestyle=dotted,linecolor=gray]{n20}{p20}
\pstGeonode[PointSymbol=Btriangle,fillcolor=lightgray](7.70,3.00){n21}
\pstProjection[PointSymbol=triangle,dotscale=1.25]{na}{nb}{n21}[p21]
\pstLineAB[linestyle=dotted,linecolor=gray]{n21}{p21}
\pstGeonode[PointSymbol=Btriangle,fillcolor=lightgray](6.10,2.60){n22}
\pstProjection[PointSymbol=triangle,dotscale=1.25]{na}{nb}{n22}[p22]
\pstLineAB[linestyle=dotted,linecolor=gray]{n22}{p22}
\pstGeonode[PointSymbol=Btriangle,fillcolor=lightgray](6.40,2.70){n23}
\pstProjection[PointSymbol=triangle,dotscale=1.25]{na}{nb}{n23}[p23]
\pstLineAB[linestyle=dotted,linecolor=gray]{n23}{p23}
\pstGeonode[PointSymbol=Btriangle,fillcolor=lightgray](5.80,2.80){n24}
\pstProjection[PointSymbol=triangle,dotscale=1.25]{na}{nb}{n24}[p24]
\pstLineAB[linestyle=dotted,linecolor=gray]{n24}{p24}
\pstGeonode[PointSymbol=Btriangle,fillcolor=lightgray](7.90,3.80){n25}
\pstProjection[PointSymbol=triangle,dotscale=1.25]{na}{nb}{n25}[p25]
\pstLineAB[linestyle=dotted,linecolor=gray]{n25}{p25}
\pstGeonode[PointSymbol=Btriangle,fillcolor=lightgray](7.70,2.60){n26}
\pstProjection[PointSymbol=triangle,dotscale=1.25]{na}{nb}{n26}[p26]
\pstLineAB[linestyle=dotted,linecolor=gray]{n26}{p26}
\pstGeonode[PointSymbol=Btriangle,fillcolor=lightgray](6.20,2.80){n27}
\pstProjection[PointSymbol=triangle,dotscale=1.25]{na}{nb}{n27}[p27]
\pstLineAB[linestyle=dotted,linecolor=gray]{n27}{p27}
\pstGeonode[PointSymbol=Btriangle,fillcolor=lightgray](6.20,3.40){n28}
\pstProjection[PointSymbol=triangle,dotscale=1.25]{na}{nb}{n28}[p28]
\pstLineAB[linestyle=dotted,linecolor=gray]{n28}{p28}
\pstGeonode[PointSymbol=Btriangle,fillcolor=lightgray](6.30,2.90){n29}
\pstProjection[PointSymbol=triangle,dotscale=1.25]{na}{nb}{n29}[p29]
\pstLineAB[linestyle=dotted,linecolor=gray]{n29}{p29}
\pstGeonode[PointSymbol=Btriangle,fillcolor=lightgray](6.70,3.30){n30}
\pstProjection[PointSymbol=triangle,dotscale=1.25]{na}{nb}{n30}[p30]
\pstLineAB[linestyle=dotted,linecolor=gray]{n30}{p30}
\pstGeonode[PointSymbol=Btriangle,fillcolor=lightgray](6.30,2.70){n31}
\pstProjection[PointSymbol=triangle,dotscale=1.25]{na}{nb}{n31}[p31]
\pstLineAB[linestyle=dotted,linecolor=gray]{n31}{p31}
\pstGeonode[PointSymbol=Btriangle,fillcolor=lightgray](6.40,3.20){n32}
\pstProjection[PointSymbol=triangle,dotscale=1.25]{na}{nb}{n32}[p32]
\pstLineAB[linestyle=dotted,linecolor=gray]{n32}{p32}
\pstGeonode[PointSymbol=Btriangle,fillcolor=lightgray](6.00,2.20){n33}
\pstProjection[PointSymbol=triangle,dotscale=1.25]{na}{nb}{n33}[p33]
\pstLineAB[linestyle=dotted,linecolor=gray]{n33}{p33}
\pstGeonode[PointSymbol=Btriangle,fillcolor=lightgray](7.40,2.80){n34}
\pstProjection[PointSymbol=triangle,dotscale=1.25]{na}{nb}{n34}[p34]
\pstLineAB[linestyle=dotted,linecolor=gray]{n34}{p34}
\pstGeonode[PointSymbol=Btriangle,fillcolor=lightgray](6.50,3.20){n35}
\pstProjection[PointSymbol=triangle,dotscale=1.25]{na}{nb}{n35}[p35]
\pstLineAB[linestyle=dotted,linecolor=gray]{n35}{p35}
\pstGeonode[PointSymbol=Btriangle,fillcolor=lightgray](6.70,3.30){n36}
\pstProjection[PointSymbol=triangle,dotscale=1.25]{na}{nb}{n36}[p36]
\pstLineAB[linestyle=dotted,linecolor=gray]{n36}{p36}
\pstGeonode[PointSymbol=Btriangle,fillcolor=lightgray](6.90,3.20){n37}
\pstProjection[PointSymbol=triangle,dotscale=1.25]{na}{nb}{n37}[p37]
\pstLineAB[linestyle=dotted,linecolor=gray]{n37}{p37}
\pstGeonode[PointSymbol=Btriangle,fillcolor=lightgray](7.60,3.00){n38}
\pstProjection[PointSymbol=triangle,dotscale=1.25]{na}{nb}{n38}[p38]
\pstLineAB[linestyle=dotted,linecolor=gray]{n38}{p38}
\pstGeonode[PointSymbol=Btriangle,fillcolor=lightgray](6.30,2.80){n39}
\pstProjection[PointSymbol=triangle,dotscale=1.25]{na}{nb}{n39}[p39]
\pstLineAB[linestyle=dotted,linecolor=gray]{n39}{p39}
\pstGeonode[PointSymbol=Btriangle,fillcolor=lightgray](6.40,3.10){n40}
\pstProjection[PointSymbol=triangle,dotscale=1.25]{na}{nb}{n40}[p40]
\pstLineAB[linestyle=dotted,linecolor=gray]{n40}{p40}
\pstGeonode[PointSymbol=Btriangle,fillcolor=lightgray](5.80,2.70){n41}
\pstProjection[PointSymbol=triangle,dotscale=1.25]{na}{nb}{n41}[p41]
\pstLineAB[linestyle=dotted,linecolor=gray]{n41}{p41}
\pstGeonode[PointSymbol=Btriangle,fillcolor=lightgray](6.40,2.80){n42}
\pstProjection[PointSymbol=triangle,dotscale=1.25]{na}{nb}{n42}[p42]
\pstLineAB[linestyle=dotted,linecolor=gray]{n42}{p42}
\pstGeonode[PointSymbol=Btriangle,fillcolor=lightgray](6.40,2.80){n43}
\pstProjection[PointSymbol=triangle,dotscale=1.25]{na}{nb}{n43}[p43]
\pstLineAB[linestyle=dotted,linecolor=gray]{n43}{p43}
\pstGeonode[PointSymbol=Btriangle,fillcolor=lightgray](7.70,2.80){n44}
\pstProjection[PointSymbol=triangle,dotscale=1.25]{na}{nb}{n44}[p44]
\pstLineAB[linestyle=dotted,linecolor=gray]{n44}{p44}
\pstGeonode[PointSymbol=Btriangle,fillcolor=lightgray](6.50,3.00){n45}
\pstProjection[PointSymbol=triangle,dotscale=1.25]{na}{nb}{n45}[p45]
\pstLineAB[linestyle=dotted,linecolor=gray]{n45}{p45}
\pstGeonode[PointSymbol=Btriangle,fillcolor=lightgray](7.20,3.60){n46}
\pstProjection[PointSymbol=triangle,dotscale=1.25]{na}{nb}{n46}[p46]
\pstLineAB[linestyle=dotted,linecolor=gray]{n46}{p46}
\pstGeonode[PointSymbol=Btriangle,fillcolor=lightgray](6.90,3.10){n47}
\pstProjection[PointSymbol=triangle,dotscale=1.25]{na}{nb}{n47}[p47]
\pstLineAB[linestyle=dotted,linecolor=gray]{n47}{p47}
\pstGeonode[PointSymbol=Btriangle,fillcolor=lightgray](6.00,3.00){n48}
\pstProjection[PointSymbol=triangle,dotscale=1.25]{na}{nb}{n48}[p48]
\pstLineAB[linestyle=dotted,linecolor=gray]{n48}{p48}
\pstGeonode[PointSymbol=Btriangle,fillcolor=lightgray](6.70,3.00){n49}
\pstProjection[PointSymbol=triangle,dotscale=1.25]{na}{nb}{n49}[p49]
\pstLineAB[linestyle=dotted,linecolor=gray]{n49}{p49}
\pstGeonode[PointSymbol=Btriangle,fillcolor=lightgray](7.70,3.80){n50}
\pstProjection[PointSymbol=triangle,dotscale=1.25]{na}{nb}{n50}[p50]
\pstLineAB[linestyle=dotted,linecolor=gray]{n50}{p50}
\pstGeonode[PointSymbol=Bo,fillcolor=lightgray](4.60,3.20){n51}
\pstProjection[PointSymbol=o,dotscale=1.25]{na}{nb}{n51}[p51]
\pstLineAB[linestyle=dotted,linecolor=gray]{n51}{p51}
\pstGeonode[PointSymbol=Bo,fillcolor=lightgray](4.70,3.20){n52}
\pstProjection[PointSymbol=o,dotscale=1.25]{na}{nb}{n52}[p52]
\pstLineAB[linestyle=dotted,linecolor=gray]{n52}{p52}
\pstGeonode[PointSymbol=Bo,fillcolor=lightgray](5.10,3.70){n53}
\pstProjection[PointSymbol=o,dotscale=1.25]{na}{nb}{n53}[p53]
\pstLineAB[linestyle=dotted,linecolor=gray]{n53}{p53}
\pstGeonode[PointSymbol=Bo,fillcolor=lightgray](5.10,3.80){n54}
\pstProjection[PointSymbol=o,dotscale=1.25]{na}{nb}{n54}[p54]
\pstLineAB[linestyle=dotted,linecolor=gray]{n54}{p54}
\pstGeonode[PointSymbol=Bo,fillcolor=lightgray](4.90,3.10){n55}
\pstProjection[PointSymbol=o,dotscale=1.25]{na}{nb}{n55}[p55]
\pstLineAB[linestyle=dotted,linecolor=gray]{n55}{p55}
\pstGeonode[PointSymbol=Bo,fillcolor=lightgray](5.00,3.40){n56}
\pstProjection[PointSymbol=o,dotscale=1.25]{na}{nb}{n56}[p56]
\pstLineAB[linestyle=dotted,linecolor=gray]{n56}{p56}
\pstGeonode[PointSymbol=Bo,fillcolor=lightgray](5.00,3.40){n57}
\pstProjection[PointSymbol=o,dotscale=1.25]{na}{nb}{n57}[p57]
\pstLineAB[linestyle=dotted,linecolor=gray]{n57}{p57}
\pstGeonode[PointSymbol=Bo,fillcolor=lightgray](5.00,3.30){n58}
\pstProjection[PointSymbol=o,dotscale=1.25]{na}{nb}{n58}[p58]
\pstLineAB[linestyle=dotted,linecolor=gray]{n58}{p58}
\pstGeonode[PointSymbol=Bo,fillcolor=lightgray](5.70,4.40){n59}
\pstProjection[PointSymbol=o,dotscale=1.25]{na}{nb}{n59}[p59]
\pstLineAB[linestyle=dotted,linecolor=gray]{n59}{p59}
\pstGeonode[PointSymbol=Bo,fillcolor=lightgray](5.00,3.50){n60}
\pstProjection[PointSymbol=o,dotscale=1.25]{na}{nb}{n60}[p60]
\pstLineAB[linestyle=dotted,linecolor=gray]{n60}{p60}
\pstGeonode[PointSymbol=Bo,fillcolor=lightgray](4.60,3.10){n61}
\pstProjection[PointSymbol=o,dotscale=1.25]{na}{nb}{n61}[p61]
\pstLineAB[linestyle=dotted,linecolor=gray]{n61}{p61}
\pstGeonode[PointSymbol=Bo,fillcolor=lightgray](5.40,3.90){n62}
\pstProjection[PointSymbol=o,dotscale=1.25]{na}{nb}{n62}[p62]
\pstLineAB[linestyle=dotted,linecolor=gray]{n62}{p62}
\pstGeonode[PointSymbol=Bo,fillcolor=lightgray](5.00,3.20){n63}
\pstProjection[PointSymbol=o,dotscale=1.25]{na}{nb}{n63}[p63]
\pstLineAB[linestyle=dotted,linecolor=gray]{n63}{p63}
\pstGeonode[PointSymbol=Bo,fillcolor=lightgray](5.10,3.80){n64}
\pstProjection[PointSymbol=o,dotscale=1.25]{na}{nb}{n64}[p64]
\pstLineAB[linestyle=dotted,linecolor=gray]{n64}{p64}
\pstGeonode[PointSymbol=Bo,fillcolor=lightgray](4.80,3.00){n65}
\pstProjection[PointSymbol=o,dotscale=1.25]{na}{nb}{n65}[p65]
\pstLineAB[linestyle=dotted,linecolor=gray]{n65}{p65}
\pstGeonode[PointSymbol=Bo,fillcolor=lightgray](5.30,3.70){n66}
\pstProjection[PointSymbol=o,dotscale=1.25]{na}{nb}{n66}[p66]
\pstLineAB[linestyle=dotted,linecolor=gray]{n66}{p66}
\pstGeonode[PointSymbol=Bo,fillcolor=lightgray](5.70,3.80){n67}
\pstProjection[PointSymbol=o,dotscale=1.25]{na}{nb}{n67}[p67]
\pstLineAB[linestyle=dotted,linecolor=gray]{n67}{p67}
\pstGeonode[PointSymbol=Bo,fillcolor=lightgray](4.40,3.00){n68}
\pstProjection[PointSymbol=o,dotscale=1.25]{na}{nb}{n68}[p68]
\pstLineAB[linestyle=dotted,linecolor=gray]{n68}{p68}
\pstGeonode[PointSymbol=Bo,fillcolor=lightgray](5.40,3.40){n69}
\pstProjection[PointSymbol=o,dotscale=1.25]{na}{nb}{n69}[p69]
\pstLineAB[linestyle=dotted,linecolor=gray]{n69}{p69}
\pstGeonode[PointSymbol=Bo,fillcolor=lightgray](5.00,3.50){n70}
\pstProjection[PointSymbol=o,dotscale=1.25]{na}{nb}{n70}[p70]
\pstLineAB[linestyle=dotted,linecolor=gray]{n70}{p70}
\pstGeonode[PointSymbol=Bo,fillcolor=lightgray](5.10,3.30){n71}
\pstProjection[PointSymbol=o,dotscale=1.25]{na}{nb}{n71}[p71]
\pstLineAB[linestyle=dotted,linecolor=gray]{n71}{p71}
\pstGeonode[PointSymbol=Bo,fillcolor=lightgray](4.90,3.10){n72}
\pstProjection[PointSymbol=o,dotscale=1.25]{na}{nb}{n72}[p72]
\pstLineAB[linestyle=dotted,linecolor=gray]{n72}{p72}
\pstGeonode[PointSymbol=Bo,fillcolor=lightgray](4.60,3.60){n73}
\pstProjection[PointSymbol=o,dotscale=1.25]{na}{nb}{n73}[p73]
\pstLineAB[linestyle=dotted,linecolor=gray]{n73}{p73}
\pstGeonode[PointSymbol=Bo,fillcolor=lightgray](5.20,3.40){n74}
\pstProjection[PointSymbol=o,dotscale=1.25]{na}{nb}{n74}[p74]
\pstLineAB[linestyle=dotted,linecolor=gray]{n74}{p74}
\pstGeonode[PointSymbol=Bo,fillcolor=lightgray](5.50,3.50){n75}
\pstProjection[PointSymbol=o,dotscale=1.25]{na}{nb}{n75}[p75]
\pstLineAB[linestyle=dotted,linecolor=gray]{n75}{p75}
\pstGeonode[PointSymbol=Bo,fillcolor=lightgray](4.60,3.40){n76}
\pstProjection[PointSymbol=o,dotscale=1.25]{na}{nb}{n76}[p76]
\pstLineAB[linestyle=dotted,linecolor=gray]{n76}{p76}
\pstGeonode[PointSymbol=Bo,fillcolor=lightgray](4.70,3.20){n77}
\pstProjection[PointSymbol=o,dotscale=1.25]{na}{nb}{n77}[p77]
\pstLineAB[linestyle=dotted,linecolor=gray]{n77}{p77}
\pstGeonode[PointSymbol=Bo,fillcolor=lightgray](4.40,2.90){n78}
\pstProjection[PointSymbol=o,dotscale=1.25]{na}{nb}{n78}[p78]
\pstLineAB[linestyle=dotted,linecolor=gray]{n78}{p78}
\pstGeonode[PointSymbol=Bo,fillcolor=lightgray](4.80,3.00){n79}
\pstProjection[PointSymbol=o,dotscale=1.25]{na}{nb}{n79}[p79]
\pstLineAB[linestyle=dotted,linecolor=gray]{n79}{p79}
\pstGeonode[PointSymbol=Bo,fillcolor=lightgray](5.40,3.90){n80}
\pstProjection[PointSymbol=o,dotscale=1.25]{na}{nb}{n80}[p80]
\pstLineAB[linestyle=dotted,linecolor=gray]{n80}{p80}
\pstGeonode[PointSymbol=Bo,fillcolor=lightgray](4.80,3.40){n81}
\pstProjection[PointSymbol=o,dotscale=1.25]{na}{nb}{n81}[p81]
\pstLineAB[linestyle=dotted,linecolor=gray]{n81}{p81}
\pstGeonode[PointSymbol=Bo,fillcolor=lightgray](4.40,3.20){n82}
\pstProjection[PointSymbol=o,dotscale=1.25]{na}{nb}{n82}[p82]
\pstLineAB[linestyle=dotted,linecolor=gray]{n82}{p82}
\pstGeonode[PointSymbol=Bo,fillcolor=lightgray](4.80,3.40){n83}
\pstProjection[PointSymbol=o,dotscale=1.25]{na}{nb}{n83}[p83]
\pstLineAB[linestyle=dotted,linecolor=gray]{n83}{p83}
\pstGeonode[PointSymbol=Bo,fillcolor=lightgray](4.90,3.00){n84}
\pstProjection[PointSymbol=o,dotscale=1.25]{na}{nb}{n84}[p84]
\pstLineAB[linestyle=dotted,linecolor=gray]{n84}{p84}
\pstGeonode[PointSymbol=Bo,fillcolor=lightgray](4.50,2.30){n85}
\pstProjection[PointSymbol=o,dotscale=1.25]{na}{nb}{n85}[p85]
\pstLineAB[linestyle=dotted,linecolor=gray]{n85}{p85}
\pstGeonode[PointSymbol=Bo,fillcolor=lightgray](4.30,3.00){n86}
\pstProjection[PointSymbol=o,dotscale=1.25]{na}{nb}{n86}[p86]
\pstLineAB[linestyle=dotted,linecolor=gray]{n86}{p86}
\pstGeonode[PointSymbol=Bo,fillcolor=lightgray](5.00,3.60){n87}
\pstProjection[PointSymbol=o,dotscale=1.25]{na}{nb}{n87}[p87]
\pstLineAB[linestyle=dotted,linecolor=gray]{n87}{p87}
\pstGeonode[PointSymbol=Bo,fillcolor=lightgray](5.20,3.50){n88}
\pstProjection[PointSymbol=o,dotscale=1.25]{na}{nb}{n88}[p88]
\pstLineAB[linestyle=dotted,linecolor=gray]{n88}{p88}
\pstGeonode[PointSymbol=Bo,fillcolor=lightgray](5.50,4.20){n89}
\pstProjection[PointSymbol=o,dotscale=1.25]{na}{nb}{n89}[p89]
\pstLineAB[linestyle=dotted,linecolor=gray]{n89}{p89}
\pstGeonode[PointSymbol=Bo,fillcolor=lightgray](5.00,3.00){n90}
\pstProjection[PointSymbol=o,dotscale=1.25]{na}{nb}{n90}[p90]
\pstLineAB[linestyle=dotted,linecolor=gray]{n90}{p90}
\pstGeonode[PointSymbol=Bo,fillcolor=lightgray](5.10,3.50){n91}
\pstProjection[PointSymbol=o,dotscale=1.25]{na}{nb}{n91}[p91]
\pstLineAB[linestyle=dotted,linecolor=gray]{n91}{p91}
\pstGeonode[PointSymbol=Bo,fillcolor=lightgray](4.80,3.10){n92}
\pstProjection[PointSymbol=o,dotscale=1.25]{na}{nb}{n92}[p92]
\pstLineAB[linestyle=dotted,linecolor=gray]{n92}{p92}
\pstGeonode[PointSymbol=Bo,fillcolor=lightgray](5.10,3.80){n93}
\pstProjection[PointSymbol=o,dotscale=1.25]{na}{nb}{n93}[p93]
\pstLineAB[linestyle=dotted,linecolor=gray]{n93}{p93}
\pstGeonode[PointSymbol=Bo,fillcolor=lightgray](5.20,4.10){n94}
\pstProjection[PointSymbol=o,dotscale=1.25]{na}{nb}{n94}[p94]
\pstLineAB[linestyle=dotted,linecolor=gray]{n94}{p94}
\pstGeonode[PointSymbol=Bo,fillcolor=lightgray](5.80,4.00){n95}
\pstProjection[PointSymbol=o,dotscale=1.25]{na}{nb}{n95}[p95]
\pstLineAB[linestyle=dotted,linecolor=gray]{n95}{p95}
\pstGeonode[PointSymbol=Bo,fillcolor=lightgray](4.90,3.10){n96}
\pstProjection[PointSymbol=o,dotscale=1.25]{na}{nb}{n96}[p96]
\pstLineAB[linestyle=dotted,linecolor=gray]{n96}{p96}
\pstGeonode[PointSymbol=Bo,fillcolor=lightgray](5.40,3.70){n97}
\pstProjection[PointSymbol=o,dotscale=1.25]{na}{nb}{n97}[p97]
\pstLineAB[linestyle=dotted,linecolor=gray]{n97}{p97}
\pstGeonode[PointSymbol=Bo,fillcolor=lightgray](5.10,3.50){n98}
\pstProjection[PointSymbol=o,dotscale=1.25]{na}{nb}{n98}[p98]
\pstLineAB[linestyle=dotted,linecolor=gray]{n98}{p98}
\pstGeonode[PointSymbol=Bo,fillcolor=lightgray](5.40,3.40){n99}
\pstProjection[PointSymbol=o,dotscale=1.25]{na}{nb}{n99}[p99]
\pstLineAB[linestyle=dotted,linecolor=gray]{n99}{p99}
\pstGeonode[PointSymbol=Bo,fillcolor=lightgray](5.10,3.40){n100}
\pstProjection[PointSymbol=o,dotscale=1.25]{na}{nb}{n100}[p100]
\pstLineAB[linestyle=dotted,linecolor=gray]{n100}{p100}
\pstGeonode[PointSymbol=Btriangle,fillcolor=lightgray](5.90,3.00){n101}
\pstProjection[PointSymbol=triangle,dotscale=1.25]{na}{nb}{n101}[p101]
\pstLineAB[linestyle=dotted,linecolor=gray]{n101}{p101}
\pstGeonode[PointSymbol=Btriangle,fillcolor=lightgray](6.90,3.10){n102}
\pstProjection[PointSymbol=triangle,dotscale=1.25]{na}{nb}{n102}[p102]
\pstLineAB[linestyle=dotted,linecolor=gray]{n102}{p102}
\pstGeonode[PointSymbol=Btriangle,fillcolor=lightgray](6.60,2.90){n103}
\pstProjection[PointSymbol=triangle,dotscale=1.25]{na}{nb}{n103}[p103]
\pstLineAB[linestyle=dotted,linecolor=gray]{n103}{p103}
\pstGeonode[PointSymbol=Btriangle,fillcolor=lightgray](6.00,2.20){n104}
\pstProjection[PointSymbol=triangle,dotscale=1.25]{na}{nb}{n104}[p104]
\pstLineAB[linestyle=dotted,linecolor=gray]{n104}{p104}
\pstGeonode[PointSymbol=Btriangle,fillcolor=lightgray](5.70,3.00){n105}
\pstProjection[PointSymbol=triangle,dotscale=1.25]{na}{nb}{n105}[p105]
\pstLineAB[linestyle=dotted,linecolor=gray]{n105}{p105}
\pstGeonode[PointSymbol=Btriangle,fillcolor=lightgray](5.70,2.80){n106}
\pstProjection[PointSymbol=triangle,dotscale=1.25]{na}{nb}{n106}[p106]
\pstLineAB[linestyle=dotted,linecolor=gray]{n106}{p106}
\pstGeonode[PointSymbol=Btriangle,fillcolor=lightgray](5.50,2.50){n107}
\pstProjection[PointSymbol=triangle,dotscale=1.25]{na}{nb}{n107}[p107]
\pstLineAB[linestyle=dotted,linecolor=gray]{n107}{p107}
\pstGeonode[PointSymbol=Btriangle,fillcolor=lightgray](5.50,2.30){n108}
\pstProjection[PointSymbol=triangle,dotscale=1.25]{na}{nb}{n108}[p108]
\pstLineAB[linestyle=dotted,linecolor=gray]{n108}{p108}
\pstGeonode[PointSymbol=Btriangle,fillcolor=lightgray](5.80,2.60){n109}
\pstProjection[PointSymbol=triangle,dotscale=1.25]{na}{nb}{n109}[p109]
\pstLineAB[linestyle=dotted,linecolor=gray]{n109}{p109}
\pstGeonode[PointSymbol=Btriangle,fillcolor=lightgray](5.10,2.50){n110}
\pstProjection[PointSymbol=triangle,dotscale=1.25]{na}{nb}{n110}[p110]
\pstLineAB[linestyle=dotted,linecolor=gray]{n110}{p110}
\pstGeonode[PointSymbol=Btriangle,fillcolor=lightgray](5.60,2.50){n111}
\pstProjection[PointSymbol=triangle,dotscale=1.25]{na}{nb}{n111}[p111]
\pstLineAB[linestyle=dotted,linecolor=gray]{n111}{p111}
\pstGeonode[PointSymbol=Btriangle,fillcolor=lightgray](5.80,2.70){n112}
\pstProjection[PointSymbol=triangle,dotscale=1.25]{na}{nb}{n112}[p112]
\pstLineAB[linestyle=dotted,linecolor=gray]{n112}{p112}
\pstGeonode[PointSymbol=Btriangle,fillcolor=lightgray](6.30,2.30){n113}
\pstProjection[PointSymbol=triangle,dotscale=1.25]{na}{nb}{n113}[p113]
\pstLineAB[linestyle=dotted,linecolor=gray]{n113}{p113}
\pstGeonode[PointSymbol=Btriangle,fillcolor=lightgray](5.60,3.00){n114}
\pstProjection[PointSymbol=triangle,dotscale=1.25]{na}{nb}{n114}[p114]
\pstLineAB[linestyle=dotted,linecolor=gray]{n114}{p114}
\pstGeonode[PointSymbol=Btriangle,fillcolor=lightgray](6.10,3.00){n115}
\pstProjection[PointSymbol=triangle,dotscale=1.25]{na}{nb}{n115}[p115]
\pstLineAB[linestyle=dotted,linecolor=gray]{n115}{p115}
\pstGeonode[PointSymbol=Btriangle,fillcolor=lightgray](5.60,2.70){n116}
\pstProjection[PointSymbol=triangle,dotscale=1.25]{na}{nb}{n116}[p116]
\pstLineAB[linestyle=dotted,linecolor=gray]{n116}{p116}
\pstGeonode[PointSymbol=Btriangle,fillcolor=lightgray](5.70,2.60){n117}
\pstProjection[PointSymbol=triangle,dotscale=1.25]{na}{nb}{n117}[p117]
\pstLineAB[linestyle=dotted,linecolor=gray]{n117}{p117}
\pstGeonode[PointSymbol=Btriangle,fillcolor=lightgray](5.70,2.90){n118}
\pstProjection[PointSymbol=triangle,dotscale=1.25]{na}{nb}{n118}[p118]
\pstLineAB[linestyle=dotted,linecolor=gray]{n118}{p118}
\pstGeonode[PointSymbol=Btriangle,fillcolor=lightgray](6.10,2.80){n119}
\pstProjection[PointSymbol=triangle,dotscale=1.25]{na}{nb}{n119}[p119]
\pstLineAB[linestyle=dotted,linecolor=gray]{n119}{p119}
\pstGeonode[PointSymbol=Btriangle,fillcolor=lightgray](5.60,2.90){n120}
\pstProjection[PointSymbol=triangle,dotscale=1.25]{na}{nb}{n120}[p120]
\pstLineAB[linestyle=dotted,linecolor=gray]{n120}{p120}
\pstGeonode[PointSymbol=Btriangle,fillcolor=lightgray](5.60,3.00){n121}
\pstProjection[PointSymbol=triangle,dotscale=1.25]{na}{nb}{n121}[p121]
\pstLineAB[linestyle=dotted,linecolor=gray]{n121}{p121}
\pstGeonode[PointSymbol=Btriangle,fillcolor=lightgray](5.40,3.00){n122}
\pstProjection[PointSymbol=triangle,dotscale=1.25]{na}{nb}{n122}[p122]
\pstLineAB[linestyle=dotted,linecolor=gray]{n122}{p122}
\pstGeonode[PointSymbol=Btriangle,fillcolor=lightgray](6.10,2.80){n123}
\pstProjection[PointSymbol=triangle,dotscale=1.25]{na}{nb}{n123}[p123]
\pstLineAB[linestyle=dotted,linecolor=gray]{n123}{p123}
\pstGeonode[PointSymbol=Btriangle,fillcolor=lightgray](6.00,2.90){n124}
\pstProjection[PointSymbol=triangle,dotscale=1.25]{na}{nb}{n124}[p124]
\pstLineAB[linestyle=dotted,linecolor=gray]{n124}{p124}
\pstGeonode[PointSymbol=Btriangle,fillcolor=lightgray](5.00,2.30){n125}
\pstProjection[PointSymbol=triangle,dotscale=1.25]{na}{nb}{n125}[p125]
\pstLineAB[linestyle=dotted,linecolor=gray]{n125}{p125}
\pstGeonode[PointSymbol=Btriangle,fillcolor=lightgray](6.40,3.20){n126}
\pstProjection[PointSymbol=triangle,dotscale=1.25]{na}{nb}{n126}[p126]
\pstLineAB[linestyle=dotted,linecolor=gray]{n126}{p126}
\pstGeonode[PointSymbol=Btriangle,fillcolor=lightgray](6.70,3.00){n127}
\pstProjection[PointSymbol=triangle,dotscale=1.25]{na}{nb}{n127}[p127]
\pstLineAB[linestyle=dotted,linecolor=gray]{n127}{p127}
\pstGeonode[PointSymbol=Btriangle,fillcolor=lightgray](5.00,2.00){n128}
\pstProjection[PointSymbol=triangle,dotscale=1.25]{na}{nb}{n128}[p128]
\pstLineAB[linestyle=dotted,linecolor=gray]{n128}{p128}
\pstGeonode[PointSymbol=Btriangle,fillcolor=lightgray](5.90,3.20){n129}
\pstProjection[PointSymbol=triangle,dotscale=1.25]{na}{nb}{n129}[p129]
\pstLineAB[linestyle=dotted,linecolor=gray]{n129}{p129}
\pstGeonode[PointSymbol=Btriangle,fillcolor=lightgray](6.20,2.20){n130}
\pstProjection[PointSymbol=triangle,dotscale=1.25]{na}{nb}{n130}[p130]
\pstLineAB[linestyle=dotted,linecolor=gray]{n130}{p130}
\pstGeonode[PointSymbol=Btriangle,fillcolor=lightgray](4.90,2.40){n131}
\pstProjection[PointSymbol=triangle,dotscale=1.25]{na}{nb}{n131}[p131]
\pstLineAB[linestyle=dotted,linecolor=gray]{n131}{p131}
\pstGeonode[PointSymbol=Btriangle,fillcolor=lightgray](7.00,3.20){n132}
\pstProjection[PointSymbol=triangle,dotscale=1.25]{na}{nb}{n132}[p132]
\pstLineAB[linestyle=dotted,linecolor=gray]{n132}{p132}
\pstGeonode[PointSymbol=Btriangle,fillcolor=lightgray](5.50,2.40){n133}
\pstProjection[PointSymbol=triangle,dotscale=1.25]{na}{nb}{n133}[p133]
\pstLineAB[linestyle=dotted,linecolor=gray]{n133}{p133}
\pstGeonode[PointSymbol=Btriangle,fillcolor=lightgray](6.30,3.30){n134}
\pstProjection[PointSymbol=triangle,dotscale=1.25]{na}{nb}{n134}[p134]
\pstLineAB[linestyle=dotted,linecolor=gray]{n134}{p134}
\pstGeonode[PointSymbol=Btriangle,fillcolor=lightgray](6.80,2.80){n135}
\pstProjection[PointSymbol=triangle,dotscale=1.25]{na}{nb}{n135}[p135]
\pstLineAB[linestyle=dotted,linecolor=gray]{n135}{p135}
\pstGeonode[PointSymbol=Btriangle,fillcolor=lightgray](6.10,2.90){n136}
\pstProjection[PointSymbol=triangle,dotscale=1.25]{na}{nb}{n136}[p136]
\pstLineAB[linestyle=dotted,linecolor=gray]{n136}{p136}
\pstGeonode[PointSymbol=Btriangle,fillcolor=lightgray](6.70,3.10){n137}
\pstProjection[PointSymbol=triangle,dotscale=1.25]{na}{nb}{n137}[p137]
\pstLineAB[linestyle=dotted,linecolor=gray]{n137}{p137}
\pstGeonode[PointSymbol=Btriangle,fillcolor=lightgray](5.20,2.70){n138}
\pstProjection[PointSymbol=triangle,dotscale=1.25]{na}{nb}{n138}[p138]
\pstLineAB[linestyle=dotted,linecolor=gray]{n138}{p138}
\pstGeonode[PointSymbol=Btriangle,fillcolor=lightgray](6.40,2.90){n139}
\pstProjection[PointSymbol=triangle,dotscale=1.25]{na}{nb}{n139}[p139]
\pstLineAB[linestyle=dotted,linecolor=gray]{n139}{p139}
\pstGeonode[PointSymbol=Btriangle,fillcolor=lightgray](5.80,2.70){n140}
\pstProjection[PointSymbol=triangle,dotscale=1.25]{na}{nb}{n140}[p140]
\pstLineAB[linestyle=dotted,linecolor=gray]{n140}{p140}
\pstGeonode[PointSymbol=Btriangle,fillcolor=lightgray](6.30,2.50){n141}
\pstProjection[PointSymbol=triangle,dotscale=1.25]{na}{nb}{n141}[p141]
\pstLineAB[linestyle=dotted,linecolor=gray]{n141}{p141}
\pstGeonode[PointSymbol=Btriangle,fillcolor=lightgray](6.70,3.10){n142}
\pstProjection[PointSymbol=triangle,dotscale=1.25]{na}{nb}{n142}[p142]
\pstLineAB[linestyle=dotted,linecolor=gray]{n142}{p142}
\pstGeonode[PointSymbol=Btriangle,fillcolor=lightgray](6.00,2.70){n143}
\pstProjection[PointSymbol=triangle,dotscale=1.25]{na}{nb}{n143}[p143]
\pstLineAB[linestyle=dotted,linecolor=gray]{n143}{p143}
\pstGeonode[PointSymbol=Btriangle,fillcolor=lightgray](5.70,2.80){n144}
\pstProjection[PointSymbol=triangle,dotscale=1.25]{na}{nb}{n144}[p144]
\pstLineAB[linestyle=dotted,linecolor=gray]{n144}{p144}
\pstGeonode[PointSymbol=Btriangle,fillcolor=lightgray](6.60,3.00){n145}
\pstProjection[PointSymbol=triangle,dotscale=1.25]{na}{nb}{n145}[p145]
\pstLineAB[linestyle=dotted,linecolor=gray]{n145}{p145}
\pstGeonode[PointSymbol=Btriangle,fillcolor=lightgray](6.00,3.40){n146}
\pstProjection[PointSymbol=triangle,dotscale=1.25]{na}{nb}{n146}[p146]
\pstLineAB[linestyle=dotted,linecolor=gray]{n146}{p146}
\pstGeonode[PointSymbol=Btriangle,fillcolor=lightgray](5.50,2.40){n147}
\pstProjection[PointSymbol=triangle,dotscale=1.25]{na}{nb}{n147}[p147]
\pstLineAB[linestyle=dotted,linecolor=gray]{n147}{p147}
\pstGeonode[PointSymbol=Btriangle,fillcolor=lightgray](6.20,2.90){n148}
\pstProjection[PointSymbol=triangle,dotscale=1.25]{na}{nb}{n148}[p148]
\pstLineAB[linestyle=dotted,linecolor=gray]{n148}{p148}
\pstGeonode[PointSymbol=Btriangle,fillcolor=lightgray](6.50,2.80){n149}
\pstProjection[PointSymbol=triangle,dotscale=1.25]{na}{nb}{n149}[p149]
\pstLineAB[linestyle=dotted,linecolor=gray]{n149}{p149}
\pstGeonode[PointSymbol=Btriangle,fillcolor=lightgray](5.50,2.60){n150}
\pstProjection[PointSymbol=triangle,dotscale=1.25]{na}{nb}{n150}[p150]
\pstLineAB[linestyle=dotted,linecolor=gray]{n150}{p150}

        \psset{dotscale=2,fillcolor=black}
    \pstGeonode[PointSymbol=none,
        fillcolor=lightgray](5.01,3.42){mu1}
    \pstProjection[PointSymbol=Bo]{na}{nb}{mu1}[m1]
    \pstGeonode[PointSymbol=none,
        fillcolor=lightgray](6.26,2.87){mu2}
    \pstProjection[PointSymbol=Btriangle]{na}{nb}{mu2}[m2]
    \end{pspicture}
    }
\end{figure}
\end{frame}



\begin{frame}{Optimal Linear Discriminant Direction}
\begin{figure}[!t]\vspace*{-14pt}
    \centering
    %\vspace{0.2in}
    \scalebox{0.8}{%
    \psset{dotscale=1.5,dotsep=0.03,arrowscale=2,PointName=none,unit=1.0in}
    \begin{pspicture}(4,1.5)(8.5,5)
    \psaxes[Dx=0.5,Dy=0.5,Ox=4.0,Oy=1.5]{->}(4.0,1.5)(8.5,5)
    %\begin{pspicture}(4,1.5)(8,4.5)
    %\psaxes[Dx=0.5,Dy=0.5,Ox=4.0,Oy=1.5]{->}(4.0,1.5)(8.0,4.5)
    \pstGeonode[PointSymbol=none](4.88,4.51){na}
    \pstGeonode[PointSymbol=none](6.89,1.47){nb}
    \pstLineAB[linewidth=2pt,arrows=->]{na}{nb}
    \uput[90](6.9,1.6){$\bw$}
    \pstGeonode[PointSymbol=Btriangle,fillcolor=lightgray](6.50,3.00){n1}
\pstProjection[PointSymbol=triangle,dotscale=1.25]{na}{nb}{n1}[p1]
\pstLineAB[linestyle=dotted,linecolor=gray]{n1}{p1}
\pstGeonode[PointSymbol=Btriangle,fillcolor=lightgray](5.80,2.70){n2}
\pstProjection[PointSymbol=triangle,dotscale=1.25]{na}{nb}{n2}[p2]
\pstLineAB[linestyle=dotted,linecolor=gray]{n2}{p2}
\pstGeonode[PointSymbol=Btriangle,fillcolor=lightgray](6.70,3.10){n3}
\pstProjection[PointSymbol=triangle,dotscale=1.25]{na}{nb}{n3}[p3]
\pstLineAB[linestyle=dotted,linecolor=gray]{n3}{p3}
\pstGeonode[PointSymbol=Btriangle,fillcolor=lightgray](6.70,2.50){n4}
\pstProjection[PointSymbol=triangle,dotscale=1.25]{na}{nb}{n4}[p4]
\pstLineAB[linestyle=dotted,linecolor=gray]{n4}{p4}
\pstGeonode[PointSymbol=Btriangle,fillcolor=lightgray](6.10,3.00){n5}
\pstProjection[PointSymbol=triangle,dotscale=1.25]{na}{nb}{n5}[p5]
\pstLineAB[linestyle=dotted,linecolor=gray]{n5}{p5}
\pstGeonode[PointSymbol=Btriangle,fillcolor=lightgray](7.20,3.20){n6}
\pstProjection[PointSymbol=triangle,dotscale=1.25]{na}{nb}{n6}[p6]
\pstLineAB[linestyle=dotted,linecolor=gray]{n6}{p6}
\pstGeonode[PointSymbol=Btriangle,fillcolor=lightgray](5.90,3.00){n7}
\pstProjection[PointSymbol=triangle,dotscale=1.25]{na}{nb}{n7}[p7]
\pstLineAB[linestyle=dotted,linecolor=gray]{n7}{p7}
\pstGeonode[PointSymbol=Btriangle,fillcolor=lightgray](6.50,3.00){n8}
\pstProjection[PointSymbol=triangle,dotscale=1.25]{na}{nb}{n8}[p8]
\pstLineAB[linestyle=dotted,linecolor=gray]{n8}{p8}
\pstGeonode[PointSymbol=Btriangle,fillcolor=lightgray](4.90,2.50){n9}
\pstProjection[PointSymbol=triangle,dotscale=1.25]{na}{nb}{n9}[p9]
\pstLineAB[linestyle=dotted,linecolor=gray]{n9}{p9}
\pstGeonode[PointSymbol=Btriangle,fillcolor=lightgray](7.20,3.00){n10}
\pstProjection[PointSymbol=triangle,dotscale=1.25]{na}{nb}{n10}[p10]
\pstLineAB[linestyle=dotted,linecolor=gray]{n10}{p10}
\pstGeonode[PointSymbol=Btriangle,fillcolor=lightgray](6.80,3.20){n11}
\pstProjection[PointSymbol=triangle,dotscale=1.25]{na}{nb}{n11}[p11]
\pstLineAB[linestyle=dotted,linecolor=gray]{n11}{p11}
\pstGeonode[PointSymbol=Btriangle,fillcolor=lightgray](5.70,2.50){n12}
\pstProjection[PointSymbol=triangle,dotscale=1.25]{na}{nb}{n12}[p12]
\pstLineAB[linestyle=dotted,linecolor=gray]{n12}{p12}
\pstGeonode[PointSymbol=Btriangle,fillcolor=lightgray](6.30,2.50){n13}
\pstProjection[PointSymbol=triangle,dotscale=1.25]{na}{nb}{n13}[p13]
\pstLineAB[linestyle=dotted,linecolor=gray]{n13}{p13}
\pstGeonode[PointSymbol=Btriangle,fillcolor=lightgray](6.80,3.00){n14}
\pstProjection[PointSymbol=triangle,dotscale=1.25]{na}{nb}{n14}[p14]
\pstLineAB[linestyle=dotted,linecolor=gray]{n14}{p14}
\pstGeonode[PointSymbol=Btriangle,fillcolor=lightgray](7.30,2.90){n15}
\pstProjection[PointSymbol=triangle,dotscale=1.25]{na}{nb}{n15}[p15]
\pstLineAB[linestyle=dotted,linecolor=gray]{n15}{p15}
\pstGeonode[PointSymbol=Btriangle,fillcolor=lightgray](7.10,3.00){n16}
\pstProjection[PointSymbol=triangle,dotscale=1.25]{na}{nb}{n16}[p16]
\pstLineAB[linestyle=dotted,linecolor=gray]{n16}{p16}
\pstGeonode[PointSymbol=Btriangle,fillcolor=lightgray](5.60,2.80){n17}
\pstProjection[PointSymbol=triangle,dotscale=1.25]{na}{nb}{n17}[p17]
\pstLineAB[linestyle=dotted,linecolor=gray]{n17}{p17}
\pstGeonode[PointSymbol=Btriangle,fillcolor=lightgray](6.30,3.30){n18}
\pstProjection[PointSymbol=triangle,dotscale=1.25]{na}{nb}{n18}[p18]
\pstLineAB[linestyle=dotted,linecolor=gray]{n18}{p18}
\pstGeonode[PointSymbol=Btriangle,fillcolor=lightgray](6.30,3.40){n19}
\pstProjection[PointSymbol=triangle,dotscale=1.25]{na}{nb}{n19}[p19]
\pstLineAB[linestyle=dotted,linecolor=gray]{n19}{p19}
\pstGeonode[PointSymbol=Btriangle,fillcolor=lightgray](6.90,3.10){n20}
\pstProjection[PointSymbol=triangle,dotscale=1.25]{na}{nb}{n20}[p20]
\pstLineAB[linestyle=dotted,linecolor=gray]{n20}{p20}
\pstGeonode[PointSymbol=Btriangle,fillcolor=lightgray](7.70,3.00){n21}
\pstProjection[PointSymbol=triangle,dotscale=1.25]{na}{nb}{n21}[p21]
\pstLineAB[linestyle=dotted,linecolor=gray]{n21}{p21}
\pstGeonode[PointSymbol=Btriangle,fillcolor=lightgray](6.10,2.60){n22}
\pstProjection[PointSymbol=triangle,dotscale=1.25]{na}{nb}{n22}[p22]
\pstLineAB[linestyle=dotted,linecolor=gray]{n22}{p22}
\pstGeonode[PointSymbol=Btriangle,fillcolor=lightgray](6.40,2.70){n23}
\pstProjection[PointSymbol=triangle,dotscale=1.25]{na}{nb}{n23}[p23]
\pstLineAB[linestyle=dotted,linecolor=gray]{n23}{p23}
\pstGeonode[PointSymbol=Btriangle,fillcolor=lightgray](5.80,2.80){n24}
\pstProjection[PointSymbol=triangle,dotscale=1.25]{na}{nb}{n24}[p24]
\pstLineAB[linestyle=dotted,linecolor=gray]{n24}{p24}
\pstGeonode[PointSymbol=Btriangle,fillcolor=lightgray](7.90,3.80){n25}
\pstProjection[PointSymbol=triangle,dotscale=1.25]{na}{nb}{n25}[p25]
\pstLineAB[linestyle=dotted,linecolor=gray]{n25}{p25}
\pstGeonode[PointSymbol=Btriangle,fillcolor=lightgray](7.70,2.60){n26}
\pstProjection[PointSymbol=triangle,dotscale=1.25]{na}{nb}{n26}[p26]
\pstLineAB[linestyle=dotted,linecolor=gray]{n26}{p26}
\pstGeonode[PointSymbol=Btriangle,fillcolor=lightgray](6.20,2.80){n27}
\pstProjection[PointSymbol=triangle,dotscale=1.25]{na}{nb}{n27}[p27]
\pstLineAB[linestyle=dotted,linecolor=gray]{n27}{p27}
\pstGeonode[PointSymbol=Btriangle,fillcolor=lightgray](6.20,3.40){n28}
\pstProjection[PointSymbol=triangle,dotscale=1.25]{na}{nb}{n28}[p28]
\pstLineAB[linestyle=dotted,linecolor=gray]{n28}{p28}
\pstGeonode[PointSymbol=Btriangle,fillcolor=lightgray](6.30,2.90){n29}
\pstProjection[PointSymbol=triangle,dotscale=1.25]{na}{nb}{n29}[p29]
\pstLineAB[linestyle=dotted,linecolor=gray]{n29}{p29}
\pstGeonode[PointSymbol=Btriangle,fillcolor=lightgray](6.70,3.30){n30}
\pstProjection[PointSymbol=triangle,dotscale=1.25]{na}{nb}{n30}[p30]
\pstLineAB[linestyle=dotted,linecolor=gray]{n30}{p30}
\pstGeonode[PointSymbol=Btriangle,fillcolor=lightgray](6.30,2.70){n31}
\pstProjection[PointSymbol=triangle,dotscale=1.25]{na}{nb}{n31}[p31]
\pstLineAB[linestyle=dotted,linecolor=gray]{n31}{p31}
\pstGeonode[PointSymbol=Btriangle,fillcolor=lightgray](6.40,3.20){n32}
\pstProjection[PointSymbol=triangle,dotscale=1.25]{na}{nb}{n32}[p32]
\pstLineAB[linestyle=dotted,linecolor=gray]{n32}{p32}
\pstGeonode[PointSymbol=Btriangle,fillcolor=lightgray](6.00,2.20){n33}
\pstProjection[PointSymbol=triangle,dotscale=1.25]{na}{nb}{n33}[p33]
\pstLineAB[linestyle=dotted,linecolor=gray]{n33}{p33}
\pstGeonode[PointSymbol=Btriangle,fillcolor=lightgray](7.40,2.80){n34}
\pstProjection[PointSymbol=triangle,dotscale=1.25]{na}{nb}{n34}[p34]
\pstLineAB[linestyle=dotted,linecolor=gray]{n34}{p34}
\pstGeonode[PointSymbol=Btriangle,fillcolor=lightgray](6.50,3.20){n35}
\pstProjection[PointSymbol=triangle,dotscale=1.25]{na}{nb}{n35}[p35]
\pstLineAB[linestyle=dotted,linecolor=gray]{n35}{p35}
\pstGeonode[PointSymbol=Btriangle,fillcolor=lightgray](6.70,3.30){n36}
\pstProjection[PointSymbol=triangle,dotscale=1.25]{na}{nb}{n36}[p36]
\pstLineAB[linestyle=dotted,linecolor=gray]{n36}{p36}
\pstGeonode[PointSymbol=Btriangle,fillcolor=lightgray](6.90,3.20){n37}
\pstProjection[PointSymbol=triangle,dotscale=1.25]{na}{nb}{n37}[p37]
\pstLineAB[linestyle=dotted,linecolor=gray]{n37}{p37}
\pstGeonode[PointSymbol=Btriangle,fillcolor=lightgray](7.60,3.00){n38}
\pstProjection[PointSymbol=triangle,dotscale=1.25]{na}{nb}{n38}[p38]
\pstLineAB[linestyle=dotted,linecolor=gray]{n38}{p38}
\pstGeonode[PointSymbol=Btriangle,fillcolor=lightgray](6.30,2.80){n39}
\pstProjection[PointSymbol=triangle,dotscale=1.25]{na}{nb}{n39}[p39]
\pstLineAB[linestyle=dotted,linecolor=gray]{n39}{p39}
\pstGeonode[PointSymbol=Btriangle,fillcolor=lightgray](6.40,3.10){n40}
\pstProjection[PointSymbol=triangle,dotscale=1.25]{na}{nb}{n40}[p40]
\pstLineAB[linestyle=dotted,linecolor=gray]{n40}{p40}
\pstGeonode[PointSymbol=Btriangle,fillcolor=lightgray](5.80,2.70){n41}
\pstProjection[PointSymbol=triangle,dotscale=1.25]{na}{nb}{n41}[p41]
\pstLineAB[linestyle=dotted,linecolor=gray]{n41}{p41}
\pstGeonode[PointSymbol=Btriangle,fillcolor=lightgray](6.40,2.80){n42}
\pstProjection[PointSymbol=triangle,dotscale=1.25]{na}{nb}{n42}[p42]
\pstLineAB[linestyle=dotted,linecolor=gray]{n42}{p42}
\pstGeonode[PointSymbol=Btriangle,fillcolor=lightgray](6.40,2.80){n43}
\pstProjection[PointSymbol=triangle,dotscale=1.25]{na}{nb}{n43}[p43]
\pstLineAB[linestyle=dotted,linecolor=gray]{n43}{p43}
\pstGeonode[PointSymbol=Btriangle,fillcolor=lightgray](7.70,2.80){n44}
\pstProjection[PointSymbol=triangle,dotscale=1.25]{na}{nb}{n44}[p44]
\pstLineAB[linestyle=dotted,linecolor=gray]{n44}{p44}
\pstGeonode[PointSymbol=Btriangle,fillcolor=lightgray](6.50,3.00){n45}
\pstProjection[PointSymbol=triangle,dotscale=1.25]{na}{nb}{n45}[p45]
\pstLineAB[linestyle=dotted,linecolor=gray]{n45}{p45}
\pstGeonode[PointSymbol=Btriangle,fillcolor=lightgray](7.20,3.60){n46}
\pstProjection[PointSymbol=triangle,dotscale=1.25]{na}{nb}{n46}[p46]
\pstLineAB[linestyle=dotted,linecolor=gray]{n46}{p46}
\pstGeonode[PointSymbol=Btriangle,fillcolor=lightgray](6.90,3.10){n47}
\pstProjection[PointSymbol=triangle,dotscale=1.25]{na}{nb}{n47}[p47]
\pstLineAB[linestyle=dotted,linecolor=gray]{n47}{p47}
\pstGeonode[PointSymbol=Btriangle,fillcolor=lightgray](6.00,3.00){n48}
\pstProjection[PointSymbol=triangle,dotscale=1.25]{na}{nb}{n48}[p48]
\pstLineAB[linestyle=dotted,linecolor=gray]{n48}{p48}
\pstGeonode[PointSymbol=Btriangle,fillcolor=lightgray](6.70,3.00){n49}
\pstProjection[PointSymbol=triangle,dotscale=1.25]{na}{nb}{n49}[p49]
\pstLineAB[linestyle=dotted,linecolor=gray]{n49}{p49}
\pstGeonode[PointSymbol=Btriangle,fillcolor=lightgray](7.70,3.80){n50}
\pstProjection[PointSymbol=triangle,dotscale=1.25]{na}{nb}{n50}[p50]
\pstLineAB[linestyle=dotted,linecolor=gray]{n50}{p50}
\pstGeonode[PointSymbol=Bo,fillcolor=lightgray](4.60,3.20){n51}
\pstProjection[PointSymbol=o,dotscale=1.25]{na}{nb}{n51}[p51]
\pstLineAB[linestyle=dotted,linecolor=gray]{n51}{p51}
\pstGeonode[PointSymbol=Bo,fillcolor=lightgray](4.70,3.20){n52}
\pstProjection[PointSymbol=o,dotscale=1.25]{na}{nb}{n52}[p52]
\pstLineAB[linestyle=dotted,linecolor=gray]{n52}{p52}
\pstGeonode[PointSymbol=Bo,fillcolor=lightgray](5.10,3.70){n53}
\pstProjection[PointSymbol=o,dotscale=1.25]{na}{nb}{n53}[p53]
\pstLineAB[linestyle=dotted,linecolor=gray]{n53}{p53}
\pstGeonode[PointSymbol=Bo,fillcolor=lightgray](5.10,3.80){n54}
\pstProjection[PointSymbol=o,dotscale=1.25]{na}{nb}{n54}[p54]
\pstLineAB[linestyle=dotted,linecolor=gray]{n54}{p54}
\pstGeonode[PointSymbol=Bo,fillcolor=lightgray](4.90,3.10){n55}
\pstProjection[PointSymbol=o,dotscale=1.25]{na}{nb}{n55}[p55]
\pstLineAB[linestyle=dotted,linecolor=gray]{n55}{p55}
\pstGeonode[PointSymbol=Bo,fillcolor=lightgray](5.00,3.40){n56}
\pstProjection[PointSymbol=o,dotscale=1.25]{na}{nb}{n56}[p56]
\pstLineAB[linestyle=dotted,linecolor=gray]{n56}{p56}
\pstGeonode[PointSymbol=Bo,fillcolor=lightgray](5.00,3.40){n57}
\pstProjection[PointSymbol=o,dotscale=1.25]{na}{nb}{n57}[p57]
\pstLineAB[linestyle=dotted,linecolor=gray]{n57}{p57}
\pstGeonode[PointSymbol=Bo,fillcolor=lightgray](5.00,3.30){n58}
\pstProjection[PointSymbol=o,dotscale=1.25]{na}{nb}{n58}[p58]
\pstLineAB[linestyle=dotted,linecolor=gray]{n58}{p58}
\pstGeonode[PointSymbol=Bo,fillcolor=lightgray](5.70,4.40){n59}
\pstProjection[PointSymbol=o,dotscale=1.25]{na}{nb}{n59}[p59]
\pstLineAB[linestyle=dotted,linecolor=gray]{n59}{p59}
\pstGeonode[PointSymbol=Bo,fillcolor=lightgray](5.00,3.50){n60}
\pstProjection[PointSymbol=o,dotscale=1.25]{na}{nb}{n60}[p60]
\pstLineAB[linestyle=dotted,linecolor=gray]{n60}{p60}
\pstGeonode[PointSymbol=Bo,fillcolor=lightgray](4.60,3.10){n61}
\pstProjection[PointSymbol=o,dotscale=1.25]{na}{nb}{n61}[p61]
\pstLineAB[linestyle=dotted,linecolor=gray]{n61}{p61}
\pstGeonode[PointSymbol=Bo,fillcolor=lightgray](5.40,3.90){n62}
\pstProjection[PointSymbol=o,dotscale=1.25]{na}{nb}{n62}[p62]
\pstLineAB[linestyle=dotted,linecolor=gray]{n62}{p62}
\pstGeonode[PointSymbol=Bo,fillcolor=lightgray](5.00,3.20){n63}
\pstProjection[PointSymbol=o,dotscale=1.25]{na}{nb}{n63}[p63]
\pstLineAB[linestyle=dotted,linecolor=gray]{n63}{p63}
\pstGeonode[PointSymbol=Bo,fillcolor=lightgray](5.10,3.80){n64}
\pstProjection[PointSymbol=o,dotscale=1.25]{na}{nb}{n64}[p64]
\pstLineAB[linestyle=dotted,linecolor=gray]{n64}{p64}
\pstGeonode[PointSymbol=Bo,fillcolor=lightgray](4.80,3.00){n65}
\pstProjection[PointSymbol=o,dotscale=1.25]{na}{nb}{n65}[p65]
\pstLineAB[linestyle=dotted,linecolor=gray]{n65}{p65}
\pstGeonode[PointSymbol=Bo,fillcolor=lightgray](5.30,3.70){n66}
\pstProjection[PointSymbol=o,dotscale=1.25]{na}{nb}{n66}[p66]
\pstLineAB[linestyle=dotted,linecolor=gray]{n66}{p66}
\pstGeonode[PointSymbol=Bo,fillcolor=lightgray](5.70,3.80){n67}
\pstProjection[PointSymbol=o,dotscale=1.25]{na}{nb}{n67}[p67]
\pstLineAB[linestyle=dotted,linecolor=gray]{n67}{p67}
\pstGeonode[PointSymbol=Bo,fillcolor=lightgray](4.40,3.00){n68}
\pstProjection[PointSymbol=o,dotscale=1.25]{na}{nb}{n68}[p68]
\pstLineAB[linestyle=dotted,linecolor=gray]{n68}{p68}
\pstGeonode[PointSymbol=Bo,fillcolor=lightgray](5.40,3.40){n69}
\pstProjection[PointSymbol=o,dotscale=1.25]{na}{nb}{n69}[p69]
\pstLineAB[linestyle=dotted,linecolor=gray]{n69}{p69}
\pstGeonode[PointSymbol=Bo,fillcolor=lightgray](5.00,3.50){n70}
\pstProjection[PointSymbol=o,dotscale=1.25]{na}{nb}{n70}[p70]
\pstLineAB[linestyle=dotted,linecolor=gray]{n70}{p70}
\pstGeonode[PointSymbol=Bo,fillcolor=lightgray](5.10,3.30){n71}
\pstProjection[PointSymbol=o,dotscale=1.25]{na}{nb}{n71}[p71]
\pstLineAB[linestyle=dotted,linecolor=gray]{n71}{p71}
\pstGeonode[PointSymbol=Bo,fillcolor=lightgray](4.90,3.10){n72}
\pstProjection[PointSymbol=o,dotscale=1.25]{na}{nb}{n72}[p72]
\pstLineAB[linestyle=dotted,linecolor=gray]{n72}{p72}
\pstGeonode[PointSymbol=Bo,fillcolor=lightgray](4.60,3.60){n73}
\pstProjection[PointSymbol=o,dotscale=1.25]{na}{nb}{n73}[p73]
\pstLineAB[linestyle=dotted,linecolor=gray]{n73}{p73}
\pstGeonode[PointSymbol=Bo,fillcolor=lightgray](5.20,3.40){n74}
\pstProjection[PointSymbol=o,dotscale=1.25]{na}{nb}{n74}[p74]
\pstLineAB[linestyle=dotted,linecolor=gray]{n74}{p74}
\pstGeonode[PointSymbol=Bo,fillcolor=lightgray](5.50,3.50){n75}
\pstProjection[PointSymbol=o,dotscale=1.25]{na}{nb}{n75}[p75]
\pstLineAB[linestyle=dotted,linecolor=gray]{n75}{p75}
\pstGeonode[PointSymbol=Bo,fillcolor=lightgray](4.60,3.40){n76}
\pstProjection[PointSymbol=o,dotscale=1.25]{na}{nb}{n76}[p76]
\pstLineAB[linestyle=dotted,linecolor=gray]{n76}{p76}
\pstGeonode[PointSymbol=Bo,fillcolor=lightgray](4.70,3.20){n77}
\pstProjection[PointSymbol=o,dotscale=1.25]{na}{nb}{n77}[p77]
\pstLineAB[linestyle=dotted,linecolor=gray]{n77}{p77}
\pstGeonode[PointSymbol=Bo,fillcolor=lightgray](4.40,2.90){n78}
\pstProjection[PointSymbol=o,dotscale=1.25]{na}{nb}{n78}[p78]
\pstLineAB[linestyle=dotted,linecolor=gray]{n78}{p78}
\pstGeonode[PointSymbol=Bo,fillcolor=lightgray](4.80,3.00){n79}
\pstProjection[PointSymbol=o,dotscale=1.25]{na}{nb}{n79}[p79]
\pstLineAB[linestyle=dotted,linecolor=gray]{n79}{p79}
\pstGeonode[PointSymbol=Bo,fillcolor=lightgray](5.40,3.90){n80}
\pstProjection[PointSymbol=o,dotscale=1.25]{na}{nb}{n80}[p80]
\pstLineAB[linestyle=dotted,linecolor=gray]{n80}{p80}
\pstGeonode[PointSymbol=Bo,fillcolor=lightgray](4.80,3.40){n81}
\pstProjection[PointSymbol=o,dotscale=1.25]{na}{nb}{n81}[p81]
\pstLineAB[linestyle=dotted,linecolor=gray]{n81}{p81}
\pstGeonode[PointSymbol=Bo,fillcolor=lightgray](4.40,3.20){n82}
\pstProjection[PointSymbol=o,dotscale=1.25]{na}{nb}{n82}[p82]
\pstLineAB[linestyle=dotted,linecolor=gray]{n82}{p82}
\pstGeonode[PointSymbol=Bo,fillcolor=lightgray](4.80,3.40){n83}
\pstProjection[PointSymbol=o,dotscale=1.25]{na}{nb}{n83}[p83]
\pstLineAB[linestyle=dotted,linecolor=gray]{n83}{p83}
\pstGeonode[PointSymbol=Bo,fillcolor=lightgray](4.90,3.00){n84}
\pstProjection[PointSymbol=o,dotscale=1.25]{na}{nb}{n84}[p84]
\pstLineAB[linestyle=dotted,linecolor=gray]{n84}{p84}
\pstGeonode[PointSymbol=Bo,fillcolor=lightgray](4.50,2.30){n85}
\pstProjection[PointSymbol=o,dotscale=1.25]{na}{nb}{n85}[p85]
\pstLineAB[linestyle=dotted,linecolor=gray]{n85}{p85}
\pstGeonode[PointSymbol=Bo,fillcolor=lightgray](4.30,3.00){n86}
\pstProjection[PointSymbol=o,dotscale=1.25]{na}{nb}{n86}[p86]
\pstLineAB[linestyle=dotted,linecolor=gray]{n86}{p86}
\pstGeonode[PointSymbol=Bo,fillcolor=lightgray](5.00,3.60){n87}
\pstProjection[PointSymbol=o,dotscale=1.25]{na}{nb}{n87}[p87]
\pstLineAB[linestyle=dotted,linecolor=gray]{n87}{p87}
\pstGeonode[PointSymbol=Bo,fillcolor=lightgray](5.20,3.50){n88}
\pstProjection[PointSymbol=o,dotscale=1.25]{na}{nb}{n88}[p88]
\pstLineAB[linestyle=dotted,linecolor=gray]{n88}{p88}
\pstGeonode[PointSymbol=Bo,fillcolor=lightgray](5.50,4.20){n89}
\pstProjection[PointSymbol=o,dotscale=1.25]{na}{nb}{n89}[p89]
\pstLineAB[linestyle=dotted,linecolor=gray]{n89}{p89}
\pstGeonode[PointSymbol=Bo,fillcolor=lightgray](5.00,3.00){n90}
\pstProjection[PointSymbol=o,dotscale=1.25]{na}{nb}{n90}[p90]
\pstLineAB[linestyle=dotted,linecolor=gray]{n90}{p90}
\pstGeonode[PointSymbol=Bo,fillcolor=lightgray](5.10,3.50){n91}
\pstProjection[PointSymbol=o,dotscale=1.25]{na}{nb}{n91}[p91]
\pstLineAB[linestyle=dotted,linecolor=gray]{n91}{p91}
\pstGeonode[PointSymbol=Bo,fillcolor=lightgray](4.80,3.10){n92}
\pstProjection[PointSymbol=o,dotscale=1.25]{na}{nb}{n92}[p92]
\pstLineAB[linestyle=dotted,linecolor=gray]{n92}{p92}
\pstGeonode[PointSymbol=Bo,fillcolor=lightgray](5.10,3.80){n93}
\pstProjection[PointSymbol=o,dotscale=1.25]{na}{nb}{n93}[p93]
\pstLineAB[linestyle=dotted,linecolor=gray]{n93}{p93}
\pstGeonode[PointSymbol=Bo,fillcolor=lightgray](5.20,4.10){n94}
\pstProjection[PointSymbol=o,dotscale=1.25]{na}{nb}{n94}[p94]
\pstLineAB[linestyle=dotted,linecolor=gray]{n94}{p94}
\pstGeonode[PointSymbol=Bo,fillcolor=lightgray](5.80,4.00){n95}
\pstProjection[PointSymbol=o,dotscale=1.25]{na}{nb}{n95}[p95]
\pstLineAB[linestyle=dotted,linecolor=gray]{n95}{p95}
\pstGeonode[PointSymbol=Bo,fillcolor=lightgray](4.90,3.10){n96}
\pstProjection[PointSymbol=o,dotscale=1.25]{na}{nb}{n96}[p96]
\pstLineAB[linestyle=dotted,linecolor=gray]{n96}{p96}
\pstGeonode[PointSymbol=Bo,fillcolor=lightgray](5.40,3.70){n97}
\pstProjection[PointSymbol=o,dotscale=1.25]{na}{nb}{n97}[p97]
\pstLineAB[linestyle=dotted,linecolor=gray]{n97}{p97}
\pstGeonode[PointSymbol=Bo,fillcolor=lightgray](5.10,3.50){n98}
\pstProjection[PointSymbol=o,dotscale=1.25]{na}{nb}{n98}[p98]
\pstLineAB[linestyle=dotted,linecolor=gray]{n98}{p98}
\pstGeonode[PointSymbol=Bo,fillcolor=lightgray](5.40,3.40){n99}
\pstProjection[PointSymbol=o,dotscale=1.25]{na}{nb}{n99}[p99]
\pstLineAB[linestyle=dotted,linecolor=gray]{n99}{p99}
\pstGeonode[PointSymbol=Bo,fillcolor=lightgray](5.10,3.40){n100}
\pstProjection[PointSymbol=o,dotscale=1.25]{na}{nb}{n100}[p100]
\pstLineAB[linestyle=dotted,linecolor=gray]{n100}{p100}
\pstGeonode[PointSymbol=Btriangle,fillcolor=lightgray](5.90,3.00){n101}
\pstProjection[PointSymbol=triangle,dotscale=1.25]{na}{nb}{n101}[p101]
\pstLineAB[linestyle=dotted,linecolor=gray]{n101}{p101}
\pstGeonode[PointSymbol=Btriangle,fillcolor=lightgray](6.90,3.10){n102}
\pstProjection[PointSymbol=triangle,dotscale=1.25]{na}{nb}{n102}[p102]
\pstLineAB[linestyle=dotted,linecolor=gray]{n102}{p102}
\pstGeonode[PointSymbol=Btriangle,fillcolor=lightgray](6.60,2.90){n103}
\pstProjection[PointSymbol=triangle,dotscale=1.25]{na}{nb}{n103}[p103]
\pstLineAB[linestyle=dotted,linecolor=gray]{n103}{p103}
\pstGeonode[PointSymbol=Btriangle,fillcolor=lightgray](6.00,2.20){n104}
\pstProjection[PointSymbol=triangle,dotscale=1.25]{na}{nb}{n104}[p104]
\pstLineAB[linestyle=dotted,linecolor=gray]{n104}{p104}
\pstGeonode[PointSymbol=Btriangle,fillcolor=lightgray](5.70,3.00){n105}
\pstProjection[PointSymbol=triangle,dotscale=1.25]{na}{nb}{n105}[p105]
\pstLineAB[linestyle=dotted,linecolor=gray]{n105}{p105}
\pstGeonode[PointSymbol=Btriangle,fillcolor=lightgray](5.70,2.80){n106}
\pstProjection[PointSymbol=triangle,dotscale=1.25]{na}{nb}{n106}[p106]
\pstLineAB[linestyle=dotted,linecolor=gray]{n106}{p106}
\pstGeonode[PointSymbol=Btriangle,fillcolor=lightgray](5.50,2.50){n107}
\pstProjection[PointSymbol=triangle,dotscale=1.25]{na}{nb}{n107}[p107]
\pstLineAB[linestyle=dotted,linecolor=gray]{n107}{p107}
\pstGeonode[PointSymbol=Btriangle,fillcolor=lightgray](5.50,2.30){n108}
\pstProjection[PointSymbol=triangle,dotscale=1.25]{na}{nb}{n108}[p108]
\pstLineAB[linestyle=dotted,linecolor=gray]{n108}{p108}
\pstGeonode[PointSymbol=Btriangle,fillcolor=lightgray](5.80,2.60){n109}
\pstProjection[PointSymbol=triangle,dotscale=1.25]{na}{nb}{n109}[p109]
\pstLineAB[linestyle=dotted,linecolor=gray]{n109}{p109}
\pstGeonode[PointSymbol=Btriangle,fillcolor=lightgray](5.10,2.50){n110}
\pstProjection[PointSymbol=triangle,dotscale=1.25]{na}{nb}{n110}[p110]
\pstLineAB[linestyle=dotted,linecolor=gray]{n110}{p110}
\pstGeonode[PointSymbol=Btriangle,fillcolor=lightgray](5.60,2.50){n111}
\pstProjection[PointSymbol=triangle,dotscale=1.25]{na}{nb}{n111}[p111]
\pstLineAB[linestyle=dotted,linecolor=gray]{n111}{p111}
\pstGeonode[PointSymbol=Btriangle,fillcolor=lightgray](5.80,2.70){n112}
\pstProjection[PointSymbol=triangle,dotscale=1.25]{na}{nb}{n112}[p112]
\pstLineAB[linestyle=dotted,linecolor=gray]{n112}{p112}
\pstGeonode[PointSymbol=Btriangle,fillcolor=lightgray](6.30,2.30){n113}
\pstProjection[PointSymbol=triangle,dotscale=1.25]{na}{nb}{n113}[p113]
\pstLineAB[linestyle=dotted,linecolor=gray]{n113}{p113}
\pstGeonode[PointSymbol=Btriangle,fillcolor=lightgray](5.60,3.00){n114}
\pstProjection[PointSymbol=triangle,dotscale=1.25]{na}{nb}{n114}[p114]
\pstLineAB[linestyle=dotted,linecolor=gray]{n114}{p114}
\pstGeonode[PointSymbol=Btriangle,fillcolor=lightgray](6.10,3.00){n115}
\pstProjection[PointSymbol=triangle,dotscale=1.25]{na}{nb}{n115}[p115]
\pstLineAB[linestyle=dotted,linecolor=gray]{n115}{p115}
\pstGeonode[PointSymbol=Btriangle,fillcolor=lightgray](5.60,2.70){n116}
\pstProjection[PointSymbol=triangle,dotscale=1.25]{na}{nb}{n116}[p116]
\pstLineAB[linestyle=dotted,linecolor=gray]{n116}{p116}
\pstGeonode[PointSymbol=Btriangle,fillcolor=lightgray](5.70,2.60){n117}
\pstProjection[PointSymbol=triangle,dotscale=1.25]{na}{nb}{n117}[p117]
\pstLineAB[linestyle=dotted,linecolor=gray]{n117}{p117}
\pstGeonode[PointSymbol=Btriangle,fillcolor=lightgray](5.70,2.90){n118}
\pstProjection[PointSymbol=triangle,dotscale=1.25]{na}{nb}{n118}[p118]
\pstLineAB[linestyle=dotted,linecolor=gray]{n118}{p118}
\pstGeonode[PointSymbol=Btriangle,fillcolor=lightgray](6.10,2.80){n119}
\pstProjection[PointSymbol=triangle,dotscale=1.25]{na}{nb}{n119}[p119]
\pstLineAB[linestyle=dotted,linecolor=gray]{n119}{p119}
\pstGeonode[PointSymbol=Btriangle,fillcolor=lightgray](5.60,2.90){n120}
\pstProjection[PointSymbol=triangle,dotscale=1.25]{na}{nb}{n120}[p120]
\pstLineAB[linestyle=dotted,linecolor=gray]{n120}{p120}
\pstGeonode[PointSymbol=Btriangle,fillcolor=lightgray](5.60,3.00){n121}
\pstProjection[PointSymbol=triangle,dotscale=1.25]{na}{nb}{n121}[p121]
\pstLineAB[linestyle=dotted,linecolor=gray]{n121}{p121}
\pstGeonode[PointSymbol=Btriangle,fillcolor=lightgray](5.40,3.00){n122}
\pstProjection[PointSymbol=triangle,dotscale=1.25]{na}{nb}{n122}[p122]
\pstLineAB[linestyle=dotted,linecolor=gray]{n122}{p122}
\pstGeonode[PointSymbol=Btriangle,fillcolor=lightgray](6.10,2.80){n123}
\pstProjection[PointSymbol=triangle,dotscale=1.25]{na}{nb}{n123}[p123]
\pstLineAB[linestyle=dotted,linecolor=gray]{n123}{p123}
\pstGeonode[PointSymbol=Btriangle,fillcolor=lightgray](6.00,2.90){n124}
\pstProjection[PointSymbol=triangle,dotscale=1.25]{na}{nb}{n124}[p124]
\pstLineAB[linestyle=dotted,linecolor=gray]{n124}{p124}
\pstGeonode[PointSymbol=Btriangle,fillcolor=lightgray](5.00,2.30){n125}
\pstProjection[PointSymbol=triangle,dotscale=1.25]{na}{nb}{n125}[p125]
\pstLineAB[linestyle=dotted,linecolor=gray]{n125}{p125}
\pstGeonode[PointSymbol=Btriangle,fillcolor=lightgray](6.40,3.20){n126}
\pstProjection[PointSymbol=triangle,dotscale=1.25]{na}{nb}{n126}[p126]
\pstLineAB[linestyle=dotted,linecolor=gray]{n126}{p126}
\pstGeonode[PointSymbol=Btriangle,fillcolor=lightgray](6.70,3.00){n127}
\pstProjection[PointSymbol=triangle,dotscale=1.25]{na}{nb}{n127}[p127]
\pstLineAB[linestyle=dotted,linecolor=gray]{n127}{p127}
\pstGeonode[PointSymbol=Btriangle,fillcolor=lightgray](5.00,2.00){n128}
\pstProjection[PointSymbol=triangle,dotscale=1.25]{na}{nb}{n128}[p128]
\pstLineAB[linestyle=dotted,linecolor=gray]{n128}{p128}
\pstGeonode[PointSymbol=Btriangle,fillcolor=lightgray](5.90,3.20){n129}
\pstProjection[PointSymbol=triangle,dotscale=1.25]{na}{nb}{n129}[p129]
\pstLineAB[linestyle=dotted,linecolor=gray]{n129}{p129}
\pstGeonode[PointSymbol=Btriangle,fillcolor=lightgray](6.20,2.20){n130}
\pstProjection[PointSymbol=triangle,dotscale=1.25]{na}{nb}{n130}[p130]
\pstLineAB[linestyle=dotted,linecolor=gray]{n130}{p130}
\pstGeonode[PointSymbol=Btriangle,fillcolor=lightgray](4.90,2.40){n131}
\pstProjection[PointSymbol=triangle,dotscale=1.25]{na}{nb}{n131}[p131]
\pstLineAB[linestyle=dotted,linecolor=gray]{n131}{p131}
\pstGeonode[PointSymbol=Btriangle,fillcolor=lightgray](7.00,3.20){n132}
\pstProjection[PointSymbol=triangle,dotscale=1.25]{na}{nb}{n132}[p132]
\pstLineAB[linestyle=dotted,linecolor=gray]{n132}{p132}
\pstGeonode[PointSymbol=Btriangle,fillcolor=lightgray](5.50,2.40){n133}
\pstProjection[PointSymbol=triangle,dotscale=1.25]{na}{nb}{n133}[p133]
\pstLineAB[linestyle=dotted,linecolor=gray]{n133}{p133}
\pstGeonode[PointSymbol=Btriangle,fillcolor=lightgray](6.30,3.30){n134}
\pstProjection[PointSymbol=triangle,dotscale=1.25]{na}{nb}{n134}[p134]
\pstLineAB[linestyle=dotted,linecolor=gray]{n134}{p134}
\pstGeonode[PointSymbol=Btriangle,fillcolor=lightgray](6.80,2.80){n135}
\pstProjection[PointSymbol=triangle,dotscale=1.25]{na}{nb}{n135}[p135]
\pstLineAB[linestyle=dotted,linecolor=gray]{n135}{p135}
\pstGeonode[PointSymbol=Btriangle,fillcolor=lightgray](6.10,2.90){n136}
\pstProjection[PointSymbol=triangle,dotscale=1.25]{na}{nb}{n136}[p136]
\pstLineAB[linestyle=dotted,linecolor=gray]{n136}{p136}
\pstGeonode[PointSymbol=Btriangle,fillcolor=lightgray](6.70,3.10){n137}
\pstProjection[PointSymbol=triangle,dotscale=1.25]{na}{nb}{n137}[p137]
\pstLineAB[linestyle=dotted,linecolor=gray]{n137}{p137}
\pstGeonode[PointSymbol=Btriangle,fillcolor=lightgray](5.20,2.70){n138}
\pstProjection[PointSymbol=triangle,dotscale=1.25]{na}{nb}{n138}[p138]
\pstLineAB[linestyle=dotted,linecolor=gray]{n138}{p138}
\pstGeonode[PointSymbol=Btriangle,fillcolor=lightgray](6.40,2.90){n139}
\pstProjection[PointSymbol=triangle,dotscale=1.25]{na}{nb}{n139}[p139]
\pstLineAB[linestyle=dotted,linecolor=gray]{n139}{p139}
\pstGeonode[PointSymbol=Btriangle,fillcolor=lightgray](5.80,2.70){n140}
\pstProjection[PointSymbol=triangle,dotscale=1.25]{na}{nb}{n140}[p140]
\pstLineAB[linestyle=dotted,linecolor=gray]{n140}{p140}
\pstGeonode[PointSymbol=Btriangle,fillcolor=lightgray](6.30,2.50){n141}
\pstProjection[PointSymbol=triangle,dotscale=1.25]{na}{nb}{n141}[p141]
\pstLineAB[linestyle=dotted,linecolor=gray]{n141}{p141}
\pstGeonode[PointSymbol=Btriangle,fillcolor=lightgray](6.70,3.10){n142}
\pstProjection[PointSymbol=triangle,dotscale=1.25]{na}{nb}{n142}[p142]
\pstLineAB[linestyle=dotted,linecolor=gray]{n142}{p142}
\pstGeonode[PointSymbol=Btriangle,fillcolor=lightgray](6.00,2.70){n143}
\pstProjection[PointSymbol=triangle,dotscale=1.25]{na}{nb}{n143}[p143]
\pstLineAB[linestyle=dotted,linecolor=gray]{n143}{p143}
\pstGeonode[PointSymbol=Btriangle,fillcolor=lightgray](5.70,2.80){n144}
\pstProjection[PointSymbol=triangle,dotscale=1.25]{na}{nb}{n144}[p144]
\pstLineAB[linestyle=dotted,linecolor=gray]{n144}{p144}
\pstGeonode[PointSymbol=Btriangle,fillcolor=lightgray](6.60,3.00){n145}
\pstProjection[PointSymbol=triangle,dotscale=1.25]{na}{nb}{n145}[p145]
\pstLineAB[linestyle=dotted,linecolor=gray]{n145}{p145}
\pstGeonode[PointSymbol=Btriangle,fillcolor=lightgray](6.00,3.40){n146}
\pstProjection[PointSymbol=triangle,dotscale=1.25]{na}{nb}{n146}[p146]
\pstLineAB[linestyle=dotted,linecolor=gray]{n146}{p146}
\pstGeonode[PointSymbol=Btriangle,fillcolor=lightgray](5.50,2.40){n147}
\pstProjection[PointSymbol=triangle,dotscale=1.25]{na}{nb}{n147}[p147]
\pstLineAB[linestyle=dotted,linecolor=gray]{n147}{p147}
\pstGeonode[PointSymbol=Btriangle,fillcolor=lightgray](6.20,2.90){n148}
\pstProjection[PointSymbol=triangle,dotscale=1.25]{na}{nb}{n148}[p148]
\pstLineAB[linestyle=dotted,linecolor=gray]{n148}{p148}
\pstGeonode[PointSymbol=Btriangle,fillcolor=lightgray](6.50,2.80){n149}
\pstProjection[PointSymbol=triangle,dotscale=1.25]{na}{nb}{n149}[p149]
\pstLineAB[linestyle=dotted,linecolor=gray]{n149}{p149}
\pstGeonode[PointSymbol=Btriangle,fillcolor=lightgray](5.50,2.60){n150}
\pstProjection[PointSymbol=triangle,dotscale=1.25]{na}{nb}{n150}[p150]
\pstLineAB[linestyle=dotted,linecolor=gray]{n150}{p150}

    \psset{dotscale=2,fillcolor=black}
    \pstGeonode[PointSymbol=none,
        fillcolor=lightgray](5.01,3.42){mu1}
    \pstProjection[PointSymbol=Bo]{na}{nb}{mu1}[m1]
    \pstGeonode[PointSymbol=none,
        fillcolor=lightgray](6.26,2.87){mu2}
    \pstProjection[PointSymbol=Btriangle]{na}{nb}{mu2}[m2]
    \end{pspicture}
    }
\end{figure}
\end{frame}



\begin{frame}{Optimal Linear Discriminant}
The mean of the projected points is given as:
\begin{align*}
    m_1  = & \bw^T\bmu_1 & 
	m_2 = & \bw^T\bmu_2
\end{align*}

To maximize the separation between the classes, we 
maximize the difference between the projected means,
$|m_1 - m_2|$. However, for good separation,
the variance of the projected points for each class should also
not be too large. 
LDA
maximizes the separation by ensuring that the {\em
  scatter} $s_i^2$ for the projected points within each class is
small, where scatter is def\/{i}ned as
\begin{align*}
  s_i^2 = \dsum_{\bx_{j} \in \bD_i} (a_{j} - m_i)^2 = n_i \sigma_i^2
\end{align*}
where $\sigma_i^2$ is the
variance for class $c_i$.
\end{frame}



\begin{frame}{Linear Discriminant Analysis: F{i}sher Objective}
  \small
We incorporate the two LDA criteria, namely, maximizing the
distance between projected means and minimizing the sum of
projected scatter, into a single maximization criterion called the
{\em F{i}sher LDA objective}: 
\begin{align*}
\tcbhighmath{
    \max_{\bw} \;\; J(\bw) = \frac{(m_1 - m_2)^2}{s_1^2 + s_2^2}
}
\end{align*}

\medskip
In matrix terms, we can rewrite $(m_1 - m_2)^2$ as follows:
\begin{align*}
  (m_1 - m_2)^2 = & \lB(\bw^T(\bmu_1 - \bmu_2)\rB)^2 = \bw^T\bB\bw
\end{align*}
where $\bB = (\bmu_1 - \bmu_2) (\bmu_1 - \bmu_2)^T$ is a $d\times
d$ rank-one matrix called the {\em between-class scatter matrix}.

\medskip
The projected scatter for class $c_i$ is given as
\begin{align*}
  s_i^2 = &
  \dsum_{\bx_j \in \bD_1} (\bw^T\bx_j - \bw^T\bmu_i)^2
  =  \bw^T \lB(\dsum_{\bx_j \in \bD_i} (\bx_j - \bmu_i)(\bx_j -
  \bmu_i)^T \rB) \bw
  =  \bw^T\bS_i\bw
\end{align*}
where $\bS_i$ is the {\em scatter matrix} for $\bD_i$.
\end{frame}


\begin{frame}{Linear Discriminant Analysis: F{i}sher Objective}
The combined scatter for both classes is given as
\begin{align*}
  s_1^2 + s_2^2 = \bw^T \bS_1 \bw + \bw^T \bS_2 \bw
  = \bw^T (\bS_1 + \bS_2) \bw= \bw^T \bS \bw
\end{align*}
where the symmetric positive semidef\/{i}nite matrix 
$\bS = \bS_1 + \bS_2$ denotes the {\em within-class scatter
matrix} for the pooled data. 

The LDA objective function in matrix form is
\begin{align*}
\tcbhighmath{
    \max_{\bw} \;\; J(\bw) = \frac{\bw^T\bB\bw}{\bw^T\bS\bw}
}
\end{align*}

\medskip
To solve for the best direction $\bw$, we differentiate the objective
function with respect to $\bw$; after simplification it yields the 
{\em generalized eigenvalue problem}
\begin{align*}
  \bB \bw  = &\; \lambda \bS \bw
\end{align*}
where $\lambda = J(\bw)$ is
a generalized
eigenvalue of $\bB$ and $\bS$.
To maximize the objective
$\lambda$ should be chosen to be the largest generalized
eigenvalue, and $\bw$ to be the corresponding eigenvector.
\end{frame}



\newcommand{\LDA}{\textsc{LinearDiscriminant}\xspace}
\begin{frame}[fragile]{Linear Discriminant Algorithm}
\begin{tightalgo}[H]{\textwidth-18pt}
\SetKwInOut{Algorithm}{\LDA($\bD = \{(\bx_i, y_i)\}_{i=1}^n$)}
\Algorithm{} 
$\bD_i \assign \bigl\{\bx_{j} \mid y_{j}=c_i, j=1,\ldots,n\bigr\}, i=1,2$ \tcp{class-specif\/{i}c subsets} 
$\bmu_i \assign \text{mean}(\bD_i), i=1,2$ \tcp{class means} 
$\bB \assign (\bmu_1-\bmu_2)(\bmu_1-\bmu_2)^T$ \tcp{between-class scatter matrix} 
$\bZ_i \assign \bD_i - \bone_{n_i} \bmu_i^T, i=1,2$ \tcp{center class matrices}
$\bS_i  \assign \bZ_i^T\bZ_i, i=1,2$ \tcp{class scatter matrices} 
$\bS \assign \bS_1 + \bS_2$ \tcp{within-class scatter matrix} 
$\lambda_1, \bw \assign \text{eigen}(\bS^{-1}\bB)$ \tcp{compute dominant eigenvector}
\end{tightalgo}
\end{frame}


\begin{frame}{Linear Discriminant Direction: Iris 2D Data}
  \begin{columns}
	\column{0.5\textwidth}
\begin{figure}[!t]\vspace*{-14pt}
    \centering
    %\vspace{0.2in}
    \scalebox{0.45}{%
    \psset{dotscale=1.5,dotsep=0.03,arrowscale=2,PointName=none,unit=1.0in}
    \begin{pspicture}(4,1.5)(8.5,5)
    \psaxes[Dx=0.5,Dy=0.5,Ox=4.0,Oy=1.5]{->}(4.0,1.5)(8.5,5)
    %\begin{pspicture}(4,1.5)(8,4.5)
    %\psaxes[Dx=0.5,Dy=0.5,Ox=4.0,Oy=1.5]{->}(4.0,1.5)(8.0,4.5)
    \pstGeonode[PointSymbol=none](4.88,4.51){na}
    \pstGeonode[PointSymbol=none](6.89,1.47){nb}
    \pstLineAB[linewidth=2pt,arrows=->]{na}{nb}
    \uput[90](6.9,1.6){$\bw$}
    \pstGeonode[PointSymbol=Btriangle,fillcolor=lightgray](6.50,3.00){n1}
\pstProjection[PointSymbol=triangle,dotscale=1.25]{na}{nb}{n1}[p1]
\pstLineAB[linestyle=dotted,linecolor=gray]{n1}{p1}
\pstGeonode[PointSymbol=Btriangle,fillcolor=lightgray](5.80,2.70){n2}
\pstProjection[PointSymbol=triangle,dotscale=1.25]{na}{nb}{n2}[p2]
\pstLineAB[linestyle=dotted,linecolor=gray]{n2}{p2}
\pstGeonode[PointSymbol=Btriangle,fillcolor=lightgray](6.70,3.10){n3}
\pstProjection[PointSymbol=triangle,dotscale=1.25]{na}{nb}{n3}[p3]
\pstLineAB[linestyle=dotted,linecolor=gray]{n3}{p3}
\pstGeonode[PointSymbol=Btriangle,fillcolor=lightgray](6.70,2.50){n4}
\pstProjection[PointSymbol=triangle,dotscale=1.25]{na}{nb}{n4}[p4]
\pstLineAB[linestyle=dotted,linecolor=gray]{n4}{p4}
\pstGeonode[PointSymbol=Btriangle,fillcolor=lightgray](6.10,3.00){n5}
\pstProjection[PointSymbol=triangle,dotscale=1.25]{na}{nb}{n5}[p5]
\pstLineAB[linestyle=dotted,linecolor=gray]{n5}{p5}
\pstGeonode[PointSymbol=Btriangle,fillcolor=lightgray](7.20,3.20){n6}
\pstProjection[PointSymbol=triangle,dotscale=1.25]{na}{nb}{n6}[p6]
\pstLineAB[linestyle=dotted,linecolor=gray]{n6}{p6}
\pstGeonode[PointSymbol=Btriangle,fillcolor=lightgray](5.90,3.00){n7}
\pstProjection[PointSymbol=triangle,dotscale=1.25]{na}{nb}{n7}[p7]
\pstLineAB[linestyle=dotted,linecolor=gray]{n7}{p7}
\pstGeonode[PointSymbol=Btriangle,fillcolor=lightgray](6.50,3.00){n8}
\pstProjection[PointSymbol=triangle,dotscale=1.25]{na}{nb}{n8}[p8]
\pstLineAB[linestyle=dotted,linecolor=gray]{n8}{p8}
\pstGeonode[PointSymbol=Btriangle,fillcolor=lightgray](4.90,2.50){n9}
\pstProjection[PointSymbol=triangle,dotscale=1.25]{na}{nb}{n9}[p9]
\pstLineAB[linestyle=dotted,linecolor=gray]{n9}{p9}
\pstGeonode[PointSymbol=Btriangle,fillcolor=lightgray](7.20,3.00){n10}
\pstProjection[PointSymbol=triangle,dotscale=1.25]{na}{nb}{n10}[p10]
\pstLineAB[linestyle=dotted,linecolor=gray]{n10}{p10}
\pstGeonode[PointSymbol=Btriangle,fillcolor=lightgray](6.80,3.20){n11}
\pstProjection[PointSymbol=triangle,dotscale=1.25]{na}{nb}{n11}[p11]
\pstLineAB[linestyle=dotted,linecolor=gray]{n11}{p11}
\pstGeonode[PointSymbol=Btriangle,fillcolor=lightgray](5.70,2.50){n12}
\pstProjection[PointSymbol=triangle,dotscale=1.25]{na}{nb}{n12}[p12]
\pstLineAB[linestyle=dotted,linecolor=gray]{n12}{p12}
\pstGeonode[PointSymbol=Btriangle,fillcolor=lightgray](6.30,2.50){n13}
\pstProjection[PointSymbol=triangle,dotscale=1.25]{na}{nb}{n13}[p13]
\pstLineAB[linestyle=dotted,linecolor=gray]{n13}{p13}
\pstGeonode[PointSymbol=Btriangle,fillcolor=lightgray](6.80,3.00){n14}
\pstProjection[PointSymbol=triangle,dotscale=1.25]{na}{nb}{n14}[p14]
\pstLineAB[linestyle=dotted,linecolor=gray]{n14}{p14}
\pstGeonode[PointSymbol=Btriangle,fillcolor=lightgray](7.30,2.90){n15}
\pstProjection[PointSymbol=triangle,dotscale=1.25]{na}{nb}{n15}[p15]
\pstLineAB[linestyle=dotted,linecolor=gray]{n15}{p15}
\pstGeonode[PointSymbol=Btriangle,fillcolor=lightgray](7.10,3.00){n16}
\pstProjection[PointSymbol=triangle,dotscale=1.25]{na}{nb}{n16}[p16]
\pstLineAB[linestyle=dotted,linecolor=gray]{n16}{p16}
\pstGeonode[PointSymbol=Btriangle,fillcolor=lightgray](5.60,2.80){n17}
\pstProjection[PointSymbol=triangle,dotscale=1.25]{na}{nb}{n17}[p17]
\pstLineAB[linestyle=dotted,linecolor=gray]{n17}{p17}
\pstGeonode[PointSymbol=Btriangle,fillcolor=lightgray](6.30,3.30){n18}
\pstProjection[PointSymbol=triangle,dotscale=1.25]{na}{nb}{n18}[p18]
\pstLineAB[linestyle=dotted,linecolor=gray]{n18}{p18}
\pstGeonode[PointSymbol=Btriangle,fillcolor=lightgray](6.30,3.40){n19}
\pstProjection[PointSymbol=triangle,dotscale=1.25]{na}{nb}{n19}[p19]
\pstLineAB[linestyle=dotted,linecolor=gray]{n19}{p19}
\pstGeonode[PointSymbol=Btriangle,fillcolor=lightgray](6.90,3.10){n20}
\pstProjection[PointSymbol=triangle,dotscale=1.25]{na}{nb}{n20}[p20]
\pstLineAB[linestyle=dotted,linecolor=gray]{n20}{p20}
\pstGeonode[PointSymbol=Btriangle,fillcolor=lightgray](7.70,3.00){n21}
\pstProjection[PointSymbol=triangle,dotscale=1.25]{na}{nb}{n21}[p21]
\pstLineAB[linestyle=dotted,linecolor=gray]{n21}{p21}
\pstGeonode[PointSymbol=Btriangle,fillcolor=lightgray](6.10,2.60){n22}
\pstProjection[PointSymbol=triangle,dotscale=1.25]{na}{nb}{n22}[p22]
\pstLineAB[linestyle=dotted,linecolor=gray]{n22}{p22}
\pstGeonode[PointSymbol=Btriangle,fillcolor=lightgray](6.40,2.70){n23}
\pstProjection[PointSymbol=triangle,dotscale=1.25]{na}{nb}{n23}[p23]
\pstLineAB[linestyle=dotted,linecolor=gray]{n23}{p23}
\pstGeonode[PointSymbol=Btriangle,fillcolor=lightgray](5.80,2.80){n24}
\pstProjection[PointSymbol=triangle,dotscale=1.25]{na}{nb}{n24}[p24]
\pstLineAB[linestyle=dotted,linecolor=gray]{n24}{p24}
\pstGeonode[PointSymbol=Btriangle,fillcolor=lightgray](7.90,3.80){n25}
\pstProjection[PointSymbol=triangle,dotscale=1.25]{na}{nb}{n25}[p25]
\pstLineAB[linestyle=dotted,linecolor=gray]{n25}{p25}
\pstGeonode[PointSymbol=Btriangle,fillcolor=lightgray](7.70,2.60){n26}
\pstProjection[PointSymbol=triangle,dotscale=1.25]{na}{nb}{n26}[p26]
\pstLineAB[linestyle=dotted,linecolor=gray]{n26}{p26}
\pstGeonode[PointSymbol=Btriangle,fillcolor=lightgray](6.20,2.80){n27}
\pstProjection[PointSymbol=triangle,dotscale=1.25]{na}{nb}{n27}[p27]
\pstLineAB[linestyle=dotted,linecolor=gray]{n27}{p27}
\pstGeonode[PointSymbol=Btriangle,fillcolor=lightgray](6.20,3.40){n28}
\pstProjection[PointSymbol=triangle,dotscale=1.25]{na}{nb}{n28}[p28]
\pstLineAB[linestyle=dotted,linecolor=gray]{n28}{p28}
\pstGeonode[PointSymbol=Btriangle,fillcolor=lightgray](6.30,2.90){n29}
\pstProjection[PointSymbol=triangle,dotscale=1.25]{na}{nb}{n29}[p29]
\pstLineAB[linestyle=dotted,linecolor=gray]{n29}{p29}
\pstGeonode[PointSymbol=Btriangle,fillcolor=lightgray](6.70,3.30){n30}
\pstProjection[PointSymbol=triangle,dotscale=1.25]{na}{nb}{n30}[p30]
\pstLineAB[linestyle=dotted,linecolor=gray]{n30}{p30}
\pstGeonode[PointSymbol=Btriangle,fillcolor=lightgray](6.30,2.70){n31}
\pstProjection[PointSymbol=triangle,dotscale=1.25]{na}{nb}{n31}[p31]
\pstLineAB[linestyle=dotted,linecolor=gray]{n31}{p31}
\pstGeonode[PointSymbol=Btriangle,fillcolor=lightgray](6.40,3.20){n32}
\pstProjection[PointSymbol=triangle,dotscale=1.25]{na}{nb}{n32}[p32]
\pstLineAB[linestyle=dotted,linecolor=gray]{n32}{p32}
\pstGeonode[PointSymbol=Btriangle,fillcolor=lightgray](6.00,2.20){n33}
\pstProjection[PointSymbol=triangle,dotscale=1.25]{na}{nb}{n33}[p33]
\pstLineAB[linestyle=dotted,linecolor=gray]{n33}{p33}
\pstGeonode[PointSymbol=Btriangle,fillcolor=lightgray](7.40,2.80){n34}
\pstProjection[PointSymbol=triangle,dotscale=1.25]{na}{nb}{n34}[p34]
\pstLineAB[linestyle=dotted,linecolor=gray]{n34}{p34}
\pstGeonode[PointSymbol=Btriangle,fillcolor=lightgray](6.50,3.20){n35}
\pstProjection[PointSymbol=triangle,dotscale=1.25]{na}{nb}{n35}[p35]
\pstLineAB[linestyle=dotted,linecolor=gray]{n35}{p35}
\pstGeonode[PointSymbol=Btriangle,fillcolor=lightgray](6.70,3.30){n36}
\pstProjection[PointSymbol=triangle,dotscale=1.25]{na}{nb}{n36}[p36]
\pstLineAB[linestyle=dotted,linecolor=gray]{n36}{p36}
\pstGeonode[PointSymbol=Btriangle,fillcolor=lightgray](6.90,3.20){n37}
\pstProjection[PointSymbol=triangle,dotscale=1.25]{na}{nb}{n37}[p37]
\pstLineAB[linestyle=dotted,linecolor=gray]{n37}{p37}
\pstGeonode[PointSymbol=Btriangle,fillcolor=lightgray](7.60,3.00){n38}
\pstProjection[PointSymbol=triangle,dotscale=1.25]{na}{nb}{n38}[p38]
\pstLineAB[linestyle=dotted,linecolor=gray]{n38}{p38}
\pstGeonode[PointSymbol=Btriangle,fillcolor=lightgray](6.30,2.80){n39}
\pstProjection[PointSymbol=triangle,dotscale=1.25]{na}{nb}{n39}[p39]
\pstLineAB[linestyle=dotted,linecolor=gray]{n39}{p39}
\pstGeonode[PointSymbol=Btriangle,fillcolor=lightgray](6.40,3.10){n40}
\pstProjection[PointSymbol=triangle,dotscale=1.25]{na}{nb}{n40}[p40]
\pstLineAB[linestyle=dotted,linecolor=gray]{n40}{p40}
\pstGeonode[PointSymbol=Btriangle,fillcolor=lightgray](5.80,2.70){n41}
\pstProjection[PointSymbol=triangle,dotscale=1.25]{na}{nb}{n41}[p41]
\pstLineAB[linestyle=dotted,linecolor=gray]{n41}{p41}
\pstGeonode[PointSymbol=Btriangle,fillcolor=lightgray](6.40,2.80){n42}
\pstProjection[PointSymbol=triangle,dotscale=1.25]{na}{nb}{n42}[p42]
\pstLineAB[linestyle=dotted,linecolor=gray]{n42}{p42}
\pstGeonode[PointSymbol=Btriangle,fillcolor=lightgray](6.40,2.80){n43}
\pstProjection[PointSymbol=triangle,dotscale=1.25]{na}{nb}{n43}[p43]
\pstLineAB[linestyle=dotted,linecolor=gray]{n43}{p43}
\pstGeonode[PointSymbol=Btriangle,fillcolor=lightgray](7.70,2.80){n44}
\pstProjection[PointSymbol=triangle,dotscale=1.25]{na}{nb}{n44}[p44]
\pstLineAB[linestyle=dotted,linecolor=gray]{n44}{p44}
\pstGeonode[PointSymbol=Btriangle,fillcolor=lightgray](6.50,3.00){n45}
\pstProjection[PointSymbol=triangle,dotscale=1.25]{na}{nb}{n45}[p45]
\pstLineAB[linestyle=dotted,linecolor=gray]{n45}{p45}
\pstGeonode[PointSymbol=Btriangle,fillcolor=lightgray](7.20,3.60){n46}
\pstProjection[PointSymbol=triangle,dotscale=1.25]{na}{nb}{n46}[p46]
\pstLineAB[linestyle=dotted,linecolor=gray]{n46}{p46}
\pstGeonode[PointSymbol=Btriangle,fillcolor=lightgray](6.90,3.10){n47}
\pstProjection[PointSymbol=triangle,dotscale=1.25]{na}{nb}{n47}[p47]
\pstLineAB[linestyle=dotted,linecolor=gray]{n47}{p47}
\pstGeonode[PointSymbol=Btriangle,fillcolor=lightgray](6.00,3.00){n48}
\pstProjection[PointSymbol=triangle,dotscale=1.25]{na}{nb}{n48}[p48]
\pstLineAB[linestyle=dotted,linecolor=gray]{n48}{p48}
\pstGeonode[PointSymbol=Btriangle,fillcolor=lightgray](6.70,3.00){n49}
\pstProjection[PointSymbol=triangle,dotscale=1.25]{na}{nb}{n49}[p49]
\pstLineAB[linestyle=dotted,linecolor=gray]{n49}{p49}
\pstGeonode[PointSymbol=Btriangle,fillcolor=lightgray](7.70,3.80){n50}
\pstProjection[PointSymbol=triangle,dotscale=1.25]{na}{nb}{n50}[p50]
\pstLineAB[linestyle=dotted,linecolor=gray]{n50}{p50}
\pstGeonode[PointSymbol=Bo,fillcolor=lightgray](4.60,3.20){n51}
\pstProjection[PointSymbol=o,dotscale=1.25]{na}{nb}{n51}[p51]
\pstLineAB[linestyle=dotted,linecolor=gray]{n51}{p51}
\pstGeonode[PointSymbol=Bo,fillcolor=lightgray](4.70,3.20){n52}
\pstProjection[PointSymbol=o,dotscale=1.25]{na}{nb}{n52}[p52]
\pstLineAB[linestyle=dotted,linecolor=gray]{n52}{p52}
\pstGeonode[PointSymbol=Bo,fillcolor=lightgray](5.10,3.70){n53}
\pstProjection[PointSymbol=o,dotscale=1.25]{na}{nb}{n53}[p53]
\pstLineAB[linestyle=dotted,linecolor=gray]{n53}{p53}
\pstGeonode[PointSymbol=Bo,fillcolor=lightgray](5.10,3.80){n54}
\pstProjection[PointSymbol=o,dotscale=1.25]{na}{nb}{n54}[p54]
\pstLineAB[linestyle=dotted,linecolor=gray]{n54}{p54}
\pstGeonode[PointSymbol=Bo,fillcolor=lightgray](4.90,3.10){n55}
\pstProjection[PointSymbol=o,dotscale=1.25]{na}{nb}{n55}[p55]
\pstLineAB[linestyle=dotted,linecolor=gray]{n55}{p55}
\pstGeonode[PointSymbol=Bo,fillcolor=lightgray](5.00,3.40){n56}
\pstProjection[PointSymbol=o,dotscale=1.25]{na}{nb}{n56}[p56]
\pstLineAB[linestyle=dotted,linecolor=gray]{n56}{p56}
\pstGeonode[PointSymbol=Bo,fillcolor=lightgray](5.00,3.40){n57}
\pstProjection[PointSymbol=o,dotscale=1.25]{na}{nb}{n57}[p57]
\pstLineAB[linestyle=dotted,linecolor=gray]{n57}{p57}
\pstGeonode[PointSymbol=Bo,fillcolor=lightgray](5.00,3.30){n58}
\pstProjection[PointSymbol=o,dotscale=1.25]{na}{nb}{n58}[p58]
\pstLineAB[linestyle=dotted,linecolor=gray]{n58}{p58}
\pstGeonode[PointSymbol=Bo,fillcolor=lightgray](5.70,4.40){n59}
\pstProjection[PointSymbol=o,dotscale=1.25]{na}{nb}{n59}[p59]
\pstLineAB[linestyle=dotted,linecolor=gray]{n59}{p59}
\pstGeonode[PointSymbol=Bo,fillcolor=lightgray](5.00,3.50){n60}
\pstProjection[PointSymbol=o,dotscale=1.25]{na}{nb}{n60}[p60]
\pstLineAB[linestyle=dotted,linecolor=gray]{n60}{p60}
\pstGeonode[PointSymbol=Bo,fillcolor=lightgray](4.60,3.10){n61}
\pstProjection[PointSymbol=o,dotscale=1.25]{na}{nb}{n61}[p61]
\pstLineAB[linestyle=dotted,linecolor=gray]{n61}{p61}
\pstGeonode[PointSymbol=Bo,fillcolor=lightgray](5.40,3.90){n62}
\pstProjection[PointSymbol=o,dotscale=1.25]{na}{nb}{n62}[p62]
\pstLineAB[linestyle=dotted,linecolor=gray]{n62}{p62}
\pstGeonode[PointSymbol=Bo,fillcolor=lightgray](5.00,3.20){n63}
\pstProjection[PointSymbol=o,dotscale=1.25]{na}{nb}{n63}[p63]
\pstLineAB[linestyle=dotted,linecolor=gray]{n63}{p63}
\pstGeonode[PointSymbol=Bo,fillcolor=lightgray](5.10,3.80){n64}
\pstProjection[PointSymbol=o,dotscale=1.25]{na}{nb}{n64}[p64]
\pstLineAB[linestyle=dotted,linecolor=gray]{n64}{p64}
\pstGeonode[PointSymbol=Bo,fillcolor=lightgray](4.80,3.00){n65}
\pstProjection[PointSymbol=o,dotscale=1.25]{na}{nb}{n65}[p65]
\pstLineAB[linestyle=dotted,linecolor=gray]{n65}{p65}
\pstGeonode[PointSymbol=Bo,fillcolor=lightgray](5.30,3.70){n66}
\pstProjection[PointSymbol=o,dotscale=1.25]{na}{nb}{n66}[p66]
\pstLineAB[linestyle=dotted,linecolor=gray]{n66}{p66}
\pstGeonode[PointSymbol=Bo,fillcolor=lightgray](5.70,3.80){n67}
\pstProjection[PointSymbol=o,dotscale=1.25]{na}{nb}{n67}[p67]
\pstLineAB[linestyle=dotted,linecolor=gray]{n67}{p67}
\pstGeonode[PointSymbol=Bo,fillcolor=lightgray](4.40,3.00){n68}
\pstProjection[PointSymbol=o,dotscale=1.25]{na}{nb}{n68}[p68]
\pstLineAB[linestyle=dotted,linecolor=gray]{n68}{p68}
\pstGeonode[PointSymbol=Bo,fillcolor=lightgray](5.40,3.40){n69}
\pstProjection[PointSymbol=o,dotscale=1.25]{na}{nb}{n69}[p69]
\pstLineAB[linestyle=dotted,linecolor=gray]{n69}{p69}
\pstGeonode[PointSymbol=Bo,fillcolor=lightgray](5.00,3.50){n70}
\pstProjection[PointSymbol=o,dotscale=1.25]{na}{nb}{n70}[p70]
\pstLineAB[linestyle=dotted,linecolor=gray]{n70}{p70}
\pstGeonode[PointSymbol=Bo,fillcolor=lightgray](5.10,3.30){n71}
\pstProjection[PointSymbol=o,dotscale=1.25]{na}{nb}{n71}[p71]
\pstLineAB[linestyle=dotted,linecolor=gray]{n71}{p71}
\pstGeonode[PointSymbol=Bo,fillcolor=lightgray](4.90,3.10){n72}
\pstProjection[PointSymbol=o,dotscale=1.25]{na}{nb}{n72}[p72]
\pstLineAB[linestyle=dotted,linecolor=gray]{n72}{p72}
\pstGeonode[PointSymbol=Bo,fillcolor=lightgray](4.60,3.60){n73}
\pstProjection[PointSymbol=o,dotscale=1.25]{na}{nb}{n73}[p73]
\pstLineAB[linestyle=dotted,linecolor=gray]{n73}{p73}
\pstGeonode[PointSymbol=Bo,fillcolor=lightgray](5.20,3.40){n74}
\pstProjection[PointSymbol=o,dotscale=1.25]{na}{nb}{n74}[p74]
\pstLineAB[linestyle=dotted,linecolor=gray]{n74}{p74}
\pstGeonode[PointSymbol=Bo,fillcolor=lightgray](5.50,3.50){n75}
\pstProjection[PointSymbol=o,dotscale=1.25]{na}{nb}{n75}[p75]
\pstLineAB[linestyle=dotted,linecolor=gray]{n75}{p75}
\pstGeonode[PointSymbol=Bo,fillcolor=lightgray](4.60,3.40){n76}
\pstProjection[PointSymbol=o,dotscale=1.25]{na}{nb}{n76}[p76]
\pstLineAB[linestyle=dotted,linecolor=gray]{n76}{p76}
\pstGeonode[PointSymbol=Bo,fillcolor=lightgray](4.70,3.20){n77}
\pstProjection[PointSymbol=o,dotscale=1.25]{na}{nb}{n77}[p77]
\pstLineAB[linestyle=dotted,linecolor=gray]{n77}{p77}
\pstGeonode[PointSymbol=Bo,fillcolor=lightgray](4.40,2.90){n78}
\pstProjection[PointSymbol=o,dotscale=1.25]{na}{nb}{n78}[p78]
\pstLineAB[linestyle=dotted,linecolor=gray]{n78}{p78}
\pstGeonode[PointSymbol=Bo,fillcolor=lightgray](4.80,3.00){n79}
\pstProjection[PointSymbol=o,dotscale=1.25]{na}{nb}{n79}[p79]
\pstLineAB[linestyle=dotted,linecolor=gray]{n79}{p79}
\pstGeonode[PointSymbol=Bo,fillcolor=lightgray](5.40,3.90){n80}
\pstProjection[PointSymbol=o,dotscale=1.25]{na}{nb}{n80}[p80]
\pstLineAB[linestyle=dotted,linecolor=gray]{n80}{p80}
\pstGeonode[PointSymbol=Bo,fillcolor=lightgray](4.80,3.40){n81}
\pstProjection[PointSymbol=o,dotscale=1.25]{na}{nb}{n81}[p81]
\pstLineAB[linestyle=dotted,linecolor=gray]{n81}{p81}
\pstGeonode[PointSymbol=Bo,fillcolor=lightgray](4.40,3.20){n82}
\pstProjection[PointSymbol=o,dotscale=1.25]{na}{nb}{n82}[p82]
\pstLineAB[linestyle=dotted,linecolor=gray]{n82}{p82}
\pstGeonode[PointSymbol=Bo,fillcolor=lightgray](4.80,3.40){n83}
\pstProjection[PointSymbol=o,dotscale=1.25]{na}{nb}{n83}[p83]
\pstLineAB[linestyle=dotted,linecolor=gray]{n83}{p83}
\pstGeonode[PointSymbol=Bo,fillcolor=lightgray](4.90,3.00){n84}
\pstProjection[PointSymbol=o,dotscale=1.25]{na}{nb}{n84}[p84]
\pstLineAB[linestyle=dotted,linecolor=gray]{n84}{p84}
\pstGeonode[PointSymbol=Bo,fillcolor=lightgray](4.50,2.30){n85}
\pstProjection[PointSymbol=o,dotscale=1.25]{na}{nb}{n85}[p85]
\pstLineAB[linestyle=dotted,linecolor=gray]{n85}{p85}
\pstGeonode[PointSymbol=Bo,fillcolor=lightgray](4.30,3.00){n86}
\pstProjection[PointSymbol=o,dotscale=1.25]{na}{nb}{n86}[p86]
\pstLineAB[linestyle=dotted,linecolor=gray]{n86}{p86}
\pstGeonode[PointSymbol=Bo,fillcolor=lightgray](5.00,3.60){n87}
\pstProjection[PointSymbol=o,dotscale=1.25]{na}{nb}{n87}[p87]
\pstLineAB[linestyle=dotted,linecolor=gray]{n87}{p87}
\pstGeonode[PointSymbol=Bo,fillcolor=lightgray](5.20,3.50){n88}
\pstProjection[PointSymbol=o,dotscale=1.25]{na}{nb}{n88}[p88]
\pstLineAB[linestyle=dotted,linecolor=gray]{n88}{p88}
\pstGeonode[PointSymbol=Bo,fillcolor=lightgray](5.50,4.20){n89}
\pstProjection[PointSymbol=o,dotscale=1.25]{na}{nb}{n89}[p89]
\pstLineAB[linestyle=dotted,linecolor=gray]{n89}{p89}
\pstGeonode[PointSymbol=Bo,fillcolor=lightgray](5.00,3.00){n90}
\pstProjection[PointSymbol=o,dotscale=1.25]{na}{nb}{n90}[p90]
\pstLineAB[linestyle=dotted,linecolor=gray]{n90}{p90}
\pstGeonode[PointSymbol=Bo,fillcolor=lightgray](5.10,3.50){n91}
\pstProjection[PointSymbol=o,dotscale=1.25]{na}{nb}{n91}[p91]
\pstLineAB[linestyle=dotted,linecolor=gray]{n91}{p91}
\pstGeonode[PointSymbol=Bo,fillcolor=lightgray](4.80,3.10){n92}
\pstProjection[PointSymbol=o,dotscale=1.25]{na}{nb}{n92}[p92]
\pstLineAB[linestyle=dotted,linecolor=gray]{n92}{p92}
\pstGeonode[PointSymbol=Bo,fillcolor=lightgray](5.10,3.80){n93}
\pstProjection[PointSymbol=o,dotscale=1.25]{na}{nb}{n93}[p93]
\pstLineAB[linestyle=dotted,linecolor=gray]{n93}{p93}
\pstGeonode[PointSymbol=Bo,fillcolor=lightgray](5.20,4.10){n94}
\pstProjection[PointSymbol=o,dotscale=1.25]{na}{nb}{n94}[p94]
\pstLineAB[linestyle=dotted,linecolor=gray]{n94}{p94}
\pstGeonode[PointSymbol=Bo,fillcolor=lightgray](5.80,4.00){n95}
\pstProjection[PointSymbol=o,dotscale=1.25]{na}{nb}{n95}[p95]
\pstLineAB[linestyle=dotted,linecolor=gray]{n95}{p95}
\pstGeonode[PointSymbol=Bo,fillcolor=lightgray](4.90,3.10){n96}
\pstProjection[PointSymbol=o,dotscale=1.25]{na}{nb}{n96}[p96]
\pstLineAB[linestyle=dotted,linecolor=gray]{n96}{p96}
\pstGeonode[PointSymbol=Bo,fillcolor=lightgray](5.40,3.70){n97}
\pstProjection[PointSymbol=o,dotscale=1.25]{na}{nb}{n97}[p97]
\pstLineAB[linestyle=dotted,linecolor=gray]{n97}{p97}
\pstGeonode[PointSymbol=Bo,fillcolor=lightgray](5.10,3.50){n98}
\pstProjection[PointSymbol=o,dotscale=1.25]{na}{nb}{n98}[p98]
\pstLineAB[linestyle=dotted,linecolor=gray]{n98}{p98}
\pstGeonode[PointSymbol=Bo,fillcolor=lightgray](5.40,3.40){n99}
\pstProjection[PointSymbol=o,dotscale=1.25]{na}{nb}{n99}[p99]
\pstLineAB[linestyle=dotted,linecolor=gray]{n99}{p99}
\pstGeonode[PointSymbol=Bo,fillcolor=lightgray](5.10,3.40){n100}
\pstProjection[PointSymbol=o,dotscale=1.25]{na}{nb}{n100}[p100]
\pstLineAB[linestyle=dotted,linecolor=gray]{n100}{p100}
\pstGeonode[PointSymbol=Btriangle,fillcolor=lightgray](5.90,3.00){n101}
\pstProjection[PointSymbol=triangle,dotscale=1.25]{na}{nb}{n101}[p101]
\pstLineAB[linestyle=dotted,linecolor=gray]{n101}{p101}
\pstGeonode[PointSymbol=Btriangle,fillcolor=lightgray](6.90,3.10){n102}
\pstProjection[PointSymbol=triangle,dotscale=1.25]{na}{nb}{n102}[p102]
\pstLineAB[linestyle=dotted,linecolor=gray]{n102}{p102}
\pstGeonode[PointSymbol=Btriangle,fillcolor=lightgray](6.60,2.90){n103}
\pstProjection[PointSymbol=triangle,dotscale=1.25]{na}{nb}{n103}[p103]
\pstLineAB[linestyle=dotted,linecolor=gray]{n103}{p103}
\pstGeonode[PointSymbol=Btriangle,fillcolor=lightgray](6.00,2.20){n104}
\pstProjection[PointSymbol=triangle,dotscale=1.25]{na}{nb}{n104}[p104]
\pstLineAB[linestyle=dotted,linecolor=gray]{n104}{p104}
\pstGeonode[PointSymbol=Btriangle,fillcolor=lightgray](5.70,3.00){n105}
\pstProjection[PointSymbol=triangle,dotscale=1.25]{na}{nb}{n105}[p105]
\pstLineAB[linestyle=dotted,linecolor=gray]{n105}{p105}
\pstGeonode[PointSymbol=Btriangle,fillcolor=lightgray](5.70,2.80){n106}
\pstProjection[PointSymbol=triangle,dotscale=1.25]{na}{nb}{n106}[p106]
\pstLineAB[linestyle=dotted,linecolor=gray]{n106}{p106}
\pstGeonode[PointSymbol=Btriangle,fillcolor=lightgray](5.50,2.50){n107}
\pstProjection[PointSymbol=triangle,dotscale=1.25]{na}{nb}{n107}[p107]
\pstLineAB[linestyle=dotted,linecolor=gray]{n107}{p107}
\pstGeonode[PointSymbol=Btriangle,fillcolor=lightgray](5.50,2.30){n108}
\pstProjection[PointSymbol=triangle,dotscale=1.25]{na}{nb}{n108}[p108]
\pstLineAB[linestyle=dotted,linecolor=gray]{n108}{p108}
\pstGeonode[PointSymbol=Btriangle,fillcolor=lightgray](5.80,2.60){n109}
\pstProjection[PointSymbol=triangle,dotscale=1.25]{na}{nb}{n109}[p109]
\pstLineAB[linestyle=dotted,linecolor=gray]{n109}{p109}
\pstGeonode[PointSymbol=Btriangle,fillcolor=lightgray](5.10,2.50){n110}
\pstProjection[PointSymbol=triangle,dotscale=1.25]{na}{nb}{n110}[p110]
\pstLineAB[linestyle=dotted,linecolor=gray]{n110}{p110}
\pstGeonode[PointSymbol=Btriangle,fillcolor=lightgray](5.60,2.50){n111}
\pstProjection[PointSymbol=triangle,dotscale=1.25]{na}{nb}{n111}[p111]
\pstLineAB[linestyle=dotted,linecolor=gray]{n111}{p111}
\pstGeonode[PointSymbol=Btriangle,fillcolor=lightgray](5.80,2.70){n112}
\pstProjection[PointSymbol=triangle,dotscale=1.25]{na}{nb}{n112}[p112]
\pstLineAB[linestyle=dotted,linecolor=gray]{n112}{p112}
\pstGeonode[PointSymbol=Btriangle,fillcolor=lightgray](6.30,2.30){n113}
\pstProjection[PointSymbol=triangle,dotscale=1.25]{na}{nb}{n113}[p113]
\pstLineAB[linestyle=dotted,linecolor=gray]{n113}{p113}
\pstGeonode[PointSymbol=Btriangle,fillcolor=lightgray](5.60,3.00){n114}
\pstProjection[PointSymbol=triangle,dotscale=1.25]{na}{nb}{n114}[p114]
\pstLineAB[linestyle=dotted,linecolor=gray]{n114}{p114}
\pstGeonode[PointSymbol=Btriangle,fillcolor=lightgray](6.10,3.00){n115}
\pstProjection[PointSymbol=triangle,dotscale=1.25]{na}{nb}{n115}[p115]
\pstLineAB[linestyle=dotted,linecolor=gray]{n115}{p115}
\pstGeonode[PointSymbol=Btriangle,fillcolor=lightgray](5.60,2.70){n116}
\pstProjection[PointSymbol=triangle,dotscale=1.25]{na}{nb}{n116}[p116]
\pstLineAB[linestyle=dotted,linecolor=gray]{n116}{p116}
\pstGeonode[PointSymbol=Btriangle,fillcolor=lightgray](5.70,2.60){n117}
\pstProjection[PointSymbol=triangle,dotscale=1.25]{na}{nb}{n117}[p117]
\pstLineAB[linestyle=dotted,linecolor=gray]{n117}{p117}
\pstGeonode[PointSymbol=Btriangle,fillcolor=lightgray](5.70,2.90){n118}
\pstProjection[PointSymbol=triangle,dotscale=1.25]{na}{nb}{n118}[p118]
\pstLineAB[linestyle=dotted,linecolor=gray]{n118}{p118}
\pstGeonode[PointSymbol=Btriangle,fillcolor=lightgray](6.10,2.80){n119}
\pstProjection[PointSymbol=triangle,dotscale=1.25]{na}{nb}{n119}[p119]
\pstLineAB[linestyle=dotted,linecolor=gray]{n119}{p119}
\pstGeonode[PointSymbol=Btriangle,fillcolor=lightgray](5.60,2.90){n120}
\pstProjection[PointSymbol=triangle,dotscale=1.25]{na}{nb}{n120}[p120]
\pstLineAB[linestyle=dotted,linecolor=gray]{n120}{p120}
\pstGeonode[PointSymbol=Btriangle,fillcolor=lightgray](5.60,3.00){n121}
\pstProjection[PointSymbol=triangle,dotscale=1.25]{na}{nb}{n121}[p121]
\pstLineAB[linestyle=dotted,linecolor=gray]{n121}{p121}
\pstGeonode[PointSymbol=Btriangle,fillcolor=lightgray](5.40,3.00){n122}
\pstProjection[PointSymbol=triangle,dotscale=1.25]{na}{nb}{n122}[p122]
\pstLineAB[linestyle=dotted,linecolor=gray]{n122}{p122}
\pstGeonode[PointSymbol=Btriangle,fillcolor=lightgray](6.10,2.80){n123}
\pstProjection[PointSymbol=triangle,dotscale=1.25]{na}{nb}{n123}[p123]
\pstLineAB[linestyle=dotted,linecolor=gray]{n123}{p123}
\pstGeonode[PointSymbol=Btriangle,fillcolor=lightgray](6.00,2.90){n124}
\pstProjection[PointSymbol=triangle,dotscale=1.25]{na}{nb}{n124}[p124]
\pstLineAB[linestyle=dotted,linecolor=gray]{n124}{p124}
\pstGeonode[PointSymbol=Btriangle,fillcolor=lightgray](5.00,2.30){n125}
\pstProjection[PointSymbol=triangle,dotscale=1.25]{na}{nb}{n125}[p125]
\pstLineAB[linestyle=dotted,linecolor=gray]{n125}{p125}
\pstGeonode[PointSymbol=Btriangle,fillcolor=lightgray](6.40,3.20){n126}
\pstProjection[PointSymbol=triangle,dotscale=1.25]{na}{nb}{n126}[p126]
\pstLineAB[linestyle=dotted,linecolor=gray]{n126}{p126}
\pstGeonode[PointSymbol=Btriangle,fillcolor=lightgray](6.70,3.00){n127}
\pstProjection[PointSymbol=triangle,dotscale=1.25]{na}{nb}{n127}[p127]
\pstLineAB[linestyle=dotted,linecolor=gray]{n127}{p127}
\pstGeonode[PointSymbol=Btriangle,fillcolor=lightgray](5.00,2.00){n128}
\pstProjection[PointSymbol=triangle,dotscale=1.25]{na}{nb}{n128}[p128]
\pstLineAB[linestyle=dotted,linecolor=gray]{n128}{p128}
\pstGeonode[PointSymbol=Btriangle,fillcolor=lightgray](5.90,3.20){n129}
\pstProjection[PointSymbol=triangle,dotscale=1.25]{na}{nb}{n129}[p129]
\pstLineAB[linestyle=dotted,linecolor=gray]{n129}{p129}
\pstGeonode[PointSymbol=Btriangle,fillcolor=lightgray](6.20,2.20){n130}
\pstProjection[PointSymbol=triangle,dotscale=1.25]{na}{nb}{n130}[p130]
\pstLineAB[linestyle=dotted,linecolor=gray]{n130}{p130}
\pstGeonode[PointSymbol=Btriangle,fillcolor=lightgray](4.90,2.40){n131}
\pstProjection[PointSymbol=triangle,dotscale=1.25]{na}{nb}{n131}[p131]
\pstLineAB[linestyle=dotted,linecolor=gray]{n131}{p131}
\pstGeonode[PointSymbol=Btriangle,fillcolor=lightgray](7.00,3.20){n132}
\pstProjection[PointSymbol=triangle,dotscale=1.25]{na}{nb}{n132}[p132]
\pstLineAB[linestyle=dotted,linecolor=gray]{n132}{p132}
\pstGeonode[PointSymbol=Btriangle,fillcolor=lightgray](5.50,2.40){n133}
\pstProjection[PointSymbol=triangle,dotscale=1.25]{na}{nb}{n133}[p133]
\pstLineAB[linestyle=dotted,linecolor=gray]{n133}{p133}
\pstGeonode[PointSymbol=Btriangle,fillcolor=lightgray](6.30,3.30){n134}
\pstProjection[PointSymbol=triangle,dotscale=1.25]{na}{nb}{n134}[p134]
\pstLineAB[linestyle=dotted,linecolor=gray]{n134}{p134}
\pstGeonode[PointSymbol=Btriangle,fillcolor=lightgray](6.80,2.80){n135}
\pstProjection[PointSymbol=triangle,dotscale=1.25]{na}{nb}{n135}[p135]
\pstLineAB[linestyle=dotted,linecolor=gray]{n135}{p135}
\pstGeonode[PointSymbol=Btriangle,fillcolor=lightgray](6.10,2.90){n136}
\pstProjection[PointSymbol=triangle,dotscale=1.25]{na}{nb}{n136}[p136]
\pstLineAB[linestyle=dotted,linecolor=gray]{n136}{p136}
\pstGeonode[PointSymbol=Btriangle,fillcolor=lightgray](6.70,3.10){n137}
\pstProjection[PointSymbol=triangle,dotscale=1.25]{na}{nb}{n137}[p137]
\pstLineAB[linestyle=dotted,linecolor=gray]{n137}{p137}
\pstGeonode[PointSymbol=Btriangle,fillcolor=lightgray](5.20,2.70){n138}
\pstProjection[PointSymbol=triangle,dotscale=1.25]{na}{nb}{n138}[p138]
\pstLineAB[linestyle=dotted,linecolor=gray]{n138}{p138}
\pstGeonode[PointSymbol=Btriangle,fillcolor=lightgray](6.40,2.90){n139}
\pstProjection[PointSymbol=triangle,dotscale=1.25]{na}{nb}{n139}[p139]
\pstLineAB[linestyle=dotted,linecolor=gray]{n139}{p139}
\pstGeonode[PointSymbol=Btriangle,fillcolor=lightgray](5.80,2.70){n140}
\pstProjection[PointSymbol=triangle,dotscale=1.25]{na}{nb}{n140}[p140]
\pstLineAB[linestyle=dotted,linecolor=gray]{n140}{p140}
\pstGeonode[PointSymbol=Btriangle,fillcolor=lightgray](6.30,2.50){n141}
\pstProjection[PointSymbol=triangle,dotscale=1.25]{na}{nb}{n141}[p141]
\pstLineAB[linestyle=dotted,linecolor=gray]{n141}{p141}
\pstGeonode[PointSymbol=Btriangle,fillcolor=lightgray](6.70,3.10){n142}
\pstProjection[PointSymbol=triangle,dotscale=1.25]{na}{nb}{n142}[p142]
\pstLineAB[linestyle=dotted,linecolor=gray]{n142}{p142}
\pstGeonode[PointSymbol=Btriangle,fillcolor=lightgray](6.00,2.70){n143}
\pstProjection[PointSymbol=triangle,dotscale=1.25]{na}{nb}{n143}[p143]
\pstLineAB[linestyle=dotted,linecolor=gray]{n143}{p143}
\pstGeonode[PointSymbol=Btriangle,fillcolor=lightgray](5.70,2.80){n144}
\pstProjection[PointSymbol=triangle,dotscale=1.25]{na}{nb}{n144}[p144]
\pstLineAB[linestyle=dotted,linecolor=gray]{n144}{p144}
\pstGeonode[PointSymbol=Btriangle,fillcolor=lightgray](6.60,3.00){n145}
\pstProjection[PointSymbol=triangle,dotscale=1.25]{na}{nb}{n145}[p145]
\pstLineAB[linestyle=dotted,linecolor=gray]{n145}{p145}
\pstGeonode[PointSymbol=Btriangle,fillcolor=lightgray](6.00,3.40){n146}
\pstProjection[PointSymbol=triangle,dotscale=1.25]{na}{nb}{n146}[p146]
\pstLineAB[linestyle=dotted,linecolor=gray]{n146}{p146}
\pstGeonode[PointSymbol=Btriangle,fillcolor=lightgray](5.50,2.40){n147}
\pstProjection[PointSymbol=triangle,dotscale=1.25]{na}{nb}{n147}[p147]
\pstLineAB[linestyle=dotted,linecolor=gray]{n147}{p147}
\pstGeonode[PointSymbol=Btriangle,fillcolor=lightgray](6.20,2.90){n148}
\pstProjection[PointSymbol=triangle,dotscale=1.25]{na}{nb}{n148}[p148]
\pstLineAB[linestyle=dotted,linecolor=gray]{n148}{p148}
\pstGeonode[PointSymbol=Btriangle,fillcolor=lightgray](6.50,2.80){n149}
\pstProjection[PointSymbol=triangle,dotscale=1.25]{na}{nb}{n149}[p149]
\pstLineAB[linestyle=dotted,linecolor=gray]{n149}{p149}
\pstGeonode[PointSymbol=Btriangle,fillcolor=lightgray](5.50,2.60){n150}
\pstProjection[PointSymbol=triangle,dotscale=1.25]{na}{nb}{n150}[p150]
\pstLineAB[linestyle=dotted,linecolor=gray]{n150}{p150}

    \psset{dotscale=2,fillcolor=black}
    \pstGeonode[PointSymbol=none,
        fillcolor=lightgray](5.01,3.42){mu1}
    \pstProjection[PointSymbol=Bo]{na}{nb}{mu1}[m1]
    \pstGeonode[PointSymbol=none,
        fillcolor=lightgray](6.26,2.87){mu2}
    \pstProjection[PointSymbol=Btriangle]{na}{nb}{mu2}[m2]
    \end{pspicture}
    }
\end{figure}

\column{0.5\textwidth} 
\scriptsize
The between-class scatter matrix is
\begin{align*}
    \bB & = \amatr{r}{1.587 & -0.693\\-0.693 & 0.303}
\end{align*}
and the within-class scatter matrix is
\begin{align*}
    \bS_1 & = \matr{6.09 & 4.91\\4.91 & 7.11}\\
    \bS_2 & = \matr{43.5 & 12.09\\12.09 & 10.96}\\
    \bS & = \matr{49.58 & 17.01\\17.01 & 18.08}
\end{align*}
The direction of most separation between $c_1$ and $c_2$ is the
dominant eigenvector corresponding to the largest eigenvalue of
the matrix $\bS^{-1} \bB$. The solution is
\begin{align*}
    J(\bw) & = \lambda_1 = 0.11\\
    \bw & = \amatr{r}{0.551\\-0.834}
\end{align*}
\end{columns}
\end{frame}


\begin{frame}{Linear Discriminant Analysis: Two Classes}
For the two class scenario, if $\bS$ is nonsingular, we can
directly solve for $\bw$ without computing the eigenvalues and
eigenvectors. 

\medskip
The between-class scatter matrix $\bB$
points in the same direction as $(\bmu_1-\bmu_2)$ because
\begin{align*}
  \bB\bw = & \Bigl( (\bmu_1 - \bmu_2) (\bmu_1 - \bmu_2)^T\Bigr) \bw\notag\\
  = & (\bmu_1 - \bmu_2) \Bigl((\bmu_1 - \bmu_2)^T \bw\Bigr)\notag\\
  = & b (\bmu_1 - \bmu_2)
\end{align*}

\medskip
The generalized eigenvectors equation can then be 
rewritten as 
\begin{align*}
  \bw = & \frac{b}{\lambda} \bS^{-1} (\bmu_1 - \bmu_2)
\end{align*}
Because ${b \over \lambda}$ is just a scalar, we can solve for the
best linear discriminant as
\begin{empheq}[box=\tcbhighmath]{align*}
\begin{split}
  \bw = & \bS^{-1} (\bmu_1 - \bmu_2)
\end{split}
\end{empheq}
We can finally normalize $\bw$ to be a unit vector.
\end{frame}



\begin{frame}{Kernel Discriminant Analysis} 
The goal of kernel LDA is to f\/{i}nd the direction vector $\bw$ in
feature space that maximizes
\begin{align*}
        \max_{\bw} \;\; J(\bw) = \frac{(m_1 - m_2)^2}{s_1^2 + s_2^2}
\end{align*}
It is well known that 
$\bw$ can be expressed as a linear combination of the
points in feature space
\begin{align*}
  \bw 
  & = \sum_{j=1}^n a_{j} \phi(\bx_{j})
\end{align*}
The mean for class $c_i$ in feature space is given as
\begin{align*}
\bmu^\phi_i = {1 \over n_i} \sum_{\bx_{j} \in \bD_i} \phi(\bx_{j})
\end{align*}
and the covariance matrix for class $c_i$ in feature space is
\begin{align*}
  \cov_i^\phi = {1 \over n_i} \sum_{\bx_{j} \in \bD_i}
  \Bigl(\phi(\bx_{j}) - \bmu_i^\phi\Bigr)
  \Bigl(\phi(\bx_{j})- \bmu_i^\phi\Bigr)^T
\end{align*}
\end{frame}



\begin{frame}{Kernel Discriminant Analysis} 
The between-class scatter matrix in feature space is
\begin{align*}
  \bB_\phi & = \Bigl(\bmu_1^\phi - \bmu_2^\phi\Bigr)
  \Bigl(\bmu_1^\phi - \bmu_2^\phi\Bigr)^T
\end{align*}
and the within-class scatter matrix in feature space is
\begin{align*}
  \bS_\phi &= n_1\cov_1^\phi + n_2\cov_2^\phi
\end{align*}
$\bS_\phi$ is a $d \times d$ symmetric, positive semidef\/{i}nite
matrix, where $d$ is the dimensionality of the feature space. 

\medskip
The best linear
discriminant vector $\bw$ in feature space is the dominant
eigenvector, which satisf\/{i}es the expression
\begin{align*}
\tcbhighmath{
  \lB(\bS_\phi^{-1}\bB_\phi\rB) \bw = \lambda \bw
}
 \label{eq:class:lda:eigeqF}
\end{align*}
where we assume that $\bS_\phi$ is non-singular. 
\end{frame}


\begin{frame}{LDA Objective via Kernel Matrix: Between-class Scatter}
\small
The projected mean for class $c_i$ is given as
\begin{align*}
   m_i & = \bw^T \bmu^\phi_i
   = {1 \over n_i} \dsum_{j=1}^n \dsum_{\bx_k \in \bD_i} a_{j}
  K(\bx_{j},\bx_k) = \ba^T \bmm_i 
\end{align*}
where $\ba = (a_1, a_2, \ldots, a_n)^T$ is the weight vector, and
\begin{align*}
  \bmm_i & = {1 \over n_i}
  \matr{
      \sum_{\bx_k \in \bD_i} K(\bx_1,\bx_k)\\
      \sum_{\bx_k \in \bD_i} K(\bx_2,\bx_k)\\
      \vdots\\
      \sum_{\bx_k \in \bD_i} K(\bx_n,\bx_k)
      }  = {1\over n_i}\bK^{c_i} \bone_{n_i}
\end{align*}
where $\bK^{c_i}$ is the $n \times n_i$ subset of the kernel
matrix, restricted to columns belonging to points only in $\bD_i$,
and $\bone_{n_i}$ is the $n_i$-dimensional vector all of whose
entries are one. 

\medskip
The separation between the projected means is thus
\begin{align*}
  (m_1 - m_2)^2  & =  \lB(\ba^T\bmm_1 - \ba^T\bmm_2\rB)^2 
     =  \ba^T\bM\ba
\end{align*}
where $\bM = (\bmm_1 - \bmm_2)(\bmm_1-\bmm_2)^T$ is the
between-class scatter matrix.
\end{frame}


\begin{frame}{LDA Objective via Kernel Matrix: Within-class Scatter}

We can compute the projected scatter for each class, $s_1^2$
and $s_2^2$, purely in terms of the kernel function, as follows
\begin{align*}
    s_1^2 & = \sum_{\bx_i \in \bD_1}
    \norm{\bw^T\phi(\bx_i) - \bw^T\bmu^\phi_1}^2
    = \ba^T \biggl(
    \Bigl(\sum_{\bx_i \in \bD_1} \bK_i\bK_i^T\Bigr)
    - n_1 \bmm_1 \bmm_1^T \biggr) \ba
     = \ba^T \bN_1 \ba
\end{align*}
where $\bK_i$ is the $i$th column of the kernel matrix, and $\bN_1$
is the class scatter matrix for $c_1$. 

\medskip
The sum of projected scatter values is then given as
\begin{align*}
  s_1^2 + s_2^2 = \ba^T (\bN_1 + \bN_2) \ba = \ba^T \bN \ba
\end{align*}
where $\bN$ is the $n\times n$ within-class scatter matrix.
\end{frame}

\begin{frame}{Kernel LDA}
The kernel LDA maximization condition is
\begin{align*}
\tcbhighmath{
    \max_{\bw} J(\bw) = \max_{\ba} J(\ba) =
    {\ba^T\bM\ba \over \ba^T\bN\ba}
}
\end{align*}


\medskip
The weight vector $\ba$ is the eigenvector
corresponding to the largest eigenvalue of the generalized
eigenvalue problem:
\begin{align*}
    \bM\ba = \lambda_1\bN\ba
\end{align*}

\medskip
When there are only two classes $\ba$ can be
obtained directly:
\begin{align*}
    \ba = \bN^{-1} (\bmm_1 - \bmm_2)
\end{align*}

\medskip
To normalize $\bw$ to be a unit vector  we
scale $\ba$ by $1 \over \sqrt{\ba^T\bK\ba}$.


\medskip
We can project any point $\bx$ onto the discriminant
direction as follows:
\begin{align*}
\tcbhighmath{
  \bw^T\phi(\bx) = \dsum_{j=1}^n a_{j}\phi(\bx_{j})^T\phi(\bx) =
  \dsum_{j=1}^n a_{j} K(\bx_{j},\bx)
}
\end{align*}
\end{frame}



\newcommand{\KLDA}{\textsc{KernelDiscriminant}}
\begin{frame}[fragile]{Kernel Discriminant Analysis Algorithm}
\begin{tightalgo}[H]{\textwidth-18pt}
\SetKwInOut{Algorithm}{\KLDA\ ($\bD = \{(\bx_i, y_i)\}_{i=1}^n, K$)} 
\Algorithm{} 
$\bK \assign \bigl\{K(\bx_i, \bx_{j}) \bigr\}_{i,j=1,\ldots,n}$ \tcp{compute $n \times n$ kernel matrix} 
$\bK^{c_i} \assign \bigl\{\bK(j,k) \mid y_k=c_i, 1\le j,k \le n\bigr\}, i=1,2$ \tcp{class kernel matrix} 
$\bmm_i \assign {1\over n_i}\bK^{c_i}  \bone_{n_i}, i=1,2$ \tcp{class means} 
$\bM \assign (\bmm_1-\bmm_2)(\bmm_1-\bmm_2)^T$ \tcp{between-class scatter matrix} 
$\bN_i  \assign \bK^{c_i} (\bI_{n_i} - {1\over n_i}\bone_{n_i \times n_i}) (\bK^{c_i})^T$, $i = 1, 2$ \tcp{class scatter matrices} 
$\bN \assign \bN_1 + \bN_2$ \tcp{within-class scatter matrix} 
$\lambda_1, \ba \assign \text{eigen}(\bN^{-1}\bM)$ \tcp{compute weight vector} 
$\ba \assign {\ba \over \sqrt{\ba^T \bK \ba}}$ \tcp{normalize $\bw$ to be unit vector} 
\end{tightalgo}
\end{frame}



\readdata{\dataPC}{CLASS/lda/figs/iris-PC.txt}
\readdata{\dataW}{CLASS/lda/figs/projw.dat} 

\begin{frame}[fragile]{Kernel Discriminant Analysis: Quadratic Homogeneous Kernel}

\begin{center}
\begin{small}
Iris 2D Data: $c_1$ (circles; {\tt iris-virginica}) and
  $c_2$ (triangles; other two Iris types).\\
  Kernel Function: $K(\bx_i, \bx_j) = (\bx_i^T\bx_j)^2$. \\
  Contours of constant projection onto optimal kernel
  discriminant. 
\end{small}
\end{center}


\begin{figure}
\scalebox{0.6}{
\psset{stepFactor=0.3} \psset{dotscale=1.5,fillcolor=lightgray,
            arrowscale=2,PointName=none}
\psset{xAxisLabel=$\bu_1$, yAxisLabel= $\bu_2$}
\psgraph[tickstyle=bottom,Ox=-4,Oy=-1.5,Dx=1,Dy=0.5]{->}(-4.0,-1.5)(4.0,1.5){4in}{3in}%
\listplot[plotstyle=dots,dotstyle=Bo,showpoints=true,
    nEnd=50,plotNo=1,plotNoMax=2]{\dataPC}
\listplot[plotstyle=dots,dotstyle=Btriangle,showpoints=true,
    nStart=51,plotNo=1,plotNoMax=2]{\dataPC}
\psset{algebraic=true}
\psplotImp[algebraic](-4,-1.5)(4,1.5){%
        0.5+0.51*x*y+0.76*x^2-0.4*y^2}
\psplotImp[algebraic](-4,-1.5)(4,1.5){%
        0.1+0.51*x*y+0.76*x^2-0.4*y^2}
\psplotImp[algebraic](-4,-1.5)(4,1.5){%
        -0.1+0.51*x*y+0.76*x^2-0.4*y^2}
\psplotImp[algebraic](-4,-1.5)(4,1.5){%
        -0.5+0.51*x*y+0.76*x^2-0.4*y^2}
\psplotImp[algebraic](-4,-1.5)(4,1.5){%
        -1+0.51*x*y+0.76*x^2-0.4*y^2}
\psplotImp[algebraic](-4,-1.5)(4,1.5){%
        -2+0.51*x*y+0.76*x^2-0.4*y^2}
\psplotImp[algebraic](-4,-1.5)(4,1.5){%
        -3+0.51*x*y+0.76*x^2-0.4*y^2}
\psplotImp[algebraic](-4,-1.5)(4,1.5){%
        -4+0.51*x*y+0.76*x^2-0.4*y^2}
\psplotImp[algebraic](-4,-1.5)(4,1.5){%
        -5+0.51*x*y+0.76*x^2-0.4*y^2}
\psplotImp[algebraic](-4,-1.5)(4,1.5){%
        -6+0.51*x*y+0.76*x^2-0.4*y^2}
\psplotImp[algebraic](-4,-1.5)(4,1.5){%
        -8+0.51*x*y+0.76*x^2-0.4*y^2}
\endpsgraph
}
\end{figure}
\end{frame}



\begin{frame}{Kernel Discriminant Analysis: Quadratic Homogeneous Kernel}

\begin{center}
\begin{small}
Iris 2D Data: $c_1$ (circles; {\tt iris-virginica}) and $c_2$ (triangles; other two Iris types).\\
  Kernel Function: $K(\bx_i, \bx_j) = (\bx_i^T\bx_j)^2$.\\
  Projection onto Optimal Kernel Discriminant.
\end{small}
\end{center}

  \begin{figure}
\scalebox{0.8}{
\psset{dotscale=1.8, arrowscale=2,PointName=none} 
\psset{xAxisLabel=$\bw$, yAxisLabel=$~$}
%\psgraph[tickstyle=bottom,Ox=-1,Oy=-0.1,Dx=2,Dy=-0.1]{->}
%(-1,-0.1)(11,0.1){5in}{1in}%
\pspicture[](-1,-1)(11,1)
%\psline[linewidth=2pt]{->}(-1,0)(11,0)
%\def\tick{\psline(0,0)(0,-0.3)}%
%\multips(-1,0)(2,0){6}{\tick}
\psaxes[Ox=-1]{->}(-1,-0.1)(11,0.1) \uput[0](11.05,0){$\bw$}
\listplot[plotstyle=dots,dotstyle=Btriangle,
    showpoints=true,nStart=51, fillcolor=lightgray]{\dataW}
%\listplot[plotstyle=dots,dotstyle=Bo,
%    showpoints=true,nEnd=50, fillcolor=lightgray]{\dataW}
\psset{dotscale=2.2} \psdot[dotstyle=Bo,fillcolor=white](0.378,0)
\psdot[dotstyle=Btriangle,fillcolor=white](4.476,0)
%\endpsgraph
\endpspicture
}
\end{figure}
\end{frame}



\readdata{\dataQHca}{CLASS/lda/figs/iris-QHc1.dat}
\readdata{\dataQHcb}{CLASS/lda/figs/iris-QHc2.dat}
\readdata{\dataMca}{CLASS/lda/figs/kdamappedPoints-c1.dat}
\readdata{\dataMcb}{CLASS/lda/figs/kdamappedPoints-c2.dat}
\begin{frame}{Kernel Feature Space and Optimal Discriminant}

\begin{center}
\begin{small}
Iris 2D Data: $c_1$ (circles; {\tt iris-virginica}) and
  $c_2$ (triangles; other two Iris types).\\ Kernel Function: 
  $K(\bx_i, \bx_j) = (\bx_i^T\bx_j)^2$.
\end{small}
\end{center}

\begin{figure}[!t]
\scalebox{0.7}{
\centering 
\psset{unit=0.4in} \psset{arrowscale=2}
\psset{Alpha=120,Beta=-150}
%\psset{Alpha=-190,Beta=-45}
\psset{nameX=$~$, nameY=$~$, nameZ=$~$} \scalebox{0.75}{
\hskip15pt\begin{pspicture}(0,-7.5)(11,4.5)
\pstThreeDCoor[xMin=-6.5, xMax= 5, yMin=0,
        yMax=15, zMin=0, zMax=2.3, Dx=2, Dy=3, Dz=.5,
        linewidth=1pt,linecolor=black]
\pstThreeDBox[linecolor=gray](-6.5,0,0)(11,0,0)(0,13,0)(0,0,2)
\pstThreeDPut(5.3,0,0){$X_1X_2$} \pstThreeDPut(0,15.2,0){$X_1^2$}
\pstThreeDPut(0,0,2.5){$X_2^2$}
\pstThreeDLine[linecolor=gray](0.074,0.261,0.01)(0.119,0.177,-0.093)
\pstThreeDLine[linecolor=gray](-1.044,2.143,0.254)(0.509,0.757,-0.398)
\pstThreeDLine[linecolor=gray](-0.338,1.088,0.052)(0.324,0.483,-0.254)
\pstThreeDLine[linecolor=gray](0.203,0.069,0.3)(0.018,0.028,-0.014)
\pstThreeDLine[linecolor=gray](0.099,0.11,0.045)(0.059,0.088,-0.046)
\pstThreeDLine[linecolor=gray](0.146,0.089,0.12)(0.048,0.072,-0.038)
\pstThreeDLine[linecolor=gray](0.158,0.027,0.462)(-0.043,-0.063,0.033)
\pstThreeDLine[linecolor=gray](0.211,0.033,0.682)(-0.071,-0.106,0.056)
\pstThreeDLine[linecolor=gray](0.13,0.052,0.162)(0.021,0.032,-0.017)
\pstThreeDLine[linecolor=gray](-0.966,0.825,0.565)(-0.047,-0.07,0.037)
\pstThreeDLine[linecolor=gray](0.036,0.002,0.338)(-0.059,-0.088,0.046)
\pstThreeDLine[linecolor=gray](0.11,0.055,0.11)(0.028,0.041,-0.022)
\pstThreeDLine[linecolor=gray](0.426,0.661,0.137)(0.34,0.506,-0.266)
\pstThreeDLine[linecolor=gray](0.328,0.435,0.124)(0.23,0.342,-0.18)
\pstThreeDLine[linecolor=gray](0.043,0.792,0.001)(0.319,0.475,-0.25)
\pstThreeDLine[linecolor=gray](0.253,0.126,0.253)(0.063,0.094,-0.05)
\pstThreeDLine[linecolor=gray](-0.159,0.094,0.133)(-0.032,-0.048,0.025)
\pstThreeDLine[linecolor=gray](0.155,0.141,0.085)(0.078,0.116,-0.061)
\pstThreeDLine[linecolor=gray](0.237,0.847,0.033)(0.385,0.573,-0.301)
\pstThreeDLine[linecolor=gray](-0.062,0.031,0.063)(-0.017,-0.026,0.014)
\pstThreeDLine[linecolor=gray](0.092,0.06,0.071)(0.033,0.049,-0.026)
\pstThreeDLine[linecolor=gray](0.401,0.345,0.234)(0.191,0.285,-0.15)
\pstThreeDLine[linecolor=gray](0.034,0.127,0.004)(0.057,0.085,-0.045)
\pstThreeDLine[linecolor=gray](0.187,0.66,0.026)(0.3,0.447,-0.235)
\pstThreeDLine[linecolor=gray](-1.008,0.5,1.017)(-0.277,-0.412,0.217)
\pstThreeDLine[linecolor=gray](-0.421,0.869,0.102)(0.207,0.309,-0.162)
\pstThreeDLine[linecolor=gray](-0.589,2.425,0.071)(0.775,1.153,-0.606)
\pstThreeDLine[linecolor=gray](-0.913,0.261,1.594)(-0.463,-0.689,0.362)
\pstThreeDLine[linecolor=gray](0.133,1.246,0.007)(0.518,0.771,-0.405)
\pstThreeDLine[linecolor=gray](0.722,0.888,0.294)(0.474,0.705,-0.371)
\pstThreeDLine[linecolor=gray](-1.064,0.565,1.002)(-0.263,-0.392,0.206)
\pstThreeDLine[linecolor=gray](-1.245,1.651,0.47)(0.22,0.328,-0.173)
\pstThreeDLine[linecolor=gray](-0.184,0.037,0.459)(-0.128,-0.19,0.1)
\pstThreeDLine[linecolor=gray](-0.44,1.2,0.081)(0.335,0.499,-0.262)
\pstThreeDLine[linecolor=gray](-0.461,1.772,0.06)(0.556,0.828,-0.435)
\pstThreeDLine[linecolor=gray](0.173,0.968,0.015)(0.419,0.623,-0.328)
\pstThreeDLine[linecolor=gray](-0.614,0.86,0.219)(0.129,0.192,-0.101)
\pstThreeDLine[linecolor=gray](-0.01,0,0.519)(-0.109,-0.162,0.085)
\pstThreeDLine[linecolor=gray](-0.152,0.51,0.023)(0.154,0.229,-0.12)
\pstThreeDLine[linecolor=gray](0.06,0.018,0.097)(0.003,0.004,-0.002)
\pstThreeDLine[linecolor=gray](0.601,1.681,0.107)(0.788,1.174,-0.617)
\pstThreeDLine[linecolor=gray](-0.704,1.489,0.166)(0.361,0.538,-0.283)
\pstThreeDLine[linecolor=gray](0.821,1.901,0.177)(0.917,1.366,-0.718)
\pstThreeDLine[linecolor=gray](0.378,0.41,0.174)(0.223,0.331,-0.174)
\pstThreeDLine[linecolor=gray](-0.419,0.809,0.109)(0.183,0.272,-0.143)
\pstThreeDLine[linecolor=gray](-0.223,0.652,0.038)(0.187,0.279,-0.147)
\pstThreeDLine[linecolor=gray](-0.07,0.005,0.494)(-0.117,-0.175,0.092)
\pstThreeDLine[linecolor=gray](-0.017,0.412,0)(0.156,0.231,-0.122)
\pstThreeDLine[linecolor=gray](-0.116,1.182,0.006)(0.428,0.637,-0.335)
\pstThreeDLine[linecolor=gray](0.438,0.214,0.448)(0.106,0.158,-0.083)
\pstThreeDLine[linecolor=gray](-0.886,8.067,0.049)(2.895,4.31,-2.266)
\pstThreeDLine[linecolor=gray](-0.561,8.351,0.019)(3.097,4.61,-2.424)
\pstThreeDLine[linecolor=gray](0.139,5.521,0.002)(2.183,3.249,-1.708)
\pstThreeDLine[linecolor=gray](1.15,2,0.331)(1.01,1.504,-0.791)
\pstThreeDLine[linecolor=gray](-0.598,5.356,0.033)(1.92,2.858,-1.502)
\pstThreeDLine[linecolor=gray](0.806,5.381,0.06)(2.29,3.409,-1.792)
\pstThreeDLine[linecolor=gray](1.583,6.468,0.194)(2.889,4.3,-2.261)
\pstThreeDLine[linecolor=gray](1.904,6.694,0.271)(3.045,4.533,-2.383)
\pstThreeDLine[linecolor=gray](0.213,1.666,0.014)(0.7,1.043,-0.548)
\pstThreeDLine[linecolor=gray](-0.403,7.149,0.011)(2.672,3.978,-2.091)
\pstThreeDLine[linecolor=gray](0.481,6.096,0.019)(2.492,3.71,-1.951)
\pstThreeDLine[linecolor=gray](0.633,6.898,0.029)(2.842,4.23,-2.224)
\pstThreeDLine[linecolor=gray](0.44,7.311,0.013)(2.955,4.399,-2.313)
\pstThreeDLine[linecolor=gray](-2.064,6.834,0.312)(2.054,3.058,-1.608)
\pstThreeDLine[linecolor=gray](0.556,1.931,0.08)(0.88,1.31,-0.689)
\pstThreeDLine[linecolor=gray](-0.196,3.112,0.006)(1.158,1.723,-0.906)
\pstThreeDLine[linecolor=gray](4.534,5.683,1.808)(3.024,4.502,-2.367)
\pstThreeDLine[linecolor=gray](0.875,0.27,1.419)(0.043,0.064,-0.034)
\pstThreeDLine[linecolor=gray](0.666,5.786,0.038)(2.416,3.597,-1.891)
\pstThreeDLine[linecolor=gray](-1.208,7.543,0.097)(2.597,3.867,-2.033)
\pstThreeDLine[linecolor=gray](-1.562,5.7,0.214)(1.765,2.627,-1.381)
\pstThreeDLine[linecolor=gray](-1,6.571,0.076)(2.278,3.391,-1.783)
\pstThreeDLine[linecolor=gray](3.034,6.878,0.669)(3.33,4.957,-2.606)
\pstThreeDLine[linecolor=gray](0.313,8.22,0.006)(3.277,4.878,-2.564)
\pstThreeDLine[linecolor=gray](1.476,1.808,0.603)(0.965,1.437,-0.756)
\pstThreeDLine[linecolor=gray](1.383,4.879,0.196)(2.218,3.302,-1.736)
\pstThreeDLine[linecolor=gray](0.809,2.328,0.141)(1.088,1.619,-0.851)
\pstThreeDLine[linecolor=gray](-0.659,4.689,0.046)(1.641,2.444,-1.285)
\pstThreeDLine[linecolor=gray](-1.461,8.597,0.124)(2.936,4.37,-2.298)
\pstThreeDLine[linecolor=gray](-0.898,7.77,0.052)(2.776,4.132,-2.173)
\pstThreeDLine[linecolor=gray](-1.265,6.846,0.117)(2.307,3.435,-1.806)
\pstThreeDLine[linecolor=gray](2.108,6.465,0.344)(2.994,4.457,-2.343)
\pstThreeDLine[linecolor=gray](2.734,4.836,0.773)(2.436,3.627,-1.907)
\pstThreeDLine[linecolor=gray](1.026,1.435,0.367)(0.751,1.118,-0.588)
\pstThreeDLine[linecolor=gray](-2.025,8.891,0.231)(2.881,4.289,-2.255)
\pstThreeDLine[linecolor=gray](0.042,6.41,0)(2.503,3.726,-1.959)
\pstThreeDLine[linecolor=gray](1.425,5.808,0.175)(2.595,3.863,-2.031)
\pstThreeDLine[linecolor=gray](-0.421,4.601,0.019)(1.675,2.493,-1.311)
\pstThreeDLine[linecolor=gray](-1.107,4.442,0.138)(1.41,2.099,-1.103)
\pstThreeDLine[linecolor=gray](-2.984,9.465,0.47)(2.805,4.175,-2.195)
\pstThreeDLine[linecolor=gray](1.262,3.167,0.251)(1.51,2.247,-1.182)
\pstThreeDLine[linecolor=gray](0.551,3.248,0.047)(1.397,2.08,-1.094)
\pstThreeDLine[linecolor=gray](1.062,7.674,0.073)(3.246,4.832,-2.541)
\pstThreeDLine[linecolor=gray](0.344,5.304,0.011)(2.15,3.201,-1.683)
\pstThreeDLine[linecolor=gray](1.209,2.513,0.291)(1.233,1.836,-0.965)
\pstThreeDLine[linecolor=gray](-0.403,7.149,0.011)(2.672,3.978,-2.091)
\pstThreeDLine[linecolor=gray](0.644,10.342,0.02)(4.185,6.23,-3.276)
\pstThreeDLine[linecolor=gray](1.192,6.969,0.102)(3,4.466,-2.348)
\pstThreeDLine[linecolor=gray](-6.265,10.448,1.878)(2.042,3.04,-1.598)
\pstThreeDLine[linecolor=gray](-1.359,14.4,0.064)(5.231,7.787,-4.094)
\pstThreeDLine[linecolor=gray](2.253,6.892,0.368)(3.193,4.753,-2.499)
\pstThreeDLine[linecolor=gray](-0.328,7.957,0.007)(3.007,4.477,-2.354)
\pstThreeDLine[linecolor=gray](-0.708,6.932,0.036)(2.503,3.726,-1.959)
\pstThreeDLine[linecolor=gray](-2.331,8.34,0.326)(2.567,3.822,-2.009)
\pstThreeDLine[linecolor=gray](0.319,1.582,0.032)(0.692,1.03,-0.541)
\pstThreeDLine[linecolor=gray](-0.932,7.375,0.059)(2.612,3.888,-2.044)
\pstThreeDLine[linecolor=gray](-0.312,3.616,0.013)(1.322,1.968,-1.035)
\pstThreeDLine[linecolor=gray](0.505,3.884,0.033)(1.635,2.435,-1.28)
\pstThreeDLine[linecolor=gray](-1.038,5.853,0.092)(1.986,2.956,-1.554)
\pstThreeDLine[linecolor=gray](2.411,5.198,0.559)(2.536,3.776,-1.985)
\pstThreeDLine[linecolor=gray](0.4,1.926,0.042)(0.845,1.258,-0.661)
\pstThreeDLine[linecolor=gray](0.08,6.829,0)(2.676,3.983,-2.094)
\pstThreeDLine[linecolor=gray](-1.418,8.99,0.112)(3.102,4.618,-2.428)
\pstThreeDLine[linecolor=gray](-0.32,3.628,0.014)(1.324,1.971,-1.036)
\pstThreeDLine[linecolor=gray](1.397,1.686,0.579)(0.902,1.343,-0.706)
\pstThreeDLine[linecolor=gray](-1.498,8.071,0.139)(2.719,4.047,-2.128)
\pstThreeDLine[linecolor=gray](-0.569,2.762,0.059)(0.913,1.36,-0.715)
\pstThreeDLine[linecolor=gray](-1.073,5.18,0.111)(1.711,2.547,-1.339)
\pstThreeDLine[linecolor=gray](-0.104,5.552,0.001)(2.131,3.173,-1.668)
\pstThreeDLine[linecolor=gray](-0.651,7.373,0.029)(2.691,4.006,-2.106)
\pstThreeDLine[linecolor=gray](-1.294,5.896,0.142)(1.926,2.866,-1.507)
\pstThreeDLine[linecolor=gray](-3.763,8.135,0.87)(2.002,2.981,-1.567)
\pstThreeDLine[linecolor=gray](-2.296,10.402,0.253)(3.393,5.051,-2.656)
\pstThreeDLine[linecolor=gray](1.289,7.445,0.112)(3.209,4.777,-2.511)
\pstThreeDLine[linecolor=gray](1.358,6.566,0.14)(2.879,4.286,-2.253)
\pstThreeDLine[linecolor=gray](4.04,6.745,1.21)(3.431,5.107,-2.685)
\pstThreeDLine[linecolor=gray](-2.629,11.539,0.299)(3.739,5.566,-2.926)
\pstThreeDLine[linecolor=gray](0.293,2.083,0.021)(0.882,1.313,-0.69)
\pstThreeDLine[linecolor=gray](-0.129,3.629,0.002)(1.377,2.049,-1.077)
\pstThreeDLine[linecolor=gray](1.15,2,0.331)(1.01,1.504,-0.791)
\pstThreeDLine[linecolor=gray](-0.493,6.29,0.019)(2.313,3.443,-1.81)
\pstThreeDLine[linecolor=gray](1.24,7.205,0.107)(3.103,4.62,-2.429)
\pstThreeDLine[linecolor=gray](-0.723,6.7,0.039)(2.408,3.585,-1.885)
\pstThreeDLine[linecolor=gray](1.832,6.439,0.26)(2.929,4.36,-2.292)
\pstThreeDLine[linecolor=gray](0.633,4.507,0.044)(1.908,2.841,-1.494)
\pstThreeDLine[linecolor=gray](3.07,7.01,0.672)(3.39,5.047,-2.653)
\pstThreeDLine[linecolor=gray](0.667,4.66,0.048)(1.976,2.942,-1.547)
\pstThreeDLine[linecolor=gray](-2.26,12.245,0.209)(4.128,6.145,-3.231)
\pstThreeDLine[linecolor=gray](4.435,6.988,1.407)(3.588,5.341,-2.808)
\pstThreeDLine[linecolor=gray](-0.403,7.149,0.011)(2.672,3.978,-2.091)
\pstThreeDLine[linecolor=gray](2.311,6.283,0.425)(2.96,4.406,-2.316)
\pstThreeDLine[linecolor=gray](1.195,7.014,0.102)(3.018,4.493,-2.362)
\pstThreeDLine[linecolor=gray](-0.112,3.8,0.002)(1.448,2.155,-1.133)
\pstThreeDLine[linecolor=gray](-3.221,8.516,0.609)(2.345,3.491,-1.836)
\pstThreeDLine[linecolor=gray](-1.113,3.696,0.168)(1.112,1.656,-0.87)
\pstThreeDLine[linecolor=gray](0.272,1.366,0.027)(0.597,0.888,-0.467)
\pstThreeDLine[linecolor=gray](1.3,5.339,0.158)(2.383,3.547,-1.865)
\pstThreeDLine[linecolor=gray](-0.515,3.779,0.035)(1.328,1.976,-1.039)
\pstThreeDLine[linecolor=gray](-5.78,12.171,1.373)(2.942,4.38,-2.303)
\pstThreeDLine[linecolor=gray](0.865,6.71,0.056)(2.823,4.203,-2.21)

\psset{dotstyle=Btriangle,dotscale=1.75,fillcolor=lightgray}
\listplotThreeD[plotstyle=dots,showpoints=true]{\dataQHcb}
\psset{dotstyle=Bo,dotscale=1.75,fillcolor=lightgray}
\listplotThreeD[plotstyle=dots,showpoints=true]{\dataQHca}
%w = (0.511, 0.761, -0.4)
\pstThreeDLine[linewidth=2pt,arrows=->](-0.509,-0.758,0.398)
(5.754, 8.566, -4.5) \pstThreeDPut(5.9,8.6,-4.5){$\bw$}
\psset{dotstyle=Btriangle,dotscale=1.75,fillcolor=white}
\listplotThreeD[plotstyle=dots,showpoints=true]{\dataMcb}
\psset{dotstyle=Bo,dotscale=1.75,fillcolor=white}
\listplotThreeD[plotstyle=dots,showpoints=true]{\dataMca}
\psset{dotscale=2,fillcolor=black}
\pstThreeDDot[dotstyle=Bo](0.173,0.257,-0.135)
\pstThreeDDot[dotstyle=Btriangle](2.287,3.405,-1.79)
\end{pspicture}
}}
\end{figure}
\end{frame}
