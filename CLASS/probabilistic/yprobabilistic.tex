\lecture{probabilistic}{probabilistic}

\date{Chapter 18: Probabilistic Classification}

\begin{frame}
\titlepage
\end{frame}


\begin{frame}{Bayes Classif\/{i}er}
Let the training dataset $\bD$ consist of $n$ points $\bx_i$ in a
$d$-dimensional space, and let $y_i$ denote the class for each
point, with $y_i \in \{c_1,c_2,\ldots,c_k\}$. 

\bigskip
The Bayes classif\/{i}er
 estimates the posterior probability
$P(c_i | \bx)$ for each class $c_i$, and chooses the class that
has the largest probability. The predicted class for $\bx$ is
given as
\begin{align*}
    \hy = \arg\max_{c_i} \{ P(c_i|\bx) \}
\end{align*}

According to the Bayes theorem, we have
\begin{align*}
\tcbhighmath{
    P(c_i|\bx) = \frac{P(\bx|c_i) \cdot P(c_i)}{P(\bx)}
}
\end{align*}
Because $P(\bx)$ is f\/{i}xed for a given point, Bayes rule 
can be rewritten as
\begin{align*}
    \hy & = \arg\max_{c_i} \{ P(c_i|\bx) \} 
    = \arg\max_{c_i} \lB\{ {P(\bx|c_i)P(c_i) \over P(\bx)}
    \rB\}
     = 
\tcbhighmath{
\arg\max_{c_i} \bigr\{ P(\bx|c_i)P(c_i) \bigl\}
}
\end{align*}
\end{frame}



\begin{frame}{Estimating the Prior Probability: $P(c_i)$}
 Let
$\bD_i$ denote the subset of points in $\bD$ that are labeled with
class $c_i$:
\begin{align*}
    \bD_i = \{\bx_{j} \in \bD \mid \bx_{j} \mbox{ has
    class } y_{j}=c_i \}
\end{align*}

\bigskip
Let the size of the dataset $\bD$ be given as $|\bD| = n$, and let
the size of each class-specif\/{i}c subset $\bD_i$ be given as
$|\bD_i| = n_i$. 

\bigskip
The prior probability for class $c_i$ can be
estimated as follows:
\begin{align*}
\tcbhighmath{
  \hP(c_i) = \frac{n_i}{n}
}
\end{align*}
\end{frame}


\begin{frame}{Estimating the Likelihood}
\framesubtitle{Numeric Attributes, Parametric Approach}
To estimate the likelihood $P(\bx|c_i)$, we have to estimate the
joint probability of $\bx$ across all the $d$ dimensions, i.e.,
we have to estimate $P\bigl(\bx=(x_1,x_2,\ldots,x_d) | c_i\bigr)$.

\medskip
In the parametric approach we assume that each class $c_i$ is
normally distributed, and we use the estimated mean $\hbmu_i$ 
and covariance matrix $\hcov_i$ to compute the probability density 
at $\bx$
\begin{align*}
\tcbhighmath{
  \hf_i(\bx) = \hf(\bx | \hbmu_i, \hcov_i) =
  \frac{1}{(\sqrt{2\pi})^{d}
    \sqrt{|\hcov_i|}} \exp \lB\{-\frac{(\bx - \hbmu_i)^T
    \hcov_i^{-1} (\bx -\hbmu_i)}{2} \rB \}
}
\end{align*}

\medskip
The posterior probability is then given as
\begin{align*}
    P(c_i|\bx) 
    = \frac{\hf_i(\bx) P(c_i)}{\sum_{j=1}^k \hf_{j}(\bx) P(c_{j})}
\end{align*}

The predicted class for $\bx$ is:
\begin{align*}
    \hy = \arg\max_{c_i} \Bigl\{ \hf_i(\bx) P(c_i) \Bigr\}
\end{align*}
\end{frame}



\newcommand{\BC}{\textsc{BayesClassif\/{i}er}\xspace}
\newcommand{\BCT}{\textsc{Testing}\xspace}
\begin{frame}[fragile]{Bayes Classifier Algorithm}
\begin{tightalgo}[H]{\textwidth-18pt}
\SetKwInOut{Algorithm}{\BC($\bD = \{(\bx_{j}, y_{j})\}_{j=1}^n$)}
\Algorithm{} 
\For{$i=1,\ldots,k$}{
  $\bD_i \assign \bigl\{\bx_{j} \mid y_{j}=c_i, j=1,\ldots,n\bigr\}$ \tcp{class-specif\/{i}c subsets} 
$n_i \assign \card{\bD_i}$ \tcp{cardinality} 
$\hP(c_i) \assign n_i/n$ \tcp{prior probability} 
$\hbmu_i \assign {1\over n_i} \sum_{\bx_{j} \in \bD_i} \bx_{j}$ \tcp{mean} 
$\bZ_i \assign \bD_i - \bone_{n_i} \hbmu_i^T$ \tcp{centered data} 
$\hcov_i  \assign {1\over n_i}\bZ_i^T\bZ_i$ \tcp{covariance matrix} 
} 
\Return{$\hP(c_i), \hbmu_i, \hcov_i$ for all $i=1,\ldots,k$} 
\BlankLine 
\BlankLine
\SetKwInOut{AlgorithmT}{\BCT($\bx$ \text{and} $\hP(c_i)$, $\hbmu_i$, $\hcov_i$, for all $i\in[1,k]$)} 
\AlgorithmT{} 
$\hy \assign \displaystyle \arg\max_{c_i} \bigl\{ f(\bx | \hbmu_i, \hcov_i) \cdot P(c_i) \bigr\}$ 
\Return{$\hy$} 
\end{tightalgo}
\end{frame}


\readdata{\dataSLW}{CLASS/probabilistic/figs/iris-slwc.txt}
\begin{frame}{Bayes Classifier: Iris Data}
  \framesubtitle{${\it X}_1$:{\tt sepal} {\tt length} versus ${\it X}_2$:{\tt sepal width}} 
\begin{figure}[!t]
   \scalebox{0.85}{
    \centering
    \psset{dotscale=1.5,fillcolor=lightgray,
        arrowscale=2,PointName=none}
        \psset{xAxisLabel=$X_1$, yAxisLabel= $X_2$}
    \psgraph[tickstyle=bottom,Dx=0.5,Dy=0.5,
    Ox=4,Oy=2]{->}(4.0,2.0)(8.5,4.5){4in}{3in}%
    \psline[linecolor=gray](5.2,2)(5.2,4.5)
    \psline[linecolor=gray](6.1,2)(6.1,4.5)
    \psline[linecolor=gray](7.0,2)(7.0,4.5)
    \psline[linecolor=gray](4,2.8)(8.5,2.8)
    \psline[linecolor=gray](4,3.6)(8.5,3.6)
    \listplot[plotstyle=dots,dotstyle=Bo,showpoints=true,
        nEnd=50,plotNo=1,plotNoMax=2]{\dataSLW}
    \psdot[dotstyle=Bo,dotscale=2,fillcolor=black](5.01,3.42)
    \listplot[plotstyle=dots,dotstyle=Btriangle,showpoints=true,
        nStart=51,plotNo=1,plotNoMax=2]{\dataSLW}
    \psdot[dotstyle=Btriangle,dotscale=2,fillcolor=black](6.26,2.87)
    \psdot[dotstyle=Bsquare,dotscale=2,fillcolor=white](6.75,4.25)
    \uput[45](6.75,4.25){$\bx=(6.75,4.25)^T$}
    \begin{psclip}{%
        \psline[](4,2)(4,4.5)(8.5,4.5)(8.5,2.0)(4,2)}
    %> ginv(S1)
    %      [,1]      [,2]
    %[1,]  18.56716 -12.82720
    %[2,] -12.82720  15.89032
        \psplotImp[linewidth=1pt,algebraic](3,1)(9,5){%
        18.57*(x-5.01)^2-2*12.83*(x-5.01)*(y-3.42)+15.89*(y-3.42)^2
        + ln(0.01)}
    %> ginv(S2)
    %      [,1]      [,2]
    %[1,]  3.316409 -3.658893
    %[2,] -3.658893 13.159501
        \psplotImp[linewidth=1pt,algebraic](3,1)(9,5){%
        3.32*(x-6.26)^2-2*3.66*(x-6.26)*(y-2.87)+13.16*(y-2.87)^2
        + ln(0.01)}
    \end{psclip}
    \endpsgraph
  }
\end{figure}
\end{frame}



\begin{frame}{Bayes Classifier}
\framesubtitle{Categorical Attributes}
Let $X_{j}$ be a
categorical attribute over the domain $dom(X_{j}) = \{a_{j1},
a_{j2}, \ldots, a_{jm_{j}}\}$. Each categorical attribute
$X_{j}$ is modeled as an $m_{j}$-dimensional multivariate Bernoulli
random variable $\bX_{j}$ that takes on $m_{j}$ distinct vector values
$\be_{j1}, \be_{j2}, \dots, \be_{jm_{j}}$, where $\be_{jr}$ is the
$r$th standard basis vector in $\setR^{m_{j}}$ and corresponds to
the $r$th value or symbol $a_{jr} \in dom(X_{j})$. 

\bigskip
The entire
$d$-dimensional dataset is modeled as the vector random variable $\bX = (\bX_1, \bX_2, \ldots,
\bX_d)^T$. Let $d' = \sum_{j=1}^d m_{j}$; a categorical point $\bx = (x_1,x_2,\dots,x_d)^T$ is
therefore represented as the $d'$-dimensional binary vector
\begin{align*}
  \bv = \matr{\bv_1\\\vdots\\\bv_d}
  = \matr{\be_{1r_1}\\\vdots\\\be_{dr_d}}
\end{align*}
where $\bv_{j} = \be_{jr_{j}}$ provided $x_{j} = a_{jr_{j}}$ is the
$r_{j}$th value in the domain of $X_{j}$. 
\end{frame}



\begin{frame}{Bayes Classifier}
\framesubtitle{Categorical Attributes}
The probability of the
categorical point $\bx$ is obtained from the joint probability
mass function (PMF) for the vector random variable $\bX$:
\begin{align*}
\tcbhighmath{
  P(\bx | c_i) = f(\bv|c_i)  =
    f\bigl(\bX_1=\be_{1r_1}, \dots, \bX_d=\be_{dr_d} | \;c_i\bigr)
}
\end{align*}
The joint PMF can be estimated directly from the data
$\bD_i$ for each class $c_i$ as follows:
\begin{align*}
    \hf(\bv|c_i) = {n_i(\bv) \over n_i}
    %\label{eq:class:prob:hfcat}
\end{align*}
where $n_i(\bv)$ is the number of times the value $\bv$ occurs in
class $c_i$.

However, to avoid zero probabilities we add a 
{\em pseudo-count} of 1 for each value 
\begin{align*}
\tcbhighmath{
    \hf(\bv|c_i) = {n_i(\bv)+1 \over n_i + \prod_{j=1}^d m_{j}}
}
\end{align*}
\end{frame}


\begin{frame}{Discretized Iris Data}
\framesubtitle{\texttt{sepal} {\tt length} and \texttt{sepal} {\tt width}}

  {\begin{tabular}{c}
%\centerline
{
%\subfloat[Discretized {\tt sepal length}]
{
    \label{tab:class:prob:slbin}
\tabcolsep12pt\renewcommand{\arraystretch}{1.1} 
\begin{tabular}{|c|l|}
\hline
Bins & \multicolumn{1}{c|}{Domain} \\ \hline
$[4.3, 5.2]$ & Very Short ($a_{11}$) \\
$(5.2, 6.1]$ & Short ($a_{12}$) \\
$(6.1, 7.0]$ & Long ($a_{13}$)\\
$(7.0, 7.9]$ & Very Long ($a_{14}$)\\ \hline
\multicolumn{2}{c}{\fontsize{8}{8}\selectfont(a) Discretized {\tt sepal length}}\\
\end{tabular}
}
%\hspace{0.2in}
%\subfloat[Discretized {\tt sepal width}]
{
\label{tab:class:prob:swbin}
\tabcolsep12pt
\renewcommand{\arraystretch}{1.1}  
\begin{tabular}{|c|l|}
  \hline
  Bins & \multicolumn{1}{c|}{Domain} \\ \hline
  $[2.0, 2.8]$ & Short ($a_{21}$) \\
  $(2.8, 3.6]$ & Medium ($a_{22}$)\\
  $(3.6, 4.4]$ & Long ($a_{23}$)\\ \hline
  \multicolumn{2}{c}{\fontsize{8}{8}\selectfont(b) Discretized {\tt sepal width}}\\
\end{tabular}
    }}
\end{tabular}}
\end{frame}



\begin{frame}{Class-specif\/{i}c Empirical Joint Probability Mass Function}
\label{tab:class:prob:jepmf}%}
\footnotesize{
{\begin{tabular}{c}
  \tabcolsep6pt\renewcommand{\arraystretch}{1.1}\begin{tabular}{|c|l||c|c|c||c|}
    \hline
    & \multirow{2}{*}{Class: $c_1$}&\multicolumn{3}{c||}{$X_2$} &
    \multirow{2}{*}{$\hf_{\!X_1}$}\\
    \cline{3-5}
        & & {\tt Short} ($\be_{21}$) & {\tt Medium} ($\be_{22}$) &
        {\tt Long} ($\be_{23}$) & \\
        \hline\hline
        \multirow{4}{*}{$X_1$}
        & {\tt Very Short ($\be_{11}$)} &  ${1/50}$  &
        ${33/50}$  & ${5/50}$ & 39/50\\
        &{\tt Short ($\be_{12}$)} & $0$
        &${3/50}$  & ${8/50}$ & 13/50\\
        &{\tt Long ($\be_{13}$)} & $0$  &
        $0$   & $0$ & 0\\
        &{\tt Very Long ($\be_{14}$)} & $0$  &
        $0$ & $0$ & 0\\
    \hline
    \multicolumn{2}{|r||}{$\hf_{\!X_2}$} & 1/50 & 36/50 & 13/50 & \\
    \hline
  \end{tabular}\\[12pt]\\
 \tabcolsep6pt\renewcommand{\arraystretch}{1.1}\begin{tabular}{|c|l||c|c|c||c|}
    \hline
    & \multirow{2}{*}{Class: $c_2$}&\multicolumn{3}{c||}{$X_2$} &
    \multirow{2}{*}{$\hf_{\!X_1}$}\\
    \cline{3-5}
        & & {\tt Short} ($\be_{21}$) & {\tt Medium} ($\be_{22}$) &
        {\tt Long} ($\be_{23}$) & \\
        \hline\hline
        \multirow{4}{*}{$X_1$}
        & {\tt Very Short ($\be_{11}$)} &  ${6/100}$  &
        ${0}$  & $0$ & 6/100\\
        &{\tt Short ($\be_{12}$)} & ${24/100}$
        &${15/100}$  &      ${0}$ & 39/100\\
        &{\tt Long ($\be_{13}$)} & ${13/100}$  &
        ${30/100}$   &      ${0}$ & 43/100\\
        &{\tt Very Long ($\be_{14}$)} & ${3/100}$  &
        ${7/100}$       & ${2/100}$ & 12/100\\
    \hline
    \multicolumn{2}{|r||}{$\hf_{\!X_2}$} & 46/100 & 52/100 & 2/100 &\\
    \hline
  \end{tabular}\\
\end{tabular}}{}
}
\end{frame}




\begin{frame}{Iris Data: Test Case}
 Consider a test point $\bx=(5.3,3.0)^T$ corresponding to
    the categorical
    point $(\texttt{Short, Medium})$, which is represented as
    $\bv = \matr{\be_{12}^T & \be_{22}^T}^T$.

	\medskip
The prior probabilities of the classes are $\hP(c_1) = 0.33$ and $\hP(c_2) = 0.67$.

 The likelihood
    and posterior probability for each class is given as
    \begin{align*}
        \hP(\bx|c_1) & = \hf(\bv|c_1) = 3/50 = 0.06\\[-2pt]
        \hP(\bx|c_2) & = \hf(\bv|c_2) = 15/100 = 0.15\\[-2pt]
        \hP(c_1|\bx) & \propto 0.06 \times 0.33 = 0.0198\\[-2pt]
        \hP(c_2|\bx) & \propto 0.15 \times 0.67 = 0.1005
    \end{align*}
    In this case the predicted class is $\hy = c_2$.
\end{frame}


\begin{frame}{Iris Data: Test Case with Pseudo-counts}
    The test point $\bx=(6.75,4.25)^T$
    corresponds to the categorical
    point $(\texttt{Long, Long})$, and it is represented as
    $\bv = \matr{\be_{13}^T & \be_{23}^T}^T$. 
	
\medskip
	Unfortunately the
    probability mass at $\bv$ is zero for both classes.
    We adjust the PMF via pseudo-counts noting that the number 
	of possible
    values are $m_1 \times m_2 = 4 \times 3 = 12$.

	\medskip
    The likelihood and prior probability
    can then be computed as
    \begin{align*}
        \hP(\bx|c_1) & = \hf(\bv|c_1) = {0+1\over 50+12} = 1.61
        \times 10^{-2}\\
        \hP(\bx|c_2) & = \hf(\bv|c_2) = {0+1 \over 100+12} =
        8.93 \times 10^{-3}\\
        \hP(c_1|\bx) & \propto (1.61 \times 10^{-2}) \times 0.33 =
        5.32 \times 10^{-3}\\
        \hP(c_2|\bx) & \propto (8.93 \times 10^{-3}) \times 0.67 =
        5.98 \times 10^{-3}
    \end{align*}
    Thus, the predicted class is $\hy = c_2$.
  \end{frame}


  \begin{frame}{Bayes Classifier: Challenges}
The main problem with the Bayes classif\/{i}er is the lack of enough
data to reliably estimate the joint probability density or mass
function, especially for high-dimensional data. 

\bigskip
For
numeric attributes we have to estimate $O(d^2)$ covariances, and
as the dimensionality increases, this requires us to estimate too
many parameters. 

\medskip
For categorical attributes we have to estimate
the joint probability for all the possible values of $\bv$, given
as $\prod_{j} |\text{dom}\bigl(X_{j}\bigr)|$. Even if each categorical attribute
has only two values, we would need to estimate the probability for
$2^d$ values. However, because there can be at most $n$ distinct
values for $\bv$, most of the counts will be zero. 

\medskip
Naive Bayes classifier addresses these concerns.
\end{frame}


\begin{frame}{Naive Bayes Classif\/{i}er}
\framesubtitle{Numeric Attributes}
The naive Bayes approach makes the simple assumption
that all the attributes are independent, which
implies that the likelihood
can be decomposed into a product of dimension-wise probabilities:
\begin{align*}
\tcbhighmath{
P(\bx | c_i) = P(x_1,x_2,\ldots,x_d | c_i) = \prod_{j=1}^d P(x_{j} |
c_i) 
}
\end{align*}

 The
likelihood for class $c_i$, for dimension $X_{j}$, is given as
\begin{align*}
    P(x_{j}|c_i) \propto f(x_{j} | \hmu_{ij}, \hsigma_{ij}^2)  = {1\over
    \sqrt{2\pi}\hsigma_{ij}}\exp\lB\{ -{(x_{j} - \hmu_{ij})^2 \over
    2\hsigma_{ij}^2}\rB\}
\end{align*}
where $\hmu_{ij}$ and
$\hsigma_{ij}^2$ denote the estimated
mean and variance for attribute $X_{j}$,
for class $c_i$.
\end{frame}


\begin{frame}{Naive Bayes Classif\/{i}er}
\framesubtitle{Numeric Attributes}
The naive assumption corresponds to setting all the
covariances to zero in $\hcov_i$, that is,
\begin{align*}
  \cov_i = \matr{
      \sigma_{i1}^2 & 0 & \ldots & 0\\
      0 & \sigma_{i2}^2 & \ldots & 0\\
      \vdots & \vdots & \ddots\\
      0 &  0 & \ldots & \sigma_{id}^2}
\end{align*}

The naive Bayes classif\/{i}er thus 
uses the sample mean $\hbmu_i =
(\hmu_{i1}, \dots, \hmu_{id})^T$
and a {\em diagonal} sample covariance matrix $\hcov_i =
\mathit{diag}(\sigma_{i1}^2, \dots, \sigma_{id}^2)$
for each class $c_i$.
In total $2d$ parameters have to be estimated,
corresponding
to the sample mean and sample variance for each
dimension $X_{j}$.
\end{frame}




\newcommand{\NB}{\textsc{NaiveBayes}\xspace}
\newcommand{\NBT}{\textsc{Testing}\xspace}
\begin{frame}[fragile]{Naive Bayes Algorithm}
\begin{tightalgo}[H]{\textwidth-18pt}
  \small
\SetKwInOut{Algorithm}{\NB($\bD = \{(\bx_{j}, y_{j})\}_{j=1}^n$)}
\Algorithm{} 
\For{$i=1,\ldots,k$}{
  $\bD_i \assign \bigl\{\bx_{j} \mid y_{j}=c_i, j=1,\ldots,n\bigr\}$
  \tcp{class-specif\/{i}c subsets}\;
  $n_i \assign \card{\bD_i}$ \tcp{cardinality}\;
  $\hP(c_i) \assign n_i/n$ \tcp{prior probability}\;
  $\hbmu_i \assign {1\over n_i} \sum_{\bx_{j} \in \bD_i} \bx_{j}$ \tcp{mean}\;
  $\bZ_i = \bD_i - \bone \cdot \hbmu_i^T$\tcp{centered data for class
  $c_i$}\;
  \For(\tcp*[h]{class-specif\/{i}c variance for $X_{j}$}){$j=1,..,d$}{
    %$Z_{j} = X_{j}  - \bone_n \cdot \hmu_{ij}$ \tcp{centered attribute}\;
    %$\hsigma_{ij}^2 \assign {1\over n_i} Z_{j}^TZ_{j}$ \tcp{variance}\;
    %$\hsigma_{ij}^2 \assign {1\over n_i}
    %  \sum_{\bx_{j} \in \bD_i} (\bx_{j} - \hbmu_i)^2$ \tcp{variance}\;
    $\hsigma_{ij}^2 \assign {1\over n_i} Z_{ij}^TZ_{ij}$
    \tcp{variance}\;
  }
  $\hbsigma_i = \matr{\hsigma_{i1}^2, \ldots, \hsigma_{id}^2}^T$
  \tcp*[h]{class-specif\/{i}c attribute variances}\;
} \Return\ {$\hP(c_i), \hbmu_i, \hbsigma_i$ for all $i=1,\ldots,k$}
\BlankLine \BlankLine 
\SetKwInOut{AlgorithmT}{\NBT($\bx$ \text{and}\ $\hP(c_i)$, $\hbmu_i$, $\hbsigma_i$, for all $i\in[1,k]$)} 
\AlgorithmT{} $\hy \assign \displaystyle \arg\max_{c_i}
\biggl\{ \hP(c_i) \prod_{j=1}^d f(x_{j} | \hmu_{ij}, \hsigma^2_{ij})
\biggr\}$\; \Return{$\hy$} 
\end{tightalgo}
\end{frame}



\begin{frame}[fragile]{Naive Bayes versus Full Bayes Classifier}
\framesubtitle{Iris 2D Data}
\framesubtitle{${\it X}_1$:{\tt sepal} {\tt length} versus ${\it X}_2$:{\tt sepal} {\tt width}}
\setcounter{subfigure}{0}
\begin{figure}[!t]
\captionsetup[subfloat]{captionskip=0.25in}
  \centerline{
  \subfloat[Naive Bayes]{
\scalebox{0.5}{
    \vspace{0.2in}
    \psset{dotscale=1.5,fillcolor=lightgray,
        arrowscale=2,PointName=none}
        \psset{xAxisLabel=$X_1$, yAxisLabel= $X_2$}
    \psgraph[tickstyle=bottom,Dx=0.5,Dy=0.5,
    Ox=4,Oy=2]{->}(4.0,2.0)(8.5,4.5){4in}{3in}%
    \psline[linecolor=gray](5.2,2)(5.2,4.5)
    \psline[linecolor=gray](6.1,2)(6.1,4.5)
    \psline[linecolor=gray](7.0,2)(7.0,4.5)
    \psline[linecolor=gray](4,2.8)(8.5,2.8)
    \psline[linecolor=gray](4,3.6)(8.5,3.6)
    \listplot[plotstyle=dots,dotstyle=Bo,showpoints=true,
        nEnd=50,plotNo=1,plotNoMax=2]{\dataSLW}
    \psdot[dotstyle=Bo,dotscale=2,fillcolor=black](5.01,3.42)
    \listplot[plotstyle=dots,dotstyle=Btriangle,showpoints=true,
        nStart=51,plotNo=1,plotNoMax=2]{\dataSLW}
    \psdot[dotstyle=Btriangle,dotscale=2,fillcolor=black](6.26,2.87)
    \psdot[dotstyle=Bsquare,dotscale=2,fillcolor=white](6.75,4.25)
    \uput[45](6.75,4.25){$\bx=(6.75,4.25)^T$}
    \begin{psclip}{%
        \psline[](4,2)(4,4.5)(8.5,4.5)(8.5,2.0)(4,2)}
    %> ginv(S1)
    %      [,1]      [,2]
    %[1,]  18.56716 -12.82720
    %[2,] -12.82720  15.89032
        \psplotImp[linewidth=1pt,algebraic](3,1)(9,5){%
        18.57*(x-5.01)^2+15.89*(y-3.42)^2
        + ln(0.01)}
    %> ginv(S2)
    %      [,1]      [,2]
    %[1,]  3.316409 -3.658893
    %[2,] -3.658893 13.159501
        \psplotImp[linewidth=1pt,algebraic](3,1)(9,5){%
        3.32*(x-6.26)^2+13.16*(y-2.87)^2
        + ln(0.01)}
    \end{psclip}
    \endpsgraph
	}}
	\hspace{0.25in}
  \subfloat[Full Bayes]{
   \scalebox{0.5}{
    \psset{dotscale=1.5,fillcolor=lightgray,
        arrowscale=2,PointName=none}
        \psset{xAxisLabel=$X_1$, yAxisLabel= $X_2$}
    \psgraph[tickstyle=bottom,Dx=0.5,Dy=0.5,
    Ox=4,Oy=2]{->}(4.0,2.0)(8.5,4.5){4in}{3in}%
    \psline[linecolor=gray](5.2,2)(5.2,4.5)
    \psline[linecolor=gray](6.1,2)(6.1,4.5)
    \psline[linecolor=gray](7.0,2)(7.0,4.5)
    \psline[linecolor=gray](4,2.8)(8.5,2.8)
    \psline[linecolor=gray](4,3.6)(8.5,3.6)
    \listplot[plotstyle=dots,dotstyle=Bo,showpoints=true,
        nEnd=50,plotNo=1,plotNoMax=2]{\dataSLW}
    \psdot[dotstyle=Bo,dotscale=2,fillcolor=black](5.01,3.42)
    \listplot[plotstyle=dots,dotstyle=Btriangle,showpoints=true,
        nStart=51,plotNo=1,plotNoMax=2]{\dataSLW}
    \psdot[dotstyle=Btriangle,dotscale=2,fillcolor=black](6.26,2.87)
    \psdot[dotstyle=Bsquare,dotscale=2,fillcolor=white](6.75,4.25)
    \uput[45](6.75,4.25){$\bx=(6.75,4.25)^T$}
    \begin{psclip}{%
        \psline[](4,2)(4,4.5)(8.5,4.5)(8.5,2.0)(4,2)}
    %> ginv(S1)
    %      [,1]      [,2]
    %[1,]  18.56716 -12.82720
    %[2,] -12.82720  15.89032
        \psplotImp[linewidth=1pt,algebraic](3,1)(9,5){%
        18.57*(x-5.01)^2-2*12.83*(x-5.01)*(y-3.42)+15.89*(y-3.42)^2
        + ln(0.01)}
    %> ginv(S2)
    %      [,1]      [,2]
    %[1,]  3.316409 -3.658893
    %[2,] -3.658893 13.159501
        \psplotImp[linewidth=1pt,algebraic](3,1)(9,5){%
        3.32*(x-6.26)^2-2*3.66*(x-6.26)*(y-2.87)+13.16*(y-2.87)^2
        + ln(0.01)}
    \end{psclip}
    \endpsgraph
	}}
	}
\end{figure}
\end{frame}



\begin{frame}{Naive Bayes}
\framesubtitle{Categorical Attributes}
The
independence assumption leads to a simplif\/{i}cation of the joint
probability mass function 
\begin{align*}
\tcbhighmath{
    P(\bx|c_i) = \prod_{j=1}^d P(x_{j} | c_i) =
  \prod_{j=1}^d   f\bigl(\bX_{j}=\be_{jr_{j}} | \;c_i\bigr)
}
\end{align*}
where $f(\bX_{j}=\be_{jr_{j}} | c_i)$ is the probability mass function
for $\bX_{j}$, estimated as:
\begin{align*}
    \hf(\bv_{j} | c_i) = {n_i(\bv_{j}) \over n_i}
\end{align*}
where $n_i(\bv_{j})$ is the observed frequency of the value
$\bv_{j}=\be_{j}r_{j}$ corresponding to the $r_{j}$th categorical value
$a_{jr_{j}}$ for the attribute $X_{j}$ for class $c_i$. 

\bigskip
If the count is zero, we can use the pseudo-count
method to obtain a prior probability. The adjusted estimates with
pseudo-counts are given as
\begin{align*}
\tcbhighmath{
    \hf(\bv_{j} | c_i) = {n_i(\bv_{j})+1 \over n_i + m_{j}}
}
\end{align*}
where $m_{j} = |dom(X_{j})|$.
\end{frame}



\begin{frame}{Nonparametric Approach}
\framesubtitle{$K$ Nearest Neighbors Classif\/{i}er}

We consider a
non-parametric approach for likelihood estimation 
using the nearest neighbors density estimation.

\medskip
Let $\bD$ be a training dataset comprising $n$ points $\bx_i \in
\setR^d$, and let $\bD_i$ denote the subset of points in $\bD$ that are
labeled with class $c_i$, with $n_i = |\bD_i|$.  

\medskip
Given a test point $\bx
\in \setR^d$, and $K$, the number of neighbors to consider, let $r$
denote the distance from $\bx$ to its $K$th nearest neighbor in $\bD$.

\medskip
Consider the $d$-dimensional hyperball of radius $r$ around the test
point $\bx$, def\/{i}ned~as
\begin{align*}
  B_d(\bx,r) = \bigl\{\bx_i \in \bD \mid
  \;\dist(\bx,\bx_i) \leq r\bigr\}
\end{align*}
Here $\dist(\bx, \bx_i)$ is the distance between $\bx$ and $\bx_i$,
which is usually assumed to be the Euclidean distance, i.e., $\dist(\bx,
\bx_i) = \|\bx - \bx_i \|_2$.  We assume that $|B_d(\bx,r)| = K$.
\end{frame}


\begin{frame}{Nonparametric Approach}
\framesubtitle{$K$ Nearest Neighbors Classif\/{i}er}
Let $K_i$ denote the number of points among the $K$ nearest neighbors of
$\bx$ that are labeled with class $c_i$, that is
\begin{align*}
  K_i = \bigl\{\bx_{j} \in B_d(\bx,r) \mid y_{j} = c_i\bigr\}
\end{align*}

The class conditional probability density at $\bx$ can be estimated as
the fraction of points from class $c_i$ that lie within the hyperball
divided by its volume, that is
\begin{align*}
\tcbhighmath{
  \hf(\bx|c_i) = \frac{K_i\text{/}n_i}{V}  = \frac{K_i}{n_i V}
}
\end{align*}
where $V = \vol(B_d(\bx,r))$ is the volume of the $d$-dimensional
hyperball.

The posterior probability $P(c_i|\bx)$
can be estimated as
\begin{align*}
    P(c_i|\bx) &
    = \frac{\hf(\bx|c_i) \hP(c_i)}{\sum_{j=1}^k \hf(\bx|c_{j}) \hP(c_{j})}
\end{align*}
However, because $\hP(c_i) = \tfrac{n_i}{n}$, we have
\begin{align*}
  \hf(\bx|c_i) \hP(c_i) = \frac{K_i}{n_i V} \cdot \frac{n_i}{n} =
  \frac{K_i}{nV}
\end{align*}
\end{frame}


\begin{frame}{Nonparametric Approach}
\framesubtitle{$K$ Nearest Neighbors Classif\/{i}er}
The posterior probability is given as
\begin{align*}
  P(c_i|\bx) =
  \frac{\frac{K_i}{n V}}{\sum_{j=1}^k \frac{K_{j}}{n V}}
  = \frac{K_i}{K}
\end{align*}
F{i}nally, the predicted class for $\bx$ is
\begin{align*}
\tcbhighmath{
  \hy = \arg\max_{c_i} \lB\{ P(c_i | \bx) \rB\} =
  \arg\max_{c_i} \lB\{ \frac{K_i}{K} \rB\} =
  \arg\max_{c_i}  \lB\{ K_i \rB\}
}
\end{align*}
Because $K$ is f\/{i}xed, the KNN classif\/{i}er predicts the class of $\bx$ as
the majority class among its $K$ nearest neighbors.
\end{frame}


\begin{frame}{Iris Data}
\framesubtitle{$K$ Nearest Neighbors Classif\/{i}er }
\begin{figure}[!b]
    \centering
  \scalebox{0.7}{
    \psset{dotscale=1.5,fillcolor=lightgray,
        arrowscale=2,PointName=none}
        \psset{xAxisLabel=$X_1$, yAxisLabel= $X_2$}
    \pspicture[](-2,-1)(2,10)
    \psgraph[tickstyle=bottom,Dx=0.5,Dy=0.5,
    Ox=4,Oy=2]{->}(4.0,2.0)(8.5,4.5){5.4in}{3in}%
    \listplot[plotstyle=dots,dotstyle=Bo,showpoints=true,
        nEnd=50,plotNo=1,plotNoMax=2]{\dataSLW}
    \listplot[plotstyle=dots,dotstyle=Btriangle,showpoints=true,
        nStart=51,plotNo=1,plotNoMax=2]{\dataSLW}
    \psdot[dotstyle=Bsquare,dotscale=2,fillcolor=white](6.75,4.25)
    \uput[45](6.75,4.25){$\bx=(6.75,4.25)^T$}
    \psplotImp[linewidth=2pt,algebraic](4,2)(8.5,7){%
    (x-6.75)^2+(y-4.25)^2-1.025 }
    \pnode(6.75,4.25){center}
    \pnode(6.2,3.4){fifth}
    %\psline[](6.75,4.25)(6.2,3.4)
    %\uput[0](6.5,3.8){$r$}
    \ncline{center}{fifth}
    \nbput{$r$}
    \endpsgraph
    \endpspicture
    }
\end{figure}
\end{frame}
%\caption{Iris Data: $K$ Nearest Neighbors Classif\/{i}er }
%\label{fig:class:prob:KNN}
%\end{figure}
%
%\begin{example}
%  Consider the 2D Iris dataset shown in F{i}gure~\ref{fig:class:prob:KNN}.
%  The two classes are: $c_1$ (circles) with $n_1=50$ points
%  and $c_2$ (triangles) with $n_2 = 100$ points.
%
%  Let us classify the test point $\bx = (6.75, 4.25)^T$ using its
%  $K=5$ nearest neighbors. The distance from $\bx$ to its 5th
%  nearest neighbor, namely $(6.2,3.4)^T$, is given as
%  $r = \sqrt{1.025} = 1.012$. The enclosing ball or circle of radius $r$
%  is shown in the f\/{i}gure. It encompasses $K_1 = 1$ point from class
%  $c_1$ and $K_2 = 4$ points from class $c_2$. Therefore, the predicted
%  class for $\bx$ is $\hy = c_2$.
%\end{example}
%
%%%%%%%%%%%%NEW
%
%\vspace*{-6pt}
%\section[Exercises]{Further Reading}
%\label{sec:class:probabilistic:ref}
%\begin{refsection}
%
%The naive Bayes classif\/{i}er is surprisingly effective even though the
%independence assumption is usually violated in real datasets.
%Comparison of the naive Bayes classif\/{i}er against other classif\/{i}cation
%approaches and reasons for why is works well have appeared in
% \citet{langley1992analysis, domingos1997optimality,
% zhang2005exploring, hand2001idiot} and \citet{rish2001empirical}.
% For the long history of naive Bayes in
% information retrieval see \citet{lewis1998naive}. The $K$ nearest neighbor classif\/{i}cation approach was f\/{i}rst proposed in F{i}x and Hodges, Jr. (1951).\vspace*{-6pt}
%
%\printbibliography[heading=emptyheading]
%\end{refsection}
%
%
%\section{Exercises}
%\label{sec:class:prob:exercise}
%
%\begin{exercises}[Q1.]
%\item \label{ex:class:prob:agecarQ}
%Consider the dataset in Table~\ref{tab:class:prob:agecardata}.
%Classify the new point: (Age=23, Car=truck)
%via the full and naive Bayes approach. You may assume that the domain of
%Car is given as $\{$sports, vintage, suv, truck$\}$.
%
%\begin{table}[!h]
%\processtable{Data for Q\ref{ex:class:prob:agecarQ}
%\label{tab:class:prob:agecardata}}
%{\tabcolsep12pt\renewcommand{\arraystretch}{1.1}\begin{tabular}{|c|cc|c|}
%    \hline
%    $\bx_i$ & Age&Car&Class\\
%    \hline
%    $\bx_1$ & 25&sports&$\textit{L}$\\
%    $\bx_2$ &20&vintage&$\textit{H}$\\
%    $\bx_3$& 25&sports&$\textit{L}$\\
%    $\bx_4$ & 45&suv&$\textit{H}$\\
%    $\bx_5$ & 20&sports&$\textit{H}$\\
%    $\bx_6$ & 25&suv&$\textit{H}$\\
%    \hline
%  \end{tabular}}{}
%\end{table}
%
%
%\comment{
%{\bf Answer:}
%  Since each attribute is independent of the other, we
%consider them separately.
%That is
%$$P((23, truck)|H) = P(23|H) \times P(truck|H)$$
%and likewise,
%$$P((23, truck)|L) = P(23|L) \times P(truck|L)$$
%
%For {\tt Age}, we have $\cD_L = \{1,3\}$ and $\cD_H = \{2,4,5,6\}$. We
%can estimate the mean and variance from these labeled subsets, as
%shown in the table below
%\begin{center}
%  \begin{tabular}{l|l}
%    H&L\\
%    \hline
%    $\mu_H = \frac{20+45+20+25}{4} = \frac{110}{4} = 27.5$ &
%    $\mu_L = \frac{25+25}{2} = 25$\\
%    $\sigma_H = \sqrt{\frac{425}{4}} = 10.31 $ & $\sigma_L =
%    \sqrt{\frac{0}{2}}= 0$\\
%  \end{tabular}
%\end{center}
%Using the univariate normal distribution, we obtain
%$P(23|H) = N(23 | \mu_H=27.5, \sigma_H=10.31) = 0.035$, and
%$P(23|L) = N(23 | \mu_L=25, \sigma_L=0) = 0$. Note that due to limited
%data we obtain $\sigma_L=0$, which leads to a zero likelihood for 23
%to come from class $L$.
%%A zero probability is too strict, and in such
%%cases it is better to assume that the standard deviation is some small
%%value instead of zero. For example, if we assume that $\sigma_L =
%%0.1$, then we obtain $P(23|L) = 5.52 \times 10^{-87}$, which is of
%%course almost zero, but not quite!
%
%For {\tt Car}, which is categorical, we immediately run into a
%problem, since the value {\tt truck} does not appear in the training
%set. We could assume that $P(truck|H)$ and $P(truck|L)$ are both
%zero. However, we desire to have some small probability of
%observing each values in the domain of the attribute. One simple way
%of obtaining non-zero probabilities is to do the {\em laplace
%  correction}, i.e., to add a count of one to the
%observed counts of each value for each class, as shown in the table
%below.
%\begin{center}
%  \begin{tabular}{l|l}
%    H&L\\
%    \hline
%    $P(sports|H) = \frac{1 (+1)}{4 (+4)} = 2/8$ &
%    $P(sports|L) = \frac{2 (+1)}{2 (+4)} = 3/6$\\
%    $P(vintage|H) = \frac{1 (+1)}{4 (+4)} = 2/8$ &
%    $P(vintage|L) = \frac{0 (+1)}{2 (+4)}  = 1/6$\\
%    $P(suv|H) = \frac{2 (+1)}{4 (+4)} = 3/8$ &
%    $P(suv|L) =  \frac{0 (+1)}{2 (+4)}  = 1/6$\\
%    $P(truck|H) = \frac{0 (+1)}{4 (+4)} = 1/8$ &
%    $P(truck|L) =  \frac{0 (+1)}{2 (+4)} = 1/6$\\
%  \end{tabular}
%\end{center}
%Note that all counts are adjusted by (+1) to obtain a non-zero
%probability in every case. Also, assuming that the domain of {\tt Car}
%consists of only the four values $\{sports, vintage, suv, truck\}$,
%this means we also need to increment the denominator by $|domain({\tt
%  Car})| = 4$. In other words, the probability of a given value is
%computed as
%\begin{align}
%  P(v|c_i) = \frac{n_v + 1}{|\cD_i| + |domain(c_i)|}
%\end{align}
%where $n_v$ is the observed count of value $v$ among the points in $\cD_i$.
%Note that instead of the laplace correction, if we have some
%prior probability estimate for each value, we can use that too
%\begin{align}
%P(v|c_i) = \frac{n_v + P_v}{|\cD_i| + \sum_{v}{P_v}}
%\end{align}
%
%Using the above probabilities, we f\/{i}nally obtain
%$$P((23,truck) | H) = P(23|H) \times P(truck|H) = 0.035 \times 1/8 = 0.0044$$
%$$P((23,truck) | L) = P(23|L) \times P(truck|L) = 0 \times 1/6 = 0$$
%We next compute
%\begin{align*}
%P(23,truck) & = P((23,truck)|H)\times P(H)+P((23,truck)|L) \times
%P(L)\\
% & =  0.0044 \times \frac{4}{6}+0 \times \frac{2}{6}  = 0.003
%\end{align*}
%We then obtain the posterior probabilities as follows
%$$P(H|(23,truck)) = \frac{P((23,truck)|H) \times P(H)}{P(23,truck)} =
%\frac{0.004\times \frac{4}{6}}{0.003} = 1$$
%and
%$$P(L|(23,truck)) = \frac{P((23,truck)|L) \times P(L)}{P(23,truck)} =
%\frac{0\times \frac{2}{6}}{0.003} = 0$$
%Thus we classify $(23, truck)$ as high risk (H).
%}
%
%\begin{table}[!t]
%\processtable{Data for Q\ref{ex:class:prob:F04HW4_2}
%\label{tab:class:prob:F04HW4_2}}
%{\tabcolsep12pt\renewcommand{\arraystretch}{1.1}\begin{tabular}{|c|ccc|c|}
%\hline
%$\bx_i$ & $a_1$ & $a_2$ & $a_3$ & Class\\
%\hline
%$\bx_1$ & $\textit{T}$ & $\textit{T}$ & 5.0 & $\textit{Y}$\\
%$\bx_2$ & $\textit{T}$ & $\textit{T}$ & 7.0 & $\textit{Y}$\\
%$\bx_3$ & $\textit{T}$ & $\textit{F}$ & 8.0 & $\textit{N}$\\
%$\bx_4$ & $\textit{F}$ & $\textit{F}$ & 3.0 & $\textit{Y}$\\
%$\bx_5$ & $\textit{F}$ & $\textit{T}$ & 7.0 & $\textit{N}$\\
%$\bx_6$ & $\textit{F}$ & $\textit{T}$ & 4.0 & $\textit{N}$\\
%$\bx_7$ & $\textit{F}$ & $\textit{F}$ & 5.0 & $\textit{N}$\\
%$\bx_8$ & $\textit{T}$ & $\textit{F}$ & 6.0 & $\textit{Y}$\\
%$\bx_9$ & $\textit{F}$ & $\textit{T}$ & 1.0 & $\textit{N}$\\
%\hline
%\end{tabular}}{}
%\end{table}
%
%\item \label{ex:class:prob:F04HW4_2}
%  Given the dataset in Table~\ref{tab:class:prob:F04HW4_2},
% use the naive Bayes classif\/{i}er to classify the new point $(T, F, 1.0)$.
%
%
%\item Consider the class means and covariance matrices for
%  classes $c_1$ and $c_2$:
%  \begin{align*}
%\bmu_1 & = (1,3) & \bmu_2 & = (5,5)\\
%\cov_1 & = \left(
%\begin{array}{cc}
%5 & 3\\
%3 & 2\\
%\end{array}
%\right) &
%\cov_2 & = \left(
%\begin{array}{cc}
%2 & 0\\
%0 & 1\\
%\end{array}
%\right)
%  \end{align*}
%Classify the point $(3,4)^T$ via the (full) Bayesian approach, assuming
%normally distributed classes, and $P(c_1)=P(c_2)=0.5$. Show all steps.
%Recall that the inverse of a $2 \times 2$ matrix
%$ A = \left(\begin{array}{cc}
%a & b\\
%c & d\\
%\end{array} \right)$
%is given as
%$A^{-1} = {1 \over \det(A)} \left(
%\begin{array}{cc}
%d & -b\\
%-c & a\\
%\end{array}
%\right)$.
%
%\comment{
%{\bf Answer:} F{i}rst, compute
%$\cov_1^{-1} = \left(
%\begin{array}{cc}
%2 & -3\\
%-3 & 5\\
%\end{array}
%\right)$
%and $\cov_2^{-1} = \left(
%\begin{array}{cc}
%{1 \over 2} & 0\\
%0 & 1\\
%\end{array}
%\right)$.
%
%Now $\bx - \bmu_1 = (3,4)-(1,3) = (2,1)$ and $\bx - \bmu_2 =
%(3,4)-(5,5) = (-2,-1)$. Computing the mahalanobis distance we get for
%$c_1$,
%$(\bx-\bmu_i)^T\;\cov_i^{-1}\;(\bx-\bmu_i) =
%(2 \; 1) \left(
%\begin{array}{cc}
%2 & -3\\
%-3 & 5\\
%\end{array}
%\right)
%\left(
%\begin{array}{c}
%2 \\ 1
%\end{array}
%\right) = 1 $
%and for $c_2$ we get
%$(-2 \; -1) \left(
%\begin{array}{cc}
%1/2 & 0\\
%0 & 1\\
%\end{array}
%\right)
%\left(
%\begin{array}{c}
%-2 \\ -1
%\end{array}
%\right) = 3$.
%Also, ${1 \over \sqrt{det(\cov_1)}} = 1$, and ${1 \over \sqrt{det(\cov_2)}} = {1
%  \over \sqrt{2}} = 0.71$
%
%Thus $P(c_1|\bx) = \frac{e^{-1/2}}{e^{-1/2} + 0.71 e^{-3/2}} =
%{0.607 \over {0.607+0.223}} = {0.607 \over 0.83} = 0.73$, and
%$P(c_2|\bx) = 0.27$. We thus classify $\bx$ as $c_1$.
%}
%
%
%
%\end{exercises}
%
%\comment{
%%F12exam2_3
%
%\item \label{ex:class:prob:F12exam2_3}
%
%Consider the class means and covariance matrices for
%  class $c_1$ and $c_2$ given below:
%  \begin{align*}
%\bmu_1 & = (1,3) &
%\cov_1 & = \left(
%\begin{array}{cc}
%5 & 3\\
%3 & 2\\
%\end{array}
%\right) &
%\bmu_2 & = (5,5) &
%\cov_2 & = \left(
%\begin{array}{cc}
%2 & 0\\
%0 & 1\\
%\end{array}
%\right)
%  \end{align*}
%Classify the point $(3,4)$ via the (full) Bayesian approach, assuming
%normally distributed classes, and $P(c_1)=P(c_2)=0.5$. Show all steps
%for full credit. Recall that the inverse of a $2 \times 2$ matrix
%$ A = \left(
%\begin{array}{cc}
%a & b\\
%c & d\\
%\end{array}
%\right)$
%is given as
%$A^{-1} =  {1 \over det(A)} \left(
%\begin{array}{cc}
%d & -b\\
%-c & a\\
%\end{array}
%\right)$
%and the bivariate normal distribution is given as:
%$f(\bx\;|\;\bmu_i, \cov_i) =
%\frac{1}{2\pi\;\sqrt{det(\cov_i)}}
%  \;\;e^{-\frac{(\bx-\bmu_i)^T\;\cov_i^{-1}\;(\bx-\bmu_i)}{2}}
%$
%
%\comment{
%{\bf Answer:} F{i}rst, compute
%$\cov_1^{-1} = \left(
%\begin{array}{cc}
%2 & -3\\
%-3 & 5\\
%\end{array}
%\right)$
%and $\cov_2^{-1} = \left(
%\begin{array}{cc}
%{1 \over 2} & 0\\
%0 & 1\\
%\end{array}
%\right)$.
%
%Now $\bx - \bmu_1 = (3,4)-(1,3) = (2,1)$ and $\bx - \bmu_2 =
%(3,4)-(5,5) = (-2,-1)$. Computing the mahalanobis distance we get for
%$c_1$,
%$(\bx-\bmu_i)^T\;\cov_i^{-1}\;(\bx-\bmu_i) =
%(2 \; 1) \left(
%\begin{array}{cc}
%2 & -3\\
%-3 & 5\\
%\end{array}
%\right)
%\left(
%\begin{array}{c}
%2 \\ 1
%\end{array}
%\right) = 1 $
%and for $c_2$ we get
%$(-2 \; -1) \left(
%\begin{array}{cc}
%1/2 & 0\\
%0 & 1\\
%\end{array}
%\right)
%\left(
%\begin{array}{c}
%-2 \\ -1
%\end{array}
%\right) = 3$.
%Also, ${1 \over \sqrt{det(\cov_1)}} = 1$, and ${1 \over \sqrt{det(\cov_2)}} = {1
%  \over \sqrt{2}} = 0.71$
%
%Thus $P(\bx|c_1) = \frac{e^{-1/2}}{e^{-1/2} + 0.71 e^{-3/2}} =
%{0.607 \over {0.607+0.223}} = {0.607 \over 0.83} = 0.73$, and
%$P(\bx|c_2) = 0.27$. We thus classify $\bx$ as $c_1$.
%}
%}
