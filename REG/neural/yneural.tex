\lecture{neural}{neural}

\date{Chapter 25: Neural Networks}

\begin{frame}
\titlepage
\end{frame}

%\chapter{Neural Networks}
%\label{ch:reg:neural}
%\index{neural networks}
%\index{artificial neural networks|see{neural networks}}
%
\begin{frame}{Artificial Neural Networks}

{\em Artificial neural networks} or simply {\em neural networks} are
inspired by biological neuronal networks. A real biological neuron, or a
{\em nerve cell}, comprises dendrites, a cell body, and an axon that leads
to synaptic terminals. A neuron transmits information via electrochemical
signals. When there is enough concentration of ions at the dendrites of a
neuron it generates an electric pulse along its axon called an {\em action
potential}, which in turn activates the synaptic terminals, releasing more
ions and thus causing the information to flow to dendrites of other
neurons. 

\medskip

A human brain has on the order of 100 billion neurons, with each
neuron having between 1,000 to 10,000 connections to other neurons. %Thus,
%a human brain is a neuronal network with 100 trillion to a quadrillion
%($10^{15}$) interconnections! Interestingly, as far as we know, learning
%happens by adjusting the synaptic strengths, since synaptic signals can be
%either excitatory or inhibitory, making the post-synaptic neuron either more
%or less likely to generate an action potential, respectively.
%
%\index{neural networks!activation function}

\medskip

Artificial neural networks are comprised of abstract neurons that try
to mimic real neurons at a very high level. They can be described via a
weighted directed graph $G=(V,E)$, with each node $v_i \in V$ representing
a neuron, and each directed edge $(v_i, v_j) \in E$ representing a synaptic
to dendritic connection from $v_i$ to $v_j$. The weight of the edge $w_{ij}$
denotes the synaptic strength. %Neural networks are characterized by the type
%of activation function used to generate an output, and the architecture of the
%network in terms of how the nodes are interconnected. For example, whether
%the graph is a directed acyclic graph or has cycles, whether the graph is
%layered or not, and so on. It is important to note that a neural network is
%designed to represent and learn information by adjusting the synaptic weights.
\end{frame}
%
\begin{frame}{Artificial neuron: aggregation and activation.}
%\begin{figure}[t!]
%% \vspace*{0.2in}
\psset{tnpos=l,tnsep=2pt,colsep=3,rowsep=0.75,mcol=c,
    ArrowInside=->, arrowscale=2}
\centerline{
\psmatrix
& \Tcircle[name=b,doubleline=true]{$x_0$}~[tnpos=l]{$1$}~[tnpos=r]{$\;\;$}\\[-2em]
\Tcircle[name=x1]{$x_1$}\\[1em]
\Tcircle[name=x2]{$x_2$} & 
\Tcircle[name=zk]{$z_k$}~[tnpos=b]{{\footnotesize$\qquad\sum_{i=1}^d w_{ik}
\cdot x_i + b_k$}} 
 & [mnode=p,name=zo]\\[-2em]
[mnode=none]{$\vdots$}\\[-2em]
\Tcircle[name=xd]{$x_{d}$}
\psset{ArrowInsidePos=.85}
\ncline{x1}{zk} \ncput*{$w_{1k}$}
\ncline{x2}{zk} \ncput*{$w_{2k}$}
\ncline{xd}{zk} \ncput*{$w_{dk}$}
\psset{ArrowInsidePos=0.95}
\ncline{b}{zk}\ncput*{$b_k$}
\psset{linecolor=gray}
\psset{ArrowInsidePos=0.25}
\ncline{zk}{zo} \ncput*{$w_{k\cdot}$}
\endpsmatrix
}
\end{frame}
%\vspace{0.2in}
%\caption{Artificial neuron: aggregation and activation.}
%\label{fig:reg:neural:neuron}
%\end{figure}
%
%
%\section{Artificial Neuron: Activation Functions}
%\label{sec:reg:neural:activations}
%\index{neural networks!activation functions}
%
%% \index{neural networks!binary neuron}
\begin{frame}{Artificial Neuron}
An artificial neuron acts as a processing unit, that first aggregates the
incoming signals via a weighted sum, and then applies some function to
generate an output. For example, a binary neuron will output a $1$ whenever
the combined signal exceeds a threshold, or $0$ otherwise.
%
%\index{neural networks!bias neuron}
%\cref{fig:reg:neural:neuron} shows the schematic of a neuron $z_k$ that
%has incoming edges from neurons $x_1, \cdots, x_d$. For simplicity, both
%the name and the (output) value of a neuron are denoted by the same symbol. Thus,
%$x_i$ denotes neuron $i$, and also the value of that neuron. The net
%input at $z_k$, denoted $\net_k$, is given as the weighted sum
\begin{align}
    \tcbhighmath{
    \net_k = b_k + \sum^{d}_{i=1} w_{ik} \cdot x_i = b_k +
\bw^T\bx}
    % \label{eq:reg:neural:net}
%    = \bw_{\cdot k}^T\bx + \bw_{\bdot k}^T\bx + b_k
\end{align}
where $\bw_k = (w_{1k}, w_{2k}, \cdots, w_{dk})^T \in \setR^d$ and $\bx
= (x_1, x_2, \cdots, x_d)^T \in \setR^d$ is an input point. 
%%Thus, $z_k$
%%computes a weighted sum of its input neurons. 
Notice that $x_0$ is a
special {\em bias neuron} whose value is always fixed at $1$, and the
weight from $x_0$ to $z_k$ is $b_k$, which specifies the bias term
for the neuron. Finally, the output value of $z_k$ is given as some {\em
    activation function}, $f(\cdot)$, applied to the net input at $z_k$
\begin{align*}
    z_k = f(\net_k)
\end{align*}
\end{frame}
%The value $z_k$ is then passed along the outgoing edges from $z_k$ to
%other neurons.
%
%
%%\begin{figure}[t!]
%%\captionsetup[subfloat]{captionskip=20pt}
%%\def\pshlabel#1{{\large $#1$}}
%%\def\psvlabel#1{{\large $#1$}}
%%\centerline{%
%%    \subfloat[Identity (or Linear)]{%
%%        \label{fig:reg:neural:identity}
%%    \scalebox{0.55}{%
%%    \centering
%%    \psset{xAxisLabel={\Large$\net_k$},yAxisLabel={\Large$z_k$},%
%%    xAxisLabelPos={c,-1.2},yAxisLabelPos={-0.6,c} }
%%    \psset{labels=none,ticks=none}
%%    \psset{linewidth=1pt,axesstyle=frame,algebraic,plotstyle=curve}
%%    \psgraph[]{->}(0,-0.001)(7,7.0001){4in}{1.9in}%
%%    \psxTick(0){-\infty}
%%    \psxTick(3.5){0}
%%    \psxTick(7){+\infty}
%%    \psyTick(0){-\infty}
%%    \psyTick(3.5){0}
%%    \psyTick(7){+\infty}
%%    \psline[linewidth=2pt,linecolor=black](0,0)(7,7)
%%    \psline[linewidth=0.5pt,linecolor=gray](3.5,0)(3.5,7)
%%    \psline[linewidth=0.5pt,linecolor=gray](0,3.5)(7,3.5)
%%    \endpsgraph
%%    }}
%%    \hspace{0.5in}
%%    \subfloat[Step]{
%%        \label{fig:reg:neural:step}
%%    \scalebox{0.55}{%
%%    \centering
%%    \psset{xAxisLabel={\Large$\net_k$},yAxisLabel={\Large$z_k$},%
%%    xAxisLabelPos={c,-0.2},yAxisLabelPos={-1.8,c} }
%%    \psset{labels=y,ticks=y}
%%    \psset{linewidth=1pt,axesstyle=frame,algebraic,plotstyle=curve}
%%    \psgraph[Dy=0.5,Ox=-7]{->}(-7,-0.001)(7,1.0001){4in}{1.9in}%
%%    \psxTick(-7){-\infty}
%%    \psxTick(0){0}
%%    \psxTick(7){+\infty}
%%    \psline[linewidth=2pt,linecolor=black](-7,0)(0,0)
%%    \psline[linewidth=2pt,linecolor=black](0,0)(0,1)
%%    \psline[linewidth=2pt,linecolor=black](0,1)(7,1)
%%    \psline[linewidth=0.5pt,linecolor=gray](-7,0.5)(7,0.5)
%%    \endpsgraph
%%    }}
%%}
%%\vspace{0.1in}
%%\centerline{
%%    \subfloat[Rectified Linear Unit]{
%%        \label{fig:reg:neural:rlu}
%%        \def\relu{(x)}
%%    \scalebox{0.55}{%
%%    \centering
%%    \psset{xAxisLabel={\Large$\net_k$},yAxisLabel={\Large$z_k$},%
%%    xAxisLabelPos={c,-0.2},yAxisLabelPos={-1.5,c} }
%%    \psset{labels=none,ticks=none}
%%    \psset{linewidth=1pt,axesstyle=frame,algebraic,plotstyle=curve}
%%    \psgraph[Oy=0,Ox=-7]{->}(-7,0)(7,1.0001){4in}{1.9in}%
%%    \psxTick(-7){-\infty}
%%    \psxTick(0){0}
%%    \psxTick(7){+\infty}
%%    \uput[180](-7.1,1){\large$+\infty$}
%%    \uput[180](-7.1,0){\large$0$}
%%    \psline[linewidth=2pt,linecolor=black](-7,0)(0,0)
%%    \psline[linewidth=2pt,linecolor=black](0,0)(7,1)
%%    \psline[linewidth=0.5pt,linecolor=gray](0,0)(0,1)
%%    \endpsgraph
%%    }}
%%    \hspace{0.5in}
%%    \subfloat[Sigmoid]{
%%        \label{fig:reg:neural:sigmoid}
%%    \def\logit{1/(1 + EXP(-x))}
%%    \scalebox{0.55}{%
%%    \centering
%%    \psset{xAxisLabel={\Large$\net_k$},yAxisLabel={\Large$z_k$},%
%%    xAxisLabelPos={c,-0.2},yAxisLabelPos={-1.8,c} }
%%    \psset{labels=y,ticks=y}
%%    \psset{linewidth=1pt,axesstyle=frame,algebraic,plotstyle=curve}
%%    \psgraph[Dy=0.5,Ox=-7]{->}(-7,0)(7,1.0001){4in}{1.9in}%
%%    \psxTick(-7){-\infty}
%%    \psxTick(0){0}
%%    \psxTick(7){+\infty}
%%    \psline[linewidth=0.5pt,linecolor=gray](0,0)(0,1.0)
%%    \psline[linewidth=0.5pt,linecolor=gray](-7,0.5)(7,0.5)
%%       \psplot[linewidth=2pt]{-7}{7}{\logit}
%%    \endpsgraph
%%    }}
%%}
%%\vspace{0.1in}
%%\centerline{
%%    \subfloat[Hyperbolic Tangent]{
%%        \label{fig:reg:neural:tanh}
%%        \def\tanhF{(1 - EXP(-2*x))/(1 + EXP(-2*x))}
%%    \scalebox{0.55}{%
%%    \centering
%%    \psset{xAxisLabel={\Large$\net_k$},yAxisLabel={\Large$z_k$},%
%%    xAxisLabelPos={c,-0.3},yAxisLabelPos={-1.5,c} }
%%    \psset{labels=y,ticks=y}
%%    \psset{linewidth=1pt,axesstyle=frame,algebraic,plotstyle=curve}
%%    \psgraph[Dy=1,Oy=-1,Ox=-7]{->}(-7,-1.0001)(7,1.0001){4in}{1.9in}%
%%    %\psxTick(-7){-\infty}
%%    %\psxTick(0){0}
%%    %\psxTick(7){+\infty}
%%    \uput[90](-7,-1.2){\large$-\infty$}
%%    \uput[90](0,-1.2){\large$0$}
%%    \uput[90](7,-1.2){\large$+\infty$}
%%    \psline[linewidth=0.5pt,linecolor=gray](0,-1)(0,1.0)
%%    \psline[linewidth=0.5pt,linecolor=gray](-7,0)(7,0)
%%       \psplot[linewidth=2pt]{-7}{7}{\tanhF}
%%    \endpsgraph
%%    }}
%%    \hspace{0.5in}
%%    \subfloat[Softmax ($\net_k$ versus $\net_{\!j}$)]{
%%\label{fig:reg:neural:softmax}
%%\psset{unit=0.15in}
%%\psset{lightsrc=30 30 10 rtp2xyz}
%%\psset{incolor=gray}
%%\psset{opacity=0.2}
%%\psset{viewpoint=30 220 10 rtp2xyz,Decran=60}
%%\psset{axisnames={{\quad\net_k}, {\net_{\!j}\quad}, z_k}}
%%\scalebox{0.7}{
%%\begin{pspicture}(-8,-2)(12,6)
%%    \axesIIID[](-4,-2,0)(4,4.1,2.75)
%%\psset{linewidth=0.5pt, dotsize=0.3}
%%\psset{showAxes=false, fillcolor=gray}
%%\psset{transform={1.0 1.0 2.5 scaleOpoint3d}}
%%\psSurface[ngrid=.3 .3,fillcolor=white,%axesboxed,
%%    linewidth=0.5\pslinewidth,
%%       color1 = {[rgb]{0 0 0}},
%%        color2 = {[rgb]{1 1 1}},
%%        hue=(color1) (color2), lightintensity=5,
%%    algebraic,Zmin=0,Zmax=1](-5,-4)(5,4){%
%%    (EXP(x))/(EXP(x) + EXP(y))}
%%\end{pspicture}
%%}}
%%}
%%\caption{Neuron activation functions.}
%%\label{fig:reg:neural:activations}
%%\end{figure}
%
%
%
%\begin{figure}[!htbp]
%\captionsetup[subfloat]{captionskip=20pt}
%\def\pshlabel#1{{\large $#1$}}
%\def\psvlabel#1{{\large $#1$}}
%\hspace{0.2in}
%\centerline{%
%    \subfloat[Linear ($z_k$ versus $\net_k$)]{%
%        \label{fig:reg:neural:identity}
%    \scalebox{0.55}{%
%    \centering
%    \psset{xAxisLabel={\Large$\net_k$},yAxisLabel={\Large$z_k$},%
%    xAxisLabelPos={c,-1.4},yAxisLabelPos={-0.6,c} }
%    \psset{labels=none,ticks=none}
%    \psset{linewidth=1pt,axesstyle=frame,algebraic,plotstyle=curve}
%    \psgraph[]{->}(0,-0.001)(7,7.0001){4in}{1.9in}%
%    \psxTick(0){-\infty}
%    \psxTick(3.5){0}
%    \psxTick(7){+\infty}
%    \psyTick(0){-\infty}
%    \psyTick(3.5){0}
%    \psyTick(7){+\infty}
%    \psline[linewidth=2pt,linecolor=black](0,0)(7,7)
%    \psline[linewidth=0.5pt,linecolor=gray](3.5,0)(3.5,7)
%    \psline[linewidth=0.5pt,linecolor=gray](0,3.5)(7,3.5)
%    \endpsgraph
%    }}
%    \hspace{0.5in}
%    \subfloat[Linear ($z_k$ versus $\bw^T\bx$)]{
%        \label{fig:reg:neural:identityB}
\begin{frame}{Linear Function}

Function:
$
      f(\net_k) = \net_k
$

Derivative:
$
    \frac{\partial f(\net_{\!j})}{\partial \net_{\!j}} = 1
$

\bigskip

\centerline{
    \scalebox{0.85}{
    \centering
    \psset{xAxisLabel={\Large$\bw^T\bx$},yAxisLabel={\Large$z_k$},%
    xAxisLabelPos={c,-1.4},yAxisLabelPos={-0.6,c} }
    \psset{labels=none,ticks=none}
    \psset{linewidth=1pt,axesstyle=frame,algebraic,plotstyle=curve}
    \psgraph[Dy=0.5,Ox=-7]{->}(0,-0.001)(7,7.0001){4in}{1.9in}%
    \psxTick(0){-\infty}
    \psxTick(3.5){0}
    \psxTick(2.5){{\Large-b_k}}
    \psxTick(7){+\infty}
    \psyTick(0){-\infty}
    \psyTick(3.5){0}
    \psyTick(7){+\infty}
    \psline[linewidth=2pt,linecolor=black](0,1)(6,7)
    \psline[linewidth=0.5pt,linecolor=gray](0,3.5)(7,3.5)
    \psline[linewidth=0.5pt,linecolor=gray,linestyle=dashed](3.5,0)(3.5,7)
    \psline[linewidth=0.5pt,linecolor=gray](2.5,0)(2.5,7)
    \endpsgraph
    }%}
}
\end{frame}
%\vspace{0.1in}
%\hspace{0.2in}
%\centerline{
%    \subfloat[Step ($z_k$ versus $\$
%        \label{fig:reg:neural:step}
%    \scalebox{0.55}{%
%    \centering
%    \psset{xAxisLabel={\Large$\net_k$},yAxisLabel={\Large$z_k$},%
%    xAxisLabelPos={c,-0.2},yAxisLabelPos={-1.8,c} }
%    \psset{labels=y,ticks=y}
%    \psset{linewidth=1pt,axesstyle=frame,algebraic,plotstyle=curve}
%    \psgraph[Dy=0.5,Ox=-7]{->}(-7,-0.001)(7,1.0001){4in}{1.9in}%
%    \psxTick(-7){-\infty}
%    \psxTick(0){0}
%    \psxTick(7){+\infty}
%    \psline[linewidth=2pt,linecolor=black](-7,0)(0,0)
%    \psline[linewidth=2pt,linecolor=black](0,0)(0,1)
%    \psline[linewidth=2pt,linecolor=black](0,1)(7,1)
%    \psline[linewidth=0.5pt,linecolor=gray](-7,0.5)(7,0.5)
%    \endpsgraph
%    }}
%    \hspace{0.5in}
%    \subfloat[Step ($z_k$ versus $\bw^T\bx$)]{
%        \label{fig:reg:neural:stepB}
\begin{frame}{Step Function}

Function:
$
            f(\net_k) = 
            \begin{cases}
                0 & \text{if } \net_k \le 0\\
                1 & \text{if } \net_k > 0
            \end{cases}
$

Derivative:
$
    \frac{\partial f(\net_{\!j})}{\partial \net_{\!j}} = 0
$

\bigskip

\centerline{
    \scalebox{0.85}{%
    \centering
    \psset{xAxisLabel={\Large$\bw^T\bx$},yAxisLabel={\Large$z_k$},%
    xAxisLabelPos={c,-0.2},yAxisLabelPos={-1.8,c} }
    \psset{labels=y,ticks=y}
    \psset{linewidth=1pt,axesstyle=frame,algebraic,plotstyle=curve}
    \psgraph[Dy=0.5,Ox=-7]{->}(-7,-0.001)(7,1.0001){4in}{1.9in}%
    \psxTick(-7){-\infty}
    \psxTick(0){0}
    \psxTick(-2){{\Large-b_k}}
    \psxTick(7){+\infty}
    \psline[linewidth=2pt,linecolor=black](-7,0)(-2,0)
    \psline[linewidth=2pt,linecolor=black](-2,0)(-2,1)
    \psline[linewidth=2pt,linecolor=black](-2,1)(7,1)
    \psline[linewidth=0.5pt,linecolor=gray,linestyle=dashed](0,0)(0,1)
    \psline[linewidth=0.5pt,linecolor=gray](-7,0.5)(7,0.5)
    \endpsgraph
    %}}
}
}
\end{frame}
%\vspace{0.1in}
%\hspace{0.2in}
%\centerline{
%    \subfloat[Rectified Linear ($z_k$ versus $\net_k$)]{
%        \label{fig:reg:neural:rlu}
%        \def\relu{(x)}
%    \scalebox{0.55}{%
%    \centering
%    \psset{xAxisLabel={\Large$\net_k$},yAxisLabel={\Large$z_k$},%
%    xAxisLabelPos={c,-0.2},yAxisLabelPos={-1.5,c} }
%    \psset{labels=none,ticks=none}
%    \psset{linewidth=1pt,axesstyle=frame,algebraic,plotstyle=curve}
%    \psgraph[Oy=0,Ox=-7]{->}(-7,0)(7,1.0001){4in}{1.9in}%
%    \psxTick(-7){-\infty}
%    \psxTick(0){0}
%    \psxTick(7){+\infty}
%    \uput[180](-7.1,1){\large$+\infty$}
%    \uput[180](-7.1,0){\large$0$}
%    \psline[linewidth=2pt,linecolor=black](-7,0)(0,0)
%    \psline[linewidth=2pt,linecolor=black](0,0)(7,1)
%    \psline[linewidth=0.5pt,linecolor=gray](0,0)(0,1)
%    \endpsgraph
%    }}
%    \hspace{0.5in}
%    \subfloat[Rectified Linear ($z_k$ versus $\bw^T\bx$)]{
%        \label{fig:reg:neural:rluB}

\begin{frame}{Rectified Linear Function}

Function:
$
            f(\net_k) = 
            \begin{cases}
                0 & \text{if } \net_k \le 0\\
                \net_k & \text{if } \net_k > 0
            \end{cases}
$

Derivative:
$
    \frac{\partial f(\net_{\!j})}{\partial \net_{\!j}} = 
    \begin{cases}
        0 & \text{if } \net_{\!j} \le 0\\
        1 & \text{if } \net_{\!j} > 0\\
    \end{cases}
$

\bigskip

\centerline{
    \scalebox{0.85}{%
    \centering
    \psset{xAxisLabel={\Large$\bw^T\bx$},yAxisLabel={\Large$z_k$},%
    xAxisLabelPos={c,-0.2},yAxisLabelPos={-1.5,c} }
    \psset{labels=none,ticks=none}
    \psset{linewidth=1pt,axesstyle=frame,algebraic,plotstyle=curve}
    \psgraph[Dy=0.5,Ox=-7]{->}(-7,-0.001)(7,1.0001){4in}{1.9in}%
    \psxTick(-7){-\infty}
    \psxTick(0){0}
    \psxTick(-2){{\Large-b_k}}
    \psxTick(7){+\infty}
    \uput[180](-7.1,1){\large$+\infty$}
    \uput[180](-7.1,0){\large$0$}
    %\psyTick(1){+\infty}
    %\psyTick(0){0}
    \psline[linewidth=2pt,linecolor=black](-7,0)(-2,0)
    \psline[linewidth=2pt,linecolor=black](-2,0)(7,1)
    \psline[linewidth=0.5pt,linecolor=gray,linestyle=dashed](0,0)(0,1)
    \psline[linewidth=0.5pt,linecolor=gray](-2,0)(-2,1)
    \endpsgraph
    }}
%}
\end{frame}
%\vspace{0.1in}
%\hspace{0.2in}
%\centerline{
%    \subfloat[Sigmoid ($z_k$ versus $\net_k$)]{
%        \label{fig:reg:neural:sigmoid}
%    \def\logit{1/(1 + EXP(-x))}
%    \scalebox{0.55}{%
%    \centering
%    \psset{xAxisLabel={\Large$\net_k$},yAxisLabel={\Large$z_k$},%
%    xAxisLabelPos={c,-0.2},yAxisLabelPos={-1.8,c} }
%    \psset{labels=y,ticks=y}
%    \psset{linewidth=1pt,axesstyle=frame,algebraic,plotstyle=curve}
%    \psgraph[Dy=0.5,Ox=-7]{->}(-7,0)(7,1.0001){4in}{1.9in}%
%    \psxTick(-7){-\infty}
%    \psxTick(0){0}
%    \psxTick(7){+\infty}
%    \psline[linewidth=0.5pt,linecolor=gray](0,0)(0,1.0)
%    \psline[linewidth=0.5pt,linecolor=gray](-7,0.5)(7,0.5)
%       \psplot[linewidth=2pt]{-7}{7}{\logit}
%    \endpsgraph
%    }}
%    \hspace{0.5in}
%    \subfloat[Sigmoid ($z_k$ versus $\bw^T\bx$)]{
%        \label{fig:reg:neural:sigmoidB}
\begin{frame}{Sigmoid Function}

Function:
$
            f(\net_k) = \frac{1}{1 + \exp\{-\net_k\}}
$

Derivative:
$
    \frac{\partial f(\net_{\!j})}{\partial \net_{\!j}} 
    = f(\net_{\!j}) \cdot (1-f(\net_{\!j}))
$

\bigskip

\centerline{
    \def\logitB{1/(1 + EXP(-x-2))}
    \scalebox{0.85}{%
    \centering
    \psset{xAxisLabel={\Large$\bw^T\bx$},yAxisLabel={\Large$z_k$},%
    xAxisLabelPos={c,-0.2},yAxisLabelPos={-1.8,c} }
    \psset{labels=y,ticks=y}
    \psset{linewidth=1pt,axesstyle=frame,algebraic,plotstyle=curve}
    \psgraph[Dy=0.5,Ox=-7]{->}(-7,0)(7,1.0001){4in}{1.9in}%
    \psxTick(-7){-\infty}
    \psxTick(0){0}
    \psxTick(-2){{\Large-b_k}}
    \psxTick(7){+\infty}
    \psline[linewidth=0.5pt,linecolor=gray](-2,0)(-2,1.0)
    \psline[linewidth=0.5pt,linecolor=gray,linestyle=dashed](0,0)(0,1.0)
    \psline[linewidth=0.5pt,linecolor=gray](-7,0.5)(7,0.5)
       \psplot[linewidth=2pt]{-7}{7}{\logitB}
    \endpsgraph
    }}
%}
\end{frame}
%\vspace{0.1in}
%\hspace{0.2in}
%\centerline{
%    \subfloat[Hyperbolic Tangent ($z_k$ versus $\net_k$)]{
%        \label{fig:reg:neural:tanh}
%        \def\tanh{(1 - EXP(-2*x))/(1 + EXP(-2*x))}
%    \scalebox{0.55}{%
%    \centering
%    \psset{xAxisLabel={\Large$\net_k$},yAxisLabel={\Large$z_k$},%
%    xAxisLabelPos={c,-0.4},yAxisLabelPos={-1.5,c} }
%    \psset{labels=y,ticks=y}
%    \psset{linewidth=1pt,axesstyle=frame,algebraic,plotstyle=curve}
%    \psgraph[Dy=1,Oy=-1,Ox=-7]{->}(-7,-1.0001)(7,1.0001){4in}{1.9in}%
%    %\psxTick(-7){-\infty}
%    %\psxTick(0){0}
%    %\psxTick(7){+\infty}
%    \uput[90](-7,-1.3){\large$-\infty$}
%    \uput[90](0,-1.3){\large$0$}
%    \uput[90](7,-1.3){\large$+\infty$}
%    \psline[linewidth=0.5pt,linecolor=gray](0,-1)(0,1.0)
%    \psline[linewidth=0.5pt,linecolor=gray](-7,0)(7,0)
%       \psplot[linewidth=2pt]{-7}{7}{\tanh}
%    \endpsgraph
%    }}
%    \hspace{0.5in}
%    \subfloat[Hyperbolic Tangent ($z_k$ versus $\bw^T\bx$)]{
%        \label{fig:reg:neural:tanhB}
\begin{frame}{Hyperbolic Tangent Function}

Function:
$
            f(\net_k) = \frac{\exp\{\net_k\} - \exp\{-\net_k\}}
                     {\exp\{\net_k\} + \exp\{-\net_k\}} 
            = \frac{\exp\{2\cdot\net_k\}-1}{\exp\{2\cdot\net_k\}+1}
$

Derivative:
$
    \frac{\partial f(\net_{\!j})}{\partial \net_{\!j}} = 1-f(\net_{\!j})^2
$

\bigskip

\centerline{
        \def\tanhB{(1 - EXP(-2*x-4))/(1 + EXP(-2*x-4))}
    \scalebox{0.85}{%
    \centering
    \psset{xAxisLabel={\Large$\bw^T\bx$},yAxisLabel={\Large$z_k$},%
    xAxisLabelPos={c,-0.4},yAxisLabelPos={-1.5,c} }
    \psset{labels=y,ticks=y}
    \psset{linewidth=1pt,axesstyle=frame,algebraic,plotstyle=curve}
    \psgraph[Dy=1,Oy=-1,Ox=-7]{->}(-7,-1.001)(7,1.0001){4in}{1.9in}%
    \uput[90](-7,-1.3){\large$-\infty$}
    \uput[90](0,-1.3){\large$0$}
    \uput[90](-2,-1.3){\Large$-b_k$}
    \uput[90](7,-1.3){\large$+\infty$}
    %\psxTick(-7){-\infty}
    %\psxTick(0){0}
    %\psxTick(-2){b_k}
   % \psxTick(7){+\infty}
    \psline[linewidth=0.5pt,linecolor=gray](-2,-1)(-2,1.0)
    \psline[linewidth=0.5pt,linecolor=gray,linestyle=dashed](0,-1)(0,1)
    \psline[linewidth=0.5pt,linecolor=gray](-7,0)(7,0)
       \psplot[linewidth=2pt]{-7}{7}{\tanhB}
    \endpsgraph
    }}
\end{frame}
%}
%\vspace{0.1in}
%\caption{Neuron activation functions; also illustrating the effect of bias.}
%\label{fig:reg:neural:activations}
%\end{figure}
%
%\begin{figure}[!b]
%    \centering
\begin{frame}{Softmax Function}

Function:
$
           f(\net_k |\; \bnet) = \frac{\exp\{\net_k\}}
           {\sum_{i=1}^p \exp\{ \net_i \} }
$

Derivative:
$
    \frac{\partial f(\net_{\!j}|\; \bnet)}{\partial \net_{\!j}} 
    = f(\net_{\!j}) \cdot (1-f(\net_{\!j}))
$

\begin{center}
\psset{unit=0.15in}
\psset{lightsrc=30 30 10 rtp2xyz}
\psset{incolor=gray}
\psset{opacity=0.2}
\psset{viewpoint=30 220 10 rtp2xyz,Decran=60}
\psset{axisnames={{\quad\net_k}, {\net_{\!j}\quad}, z_k}}
\scalebox{1}{
\begin{pspicture}(-8,-2)(12,6)
    \axesIIID[](-4,-2,0)(4,4.1,2.75)
\psset{linewidth=0.5pt, dotsize=0.3}
\psset{showAxes=false, fillcolor=gray}
\psset{transform={1.0 1.0 2.5 scaleOpoint3d}}
\psSurface[ngrid=.3 .3,fillcolor=white,%axesboxed,
    linewidth=0.5\pslinewidth,
       color1 = {[rgb]{0 0 0}},
        color2 = {[rgb]{1 1 1}},
        hue=(color1) (color2), lightintensity=5,
    algebraic,Zmin=0,Zmax=1](-5,-4)(5,4){%
    (EXP(x))/(EXP(x) + EXP(y))}
\end{pspicture}
}
\end{center}
\end{frame}
%\vspace{0.2in}
%\caption{Softmax ($\net_k$ versus $\net_{\!j}$).}
%\label{fig:reg:neural:softmax}
%\end{figure}
%
%% \begin{figure}
%% \centerline{
%%     %\def\softmax{EXP(x)/(EXP(x) + EXP(y))}
%%     %\def\softmax{1/(1 + EXP(7-x))}
%%     %\def\softmaxb{1/(1 + EXP(5-x) + EXP(-2-x))}
%%     \scalebox{0.55}{%
%%     \centering
%%     \psset{xAxisLabel={\Large$\net_k$},yAxisLabel={\Large$z_k$},%
%%     xAxisLabelPos={c,-0.3},yAxisLabelPos={-1.5,c} }
%%     \psset{labels=y,ticks=y}
%%     \psset{linewidth=1pt,axesstyle=frame,algebraic,plotstyle=curve}
%%     \psgraph[Dy=0.25,Oy=0,Ox=0]{->}(-7,0)(14,1.0001){4in}{2.5in}%
%%        \psplot[linewidth=2pt]{-7}{14}{\softmax}
%%     \endpsgraph
%%     }}
%% \end{figure}
%
%Neurons differ based on the type of activation function used.
%Some commonly used activation functions, illustrated in
%\cref{fig:reg:neural:activations}, are:
%\begin{description}
%\index{neural networks!linear function}
%\index{neural networks!identity function}
%% \index{identity activation function|see{linear activation function}}
%    \item[{\bf Linear/Identity Function}:] Identity is the simplest activation
%        function; it simply returns its argument, and is also 
%        called a
%        {\em linear} activation function:
%        \begin{align}
%            f(\net_k) = \net_k
%            \label{eq:reg:neural:identity}
%        \end{align}
%        \cref{fig:reg:neural:identity} plots the identity function.
%        To examine the role of the bias term, note that
%        $\net_k > 0$ is equivalent to 
%        $\bw^T\bx > -b_k$. That is, the output
%        transitions from negative to positive when the weighted sum of the
%        inputs exceeds $-b_k$, as shown in
%        \cref{fig:reg:neural:identityB}. 
%
%       % The
%       % bias term $b_k$ simply shifts the function to the left as
%       % seen in \cref{fig:reg:neural:identityB}.
%
%\index{step function}
%\index{neural networks!step function}
%    \item[{\bf Step Function}:] This is a binary activation function,
%        where the neuron outputs a $0$ if the net value is negative (or
%        zero),
%        and $1$ if the net value is positive (see
%        \cref{fig:reg:neural:step}).
%        \begin{align}
%            f(\net_k) = 
%            \begin{cases}
%                0 & \text{if } \net_k \le 0\\
%                1 & \text{if } \net_k > 0
%            \end{cases}
%            \label{eq:reg:neural:step}
%        \end{align}
%        % Since $\net_k = b_k + \sum^{d}_{i=1} w_{ik} \cdot x_i$, we
%        % can see that $\net_k > 0$ is equivalent to 
%        % $\sum^{d}_{i=1} w_{ik} \cdot x_i > -b_k$. That is, 
%        It is interesting to note that the
%        transition from $0$ to $1$ happens when the weighted sum of the
%        inputs exceeds $-b_k$, as shown in
%        \cref{fig:reg:neural:stepB}. 
%
%\index{ReLU function}
%\index{neural networks!ReLU function}
%\index{rectified linear unit|see{ReLU function}}
%    \item[{\bf Rectified Linear Unit (ReLU)}:] Here, the neuron 
%        remains inactive if the net input is less than or equal to zero, and then 
%        increases linearly with $\net_k$, as shown
%        in \cref{fig:reg:neural:rlu}.
%        \begin{align}
%            f(\net_k) = 
%            \begin{cases}
%                0 & \text{if } \net_k \le 0\\
%                \net_k & \text{if } \net_k > 0
%            \end{cases}
%            \label{eq:reg:neural:relu}
%        \end{align}
%        An alternative expression for the ReLU activation is
%        given as $f(\net_k) = \max\{0, \net_k\}$. The transition from
%        zero to linear output happens when the weighted sum of the
%        inputs exceeds $-b_k$ (see \cref{fig:reg:neural:rluB}).
%        
%
%\index{sigmoid function}
%\index{neural networks!sigmoid function}
%    \item[{\bf Sigmoid}:] The sigmoid function, illustrated in
%        \cref{fig:reg:neural:sigmoid} squashes its input so
%        that the output ranges between $0$ and $1$
%        \begin{align}
%            f(\net_k) = \frac{1}{1 + \exp\{-\net_k\}}
%            \label{eq:reg:neural:sigmoid}
%        \end{align}
%       % Note the difference between the sigmoid neuron without and with
%       %  the bias term as shown in
%       %  \cref{fig:reg:neural:sigmoid,fig:reg:neural:sigmoidB}.
%        When $\net_k = 0$, we have $f(\net_k) = 0.5$, which implies that 
%        the transition point where the output crosses $0.5$ happens when 
%        the weighted sum of the inputs
%        exceeds $-b_k$ (see \cref{fig:reg:neural:sigmoidB}).
%
%\index{tanh function}
%\index{neural networks!tanh function}
%\index{hyperbolic tangent|see{tanh function}}
%    \item[{\bf Hyperbolic Tangent (tanh)}:] The hyperbolic tangent or
%        tanh
%        function is similar to the sigmoid, but its output ranges between
%        $-1$ and $+1$ (see \cref{fig:reg:neural:tanh}).
%        \begin{align}
%            f(\net_k) = \frac{\exp\{\net_k\} - \exp\{-\net_k\}}
%                     {\exp\{\net_k\} + \exp\{-\net_k\}} 
%            = \frac{\exp\{2\cdot\net_k\}-1}{\exp\{2\cdot\net_k\}+1}
%            \label{eq:reg:neural:tanh}
%        \end{align}
%        When $\net_k = 0$, we have $f(\net_k) = 0$, 
%        which implies that 
%        the output transitions from negative to positive when 
%        the weighted sum of the inputs
%        exceeds $-b_k$, as shown in \cref{fig:reg:neural:tanhB}.
%
%\index{softmax function}
%\index{neural networks!softmax function}
%    \item[{\bf Softmax}:] Softmax is a generalization of the sigmoid or
%        logistic activation function. 
%       Softmax is mainly used at the output layer in a neural
%       network, and unlike the other functions it depends not only on
%       the net input at neuron $k$, but it depends on the net signal at
%       all other neurons in the output layer. Thus, given the net input
%       vector, $\bnet = 
%       (\net_1,
%       \net_2, \cdots, \net_p)^T$, for all the $p$ output neurons, the
%       output of the softmax function for the $k$th neuron is given as
%       \begin{align}
%           f(\net_k |\; \bnet) = \frac{\exp\{\net_k\}}
%           {\sum_{i=1}^p \exp\{ \net_i \} }
%            \label{eq:reg:neural:softmax}
%       \end{align}
%       \cref{fig:reg:neural:softmax} plots the softmax activation for 
%       $\net_k$ versus the net signal $\net_{\!j}$, 
%       with all other net values fixed at zero. The
%       output behaves similar to a sigmoid curve for any given value of
%       $\net_{\!j}$.
%       % \begin{align*}
%       %     f(\net_k |\; \net_{\!j}) & = \frac{\exp\{\net_k\}}
%       %     {\sum_{i=1}^p \exp\{ \net_i \} } = 
%       %  \frac{\exp\{\net_k\}}
%       %  {\exp\{ \net_k \} + \exp\{ \net_{\!j}\} }
%       %  = \frac{1}{1 + \exp\{-(\net_k - \net_{\!j})\}}
%       % \end{align*}
%\end{description}
%
%\subsection{Derivatives for Activation Functions}
%\label{sec:reg:neural:deriv_activations}
%For learning using a neural network, we need
%to consider the derivative of an activation function with respect
%to its argument. The derivatives for the activation functions are
%given as follows:
%\begin{description}
%\index{linear function!derivative}
%    \item [\bf Identity/Linear:] The identity (or linear) activation function
%        has a derivative of $1$ with respect to its argument, giving us:
%\begin{align}
%    \frac{\partial f(\net_{\!j})}{\partial \net_{\!j}} = 1
%    \label{eq:reg:neural:deriv_identity}
%\end{align}
%
%\index{step function!derivative}
%\item [\bf Step:] The step function has a derivative of $0$ everywhere
%    except for the discontinuity at $0$, where the derivative is
%    $\infty$. 
%    % Since a step function is zero everywhere (except at $0$) 
%    % it is never
%    % really used with backpropagation. 
%
%% A simple approximation to the step
%    % function can be obtained via a sigmoid activation with an additional
%    % steepness parameter $\beta \ge 1$, defined as
%    % \begin{align*}
%    %     f(\net_{\!j}) = \frac{1}{1 + \exp\{ -\beta \cdot \net_{\!j} \}}
%    % \end{align*}
%    % The larger $\beta$ is, the better the approximation to the step
%    % function.
%
%\index{ReLU function!derivative}
%\item[\bf ReLU:]
%    The ReLU function [\cref{eq:reg:neural:relu}] is non-differentiable at 0, nevertheless for other
%    values its derivative is $0$ if $net_{\!j} < 0$ and $1$ if
%    $net_{\!j} > 0$. 
%    At $0$, we can set the derivative to be any value
%    in the range $[0,1]$, a simple choice being 0. Putting it all
%    together, we have
%\begin{align}
%    \frac{\partial f(\net_{\!j})}{\partial \net_{\!j}} = 
%    \begin{cases}
%        0 & \text{if } \net_{\!j} \le 0\\
%        1 & \text{if } \net_{\!j} > 0\\
%    \end{cases}
%    \label{eq:reg:neural:deriv_relu}
%\end{align}
%    Even though ReLU has a discontinuity at $0$ it is a popular choice
%    for training deep neural networks.
%
%\index{sigmoid function!derivative}
%\item[\bf Sigmoid:]
%    The derivative of the sigmoid function
%    [\cref{eq:reg:neural:sigmoid}] 
%    is given as
%\begin{align}
%    \frac{\partial f(\net_{\!j})}{\partial \net_{\!j}} 
%    = f(\net_{\!j}) \cdot (1-f(\net_{\!j}))
%    \label{eq:reg:neural:deriv_sigmoid}
%\end{align}
%
%\index{tanh function!derivative}
%\item[\bf Hyperbolic Tangent:]
%    The derivative of the tanh function [\cref{eq:reg:neural:tanh}] is given as
%\begin{align}
%    \frac{\partial f(\net_{\!j})}{\partial \net_{\!j}} = 1-f(\net_{\!j})^2
%    \label{eq:reg:neural:deriv_tanh}
%\end{align}
%
%\index{softmax function!derivative}
%\item[\bf Softmax:]
%    The softmax activation function [\cref{eq:reg:neural:softmax}] is
%     a vector valued function, which maps a vector input 
%    $\bnet = (\net_1, \net_2, \cdots, \net_{p})^T$ to a vector of
%    probability values. Softmax is typically used only for the output
%    layer.
%    The partial derivative of $f(\net_{\!j})$ with respect to
%        $\net_{\!j}$ is given as
%\begin{align*}
%    \frac{\partial f(\net_{\!j}|\; \bnet)}{\partial \net_{\!j}} 
%    = f(\net_{\!j}) \cdot (1-f(\net_{\!j}))
%\end{align*}
%whereas the partial derivative of $f(\net_{\!j})$ with respect to
%        $\net_{k}$, with $k\ne j$ is given as
%\begin{align*}
%    \frac{\partial f(\net_{\!j}|\; \bnet)}{\partial \net_{k}} 
%    = -f(\net_k) \cdot f(\net_{\!j})
%\end{align*}
%Since softmax is used at the output layer, if we denote the $i$th output
%neuron as $o_i$, then
%$f(\net_i) = o_i$, and we can write the derivative as:
%\begin{align}
%    \frac{\partial f(\net_{\!j}|\; \bnet)}{\partial \net_{k}} =
%    \frac{\partial o_{\!j}}{\partial \net_{k}} =
%    \begin{cases}
%        o_{\!j} \cdot (1-o_{\!j}) & \text{if } k=j\\
%        -o_{\!k} \cdot o_{\!j} & \text{if } k \ne j\\
%    \end{cases}
%    \label{eq:reg:neural:softmax_der}
%\end{align}
%\end{description}
%
%
%
%\section{Neural Networks: Regression and Classification}
%Networks of (artificial) neurons are capable of representing and 
%learning arbitrarily
%complex functions for both regression and classification tasks. 
%
%\subsection{Regression}
%\index{neural networks!regression}
%Consider the multiple (linear) regression problem, where given an input
%$\bx_i \in \setR^d$, the goal is to predict the response as follows
%\begin{align*}
%    \hy_i = b + w_1 x_{i1} + w_2 x_{i2} + \cdots + w_d x_{id}
%%    = \bw^T\bx_i
%\end{align*}
%Here, $b$ is the bias term, and 
%$w_{\!j}$ is the regression coefficient or weight for attribute 
%$X_{\!j}$.
%% Also, we have added an extra dimension fixed at
%% $1$ to each
%% point so that $\bx_k = (1,x_{k1},x_{k2}, \cdots, x_{kd})^T \in
%% \setR^{d+1}$, and we have incorporated the bias term as just another
%% weight for that fixed dimension, so that
%% $\bw = (b, w_1, w_2, \cdots, w_d)^T \in \setR^{d+1}$.
%Given a training data $\bD$ comprising $n$ points 
%$\bx_i$ in a $d$-dimensional space, along with their
%corresponding true response value $y_i$, 
%the bias and weights for linear regression 
%are chosen so as to minimize the sum of squared
%errors between the true and predicted response over all data points
%\begin{align*}
%SSE = \sum_{i=1}^n (y_i - \hy_i)^2 
%    %= \sum_{i=1}^n (y_i - b - \bw^T\bx_i)^2
%\end{align*}
%
%\begin{figure}[!t]
\begin{frame}{Linear and Logistic Regression via Neural Networks}
\psset{tnpos=l,tnsep=2pt,colsep=3,rowsep=0.75,mcol=c,
    ArrowInside=->, arrowscale=2}
\centerline{
%\subfloat[Single Output]{
%\label{fig:reg:neural:ann_multiple_regression}
\psmatrix
\Tcircle[name=b,doubleline=true]{$x_0$}~[tnpos=l]{$1$}~[tnpos=r]{$\;\;$}\\[1em]
\Tcircle[name=x1]{$x_1$} & \Tcircle[name=zk]{$o$}\\[-1em]
[mnode=none]{$\vdots$}\\[-1em]
\Tcircle[name=xd]{$x_{d}$}
\psset{ArrowInsidePos=.85}
\ncline{b}{zk}\ncput*{$b$}
\ncline{x1}{zk} \ncput*{$w_{1}$}
\ncline{xd}{zk} \ncput*{$w_{d}$}
\endpsmatrix
%}
\hspace{0.25in}
%\subfloat[Multiple Outputs]{
%\label{fig:reg:neural:ann_multivariate_regression}
\psmatrix
\Tcircle[name=b,doubleline=true]{$x_0$}~[tnpos=l]{$1$}~[tnpos=r]{$\;\;$}\\[1em]
\Tcircle[name=x1]{$x_1$} & \Tcircle[name=o1]{$o_1$}\\[-1em]
[mnode=none]{$\vdots$} & [mnode=none]{$\vdots$}\\[-1em]
\Tcircle[name=xd]{$x_{d}$} & \Tcircle[name=op]{$o_{p}$}
\psset{ArrowInsidePos=.85,npos=0.25}
\ncline{b}{o1}\ncput*{$b_1$}
\ncline{b}{op}\ncput*{$b_p$}
\ncline{x1}{o1}\ncput*{$w_{11}$}
\ncline{x1}{op}\ncput*{$w_{1p}$}
\ncline{xd}{o1}\ncput*{$w_{d1}$}
\ncline{xd}{op}\ncput*{$w_{dp}$}
\endpsmatrix
%}
}
\end{frame}
%\vspace{0.2in}
%\caption{Linear and logistic regression via neural networks.}
%\vspace{-0.2in}
%\end{figure}
%
%
%As
%shown in \cref{fig:reg:neural:ann_multiple_regression}, 
%a neural network with $d+1$ input neurons $x_0, x_1, \cdots, x_d$,
%including the bias neuron $x_0=1$, and a single output neuron $o$, all
%with identity activation functions and with $\hy=o$, represents the
%exact same model as multiple linear regression.
%Whereas the multiple
%regression problem has a closed form solution, neural networks learn the bias
%and weights via a gradient descent approach that minimizes the squared error.
%
%Neural networks can just as easily model the
%multivariate (linear) regression task, where we have a $p$-dimensional 
%response vector $\by_i \in \setR^p$ instead of a single value $y_i$. 
%That is, the training data $\bD$ comprises $n$ points
%$\bx_i \in \setR^d$ and their true response vectors $\by_i \in \setR^p$.
%As shown in \cref{fig:reg:neural:ann_multivariate_regression},
%multivariate regression can be modeled by a neural network with $d+1$
%input neurons, and $p$ output neurons $o_1, o_2, \cdots, o_p$, with all
%input and output neurons using the identity activation function.
%A neural network learns the weights by comparing its 
%predicted output $\hby = \bo = (o_1, o_2, \cdots, o_p)^T$ with the true
%response vector $\by = (y_1, y_2, \cdots, y_p)^T$.
%That is, training happens by first
%computing the {\em error function} 
%or {\em loss function} between $\bo$ and $\by$.
%Recall that a loss function assigns a score or penalty for predicting
%the output to be $\bo$ when the desired output is $\by$. When the
%prediction matches the true output the loss should be zero.
%The most common loss function for regression is the squared error
%function
%\begin{align*}
%    \cE_\bx = \frac{1}{2} \norm{\by - \bo}^2 
%    = \frac{1}{2} \sum_{j=1}^p (y_{\!j} - o_{\!j})^2
%\end{align*}
%where $\cE_\bx$ denotes the error on input $\bx$.
%Across all the points in a dataset, the total sum of squared errors is
%\begin{align*}
%    \cE = \sum_{i=1}^n \cE_{\bx_i} = \frac{1}{2} \cdot 
%    \sum^{n}_{i=1} \norm{\by_i - \bo_i}^2  
%\end{align*}
%
%
%
%% neural -- linear run plwsl
%%$ python3 ./neuralnet.py iris-plwsl.txt iris-plwsl.txt 0 0.0001 10000 sigmoid
%% Weights: Input to Output
%% [[-0.08733496]
%%  [ 0.4516477 ]
%%  [ 0.00959493]]
%
%% accuracy 6.18028272339 0.0412018848226
%
%%optimal multiple regression model
%%weights [-0.08190841  0.44992999 -0.01385201] 0.209337858118
%%accuracy, sse 0.0 6.178954243
%
%% multivariate regression iris-4d.txt
%% $ python3 ./neuralnet.py iris-4d.txt iris-4d.txt 0 0.0001 10000 sigmoid identity
%% epoch 9999 84.9019338059 0.566012892039
%
%% Weights: Input to Output
%% [[ 1.72077749  0.71685135]
%%  [-1.45859672 -0.49573522]
%%  [-1.82704033 -1.46986061]]
%
%% accuracy 84.9019338059 0.566012892039
%% skikit linear [[ 1.78081362 -1.33924232  0.        ]
%%  [ 0.72659367 -0.47578234  0.        ]] [-2.55717488 -1.59402307]
%% SCIKIT accuracy 84.1610345717 0.561073563811
%
%\enlargethispage{1\baselineskip}
%\begin{example}[Neural Networks for Multiple and Multivariate Regression]
\begin{frame}{ANN for Multiple and Multivariate Regression}
\framesubtitle{Example}
Consider the multiple regression of {\tt sepal length} and {\tt petal
length} on the dependent attribute {\tt petal width} for the Iris
dataset with $n=150$ points. 
%From we find
%that 
The solution is given as
   \begin{align*}
       \hy = -0.014 -0.082 \cdot x_{1} + 0.45\cdot x_{2}
\end{align*}
    The squared error for this optimal solution is $6.179$ on the
    training data.

Using the presented neural network,% in
%\cref{fig:reg:neural:ann_multiple_regression}, 
with linear activation
for the output and minimizing the squared error via gradient descent,
results in the following learned parameters, $b = 0.0096$, $w_1 =
-0.087$ and $w_2 = 0.452$, yielding the regression model
\begin{align*}
    o = 0.0096 - 0.087 \cdot x_{1} + 0.452 \cdot x_{2}
\end{align*}
with a squared error of $6.18$, which is very close to the optimal
solution. 
%However, it is important to note that this is the best result
%over multiple runs of the neural network, since the learning uses a
%random set of initial weights and bias.
\end{frame}

\begin{frame}{ANN for Multiple and Multivariate Regression}
\framesubtitle{Example}

\textbf{Multivariate Linear Regression}
For multivariate regression, we use the neural network architecture 
presented
%in \cref{fig:reg:neural:ann_multivariate_regression} 
to learn the weights
and bias for the Iris dataset, where we use {\tt sepal
length} and {\tt sepal width} as the independent attributes, and {\tt
petal length} and {\tt petal width} as the response or dependent
attributes. Therefore, each input point $\bx_i$ is 2-dimensional, and
the true response vector $\by_i$ is also 2-dimensional. That is, $d=2$
and $p=2$ specify the size of the input and output layers. Minimizing
the squared error via gradient descent, yields the following parameters:
\begin{align*}
    \matr{%
        b_1 & b_2\\
        w_{11} & w_{12}\\
        w_{21} & w_{22}
    } & = 
    \amatr{r}{
  -1.83 & -1.47\\
  1.72 &  0.72\\
  -1.46 & -0.50
  } &
    \matr{o_1 \\ o_2} & =
    \matr{
        -1.83 + 1.72 \cdot x_{1} - 1.46 \cdot x_2\\
        -1.47 + 0.72 \cdot x_{1} - 0.50 \cdot x_2
    }
\end{align*}
The squared error on the training set is $84.9$. Optimal least squared
multivariate regression yields a squared error of $84.16$ with the
following parameters
\begin{align*}
    \matr{\hy_1 \\ \hy_2} =
    \matr{
        -2.56 + 1.78 \cdot x_{1} - 1.34 \cdot x_2\\
        -1.59 + 0.73 \cdot x_{1} - 0.48 \cdot x_2
    }
    % \matr{%
    %     b_1 & b_2\\
    %     w_{11} & w_{12}\\
    %     w_{21} & w_{22}
    % } = 
    % \amatr{r}{
    %     -2.56 & -1.59\\
    %     1.78 &  0.73\\
    %     -1.34 & -0.48
  % }
\end{align*}
\end{frame}
%\end{example}
%
%
%
%\subsection{Classification}
%\index{neural networks!logistic regression}
%\index{neural networks!classification}
%% The cross-entropy error is typically
%% used for classification tasks, along with either sigmoid or
%% softmax activation functions at the output layer.
%Networks of artificial neurons can also learn to classify the inputs.
%Consider the binary
%classification problem, where $y=1$ denotes that the point
%belongs to the positive class, and $y=0$ means that it belongs
%to the negative class. 
%Recall that in logistic regression, we model the probability of the positive
%class via the logistic (or sigmoid) function:
%$$\pi(\bx) = P(y = 1|\bx) = \frac{1}{1 + \exp\{ -(b + \bw^T\bx) \}}$$
%where $b$ is the bias term and $\bw=(w_1,w_2,\cdots,w_d)^T$ is the
%vector of estimated weights or regression coefficients.
%On the other hand
%$$P(y = 0 | \bx) = 1 - P(y=1|\bx) = 1 - \pi(\bx)$$ 
%
%A simple change to the neural network shown in
%\cref{fig:reg:neural:ann_multiple_regression} allows it to solve the
%logistic regression problem. All we have to do is use a sigmoid
%activation function at the output neuron $o$, and use the cross-entropy
%error instead of squared error. 
%Given input $\bx$, true response $y$, and predicted response $o$, recall
%that the {\em cross-entropy error} (see \cref{eq:reg:logit:crossentropy}) 
%is defined as
%\index{classification!cross-entropy error}
%\begin{align*}
%    \cE_\bx = - \bigl( y \cdot
%        \ln(o) + (1-y) \cdot \ln(1-o) \bigr) 
%\end{align*}
%Thus, with sigmoid activation, the output of the neural network in
%\cref{fig:reg:neural:ann_multiple_regression} is given as
%$$o = f(\net_o) = \text{sigmoid}(b + \bw^T\bx) = 
%\frac{1}{1 + \exp\{ -(b + \bw^T\bx) \}} = \pi(\bx)$$
%which is the same as the logistic regression model.
%
%\paragraph{Multiclass Logistic Regression}
%In a similar manner, the multiple output neural network architecture
%shown in \cref{fig:reg:neural:ann_multivariate_regression} can be used
%for multiclass or nominal logistic regression.
%For the general classification problem with $K$ classes
%$\{c_1, c_2,\cdots,c_K\}$, the true
%response $y$ is encoded as a one-hot vector. Thus, class
%$c_1$ is encoded as $\be_1 = (1,0,\cdots,0)^T$, class 
%$c_2$ is encoded as $\be_2 = (0,1,\cdots,0)^T$, and so on, with
%$\be_i \in \{0,1\}^K$ for $i=1,2,\cdots,K$. 
%Thus, we encode $y$ as a multivariate vector $\by \in \{\be_1, \be_2,
%\cdots, \be_K\}$.
%Recall that
%in multiclass logistic regression (see \cref{sec:reg:logit:multiclass})
%the task is to estimate the per class bias $b_i$ and weight
%vector $\bw_i \in \setR^{d}$, with the last class 
%$c_K$ used as the base class with fixed bias $b_K=0$ and
%fixed weight vector $\bw_K = (0,0,\cdots,0)^T\in \setR^d$. The probability
%vector across all $K$ classes is modeled via the softmax
%function (see \cref{eq:reg:logit:softmax}):
%\begin{align*}
%    \pi_i(\bx) = \frac{\exp\{b_i + \bw_i^T\bx\}}{\sum_{j=1}^K
%    \exp\{b_j + \bw_{\!j}^T\bx\} }, \qquad
%    \text{ for all } i = 1, 2, \cdots, K
%\end{align*}
%
%Therefore, 
%the neural network shown in 
%\cref{fig:reg:neural:ann_multivariate_regression} (with $p=K$) can solve
%the multiclass logistic regression task, provided we use a softmax
%activation at the outputs, and use the $K$-way cross-entropy error 
%%for a given input pair $(\bx, \by) \in \bD$
%(see
%\cref{eq:reg:logit:multi_likelihood}),
%defined as
%\begin{align*}
%    \cE_\bx = - \Bigl( y_1 \cdot \ln(o_1) + \cdots 
%            + y_K \cdot \ln(o_K) \Bigr) 
%\end{align*}
%where $\bx$ is an input vector, $\bo = (o_1, o_2, \cdots, o_K)^T$ is the
%predicted response vector, and $\by = (y_1, y_2, \cdots, y_K)^T$ is the
%true response vector. Note that 
%only one element of $\by$ is $1$, and the rest are $0$,
%due to the one-hot encoding.
%
%With softmax activation, 
%the output of the neural network in 
%\cref{fig:reg:neural:ann_multivariate_regression} (with $p=K$) 
%is given as
%$$o_i = P(\by = \be_i | \bx) = f(\net_i | \bnet) =
%            \frac{\exp\{\net_i\}}
%            {\sum_{j=1}^p \exp\{ \net_j \} } = \pi_i(\bx)$$
%which matches the multiclass logistic regression task.
%The only restriction we have to impose on the neural network is that the
%weights on edges into the last output neuron should be zero to model the
%base class weights $\bw_K$. However, in practice, we
%can relax this restriction, and just learn a regular weight vector for
%class $c_K$.
%
%
%
\begin{frame}{Neural networks for multiclass logistic regression}

Iris principal components
data. Misclassified point are shown in dark gray color. Points in class
$c_1$ and $c_2$ are shown displaced with respect to the base class $c_3$
only for illustration.
%
%\begin{figure}[t!]
\psset{unit=0.25in}
\psset{lightsrc=viewpoint}
\psset{incolor=gray}
\psset{opacity=0.2}
\def\PsEuler{2.71828182846}
\psset{viewpoint=30 120 10 rtp2xyz,Decran=60}
\psset{axisnames={X_1, X_2, Y}}
\centerline{
\scalebox{1}{%
\begin{pspicture}(-6,-2)(6,6)
\axesIIID[](-4,-2,0)(4.5,2.5,2.5)
\psset{linewidth=0.5pt, dotsize=0.3}
\psset{showAxes=false, fillcolor=gray}
%% \psPoint(0.75,0.5,2.5){p1}
%% \uput[l](p1){$\pi_1$}
%% \psPoint(-3,-3,2.5){p2}
%% \uput[l](p2){$\pi_2$}
%% \psPoint(-0.5,-0.5,2.0){p3}
%% \uput[l](p3){$\pi_3$}
\psset{transform={1.0 1.0 2.0 scaleOpoint3d}}
\rput(-4,5.1){$\pi_1(\bx)$}
\rput(-0.75,4.95){$\pi_3(\bx)$}
\rput(5,4.75){$\pi_2(\bx)$}
\psSolid[object=parallelepiped,a=8,b=5,c=1.0,action=draw](0,-0.5,0.5)
\psSurface[ngrid=.2 .2,fillcolor=white,axesboxed=false,
    linewidth=0.5\pslinewidth, 
       color1 = {[rgb]{0 0 0}},
        color2 = {[rgb]{1 1 1}},
        hue=(color1) (color2), lightintensity=5,
        algebraic,Zmin=0,Zmax=1](-4,-3)(4,2){%
        ((\PsEuler^((4.54*x)+(1.96*y)-0.89))/%
        (\PsEuler^((4.54*x)+(1.96*y)-0.89) +%
            \PsEuler^((-5.11*x)+(-2.88*y)-3.38) +%
            \PsEuler^((0.52*x)+(0.92*y)+4.24)))}
\psset{transform={1.0 1.0 2.0 scaleOpoint3d}}
\psSurface[ngrid=.2 .2,fillcolor=gray,axesboxed=false,
    linewidth=0.5\pslinewidth, 
       color1 = {[rgb]{0 0 0}},
        color2 = {[rgb]{1 1 1}},
        hue=(color1) (color2), lightintensity=5,
    algebraic,Zmin=0,Zmax=1](-4,-3)(4,2){%
        ((\PsEuler^((-5.11*x)+(-2.88*y)-3.38))/%
        (\PsEuler^((4.54*x)+(1.96*y)-0.89) +%
         \PsEuler^((-5.11*x)+(-2.88*y)-3.38) +%
         \PsEuler^((0.52*x)+(0.92*y)+4.24)))}
\psset{transform={1.0 1.0 2.0 scaleOpoint3d}}
\psSurface[ngrid=.2 .2,fillcolor=darkgray,axesboxed=false,
    linewidth=0.5\pslinewidth, 
       color1 = {[rgb]{0 0 0}},
        color2 = {[rgb]{1 1 1}},
        hue=(color1) (color2), lightintensity=5,
    algebraic,Zmin=0,Zmax=1](-4,-3)(4,2){%
        ((\PsEuler^((0.52*x)+(0.92*y)+4.24))/%
        (\PsEuler^((4.54*x)+(1.96*y)-0.89) +%
         \PsEuler^((-5.11*x)+(-2.88*y)-3.38) +%
        \PsEuler^((0.52*x)+(0.92*y)+4.24)))}
\psset{dotstyle=Bsquare,fillcolor=white}
\psPoint(2.84032129682701,-0.220576338276476,2.0){p0}
\psdot(p0)
\psPoint(2.88981953961792,-0.137345609605027,2.0){p1}
\psdot(p1)
\psPoint(2.5432349073037,0.440031754659816,2.0){p2}
\psdot(p2)
\psPoint(2.58734618891774,0.520473638805969,2.0){p3}
\psdot(p3)
\psPoint(2.67384468671912,-0.106691703752738,2.0){p4}
\psdot(p4)
\psPoint(2.46905599754512,0.137887314590418,2.0){p5}
\psdot(p5)
\psPoint(2.62648199332382,0.170405348960291,2.0){p6}
\psdot(p6)
\psPoint(2.70391231486365,0.115010852170506,2.0){p7}
\psdot(p7)
\psPoint(2.38386932379938,1.34475434455986,2.0){p8}
\psdot(p8)
\psPoint(2.4055141012847,0.195917257696061,2.0){p9}
\psdot(p9)
\psPoint(2.74643719730873,-0.311124315751991,2.0){p10}
\psdot(p10)
\psPoint(2.62252620312581,0.818089674596596,2.0){p11}
\psdot(p11)
\psPoint(2.86699984693253,0.0771930957235878,2.0){p12}
\psdot(p12)
\psPoint(2.20883295417671,0.442696030421004,2.0){p13}
\psdot(p13)
\psPoint(2.7874339759971,-0.227740188871106,2.0){p14}
\psdot(p14)
\psPoint(2.54268575707705,0.586281025343989,2.0){p15}
\psdot(p15)
\psPoint(2.19907796143076,0.879244088091737,2.0){p16}
\psdot(p16)
\psPoint(2.98184266485391,-0.480250048856076,2.0){p17}
\psdot(p17)
\psPoint(2.4100773371216,0.418080008247617,2.0){p18}
\psdot(p18)
\psPoint(2.77013891074632,0.271059419765168,2.0){p19}
\psdot(p19)
\psPoint(2.30312853766388,0.105522678429983,2.0){p20}
\psdot(p20)
\psPoint(2.67384468671912,-0.106691703752738,2.0){p21}
\psdot(p21)
\psPoint(3.21585769490011,0.141615571625584,2.0){p22}
\psdot(p22)
\psPoint(2.63982126837658,0.319290065960322,2.0){p23}
\psdot(p23)
\psPoint(2.62522846468042,0.606800008421587,2.0){p24}
\psdot(p24)
\psPoint(2.82089068218063,-0.0821045110246802,2.0){p25}
\psdot(p25)
\psPoint(2.63284790803076,-0.190075830633623,2.0){p26}
\psdot(p26)
\psPoint(2.88795856533563,-0.570798026331591,2.0){p27}
\psdot(p27)
\psPoint(2.71566519074745,-0.242681482898112,2.0){p28}
\psdot(p28)
\psPoint(2.2798973610096,0.747782713225134,2.0){p29}
\psdot(p29)
\psPoint(2.61314271827106,0.0215206319602587,2.0){p30}
\psdot(p30)
\psPoint(2.99829644283235,-0.334307574590776,2.0){p31}
\psdot(p31)
\psPoint(2.3561710866839,-0.0312095890683376,2.0){p32}
\psdot(p32)
\psPoint(2.71539061563413,-0.169556847556025,2.0){p33}
\psdot(p33)
\psPoint(2.85221108156639,-0.932865367469544,2.0){p34}
\psdot(p34)
\psPoint(3.22520044627498,-0.503279909485425,2.0){p35}
\psdot(p35)
\psPoint(2.72859298183131,0.333924563568455,2.0){p36}
\psdot(p36)
\psPoint(2.56239094683675,0.374684562750107,2.0){p37}
\psdot(p37)
\psPoint(2.59715947707592,1.10002192800727,2.0){p38}
\psdot(p38)
\psPoint(2.50791722683788,-0.139056339913173,2.0){p39}
\psdot(p39)
\psPoint(2.68420712510395,0.326607314764389,2.0){p40}
\psdot(p40)
\psPoint(2.58846205130339,-0.19739307943769,2.0){p41}
\psdot(p41)
\psPoint(2.53757337101351,0.510367545476607,2.0){p42}
\psdot(p42)
\psPoint(2.6476366733969,0.819982632559508,2.0){p43}
\psdot(p43)
\psPoint(2.64354321694115,1.18619489941345,2.0){p44}
\psdot(p44)
\psPoint(2.67384468671912,-0.106691703752738,2.0){p45}
\psdot(p45)
\psPoint(2.5065267893389,0.651935013672573,2.0){p46}
\psdot(p46)
\psPoint(2.64832273247913,0.319136667750885,2.0){p47}
\psdot(p47)
\psPoint(2.31053170131313,0.397867821588893,2.0){p48}
\psdot(p48)
\psPoint(2.59032302558567,0.236059337288874,2.0){p49}
\psdot(p49)

\psset{dotstyle=Bo,fillcolor=white}
\psPoint(-2.34975798395094,-0.0418825496548417,2.0){p0}
\psdot(p0)
\psPoint(-1.41407222517376,-0.574925055912304,2.0){p1}
\psdot(p1)
\psPoint(-2.31430339462955,0.182608850793699,2.0){p2}
\psdot(p2)
\psPoint(-2.31967279387405,-0.24554817060691,2.0){p3}
\psdot(p3)
\psPoint(-1.2906696477376,-0.116425251829372,2.0){p4}
\psdot(p4)
\psPoint(-2.61419383069833,0.558366950278822,2.0){p5}
\psdot(p5)
\psPoint(-1.38966613331942,-0.282886709172269,2.0){p6}
\psdot(p6)
\psPoint(-1.76404593552698,0.0785191864293577,2.0){p7}
\psdot(p7)
\psPoint(-0.519383245084936,-1.19135168905065,2.0){p8}
\psdot(p8)
\psPoint(-2.38756440236903,0.462519250726261,2.0){p9}
\psdot(p9)
\psPoint(-2.56332271234081,0.275974502239838,2.0){p10}
\psdot(p10)
\psPoint(-1.34459421750982,-0.776415425176825,2.0){p11}
\psdot(p11)
\psPoint(-1.52566363131387,-0.375020848191823,2.0){p12}
\psdot(p12)
\psPoint(-2.1653788629534,0.215280283373007,2.0){p13}
\psdot(p13)
\psPoint(-2.9320050969913,0.35237700618086,2.0){p14}
\psdot(p14)
\psPoint(-2.61648460828404,0.341935286987295,2.0){p15}
\psdot(p15)
\psPoint(-1.19809737227496,-0.605578961764594,2.0){p16}
\psdot(p16)
\psPoint(-2.53172698043956,-0.0118422366403003,2.0){p17}
\psdot(p17)
\psPoint(-2.1449868567093,0.138906608877315,2.0){p18}
\psdot(p18)
\psPoint(-2.10765373117327,0.371482249177106,2.0){p19}
\psdot(p19)
\psPoint(-3.07652162102066,0.685764442246057,2.0){p20}
\psdot(p20)
\psPoint(-1.77964010689857,-0.501464794706015,2.0){p21}
\psdot(p21)
\psPoint(-1.80234045266073,-0.216154606626536,2.0){p22}
\psdot(p22)
\psPoint(-1.58526729930864,-0.539307053847172,2.0){p23}
\psdot(p23)
\psPoint(-3.23234289829587,1.37052403597633,2.0){p24}
\psdot(p24)
\psPoint(-3.79468686120997,0.253265570972535,2.0){p25}
\psdot(p25)
\psPoint(-1.25762518292514,-0.179136997423223,2.0){p26}
\psdot(p26)
\psPoint(-1.90162907528823,0.115876748247966,2.0){p27}
\psdot(p27)
\psPoint(-1.97081494590648,-0.181125694704913,2.0){p28}
\psdot(p28)
\psPoint(-2.41939121980498,0.303503937702629,2.0){p29}
\psdot(p29)
\psPoint(-1.38766825018156,-0.204030986570155,2.0){p30}
\psdot(p30)
\psPoint(-1.90474357821391,0.118818990982699,2.0){p31}
\psdot(p31)
\psPoint(-1.29832982456782,-0.761013936521508,2.0){p32}
\psdot(p32)
\psPoint(-2.84096092528538,0.372742591049021,2.0){p33}
\psdot(p33)
\psPoint(-1.66193494701949,0.242038401037523,2.0){p34}
\psdot(p34)
\psPoint(-2.27585364930569,0.333386525756642,2.0){p35}
\psdot(p35)
\psPoint(-2.42816725902085,0.376781971254153,2.0){p36}
\psdot(p36)
\psPoint(-3.39687992067814,0.547168046230153,2.0){p37}
\psdot(p37)
\psPoint(-1.44310260437597,-0.143801289082726,2.0){p38}
\psdot(p38)
\psPoint(-1.90486292513713,0.0480475082318348,2.0){p39}
\psdot(p39)
\psPoint(-1.41407222517376,-0.574925055912304,2.0){p40}
\psdot(p40)
\psPoint(-2.12285398050831,-0.21085488454949,2.0){p41}
\psdot(p41)
\psPoint(-2.15873837313313,-0.218325531562993,2.0){p42}
\psdot(p42)
\psPoint(-3.49926548429896,0.456773466964075,2.0){p43}
\psdot(p43)
\psPoint(-1.9492487818645,0.0407302594277681,2.0){p44}
\psdot(p44)
\psPoint(-2.91813423364425,0.780380629372032,2.0){p45}
\psdot(p45)
\psPoint(-1.92245088483575,0.409271176178696,2.0){p46}
\psdot(p46)
\psPoint(-1.1688534694704,-0.16450249981509,2.0){p47}
\psdot(p47)
\psPoint(-1.94401704887774,0.187415222046014,2.0){p48}
\psdot(p48)
\psPoint(-3.48876537965639,1.17154454426356,2.0){p49}
\psdot(p49)

\psset{dotstyle=Btriangle,fillcolor=white}
\psPoint(-0.511098060683477,-0.10228410504597,0){p0}
\psdot(p0)
\psPoint(-1.46406132277908,0.504189832972453,0){p1}
\psdot(p1)
\psPoint(-1.04329778070628,0.228956908777007,0){p2}
\psdot(p2)
\psPoint(-0.26233575615318,-0.547893298025336,0){p3}
\psdot(p3)
\psPoint(-0.331126947332723,-0.211180140662626,0){p4}
\psdot(p4)
\psPoint(-0.297807907406934,-0.347016521598564,0){p5}
\psdot(p5)
\psPoint(-0.164513428369255,-0.679661469310813,0){p6}
\psdot(p6)
\psPoint(-0.180967206347698,-0.825603943576113,0){p7}
\psdot(p7)
\psPoint(-0.228879049969647,-0.402257620178911,0){p8}
\psdot(p8)
\psPoint(0.908463333123496,-0.751568725169485,0){p9}
\psdot(p9)
\psPoint(-0.0432464003287043,-0.581489446612358,0){p10}
\psdot(p10)
\psPoint(-0.234540586259835,-0.33192182936212,0){p11}
\psdot(p11)
\psPoint(-0.812867790369069,-0.370678998318909,0){p12}
\psdot(p12)
\psPoint(-0.6595927890562,-0.351976291060316,0){p13}
\psdot(p13)
\psPoint(-0.890160445651156,-0.0338124427467624,0){p14}
\psdot(p14)
\psPoint(-0.355533039187063,-0.503218487402662,0){p15}
\psdot(p15)
\psPoint(0.307334755663405,-0.36508661276616,0){p16}
\psdot(p16)
\psPoint(-0.375238228946766,-0.291622024808779,0){p17}
\psdot(p17)
\psPoint(-0.920502648909009,-0.182390363327921,0){p18}
\psdot(p18)
\psPoint(0.174864001965697,-0.25181557108017,0){p19}
\psdot(p19)
\psPoint(-0.245195161690346,-0.266728035661848,0){p20}
\psdot(p20)
\psPoint(-0.587274853579911,-0.483284267717482,0){p21}
\psdot(p21)
\psPoint(-0.356786567830461,-0.0668238279413658,0){p22}
\psdot(p22)
\psPoint(-0.812455548998002,-0.162331574878633,0){p23}
\psdot(p23)
\psPoint(0.707081283920861,-1.00842476177846,0){p24}
\psdot(p24)
\psPoint(-0.932410752982917,0.319198089833649,0){p25}
\psdot(p25)
\psPoint(-1.5573962720707,0.267392584812765,0){p26}
\psdot(p26)
\psPoint(0.511086195895091,-1.26249195386214,0){p27}
\psdot(p27)
\psPoint(-1.11624072375382,-0.0842140138783732,0){p28}
\psdot(p28)
\psPoint(-0.942361707398841,-0.541822258150014,0){p29}
\psdot(p29)
\psPoint(0.751467140648227,-1.00110751297439,0){p30}
\psdot(p30)
\psPoint(-1.28479458784507,0.68543918613292,0){p31}
\psdot(p31)
\psPoint(0.191884492103146,-0.677490544374357,0){p32}
\psdot(p32)
\psPoint(-1.09522370993848,0.283891210938481,0){p33}
\psdot(p33)
\psPoint(-1.33104141885472,0.24466952060229,0){p34}
\psdot(p34)
\psPoint(-0.984044545169431,-0.124360420222278,0){p35}
\psdot(p35)
\psPoint(-0.927572942032699,0.468236205043117,0){p36}
\psdot(p36)
\psPoint(0.0101900707278536,-0.720574866701901,0){p37}
\psdot(p37)
\psPoint(-0.714008213642833,0.150379153148438,0){p38}
\psdot(p38)
\psPoint(-0.134994950451373,-0.311709642703396,0){p39}
\psdot(p39)
\psPoint(-1.29646885028553,-0.327561519794944,0){p40}
\psdot(p40)
\psPoint(-1.22042896624468,0.408035337001018,0){p41}
\psdot(p41)
\psPoint(-1.37873698277558,-0.421205138214628,0){p42}
\psdot(p42)
\psPoint(-0.640436749523145,-0.417323482970025,0){p43}
\psdot(p43)
\psPoint(-0.899640863283776,0.329610979581884,0){p44}
\psdot(p44)
\psPoint(-0.807205496676716,0.195053963771112,0){p45}
\psdot(p45)
\psPoint(0.070342888949272,-0.702537931730725,0){p46}
\psdot(p46)
\psPoint(-0.641690278166543,0.0190711764912711,0){p47}
\psdot(p47)
\psPoint(-1.087134487207,0.0753903892887678,0){p48}
\psdot(p48)
\psPoint(-0.463030988871424,-0.669526546536122,0){p49}
\psdot(p49)

%%misclassified points
\psset{dotstyle=Bo, fillcolor=darkgray}
\psPoint(-0.52,-1.19,2.0){x0} %point58
\psdot(x0)
\psPoint(-1.17,-0.16,2.0){x1} %point97
\psdot(x1)
\psset{dotstyle=Btriangle, fillcolor=darkgray}
\psPoint(-1.55,0.26,0){x2} %point126
\psdot(x2)
\psPoint(-1.296,-0.33,0){x3} %point140
\psdot(x3)
\psPoint(-1.38,-0.42,0){x4} %point142
\psdot(x4)
\end{pspicture}
}}
\end{frame}
%\vspace{0.2in}
%\caption{Neural networks for multiclass logistic regression: Iris principal components
%data. Misclassified point are shown in dark gray color. Points in class
%$c_1$ and $c_2$ are shown displaced with respect to the base class $c_3$
%only for illustration.}
%\label{fig:reg:neural:iris-3K-multiclass}
%\end{figure}
%
%\begin{example}[Logistic Regression: Binary and Multiclass]
\begin{frame}{Logistic Regression: Binary and Multiclass}
\framesubtitle{Example}
    We applied the neural network presented,
%in \cref{fig:reg:neural:ann_multiple_regression}, 
with logistic
    activation at the output neuron and cross-entropy error function, 
    on the Iris principal components dataset. The output is
a binary response indicating {\tt Iris-virginica} ($Y=1$) or one of
the other Iris types ($Y=0$). As expected, 
the neural network learns an identical set
of weights and bias as shown for the logistic regression model,
% in \cref{ex:reg:logit:logit}, 
namely:
    \begin{align*}
        o = -6.79 -5.07 \cdot x_1 - 3.29 \cdot x_2
    \end{align*}


%python3 neuralnet.py iris-PC-3K.txt iris-PC-3K.txt 0 -eta 0.001 -eps 0.001 -t 100000 -of softmax -ef CE  -c -d " " -l
% epoch 3626 145.0 0.9666666666666667 14.0046809394 0.000999892559042

% Weights: Input to Output
% [[ 3.61439879 -5.17714895  0.        ]
%  [ 2.65259494 -3.3970625   0.        ]
%  [-3.49440596 -6.94833725  0.        ]]
% YMAP {0.0: 0, 1.0: 1, 2.0: 2}

% accuracy 145.0 0.9666666666666667 14.0046809394
    Next, we we applied the neural network presented,
%in \cref{fig:reg:neural:ann_multivariate_regression}, 
using a softmax
activation and cross-entropy error function,
to the Iris
principal components data with three classes: {\tt Iris-setosa}
($Y=1$), {\tt Iris-versicolor} ($Y=2$) and {\tt Iris-virginica}
($Y=3$). 
Thus, we need $K=3$ output neurons, $o_1$, $o_2$, and $o_3$.
Further, to obtain the
same model as in the multiclass logistic regression example,
%from \cref{ex:reg:logit:multi-logit}, 
we fix the incoming weights and bias for
output neuron $o_3$ to be zero. The model is given as
    \begin{align*}
        o_1 & = -3.49 + 3.61 \cdot x_1 + 2.65 \cdot x_2\\
        o_2 & = -6.95 -5.18 \cdot x_1 -3.40 \cdot x_2\\
        o_3 & = 0 + 0 \cdot x_1 + 0 \cdot x_2
    \end{align*}
which is essentially the same as presented before.
%in \cref{ex:reg:logit:multi-logit}.
\end{frame}

\begin{frame}{Logistic Regression: Binary and Multiclass}
\framesubtitle{Example}
If we do not constrain the weights and bias for $o_3$ we obtain the
following model:
    \begin{align*}
        o_1 & = -0.89 + 4.54 \cdot x_1 + 1.96 \cdot x_2\\
        o_2 & = -3.38 -5.11 \cdot x_1 -2.88 \cdot x_2\\
        o_3 & = 4.24 + 0.52 \cdot x_1 + 0.92 \cdot x_2
    \end{align*}
The classification decision surface for each class is illustrated in
%\cref{fig:reg:neural:iris-3K-multiclass}. 
the figure.
The points in class $c_1$ are shown as squares, $c_2$ as circles, and
$c_3$ as triangles. 
This figure should be
contrasted with the decision boundaries shown for multiclass logistic
regression,
% in \cref{fig:reg:logit:iris-3K-multiclass}, 
which 
has the weights and bias set to $0$ for the base class $c_3$.

\end{frame}
%\end{example}
%
%% python3 neuralnet.py iris-PC-3K.txt iris-PC-3K.txt 0 -eta 0.001 -eps 0.001 -t 100000 -of softmax -ef CE  -c -d " "
%% Weights: Input to Output
%% [[ 4.54090129 -5.11417967  0.51748127]
%%  [ 1.9603263  -2.88048547  0.92212899]
%%  [-0.88670555 -3.3759307   4.24191964]]
%% YMAP {0.0: 0, 1.0: 1, 2.0: 2}
%
%% accuracy 145.0 0.9666666666666667 13.2631586158
%
%
%\subsection{Error Functions}
%\index{neural networks!error functions}
%\label{sec:reg:neural:errorDERIV}
\begin{frame}{Error Functions}
%Typically, for a regression task, we use squared error as the loss
%function, whereas for classification, we use the cross-entropy loss
%function. Furthermore, when learning from neural networks, we will
%require the partial derivatives of the error function with respect to
%the output neurons.
%Thus, the commonly used error functions and their derivatives are listed
%below:
%
%\begin{description}
%\index{neural networks!squared error}

    {\bf Squared Error:} Given an input vector $\bx \in \setR^d$,
        the squared error loss function measures the squared deviation
        between the predicted output vector $\bo \in \setR^p$ and the
        true response $\by \in \setR^p$, defined as follows: 
    \begin{align}
        \tcbhighmath{
    \cE_\bx = \frac{1}{2} \norm{\by - \bo}^2 
= \frac{1}{2} \sum_{j=1}^p (y_{\!j} - o_{\!j})^2}
    \label{eq:reg:neural:SSE}
\end{align}
where $\cE_\bx$ denotes the error on input $\bx$.
%        
        The partial derivative of the squared error function with
        respect to a particular output neuron $o_{\!j}$ is
        \begin{align}
            \frac{\partial \cE_\bx}{\partial o_{\!j}} = 
            \frac{1}{2} \cdot 2 \cdot (y_{\!j} - o_{\!j}) \cdot -1 = 
            o_{\!j} -
            y_{\!j}
        \label{eq:reg:neural:deriv_SE_single}
        \end{align}
        Across all the output neurons, we can write this as
        \begin{align}
            \frac{\partial\cE_\bx}{\partial \bo} = \bo - \by
        \label{eq:reg:neural:deriv_SE}
        \end{align}
%
%\index{neural networks!cross-entropy error}
\end{frame}

\begin{frame}{Error Functions}

    {\bf Cross-Entropy Error:} For classification tasks, with $K$
        classes $\{c_1, c_2, \cdots, c_K\}$, we usually set the number
        of output neurons $p=K$, with one output neuron 
        per class. Furthermore, each of
        the classes is coded as a one-hot vector, with class $c_i$
        encoded as the $i$th standard basis vector 
        $\be_i  = (e_{i1}, e_{i2}, \cdots, e_{iK})^T \in \{0,1\}^{K}$,
        with $e_{ii}=1$ and $e_{ij} = 0$ for all $j\ne i$. Thus,
        given input $\bx \in \setR^d$, with 
        the true response $\by = (y_1,
        y_2, \cdots, y_K)^T$, where $\by \in \{\be_1, \be_2,
        \cdots, \be_K\}$, the cross-entropy loss is defined as 
\begin{align}
    \tcbhighmath{
    \cE_\bx = - \sum_{i=1}^K y_i \cdot \ln(o_i) = 
    - \Bigl( y_1 \cdot \ln(o_1) + \cdots 
+ y_K \cdot \ln(o_K) \Bigr) }
            \label{eq:reg:neural:CE}
\end{align}
Note that only one element of $\by$ is $1$ and the rest are $0$
due to the one-hot encoding. That is, if $\by = \be_i$, then only $y_i =
1$, and the other elements $y_j = 0$ for $j \ne i$.

The partial derivative of the cross-entropy error function with
respect to a particular output neuron $o_{\!j}$ is
\begin{align}
    \frac{\partial \cE_\bx}{\partial o_{\!j}} = 
    -\frac{y_{\!j}}{o_{\!j}}
            \label{eq:reg:neural:deriv_CE}
\end{align}
\end{frame}

\begin{frame}{Error Functions}

The vector of partial derivatives of the error function with respect to
the output neurons
is therefore given as
\begin{align}
    \frac{\partial\cE_\bx}{\partial \bo} & = 
    \lB(\frac{\partial \cE_\bx}{\partial o_{1}},
        \frac{\partial \cE_\bx}{\partial o_{2}},
    \;\cdots,\; \frac{\partial \cE_\bx}{\partial o_{K}}\rB)^T
     = \lB( -\frac{y_1}{o_1}, -\frac{y_2}{o_2}, \cdots, 
             -\frac{y_K}{o_K}
    \rB)^T
    \label{eq:reg:neural:pd_ef_o}
\end{align}

\end{frame}

\begin{frame}{Error Functions}
\index{neural networks!binary cross-entropy error}
{\bf Binary Cross-Entropy Error:} For classification tasks with
    binary classes, it is typical to encode the positive class as $1$
    and the negative class as $0$, as opposed to using a one-hot
    encoding as in the general $K$-class case. Given an input $\bx \in
    \setR^d$, with true response $y \in \{0,1\}$, there is only
    one output neuron $o$. Therefore, the binary cross-entropy error is
    defined as
    \begin{align}
        \tcbhighmath{
        \cE_\bx = - \bigl( y \cdot \ln(o) 
            + (1-y) \cdot \ln(1-o) \bigr) 
        }
        \label{eq:reg:neural:CEbinary}
    \end{align}
    Here $y$ is either $1$ or $0$.
The partial derivative of the binary cross-entropy error function with
respect to the output neuron $o$ is
\begin{align}
    \frac{\partial \cE_\bx}{\partial o} & =
    \frac{\partial}{\partial o} \Bigl\{- y\cdot\ln(o) -
    (1-y)\cdot\ln(1-o) \Bigr\} \notag\\
    & = -\lB(\frac{y}{o} + \frac{1-y}{1-o} \cdot -1\rB)
         = \frac{-y \cdot (1-o) + (1-y)\cdot o} {o\cdot (1-o)} \notag\\
         & = \frac{o-y}{o \cdot (1-o)}
            \label{eq:reg:neural:deriv_CE_binary}
\end{align}


\end{frame}
%\section{Multilayer Perceptron: One Hidden Layer}
%
%\index{MLP! see {multilayer perceptron}}
%\index{multilayer perceptron}
%\index{neural networks!multilayer perceptron}
%
\begin{frame}{Multilayer Perceptron: One Hidden Layer}
A multilayer perceptron (MLP) is a neural network that has distinct layers of
neurons. 

\medskip 
The inputs to the neural network comprise the {\em input
layer}, and the final outputs from the MLP comprise the {\em output
layer}. Any intermediate layer is called a {\em hidden layer}, and an MLP
can have one or many hidden layers. Networks with many hidden layers are
called {\em deep neural networks}. 


\medskip 
An MLP is also a feed-forward network. That is, information flows
in the forward direction, and from a layer only to the subsequent layer. 
\end{frame}
%Thus, information flows from the input to the first hidden layer, from the
%first to the second hidden layer, and so on, until it reaches the output
%layer from the last hidden layer. 
%Typically, MLPs are fully connected
%between layers. That is, each neuron in the input layer is connected to
%all the neurons in the first hidden layer, and each
%neuron in the first hidden layer is connected to all neurons in the
%second hidden layer, and so on, and finally,
%each neuron in the last hidden layer
%is connected to all neurons in the output layer. 
%
%\begin{figure}[!t]
\begin{frame}{Multilayer perceptron with one hidden layer}
\psset{tnpos=l,tnsep=2pt,colsep=3,rowsep=0.75,mcol=c,
    ArrowInside=->, arrowscale=2}
\centerline{
\psmatrix
[mnode=none]{Input} & [mnode=none]{Hidden} &
[mnode=none]{Output}\\[-2.5em]
[mnode=none]{Layer} & [mnode=none]{Layer} &
[mnode=none]{Layer}\\[-2em]
\Tcircle[name=x0,doubleline=true]{$x_0$}~[tnpos=l]{$1$} & 
\Tcircle[name=z0,doubleline=true]{$z_0$}~[tnpos=l]{$1$}\\[1em] 
\Tcircle[name=x1]{$x_1$} & \Tcircle[name=z1]{$z_1$} & 
    \Tcircle[name=y1]{$o_1$}\\[-1em]
[mnode=none]{$\vdots$} & [mnode=none]{$\vdots$} & 
[mnode=none]{$\vdots$}\\[-1em]
\Tcircle[name=xi]{$x_i$}& \Tcircle[name=zk]{$z_k$} &
\Tcircle[name=yj]{$o_{\!j}$} \\[-1em]
[mnode=none]{$\vdots$} & [mnode=none]{$\vdots$} &
[mnode=none]{$\vdots$}\\[-1em]
\Tcircle[name=xd]{$x_{d}$} & \Tcircle[name=zh]{$z_{m}$} &
\Tcircle[name=yp]{$o_{\!p}$}
%weights into zk and out of zk
% \psset{ArrowInsidePos=.85}
% \ncline[linecolor=gray,linewidth=0.5pt]{x0}{z0}\ncput*[npos=0.5]{$1$}
\psset{ArrowInsidePos=.85}
\psset{linecolor=black,linewidth=1.5pt,labelsep=0pt}
\ncline{x0}{zk}\ncput*[npos=0.25]{$b_k$}
\ncline{x1}{zk}\ncput*[npos=0.25]{$w_{1k}$}
\ncline{xi}{zk}\ncput*[npos=0.25]{$w_{ik}$}
\ncline{xd}{zk}\ncput*[npos=0.25]{$w_{dk}$}
\ncline{zk}{y1}\ncput*[npos=0.25]{$w_{k1}$}
\ncline{zk}{yj}\ncput*[npos=0.25]{$w_{k\!j}$}
\ncline{zk}{yp}\ncput*[npos=0.25]{$w_{kp}$}
%input to hidden
\psset{linecolor=gray,linewidth=0.5pt}
\ncline{x0}{z1}%\ncput*{$b_1$}
\ncline{x0}{zh}%\ncput*{$b_h$}
\ncline{x1}{z1}%\ncput*{$w_{11}$}
\ncline{x1}{zh}%\ncput*{$w_{1h}$}
\ncline{xi}{z1}%\ncput*{$w_{i1}$}
\ncline{xi}{zh}%\ncput*{$w_{ih}$}
\ncline{xd}{z1}%\ncput*{$w_{d1}$}
\ncline{xd}{zh}%\ncput*{$w_{dh}$}
%#hidden to output
\ncline{z0}{y1}%\ncput*{$w'_{01}$}
\ncline{z0}{yj}%\ncput*{$w'_{02}$}
\ncline{z0}{yp}%\ncput*{$w'_{0p}$}
\ncline{z1}{y1}%\ncput*{$w'_{11}$}
\ncline{z1}{yj}%\ncput*{$w'_{12}$}
\ncline{z1}{yp}%\ncput*{$w'_{1p}$}
\ncline{z2}{y1}%\ncput*{$w'_{21}$}
\ncline{z2}{yj}%\ncput*{$w'_{22}$}
\ncline{z2}{yp}%\ncput*{$w'_{2p}$}
\ncline{zh}{y1}%\ncput*{$w'_{h1}$}
\ncline{zh}{yj}%\ncput*{$w'_{h2}$}
\ncline{zh}{yp}%\ncput*{$w'_{hp}$}
\endpsmatrix
}
\end{frame}
%\vspace{0.2in}
%\caption{Multilayer perceptron with one hidden layer. Input and output
%    links for neuron $z_k$ are shown in bold. Neurons $x_0$ and $z_0$
%are bias neurons.}
%\label{fig:reg:neural:MLP}
%\end{figure}
%
%
%% \begin{figure}[!t]
%% \centerline{
%% \begin{neuralnetwork}[height=4]
%% 		\newcommand{\nodetextclear}[2]{}
%% 		\newcommand{\nodetextx}[2]{$x_#2$}
%% 		\newcommand{\nodetexty}[2]{$y_#2$}
%% 		\newcommand{\nodetextz}[2]{$z_#2$}
%% 		\inputlayer[count=4, bias=false, title=Input\\layer, text=\nodetextx]
%% 		\hiddenlayer[count=5, bias=false, title=Hidden\\layer,
%%         text=\nodetextz] \linklayers
%% 		\outputlayer[count=3, title=Output\\layer, text=\nodetexty] \linklayers
%% \end{neuralnetwork}
%% }
%%\end{figure}
%
%
%
%
%
%For ease of explanation,  in this section,
%we will consider an MLP with only one
%hidden layer, and we will later generalize the discussion to deep MLPs.
%For example, \cref{fig:reg:neural:MLP} shows an 
%MLP with one hidden layer. The input layer has $d$ neurons, $x_1, x_2,
%\cdots, x_d$,
%and an additional neuron $x_0$ that specifies the biases for the hidden
%layer. The hidden layer has $m$ neurons, $z_1, z_2, \cdots, z_m$, and an additional neuron $z_0$
%that specifies the biases for the output neurons. Finally, the output
%layer has $p$ neurons, $o_1, o_2, \cdots, o_p$. 
%The bias neurons have no incoming edges, 
%since their value is
%always fixed at $1$.
%Thus, in total there are $d \times m + m \times p$ 
%weight parameters ($w_{ij}$) and a further $m + p$ bias parameters ($b_i$)
%that need to be learned by the neural network. These parameters also
%correspond to the total number of edges in the MLP.
%
%
%\subsection{Feed-forward Phase}
\begin{frame}{Feed-forward Phase}
%\index{multilayer perceptron!feed-forward phase}
%\index{neural networks!feed-forward phase}
%
%Let $\bD$ denote the training dataset, 
%comprising $n$ input points $\bx_i \in \setR^d$ and
%corresponding true response vectors $\by_i \in \setR^p$.
%For each pair $(\bx, \by)$ in the data,
%in the feed-forward phase, the point 
%$\bx =  (x_1, x_2, \cdots, x_d)^T \in \setR^d$ 
%is supplied as an input to the MLP. 
%The input neurons do not use any activation
%function, and simply pass along the supplied input values as their
%output value. This is equivalent to saying that 
%the net at input neuron $k$ is $\net_k = x_k$, and the activation function is
%the identity function $f(\net_k) = \net_k$, 
%so that the output value of neuron $k$ is simply $x_k$.
%
Given the input neuron values,  
we compute the output value for each hidden neuron $z_k$
as follows:
\begin{align*}
    z_k = f(\net_k) = f\lB(b_k + \sum_{i=1}^d w_{ik} \cdot x_i\rB)
\end{align*}
where $f$ is some activation function, and $w_{ik}$ denotes the weight
between input neuron $x_i$ and hidden neuron $z_k$.
Next, given the hidden
neuron values, we compute the value for each output
neuron $o_{\!j}$ as follows:
\begin{align*}
    o_{\!j} = f(\net_{\!j}) = f\lB(b_{\!j} + \sum_{i=1}^m w_{i\!j}
    \cdot z_i\rB)
\end{align*}
where $w_{ij}$ denotes the weight between hidden neuron $z_i$ and output
neuron $o_{\!j}$.
\end{frame}
%
%We can write the feed-forward phase as a series of matrix-vector
%operations. For this we define
%the $d\times m$ matrix 
%$\bW_{\!h}$
%comprising the
%weights between input and hidden layer neurons, 
%and vector $\bb_{h} \in \setR^m$ comprising 
%the bias terms for hidden layer neurons,
%given as
%\begin{align}
%    \bW_{\!h} & = \matr{
%        w_{11} & w_{12} & \cdots & w_{1m}\\
%         w_{21} & w_{22} & \cdots & w_{2m}\\
%         \vdots & \vdots & \cdots & \vdots\\
%         w_{d1} & w_{d2} & \cdots & w_{dm}\\
%    } & 
%    \bb_{h} & = \matr{b_1\\  b_2\\ \vdots\\ b_m} 
%    \label{eq:reg:neural:Wi}
%\end{align}
%where $w_{ij}$ denotes the weight on the edge between input neuron
%$x_i$ and hidden neuron $z_{\!j}$, and 
%$b_i$ denotes the bias weight from $x_0$ to $z_i$.
%The net input and the output for all the hidden layer neurons can be 
%computed via a matrix-vector multiplication operation, as follows:
%% \begin{align*}
%% \tikzmarkin[tmark]{bneth}
%% \bnet_{h} & = \bb_{h} + \bW_{\!h}^T \bx 
%% \tikzmarkend{bneth}
%% &
%% \tikzmarkin[tmark]{bz}(0.2,-1)(-0.2,1)
%% \bz & = f(\bnet_h) = f\lB(\bb_{h} + \bW_{\!h}^T \bx \rB)
%% \tikzmarkend{bz}
%% \end{align*}
%\begin{empheq}[box=\tcbhighmath]{gather}
%\bnet_{h}  = \bb_{h} + \bW_{\!h}^T \bx \\
%\bz  = f(\bnet_h) = f\lB(\bb_{h} + \bW_{\!h}^T \bx \rB)
%\end{empheq}
% Here, $\bnet_h = (\net_1, \cdots, \net_m)^T$ 
%denotes the net input at each hidden neuron (excluding the bias neuron
%$z_0$ whose value is always fixed at $z_0 = 1$), and
%$\bz = (z_1, z_2, \cdots, z_m)^T$ denotes the vector of hidden neuron
%values.
%The activation function $f(\bdot)$ applies to, or distributes over, each
%element of $\bnet_h$, i.e., $f(\bnet_h) = 
%\lB(f(\net_1), \cdots, f(\net_m)\rB)^T \in \setR^{m}$.
%Typically, all neurons in a given layer use the same activation
%function, but they can also be different if desired.
%
%Likewise, let $\bW_{\!o} \in \setR^{m \times p}$ 
%denote the weight matrix
%between the hidden and output layers, and let $\bb_{o} \in
%\setR^p$ be the bias
%vector for output neurons, given as
%\begin{align}
%    \bW_{\!o} & = \matr{
%         w_{11} & w_{12} & \cdots & w_{1p}\\
%         w_{21} & w_{22} & \cdots & w_{2p}\\
%         \vdots & \vdots & \cdots & \vdots\\
%         w_{m1} & w_{m2} & \cdots & w_{mp}\\
%    } &
%    \bb_{o} & = \matr{b_1\\  b_2\\ \vdots\\ b_p} 
%    \label{eq:reg:neural:Wo}
%\end{align}
%where $w_{ij}$ denotes the weight on the edge between hidden neuron
%$z_i$ and output neuron $o_{\!j}$, and $b_i$ the bias weight
%between $z_0$ and output neuron $o_{\!i}$.
\begin{frame}{Feed-Forward Phase}
The output vector can then be
computed as follows:
\begin{empheq}[box=\tcbhighmath]{gather}
    \bnet_o = \bb_{o} + \bW_{\!o}^T\bz\\
    \bo = f(\bnet_o) = f\lB(\bb_{o} + \bW_{\!o}^T\bz\rB)
\end{empheq}
To summarize, for a given input $\bx \in \bD$ with desired response $\by$,
an MLP computes the output vector via the
feed-forward process, as follows:
\begin{align}
    \bo = f\Bigl(\bb_{o} + \bW_{\!o}^T\bz\Bigr) 
    = f\Bigl(\bb_{o} + \bW_{\!o}^T \cdot 
    f\lB(\bb_{h} + \bW_{\!h}^T \bx \rB) \Bigr)
    \label{eq:reg:neural:FFo}
\end{align}
where, $\bo = (o_1, o_2, \cdots, o_p)^T$ is the vector of predicted
outputs from the single hidden layer MLP.
\end{frame}
%
%
%\subsection{Backpropagation Phase}
%\index{multilayer perceptron!backpropagation phase}
%\index{neural networks!backpropagation phase}
%\index{backpropagation!neural networks}
\begin{frame}{Backpropagation Phase}
%
Backpropagation is the algorithm used to learn the weights between
successive layers in an MLP. The name comes from the manner in which the
{\em error gradient} is propagated backwards from the output to input layers via
the hidden layers. %For simplicity of exposition, we will consider
%backpropagation for an MLP with a single hidden layer with $m$ neurons, 
%with squared error
%function, and with sigmoid activations for all neurons.
%We will later generalize to multiple hidden layers, and
%other error and activation functions.
%
%Let $\bD$
%%= \lB\{\bx_i^T, \by_i\rB\}_{i=1}^n$, 
%denote the training dataset, 
%comprising $n$ input points 
%$\bx_i = (x_{i1}, x_{i2}, \cdots, x_{id})^T \in \setR^d$ and
%corresponding true response vectors $\by_i \in \setR^p$.
%Let $\bW_{\!h} \in\setR^{d\times m}$ 
%denote the weight matrix between the input and hidden layer,
%and $\bb_{h} \in \setR^m$ 
%the vector of bias terms for the hidden neurons from
%\cref{eq:reg:neural:Wi}. Likewise,
%let $\bW_{\!o} \in \setR^{m \times p}$ 
%denote the weight matrix between the hidden and output layer,
%and $\bb_{o} \in \setR^p$ the bias vector for output neurons from
%\cref{eq:reg:neural:Wo}. 


\medskip 
%
For a given input pair $(\bx, \by)$ in the training data, the MLP first computes the
output vector  $\bo$ via the feed-forward step.% in
%\cref{eq:reg:neural:FFo}.
Next, it computes the error in the predicted output {\em vis-a-vis} 
the true
response $\by$ using the squared error function
\begin{align}
    \cE_\bx = \frac{1}{2} \norm{\by - \bo}^2 = 
    \frac{1}{2} \sum^{p}_{j=1} (y_{\!j} - o_{\!j})^2
    \label{eq:reg:neural:SSEderiv}
\end{align}
\end{frame}

\begin{frame}{Backpropagation Phase}
The basic idea behind backpropagation is to examine the extent to which
an output neuron, say $o_{\!j}$, deviates from the corresponding 
target response $y_{\!j}$, and to modify the
weights $w_{ij}$ between each hidden neuron $z_i$ and $o_{\!j}$ as some
function of the error -- large error should cause a correspondingly
large change in the weight, and small error should result in smaller
changes. 
%Likewise, the weights between all input and hidden
%neurons should also be updated as some function of the error at the
%output, as well as changes already computed for the weights between the hidden
%and output layers. That is, the error propagates backwards.
%
%\index{multilayer perceptron!weight update}
%\index{multilayer perceptron!weight gradient}
%\index{multilayer perceptron!gradient descent}
%\index{neural networks!weight update}
%\index{neural networks!weight gradient}
%\index{neural networks!gradient descent}
The weight update is done via a gradient descent approach to minimize
the error. 
Let $\grad_{w_{ij}}$ be the gradient of the error function with
respect to $w_{ij}$, or simply the {\em weight gradient} at
$w_{ij}$.
Given the previous weight estimate $w_{ij}$, 
a new weight is computed by taking a
small step $\eta$ in a direction that is opposite to the weight gradient 
at $w_{ij}$
\begin{align}
    w_{ij} = w_{ij} - \eta \cdot \grad_{w_{ij}}
\end{align}
In a similar manner, the bias term $b_{\!j}$ is also updated via gradient
descent
\begin{align}
    b_{\!j} = b_{\!j} - \eta \cdot \grad_{b_{\!j}}
\end{align}
where 
$\grad_{b_{\!j}}$ is the gradient of the error function with
respect to $b_{\!j}$, which we call the {\em bias gradient} at
$b_{\!j}$.
\end{frame}
%\index{multilayer perceptron!bias gradient}
%\index{neural networks!bias gradient}
%
%
%\subsubsection{Updating Parameters Between Hidden and Output Layer}
\begin{frame}{Updating Parameters Between Hidden and Output Layer}
%
Consider the weight $w_{ij}$ between hidden neuron $z_i$ and
output neuron $o_{\!j}$, and the bias term $b_{\!j}$
between $z_0$ and $o_{\!j}$. 
Using the chain rule of differentiation, we
compute the weight gradient at $w_{ij}$ and bias gradient at
$b_{\!j}$, as
follows:
\begin{align}
    \label{eq:reg:neural:gradwij}
    \begin{aligned}
    \grad_{w_{ij}} & = 
 \frac{\partial \cE_\bx}{\partial w_{ij}}
     = \frac{\partial \cE_\bx}{\partial \net_{\!j}} \cdot 
    \frac{\partial \net_{\!j}}{\partial w_{ij}}
    = \delta_{\!j}  \cdot z_i\\
    \grad_{b_{\!j}} & = 
 \frac{\partial \cE_\bx}{\partial b_{\!j}}
     = \frac{\partial \cE_\bx}{\partial \net_{\!j}} \cdot 
    \frac{\partial \net_{\!j}}{\partial b_{\!j}}
    = \delta_{\!j}
\end{aligned}
\end{align}
where we use the symbol $\delta_{\!j}$ to denote the partial derivative of
the error with respect to net signal at $o_{\!j}$, which we also call the
{\em net gradient} at $o_{\!j}$
%\index{multilayer perceptron!net gradient}
%\index{neural networks!net gradient}
%\begin{align}
%    \delta_{\!j} = \frac{\partial \cE_\bx}{\partial \net_{\!j}}
%\end{align}
%Furthermore, the partial derivative of $\net_{\!j}$ with respect to
%$w_{ij}$ and $b_{\!j}$ is given as
%\begin{align*}
%   \frac{\partial \net_{\!j}}{\partial w_{ij}} & = 
%   \frac{\partial}{\partial w_{ij}} 
%   \lB\{b_{\!j} + \sum_{k=1}^m w_{k\!j} \cdot z_{k} \rB\}
%     = z_i &
%   % = 
%   % \frac{\partial}{\partial w_{ij}}
%   %  \lB\{b_{\!j} + \sum_{k\ne i}^m w_{k\!j} \cdot z_{k} + 
%   %  w_{i\!j} \cdot z_i \rB\}
%   \frac{\partial \net_{\!j}}{\partial b_{\!j}} & = 
%   \frac{\partial}{\partial b_{\!j}} 
%   \lB\{b_{\!j} + \sum_{k=1}^m w_{k\!j} \cdot z_{k} \rB\}
%     = 1
%\end{align*}
%where we used the fact that $b_{\!j}$ and 
%all $w_{k\!j}$ for $k \ne i$
%are constants with respect to $w_{ij}$.
%
Next, we need to compute $\delta_{\!j}$, the net gradient at
$o_{\!j}$. This can also be computed via
the chain rule
\begin{align}
    \delta_{\!j} = \frac{\partial \cE_\bx}{\partial \net_{\!j}} =
   \frac{\partial \cE_\bx}{\partial f(\net_{\!j})} \cdot
\frac{\partial f(\net_{\!j})}{\partial \net_{\!j}}
\label{eq:reg:neural:deltaj_chain}
\end{align}
Note that $f(\net_{\!j}) = o_{\!j}$.
\end{frame}

\begin{frame}{Updating Parameters Between Hidden and Output Layer}
%Thus, $\delta_{\!j}$ is composed of two terms, namely the partial derivative
%of the error term with respect to the output or activation function
%applied to the net signal, 
%and the
%derivative of the activation function with respect to its argument.
Using the squared error function %and from
%\cref{eq:reg:neural:deriv_SE_single}, 
for the former, we have
\begin{align*}
   \frac{\partial \cE_\bx}{\partial f(\net_{\!j})}  =
   \frac{\partial \cE_\bx}{\partial o_{\!j}}
   = \frac{\partial}{\partial o_{\!j}} \lB\{\frac{1}{2} \sum^{p}_{k=1}
   (y_{k} - o_{k})^2\rB\} \notag
   = \lB( o_{\!j} - y_{\!j} \rB)
\end{align*}
where we used the observation that all $o_k$ for $k \ne j$ are constants
with respect to $o_{\!j}$.
Since we assume a sigmoid activation function, for the latter, we have
%via \cref{eq:reg:neural:deriv_sigmoid}
\begin{align*}
   \frac{\partial f(\net_{\!j})}{\partial \net_{\!j}}  
   = o_{\!j} \cdot \lB(1 - o_{\!j}\rB)
\end{align*}
Putting it all together, we get
\begin{align*}
    \delta_{\!j} & =\lB( o_{\!j} - y_{\!j} \rB) \cdot o_{\!j} \cdot \lB(1 -
   o_{\!j}\rB)
\end{align*}
\end{frame}
%
%Let $\bdelta_o = (\delta_1, \delta_2, \ldots, \delta_p)^T$ denote the
%vector of net gradients 
%at each output neuron, which we call the {\em net gradient vector} for
%the output layer. We can write $\bdelta_o$ as
%\begin{empheq}[box=\tcbhighmath]{align}
%    \bdelta_o = \bo \;\odot\; (\bone - \bo) \;\odot\; (\bo - \by)
%\end{empheq}
%where $\odot$ denotes the element-wise product (also called the {\em
%Hadamard product}) between the vectors, and where
%$\bo = (o_1,
%o_2, \cdots, o_p)^T$ is the predicted output vector, $\by
%= (y_1, y_2, \cdots, y_p)^T$ is the (true) response vector, and
%$\bone = (1,\cdots,1)^T \in \setR^p$ is the $p$-dimensional 
%vector of all ones. 
%% Instead of computing the net gradient $\delta_{\!j}$ at each output
%% neuron, we could have derived an expression for the net gradient vector
%% $\bdelta_o$ via vector derivatives, as follows:
%% \begin{align*}
%%     \bdelta_o  = \frac{\partial \cE_\bx}{\partial \bnet_o} & = 
%%     \frac{\partial \bo}{\partial \bnet_o} \odot \frac{\partial
%%     \cE_\bx}{\partial \bo}\\
%%     & = \bo (\bone - \bo) \odot (\bo - \by)
%% \end{align*}
%
%Let $\bz = (z_1, z_2, \cdots, z_m)^T$ denote the vector comprising
%the values of all hidden layer neurons (after applying the activation
%function).
%Based on \cref{eq:reg:neural:gradwij},
%we can compute the gradients $\grad_{w_{ij}}$ for all
%hidden to output neuron connections
%via the outer product of $\bz$ and $\bdelta_o$:
%\begin{align}
%    \tcbhighmath{
%    \bgrad_{\bW_{\!o}}  
%    = \matr{
%        \grad_{w_{11}} & \grad_{w_{12}} & \cdots &\grad_{w_{1p}}\\
%        \grad_{w_{21}} & \grad_{w_{22}} & \cdots &\grad_{w_{2p}}\\
%        \vdots & \vdots & \cdots & \vdots\\
%        \grad_{w_{m1}} & \grad_{w_{m2}} & \cdots &\grad_{w_{mp}}\\
%    }
%= \bz \cdot \bdelta_o^T}
%\end{align}
%where $\bgrad_{\bW_{\!o}} \in \setR^{m \times p}$ is the 
%matrix of weight gradients.
%The vector of bias gradients is given as:
%\begin{align}
%    \tcbhighmath{
%    \bgrad_{\!\bb_{o}} = 
%    \lB(\grad_{b_1}, \grad_{b_2}, \cdots,
%\grad_{b_p}\rB)^T = \bdelta_{\!o}}
%\end{align}
%where $\bgrad_{\!\bb_{o}} \in \setR^p$.
%
%Once the gradients have been computed, we can 
%update all the weights and biases as follows
%\begin{empheq}[box=\tcbhighmath]{align}
%    \begin{aligned}
%        \bW_{\!o} & = \bW_{\!o} - \eta \cdot \bgrad_{\bW_{\!o}}\\
%        \bb_{o} & = \bb_{o} - \eta \cdot \bgrad_{\!\bb_{o}}
%\end{aligned}
%\end{empheq}
%where $\eta$ is the step size (also called the {\em learning rate}) 
%for gradient descent.
%
%
%\subsubsection{Updating Parameters Between Input and Hidden Layer}
\begin{frame}{Updating Parameters Between Input and Hidden Layer}
Consider the weight $w_{ij}$ between input neuron $x_i$ and
hidden neuron $z_{\!j}$, and the bias term between $x_0$ and
$z_{\!j}$. 
%The weight gradient at $w_{ij}$ and bias gradient at $b_{\!j}$ 
%is computed similarly to
%\cref{eq:reg:neural:gradwij}
\begin{align}
    \begin{aligned}
    \grad_{w_{ij}} & = 
 \frac{\partial \cE_\bx}{\partial w_{ij}}
     = \frac{\partial \cE_\bx}{\partial \net_{\!j}} \cdot 
    \frac{\partial \net_{\!j}}{\partial w_{ij}}
    = \delta_{\!j}  \cdot x_i\\
    \grad_{b_{\!j}} & = 
 \frac{\partial \cE_\bx}{\partial b_{\!j}}
     = \frac{\partial \cE_\bx}{\partial \net_{\!j}} \cdot 
    \frac{\partial \net_{\!j}}{\partial b_{\!j}}
    = \delta_{\!j}
\end{aligned}
\label{eq:reg:neural:gradwij_ih}
\end{align}
%which follows from
%\begin{align*}
%   \frac{\partial \net_{\!j}}{\partial w_{ij}} & = 
%   \frac{\partial}{\partial w_{ij}} 
%   \lB\{b_{\!j} + \sum_{k=1}^m w_{k\!j} \cdot x_{k} \rB\}
%    = x_i &
%  % = \frac{\partial}{\partial w_{ij}} 
%  %   \lB\{b_{\!j} + \sum_{k\ne i}^m w_{k\!j} \cdot x_{k} +
%  %   w_{i\!j} \cdot x_i \rB\}
%   \frac{\partial \net_{\!j}}{\partial b_{\!j}} & = 
%   \frac{\partial}{\partial b_{\!j}} 
%   \lB\{b_{\!j} + \sum_{k=1}^m w_{k\!j} \cdot x_{k} \rB\}
%    = 1
%\end{align*}
%
%\enlargethispage{1\baselineskip}
To compute the net gradient $\delta_{\!j}$ 
at the hidden neuron $z_j$ we have to consider the error gradients that
flow back from all the output neurons to $z_j$.
Applying the chain rule, we get:
\begin{align*}
    \delta_{\!j} = \frac{\partial \cE_\bx}{\partial \net_{\!j}}
    & =
   \sum^{p}_{k=1} \frac{\partial \cE_\bx}{\partial \net_k} \cdot
   \frac{\partial \net_{k}}{\partial z_{\!j}} \cdot
\frac{\partial z_{\!j}}{\partial \net_{\!j}}
    =
\frac{\partial z_{\!j}}{\partial \net_{\!j}} \cdot
   \sum^{p}_{k=1} \frac{\partial \cE_\bx}{\partial \net_k} \cdot
   \frac{\partial \net_{k}}{\partial z_{\!j}} \\
& = z_{\!j} \cdot (1-z_{\!j}) \cdot \sum^{p}_{k=1} \delta_k \cdot w_{\!jk}
\end{align*}
where $\tfrac{\partial z_{\!j}}{\partial \net_{\!j}} =
z_{\!j}\cdot(1-z_{\!j})$, since we assume a sigmoid activation function
for the hidden neurons.
\end{frame}
%The chain rule leads to a natural interpretation for backpropagation,
%namely, to find the net gradient at $z_{\!j}$ we have to consider
%the net gradients at each of the output neurons $\delta_k$ but weighted
%by the strength of the connection $w_{\!jk}$ between $z_{\!j}$ and
%$o_k$, as illustrated in \cref{fig:reg:neural:backpropOH}. 
%That is, we compute the weighted sum of gradients 
%$\sum_{k=1}^p \delta_k \cdot w_{\!jk}$, 
%which is used to compute $\delta_{\!j}$, the net gradient at hidden
%neuron $z_{\!j}$.
%
%\begin{figure}[!t]
\begin{frame}{Backpropagation of gradients from output to hidden layer}
\psset{tnpos=l,tnsep=2pt,colsep=2,rowsep=0.75,mcol=c,
ArrowInside=->, arrowscale=1.5}
\centerline{
    \scalebox{0.9}{%
\psmatrix
[mnode=none]{Input} & [mnode=none]{Hidden} &
[mnode=none]{Output}\\[-2.5em]
[mnode=none]{Layer} & [mnode=none]{Layer} &
[mnode=none]{Layer}\\[-2em]
\Tcircle[name=x0,doubleline=true]{$x_0$}~[tnpos=l]{$1$}
& & \Tcircle[name=y1]{$o_1$}~[tnpos=r]{$\delta_1$}\\[-1em]
 &  & 
[mnode=none]{$\vdots\quad$}\\[-1em]
\Tcircle[name=xi]{$x_i$}&
\Tcircle[name=zj]{$z_{\!j}$}~[tnpos=b]{{%
        \footnotesize%
    $\displaystyle\sum_{k=1}^p \delta_k \cdot w_{\!jk}$}} &
\Tcircle[name=yk]{$o_{k}$}~[tnpos=r]{$\delta_k$} \\[-4em]
& & [mnode=none]{$\vdots\quad$}\\[-1em]
 &  & \Tcircle[name=yp]{$o_{\!p}$}~[tnpos=r]{$\delta_p$}
%gradients into zj
\psset{ArrowInsidePos=.9}
\psset{linecolor=black,linewidth=1.5pt,labelsep=0pt}
\ncline{zj}{x0}\ncput*[npos=0.5]{$b_{\!j}$}
\ncline{zj}{xi}\ncput*[npos=0.5]{$w_{i\!j}$}
\ncline{y1}{zj}\ncput*[npos=0.5]{$w_{\!j1}$}
\ncline{yk}{zj}\ncput*[npos=0.5]{$w_{\!jk}$}
\ncline{yp}{zj}\ncput*[npos=0.5]{$w_{\!jp}$}
\endpsmatrix
}}
\end{frame}
%\vspace{0.2in}
%\caption{Backpropagation of gradients from output to hidden layer.}
%\label{fig:reg:neural:backpropOH}
%\vspace{-0.2in}
%\end{figure}
%
%
%
%Let $\bdelta_o = (\delta_1, \delta_2, \ldots, \delta_p)^T$ denote the
%vector of net gradients at
%the output neurons, and
%$\bdelta_h = (\delta_1, \delta_2, \ldots, \delta_m)^T$ the
%net
%gradients at the hidden layer neurons.
%We can write $\bdelta_h$ compactly as
%\begin{empheq}[box=\tcbhighmath]{align}
%    \bdelta_h = \bz \;\odot\; (\bone - \bz) \;\odot\; (\bW_{\!o} \cdot \bdelta_o)
%\end{empheq}
%where $\odot$ is the element-wise product, 
%$\bone = (1,1,\cdots,1) \in \setR^{m}$ is the vector
%of all ones and
%$\bz = (z_1, z_2, \cdots, z_m)^T$ is the vector of
%hidden layer outputs. Furthermore, 
%$\bW_{\!o} \cdot \bdelta_o \in \setR^m$ is the vector of
%weighted gradients at each hidden neuron, since
%\begin{align*}
%   \bW_{\!o} \cdot \bdelta_o = 
%   \matr{ 
%    \displaystyle
%    \sum_{k=1}^p \delta_k \cdot w_{1k}, & 
%    \displaystyle
%    \sum_{k=1}^p \delta_k \cdot w_{2k}, & 
%    \cdots, & 
%    \displaystyle
%       \sum_{k=1}^p \delta_k \cdot w_{mk}
%       }^T
%\end{align*}
%
%% We could have directly computed the net gradient vector at the hidden
%% layer via vector derivatives, as follows
%% \begin{align*}
%%     \bdelta_h  = \frac{\partial \cE_\bx}{\partial \bnet_h} & = 
%%     \frac{\partial \bz}{\partial \bnet_h} \odot 
%%     \lB(\frac{\partial \bnet_o}{\partial \bz} \cdot
%%         \frac{\partial \cE_\bx}{\partial \bnet_o}\rB)\\
%%         & = \bz (\bone - \bz) \odot \lB( \bW_o \cdot \bdelta_o\rB)
%% \end{align*}
%
%
%
%
%Let $\bx = (x_1, x_2, \cdots, x_d)^T$ denote the
%input vector, then based on \cref{eq:reg:neural:gradwij_ih} we can compute the gradients
%$\grad_{w_{ij}}$ for all input to hidden layer connections via the outer
%product:
%\begin{align}
%    \tcbhighmath{
%    \bgrad_{\bW_{\!h}} 
%    = \matr{
%         \grad_{w_{11}} & \cdots &\grad_{w_{1m}}\\
%         \grad_{w_{21}} & \cdots &\grad_{w_{2m}}\\
%         \vdots & \cdots & \vdots\\
%         \grad_{w_{d1}} & \cdots &\grad_{w_{dm}}\\
%    }
%= \bx \cdot \bdelta_h^T}
%\end{align}
%where $\bgrad_{\bW_{\!h}} \in \setR^{d \times m}$ is the 
%matrix of weight gradients. 
%\enlargethispage{1\baselineskip}
%The vector of bias gradients is given as:
%\begin{align}
%    \tcbhighmath{
%    \bgrad_{\!\bb_{h}} = 
%    \lB(\grad_{b_1}, \grad_{b_2}, \cdots,
%\grad_{b_m}\rB)^T = \bdelta_h}
%\end{align}
%where $\bgrad_{\!\bb_{h}} \in \setR^m$.
%
%
%Once the gradients have been computed, we can 
%update all the weights and biases as follows
%\begin{empheq}[box=\tcbhighmath]{align}
%    \begin{aligned}
%        \bW_{\!h} & = \bW_{\!h} - \eta \cdot \bgrad_{\bW_{\!h}}\\
%        \bb_{h} & = \bb_{h} - \eta \cdot \bgrad_{\!\bb_{h}}
%\end{aligned}
%\end{empheq}
%where $\eta$ is the step size (or learning rate).
%
%

\begin{frame}{MLP Training: Stochastic Gradient Descent}
\SetKwFunction{RandomMatrix}{RandomMatrix}
\begin{tightalgo}[H]{\textwidth-18pt}
    \SetKwInput{AlgorithmRM}{\textsc{RandomMatrix} ($r,
    c$)}
    \SetKwInOut{Algorithm}{\textsc{MLP-Training} ($\bD, m, \eta,
    \texttt{maxiter}$)}
\Algorithm{}
\tcp{Initialize bias vectors}
$\bb_{h} \assign$ random $m$-dimensional vector with small
values\; 
$\bb_{o} \assign$ random $p$-dimensional vector with small
values\;
\tcp{Initialize weight matrices}
$\bW_{\!h} \assign \text{ random } d \times m \text{ matrix with small
values}$\;
$\bW_{\!o} \assign  \text{ random } m \times p \text{ matrix with small
values}$\;
$t \assign 0$ \tcp*[h]{iteration counter}\;
\end{tightalgo}
\end{frame}

\begin{frame}{MLP Training: Stochastic Gradient Descent}
\SetKwFunction{RandomMatrix}{RandomMatrix}
\begin{tightalgo}[H]{\textwidth-18pt}
\setcounter{AlgoLine}{5}
\Repeat{$t \ge \text{\tt maxiter}$}
{%
    \ForEach{$(\bx_i,\by_i) \in \bD$ in random order}{%
        \tcp{Feed-forward phase}
        $\bz_i \assign f\lB(\bb_h + \bW_{\!h}^T \bx_i \rB)$\;
        $\bo_i \assign f\Bigl(\bb_o + \bW_{\!o}^T\bz_i \Bigr)$\;
        \tcp{Backpropagation phase: net gradients}
        $\bdelta_o \assign \bo_i \;\odot\; (\bone - \bo_i) \;\odot\;
        (\bo_i-\by_i)$ \;
        $\bdelta_h \assign \bz_i \;\odot\; (\bone - \bz_i) \;\odot\; (\bW_{\!o}
        \cdot \bdelta_o)$ \;
        \tcp{Gradient descent for bias vectors}
        $\bgrad_{\!\bb_{o}} \assign \bdelta_o; \quad
        \bb_{o} \assign \bb_{o} - 
                \eta \cdot \bgrad_{\bb_{o}}$\;   
        $\bgrad_{\!\bb_{h}} \assign \bdelta_h; \quad
        \bb_{h} \assign \bb_{h} - 
                \eta \cdot \bgrad_{\bb_{h}}$\; 
        \tcp{Gradient descent for weight matrices}
        $\bgrad_{\bW_{\!o}} \assign \bz_i \cdot \bdelta_o^T; \quad  
        \bW_{\!o} \assign \bW_{\!o} - \eta \cdot \bgrad_{\bW_{\!o}}$ \;
        $\bgrad_{\bW_{\!h}} \assign \bx_i \cdot \bdelta_h^T; \quad
        \bW_{\!h} \assign \bW_{\!h} - \eta \cdot \bgrad_{\bW_{\!h}}$ \;
    }
    $t \assign t+1$\;
}%
%\caption{MLP Training: Stochastic Gradient Descent}
%\label{alg:reg:neural:mlp}
\end{tightalgo}
\end{frame}
%
%% \vspace*{-0.2in}
%\subsection{MLP Training}
%\index{multilayer perceptron!training}
%\index{neural networks!training}
%\index{multilayer perceptron!stochastic gradient descent}
%\index{stochastic gradient descent}
%\index{neural networks!stochastic gradient descent}
%%So far we have discussed how the feed-forward and backpropagation steps
%%handle a single input vector. 
%\cref{alg:reg:neural:mlp} shows the
%pseudo-code for 
%learning the weights considering all of the input points
%in $\bD$ via {\em stochastic gradient descent}. 
%The code is shown for an MLP with a single hidden layer, using
%a squared error function, and sigmoid activations for all hidden and
%output neurons. The approach is called stochastic gradient descent since
%we compute the weight and bias gradients after observing each training
%point (in random order).
%
%The MLP algorithm takes as input the dataset $\bD$ (with points $\bx_i$
%and desired responses $\by_i$ for $i=1,2,\cdots,n$), 
%%= \lB\{\bx_i^T, \by_i\rB\}_{i=1,\cdots,n}$, 
%the number of hidden layer neurons $m$, the
%learning rate $\eta$, and an
%integer threshold {\tt maxiter} that specifies the maximum number of
%iterations. The size of the input ($d$) and output ($p$) layers is
%determined automatically from $\bD$. The MLP first initializes the $d
%\times m$ input to hidden layer weight matrix $\bW_{\!h}$, and the $m
%\times p$ hidden to output layer matrix $\bW_{\!o}$ to small values, for
%example, uniformly random in the range $[-0.01,0.01]$. It is important to note that
%weights should not be set to $0$, otherwise, all hidden neurons will be
%identical in their values, and so will be the output neurons.
%
%The MLP training takes multiple iterations over the input points. For
%each input $\bx_i$, the MLP computes the output vector $\bo_i$ via the
%feed-forward step. In the backpropagation phase, we compute the error
%gradient vector $\bdelta_o$ with respect to the net at output neurons,
%followed by $\bdelta_h$ for hidden neurons. In the stochastic gradient descent
%step, we compute the error gradients with respect to the weights and
%biases, which are used to update the weight matrices and bias vectors.
%Thus, for every input vector $\bx_i$, all the weights and biases are
%updated based on the error incurred between the predicted output $\bo_i$
%and the true response $\by_i$. After each input has been processed, that
%completes one iteration of training, called an {\em epoch}. 
%Training stops when the maximum
%number of iterations, {\tt maxiter}, has been reached.
%On the other hand, during testing, for any input $\bx$, we apply the
%feed-forward steps and print the predicted output $\bo$.
%\index{multilayer perceptron!epoch}
%\index{neural networks!epoch}
%
%In terms of computational complexity, each iteration of the MLP training
%algorithm takes $O(dm+mp)$ time
%for the feed-forward phase, 
%$p + mp + m = O(mp)$ time for the backpropagation of error gradients, and
%$O(dm + mp)$ time for updating the weight matrices and bias vectors.
%Thus, the total training time per iteration is
%$O(dm+mp)$.
\begin{frame}{MLP with one hidden layer}
\framesubtitle{Example}
    %In the examples above we used neural networks for linear and 
    %logistic regression, which did not require the use of the hidden
    %layer. 
    We now illustrate an MLP with a hidden layer using a
    non-linear activation function to learn the sine curve.
    %\cref{fig:reg:neural:sine1} 
Figure shows the training data (the gray points
    on the curve), which comprises $n=25$ points $x_i$ sampled randomly
    in the range $[-10,10]$, with $y_i = \sin(x_i)$. 
    The testing data comprises 1000 points sampled uniformly from 
    the same range.
    The figure also shows the desired output curve (thin line).
 

\medskip 
   We used an MLP with
    one input neuron ($d=1$),
    ten hidden neurons ($m=10$) and one output neuron ($p=1$). The hidden
    neurons use tanh activations, whereas the output unit
    uses an identity activation. The step size is $\eta=0.005$. 


\medskip 

    The input to hidden weight matrix $\bW_{\!h} \in \setR^{1\times 10}$ and the
    corresponding bias vector $\bb_{h} \in \setR^{10 \times 1}$ 
    are given as:
    \begin{align*}        
        \bW_{\!h} & = (-0.68,  0.77, -0.42, -0.72, -0.93, -0.42,
        -0.66, -0.70, -0.62, -0.50)\\
        \bb_{h} & = (-4.36,  2.43, -0.52,  2.35  -1.64,  3.98,
        0.31,  4.45,  1.03, -4.77)^T
    \end{align*}
\end{frame}

\begin{frame}{MLP with one hidden layer}
\framesubtitle{Example}
    The hidden to output weight matrix $\bW_{\!o} \in \setR^{10 \times 1}$ and the bias term
    $\bb_{o} \in \setR$ are given as:
    \begin{align*}
        \bW_{\!o} & = (-1.82,
       -1.69,
       -0.82,
       1.37,
       0.14,
       2.37,
       -1.64,
       -1.92,
       0.78,
       2.17)^T\\
       \bb_{o} & = -0.16
    \end{align*}


    Figures 
%    \cref{fig:reg:neural:sine1} 
show the output of the MLP on the test for various numbers of iterations.
%    set, after the 
%    first iteration of training ($t=1$). We can see that initially the
%    predicted response deviates significantly from the true sine
%    response. 
%\cref{fig:reg:neural:sine}\protect\subref{fig:reg:neural:sine1}--\protect\subref{fig:reg:neural:sine30k} show the output from the MLP
%    after different number of training iterations. By $t=15000$
%    iterations the output on the test set 
%    comes close to the sine curve, but it takes
%    another $15000$ iterations to get a closer fit. 
The final SSE is $1.45$ over the 1000 test points.


\medskip 
    
    We
    can observe that, even with a very small training data of 25 points
    sampled randomly from the sine curve, the MLP is able to learn the
    desired function. 

\medskip

However, it is also important to recognize that
    the MLP model has not really learned the sine function; rather, it
    has learned to approximate it only in the specified
    range $[-10,10]$. We can also see %in \cref{fig:reg:neural:sine20test}
    that when we try to predict values outside this range, the MLP does
    not yield a good fit.
    \label{ex:reg:neural:mlp_sine}
\end{frame}
%\end{example}
%
%
\readdata{\dataS}{REG/neural/figs/sine-train_n25.txt}
\readdata{\dataPzero}{REG/neural/figs/n25_h10_100k/foo_e0.txt}
\readdata{\dataPoneK}{REG/neural/figs/n25_h10_100k/foo_e1000.txt}
\readdata{\dataPfiveK}{REG/neural/figs/n25_h10_100k/foo_e5000.txt}
\readdata{\dataPtenK}{REG/neural/figs/n25_h10_100k/foo_e10000.txt}
\readdata{\dataPfifteenK}{REG/neural/figs/n25_h10_100k/foo_e15000.txt}
\readdata{\dataPthirtyK}{REG/neural/figs/n25_h10_100k/foo_e30000.txt}
\readdata{\dataPsinM}{REG/neural/figs/n25_h10_100k/sine_n25_h10_test20_predict.txt}
%\begin{figure}[htbp!]
%    \captionsetup[subfloat]{captionskip=20pt}
%    \centering
    \def\pshlabel#1{\scriptsize {$#1$}}
    \def\psvlabel#1{\scriptsize {$#1$}}
    \psset{arrowscale=2}
    \psset{xAxisLabel=$X$,yAxisLabel= $Y$,
        xAxisLabelPos={0in,-0.35in},yAxisLabelPos={-0.35in,0in}}
        %xAxisLabelPos={c,-0.35in},yAxisLabelPos={-0.35in,c}}
\begin{frame}{MLP for sine curve}
\framesubtitle{$t=1$}
    \centerline{%
%    \subfloat[$t=1$]{%
%        \label{fig:reg:neural:sine1}
    \scalebox{1.0}{%
%    % \psset{yAxisLabelPos={-0.35in,0in}}
        \psgraph[tickstyle=bottom,Dx=2,Dy=0.5,Ox=-12,Oy=-2.5]{->}(-12.0,-2.5)(12.5,1.25){4in}{2.25in}%
    \psset{dotstyle=Bo,dotscale=1.5,fillcolor=lightgray}
    \dataplot[plotstyle=dots,showpoints=true]{\dataS}
    \psset{dotstyle=Bo,dotscale=0.5,fillcolor=black}
    \listplot[plotstyle=dots,showpoints=true,plotNoMax=3,plotNo=3]{\dataPzero}
    \psplot[linewidth=0.5pt,plotpoints=200]{-10}{10}{x RadtoDeg sin} 
    \psline[linewidth=0.5pt,linestyle=dashed](-12,0)(12.5,0)
    \endpsgraph
    }}
\end{frame}
%    % \psset{yAxisLabelPos={-0.35in,c}}
%    \hspace{0.5in}
%    \subfloat[$t=1000$]{%
%        \label{fig:reg:neural:sine1k}
\begin{frame}{MLP for sine curve}
\framesubtitle{$t=1000$}
\hspace*{1cm}
        \scalebox{1.00}{%
    \psgraph[tickstyle=bottom,Dx=2,Dy=0.25,Ox=-12,Oy=-1.25]{->}(-12.0,-1.25)(12.5,1.25){4in}{2.25in}%
    \psset{dotstyle=Bo,dotscale=1.5,fillcolor=lightgray}
    \dataplot[plotstyle=dots,showpoints=true]{\dataS}
    \psset{dotstyle=Bo,dotscale=0.5,fillcolor=black}
    \listplot[plotstyle=dots,showpoints=true,plotNoMax=3,plotNo=3]{\dataPoneK}
    \psplot[linewidth=0.5pt,plotpoints=200]{-10}{10}{x RadtoDeg sin} 
    \psline[linewidth=0.5pt,linestyle=dashed](-12,0)(12.5,0)
    \endpsgraph
%    }}
    }
\end{frame}
%    \vspace{-0.1in}
%    \centerline{%
%    \subfloat[$t=5000$]{%
%        \label{fig:reg:neural:sine5k}
\begin{frame}{MLP for sine curve}
\framesubtitle{$t=5000$}
\hspace*{1cm}
    \scalebox{1.00}{%
    \psgraph[tickstyle=bottom,Dx=2,Dy=0.25,Ox=-12,Oy=-1.25]{->}(-12.0,-1.25)(12.5,1.25){4in}{2.25in}%
    \psset{dotstyle=Bo,dotscale=1.5,fillcolor=lightgray}
    \dataplot[plotstyle=dots,showpoints=true]{\dataS}
    \psset{dotstyle=Bo,dotscale=0.5,fillcolor=black}
    \listplot[plotstyle=dots,showpoints=true,plotNoMax=3,plotNo=3]{\dataPfiveK}
    \psplot[linewidth=0.5pt,plotpoints=200]{-10}{10}{x RadtoDeg sin} 
    \psline[linewidth=0.5pt,linestyle=dashed](-12,0)(12.5,0)
    \endpsgraph
    }%}
\end{frame}
%    \hspace{0.5in}
%    \subfloat[$t=10000$]{%
%        \label{fig:reg:neural:sine10k}
\begin{frame}{MLP for sine curve}
\framesubtitle{$t=10000$}
\hspace*{1cm}
        \scalebox{1.00}{%
    \psgraph[tickstyle=bottom,Dx=2,Dy=0.25,Ox=-12,Oy=-1.25]%
                {->}(-12.0,-1.25)(12.5,1.5){4in}{2.5in}%
    \psset{dotstyle=Bo,dotscale=1.5,fillcolor=lightgray}
    \dataplot[plotstyle=dots,showpoints=true]{\dataS}
    \psset{dotstyle=Bo,dotscale=0.5,fillcolor=black}
    \listplot[plotstyle=dots,showpoints=true,plotNoMax=3,plotNo=3]{\dataPtenK}
    \psplot[linewidth=0.5pt,plotpoints=200]{-10}{10}{x RadtoDeg sin} 
    \psline[linewidth=0.5pt,linestyle=dashed](-12,0)(12.5,0)
    \endpsgraph
%    }}
    }
\end{frame}
%    \vspace{-0.1in}
%    \centerline{%
%    \subfloat[$t=15000$]{%
%        \label{fig:reg:neural:sine15k}
\begin{frame}{MLP for sine curve}
\framesubtitle{$t=15000$}
\hspace*{1cm}
    \scalebox{1.00}{%
    \psgraph[tickstyle=bottom,Dx=2,Dy=0.25,Ox=-12,Oy=-1.25]{->}(-12.0,-1.25)(12.5,1.25){4in}{2.25in}%
    \psset{dotstyle=Bo,dotscale=1.5,fillcolor=lightgray}
    \dataplot[plotstyle=dots,showpoints=true]{\dataS}
    \psset{dotstyle=Bo,dotscale=0.5,fillcolor=black}
    \listplot[plotstyle=dots,showpoints=true,plotNoMax=3,plotNo=3]{\dataPfifteenK}
    \psplot[linewidth=0.5pt,plotpoints=200]{-10}{10}{x RadtoDeg sin} 
    \psline[linewidth=0.5pt,linestyle=dashed](-12,0)(12.5,0)
    \endpsgraph
    }%}
\end{frame}
%    \hspace{0.5in}
%    \subfloat[$t=30000$]{%
%        \label{fig:reg:neural:sine30k}
\begin{frame}{MLP for sine curve}
\framesubtitle{$t=30000$}
\hspace*{1cm}
        \scalebox{1.00}{%
    \psgraph[tickstyle=bottom,Dx=2,Dy=0.25,Ox=-12,Oy=-1.25]{->}(-12.0,-1.25)(12.5,1.25){4in}{2.25in}%
    \psset{dotstyle=Bo,dotscale=1.5,fillcolor=lightgray}
    \dataplot[plotstyle=dots,showpoints=true]{\dataS}
    \psset{dotstyle=Bo,dotscale=0.5,fillcolor=black}
    \listplot[plotstyle=dots,showpoints=true,plotNoMax=3,plotNo=3]{\dataPthirtyK}
    \psplot[linewidth=0.5pt,plotpoints=200]{-10}{10}{x RadtoDeg sin} 
    \psline[linewidth=0.5pt,linestyle=dashed](-12,0)(12.5,0)
    \endpsgraph
%    }}
    }
\end{frame}
%    \vspace{-0.1in}
%    \centerline{%
%    \subfloat[Test range ${[-20,20]}$]{%
%    \label{fig:reg:neural:sine20test}
\begin{frame}{MLP for sine curve}
\framesubtitle{Test range ${[-20,20]}$}
\hspace*{1cm}
    \scalebox{1.0}{%
        \psgraph[tickstyle=bottom,Dx=5,Dy=0.5,Ox=-20,Oy=-2.5]{->}(-20.0,-2.5)(24,2.5){4in}{2.25in}%
    \psset{dotstyle=Bo,dotscale=1.5,fillcolor=lightgray}
    \dataplot[plotstyle=dots,showpoints=true]{\dataS}
    \psset{dotstyle=Bo,dotscale=0.5,fillcolor=black}
    \listplot[plotstyle=dots,showpoints=true,plotNoMax=2,plotNo=2]{\dataPsinM}
    \psplot[linewidth=0.5pt,plotpoints=250]{-20}{20}{x RadtoDeg sin} 
    \psline[linewidth=0.5pt,linestyle=dashed](-20,0)(20.5,0)
    %\psline[linewidth=0.5pt](-10.1,-1.1)(10.1,-1.1)
    %\psline[linewidth=0.5pt](10.1,-1.1)(10.1,1.1)
    %\psline[linewidth=0.5pt](10.1,1.1)(-10.1,1.1)
    %\psline[linewidth=0.5pt](-10.1,1.1)(-10.1,-1.1)
    \psframe[linewidth=0.5pt](-10.1,-1.1)(10.1,1.1)
    \endpsgraph
%    }}
    }
\end{frame}
%\vspace{0.2in}
%\caption{MLP for sine curve: 10 hidden neurons with hyperbolic tangent
%activation functions. The gray dots represent the training data. The
%bold line is the predicted response, whereas the thin line is the true
%response. \protect\subref{fig:reg:neural:sine1}--\protect\subref{fig:reg:neural:sine30k}:
%Predictions after different number of iterations.
%\protect\subref{fig:reg:neural:sine20test}: 
%Testing outside the training range. Good fit within the training
%    range $[-10,10]$ shown in the box.}
%\label{fig:reg:neural:sine}
%\end{figure}
%
%
%%\readdata{\dataPsinM}{REG/neural/figs/sin20.txt}
%%\begin{figure}[t!]
%%    %\captionsetup[subfloat]{captionskip=20pt}
%%    \centering
%%    \def\pshlabel#1{\scriptsize {$#1$}}
%%    \def\psvlabel#1{\scriptsize {$#1$}}
%%    \psset{arrowscale=2}
%%    \psset{xAxisLabel=$X$,yAxisLabel= $Y$,
%%        xAxisLabelPos={c,-0.35in},yAxisLabelPos={-0.35in,c}}
%%    \centerline{%
%%    \scalebox{0.8}{%
%%        \psgraph[tickstyle=bottom,Dx=5,Dy=0.5,Ox=-20,Oy=-2.5]{->}(-20.0,-2.5)(24,2.5){4in}{2.5in}%
%%    \psset{dotstyle=Bo,dotscale=1.5,fillcolor=lightgray}
%%    \dataplot[plotstyle=dots,showpoints=true]{\dataS}
%%    \psset{dotstyle=Bo,dotscale=0.5,fillcolor=black}
%%    \listplot[plotstyle=dots,showpoints=true,plotNoMax=2,plotNo=2]{\dataPsinM}
%%    \psplot[linewidth=0.5pt,plotpoints=250]{-20}{20}{x RadtoDeg sin} 
%%    \psline[linewidth=0.5pt](-20,0)(20.5,0)
%%    %\psline[linewidth=0.5pt](-10.1,-1.1)(10.1,-1.1)
%%    %\psline[linewidth=0.5pt](10.1,-1.1)(10.1,1.1)
%%    %\psline[linewidth=0.5pt](10.1,1.1)(-10.1,1.1)
%%    %\psline[linewidth=0.5pt](-10.1,1.1)(-10.1,-1.1)
%%    \psframe[linewidth=0.5pt](-10.1,-1.1)(10.1,1.1)
%%    \endpsgraph
%%    }
%%    }
%%    \vspace{0.2in}
%%    \caption{MLP: Testing outside range. Good fit within the training
%%    range $[-10,10]$ shown in the box.}
%%    \label{fig:reg:neural:sine20test}
%%\end{figure}
%
%
%
%% $ python3 ./neuralnet.py ./sine-train_n25.txt ./sine-test.txt 10 -of identity -hf tanh -ef SSE -d " " -eta 0.005 -eps 0.0001 -predfile foo -ws 1 -t 100000
%% Using TensorFlow backend.
%% args Namespace(Nh=10, No=1, alpha=0.0, classification=False, delim=' ', emulate_logistic=False, eps=0.0001, error_func='SSE', eta=0.005, fname_test='./sine-test.txt', fname_train='./sine-train_n25.txt', h_fact='tanh', max_epochs=100000, o_fact='identity', predfile=<_io.TextIOWrapper name='foo' mode='w' encoding='cp1252'>, print_weights=True, scikit=False, weight_scale=1.0)
%% run:  ./sine-train_n25.txt 10 0.005 100000 0.0 tanh identity SSE
%% 25 2
%% 1 10 1
%% YMAP {}
%% test epoch 0 1747.08467762 1.74708467762 0.0
%% epoch 0 49.5311909913 1.98124763965 0.0 0.630710930991
%% test epoch 1000 268.742876194 0.268742876194 0.0
%% epoch 29000 0.0146660761746 0.000586643046984 0.0 0.0061037355853
%% test epoch 30000 1.45026161969 0.00145026161969 0.0
%% epoch 30000 0.00516831481532 0.000206732592613 0.0 0.00569482033655
%% test epoch 31000 1.64468151983 0.00164468151983 0.0
%% epoch 31000 0.00517583092402 0.000207033236961 0.0 0.00765361623064
%% test epoch 32000 1.3008066804 0.0013008066804 0.0
%% test epoch 99000 0.808919201104 0.000808919201104 0.0
%% epoch 99000 0.00131739910771 5.26959643083e-05 0.0 0.00112253767874
%% test epoch 100000 1.27890965513 0.00127890965513 0.0
%% epoch 100000 0.00664113782447 0.000265645512979 0.0 0.00369288465959
%
%% Weights: Input to Hidden
%% [[-0.67602583  0.77173744 -0.41654405 -0.72181249 -0.93181399 -0.41730082
%%   -0.65974388 -0.69908068 -0.62293081 -0.50419322  0.        ]
%%  [-4.36082592  2.42651427 -0.51500966  2.3490174  -1.63554714  3.98393186
%%    0.30621808  4.45275316  1.02691938 -4.76767523  0.        ]]
%
%% Weights: Hidden to Output
%% [[-1.81652886]
%%  [-1.68606964]
%%  [-0.81960689]
%%  [ 1.36510596]
%%  [ 0.13842468]
%%  [ 2.37200062]
%%  [-1.64174947]
%%  [-1.92307993]
%%  [ 0.78482387]
%%  [ 2.16741996]
%%  [-0.15689627]]
%
%% accuracy 1.27890965513 0.00127890965513 0.0
%
%%% see trained_nn_example.R
%% Wi = c(-0.67602583,  0.77173744, -0.41654405, -0.72181249, -0.93181399, -0.41730082,
%%        -0.65974388, -0.69908068, -0.62293081, -0.50419322)
%% Ti = c(-4.36082592,  2.42651427, -0.51500966,  2.3490174,  -1.63554714,  3.98393186,
%%        0.30621808,  4.45275316,  1.02691938, -4.76767523)
%% Wo = c(-1.81652886,
%%        -1.68606964,
%%        -0.81960689,
%%        1.36510596,
%%        0.13842468,
%%        2.37200062,
%%        -1.64174947,
%%        -1.92307993,
%%        0.78482387,
%%        2.16741996
%% )
%% To = -0.15689627
%
%
%
%\begin{example}[MLP with one hidden layer]
% WMJ Moved above prior to figures
%
%
%
%% \begin{figure}[!h]
%%     \centerline{
%%     \subfloat[$0$]{%
%%     \scalebox{0.25}{%
%%     \begin{tikzpicture}
%%         \begin{axis}[
%%             colormap/blackwhite,
%%             enlargelimits=false,
%%             axis on top,
%%             point meta min=0,
%%             point meta max=255,
%%         ]
%%        \addplot [matrix plot,point meta=explicit] file [meta=index 2]
%%             {REG/neural/figs/mnist_digits/digit_M0_3.txt};
%%         \end{axis}
%%     \end{tikzpicture}}}
%%     \subfloat[$1$]{%
%%     \scalebox{0.25}{%
%%     \begin{tikzpicture}
%%         \begin{axis}[
%%             colormap/blackwhite,
%%             enlargelimits=false,
%%             axis on top,
%%             point meta min=0,
%%             point meta max=255,
%%         ]
%%        \addplot [matrix plot,point meta=explicit] file [meta=index 2]
%%             {REG/neural/figs/mnist_digits/digit_M1_2.txt};
%%         \end{axis}
%%     \end{tikzpicture}}}
%%     \subfloat[$2$]{%
%%     \scalebox{0.25}{%
%%     \begin{tikzpicture}
%%         \begin{axis}[
%%             colormap/blackwhite,
%%             enlargelimits=false,
%%             axis on top,
%%             point meta min=0,
%%             point meta max=255,
%%         ]
%%        \addplot [matrix plot,point meta=explicit] file [meta=index 2]
%%             {REG/neural/figs/mnist_digits/digit_M2_1.txt};
%%         \end{axis}
%%     \end{tikzpicture}}}
%%     \subfloat[$3$]{%
%%     \scalebox{0.25}{%
%%     \begin{tikzpicture}
%%         \begin{axis}[
%%             % view={0}{90},   % not needed for `matrix plot*' variant
%%             colormap/blackwhite,
%%             enlargelimits=false,
%%             axis on top,
%%             point meta min=0,
%%             point meta max=255,
%%         ]
%%        \addplot [matrix plot,point meta=explicit] file [meta=index 2]
%%             {REG/neural/figs/mnist_digits/digit_M3_18.txt};
%%         \end{axis}
%%     \end{tikzpicture}}}
%%     \subfloat[$4$]{%
%%     \scalebox{0.25}{%
%%     \begin{tikzpicture}
%%         \begin{axis}[
%%             % view={0}{90},   % not needed for `matrix plot*' variant
%%             colormap/blackwhite,
%%             enlargelimits=false,
%%             axis on top,
%%             point meta min=0,
%%             point meta max=255,
%%         ]
%%        \addplot [matrix plot,point meta=explicit] file [meta=index 2]
%%             {REG/neural/figs/mnist_digits/digit_M4_4.txt};
%%         \end{axis}
%%     \end{tikzpicture}}}
%%     }
%%     \centerline{%
%%     \subfloat[$5$]{%
%%     \scalebox{0.25}{%
%%     \begin{tikzpicture}
%%         \begin{axis}[
%%             % view={0}{90},   % not needed for `matrix plot*' variant
%%             colormap/blackwhite,
%%             enlargelimits=false,
%%             axis on top,
%%             point meta min=0,
%%             point meta max=255,
%%         ]
%%        \addplot [matrix plot,point meta=explicit] file [meta=index 2]
%%             {REG/neural/figs/mnist_digits/digit_M5_8.txt};
%%         \end{axis}
%%     \end{tikzpicture}}}
%%     \subfloat[$6$]{%
%%     \scalebox{0.25}{%
%%     \begin{tikzpicture}
%%         \begin{axis}[
%%             % view={0}{90},   % not needed for `matrix plot*' variant
%%             colormap/blackwhite,
%%             enlargelimits=false,
%%             axis on top,
%%             point meta min=0,
%%             point meta max=255,
%%         ]
%%        \addplot [matrix plot,point meta=explicit] file [meta=index 2]
%%             {REG/neural/figs/mnist_digits/digit_M6_11.txt};
%%         \end{axis}
%%     \end{tikzpicture}}}
%%     \subfloat[$7$]{%
%%     \scalebox{0.25}{%
%%     \begin{tikzpicture}
%%         \begin{axis}[
%%             % view={0}{90},   % not needed for `matrix plot*' variant
%%             colormap/blackwhite,
%%             enlargelimits=false,
%%             axis on top,
%%             point meta min=0,
%%             point meta max=255,
%%         ]
%%        \addplot [matrix plot,point meta=explicit] file [meta=index 2]
%%             {REG/neural/figs/mnist_digits/digit_M7_0.txt};
%%         \end{axis}
%%     \end{tikzpicture}}}
%%     \subfloat[$8$]{%
%%     \scalebox{0.25}{%
%%     \begin{tikzpicture}
%%         \begin{axis}[
%%             % view={0}{90},   % not needed for `matrix plot*' variant
%%             colormap/blackwhite,
%%             enlargelimits=false,
%%             axis on top,
%%             point meta min=0,
%%             point meta max=255,
%%         ]
%%        \addplot [matrix plot,point meta=explicit] file [meta=index 2]
%%             {REG/neural/figs/mnist_digits/digit_M8_61.txt};
%%         \end{axis}
%%     \end{tikzpicture}}}
%%     \subfloat[$9$]{%
%%     \scalebox{0.25}{%
%%     \begin{tikzpicture}
%%         \begin{axis}[
%%             % view={0}{90},   % not needed for `matrix plot*' variant
%%             colormap/blackwhite,
%%             enlargelimits=false,
%%             axis on top,
%%             point meta min=0,
%%             point meta max=255,
%%         ]
%%        \addplot [matrix plot,point meta=explicit] file [meta=index 2]
%%             {REG/neural/figs/mnist_digits/digit_M9_7.txt};
%%         \end{axis}
%%     \end{tikzpicture}}}
%%     }
%%     \caption{MNIST Digits}
%%     \end{figure}
%
%
%%see mlp_mnist_output.txt for results
%\begin{example}[MLP for handwritten digit classification]
\begin{frame}{MLP for handwritten digit classification}
\framesubtitle{Example}
\small
    % We now illustrate the use of a hidden layer with a
    % non-linear activation function.
    In this example, we apply an MLP with one hidden layer for the 
    task of predicting the correct label for a hand-written digit
    from the MNIST database, which contains 60,000 training images that
    span the 10 digit labels, from $0$ to $9$. 
    Each (grayscale)
    image is a $28\times28$
    matrix of pixels, with values between 0 and 255. Each pixel is
    converted to a value in the interval $[0,1]$ by dividing by 255.
    %\cref{fig:reg:neural:mnist_ex} 
Figure shows an example of each digit from
    the MNIST dataset.

\medskip

    
    Since images are 2-dimensional matrices, we first {\em
    flatten} them into a vector $\bx \in \setR^{784}$ with
    dimensionality $d= 28\times 28=784 $. This is done by simply
    concatenating all of the rows of the images to obtain one long
    vector. 

\medskip

Next, since the output labels are categorical values that
    denote the digits from $0$ to $9$, we need to convert them into
    binary (numerical) vectors, using {\em one-hot} encoding. Thus, the
    label $0$ is encoded as $\be_1 = (1,0,0,0,0,0,0,0,0,0)^T \in
    \setR^{10}$, the label $1$ as $\be_2 =(0,1,0,0,0,0,0,0,0,0)^T \in
    \setR^{10}$, and so on, and finally the label $9$ is encoded as 
    $\be_{10} = (0,0,0,0,0,0,0,0,0,1)^T \in
    \setR^{10}$. That is, each input image vector $\bx$ has a
    corresponding target response vector $\by \in \{\be_1, \be_2,
    \cdots, \be_{10}\}$. 

\medskip

Thus, the input layer for the MLP has $d=784$
    neurons, and the output layer has $p=10$ neurons.
\end{frame}

\begin{frame}{MLP for handwritten digit classification}
\framesubtitle{Example}
    For the hidden layer, we consider several MLP models, each
    with a different number of hidden neurons $m$. We try 
    $m=0, 7, 49, 98, 196, 392$, to study the effect of increasing the
    number of hidden neurons, from small to large. For the hidden layer,
    we use ReLU activation function, and for the output layer, we use
    softmax activation, since the target response vector 
    has only one neuron with value $1$, with the rest being $0$.

\medskip

    Note that $m=0$ means that there is no hidden layer -- the input
    layer is directly connected to the output layer, which is equivalent
    to a multiclass logistic regression model.
    We train each MLP for $t=15$ epochs, using step size $\eta=0.25$.

\medskip

    During training, we plot the number of misclassified images after
    each epoch, on the 
    separate MNIST test set comprising 10,000 images. 
    %\cref{fig:reg:neural:mnist_errors_mlp} 

\medskip

Figure shows the number of errors
    from each of the models (with a different number of hidden neurons
    $m$), after each epoch. 

\end{frame}

\begin{frame}{MLP for handwritten digit classification}
\framesubtitle{Example}

The final test error at the end of training
    is given as
    \begin{center}
        \begin{tabular}{|c||c|c|c|c|c|c|c|}
            \hline
            $m$ & $0$ & $7$ & $10$ & $49$ & $98$ & $196$ &
            $392$\\
            \hline
            errors & 1677 & 901 & 792 & 546 & 495 & 470 & 454\\
            \hline
        \end{tabular}
    \end{center}

\medskip

    We can observe that adding a hidden layer significantly improves the
    prediction accuracy. Using even a small number of hidden neurons helps,
    compared to the logistic regression model ($m=0$).
    For example, using $m=7$ results in 901 errors (or error rate 9.01\%)
    compared to using $m=0$, which results in $1677$ errors (or error rate
    16.77\%).

\medskip

    On the other hand, as we increase the number of hidden neurons, the
    error rate decreases, though with diminishing returns. Using $m=196$,
    the error rate is $4.70\%$, but even after doubling the number of
    hidden neurons ($m=392$), the error rate goes down to only 4.54\%. 
    Further increasing $m$ does not reduce the error rate.
\end{frame}
%    \label{ex:reg:neural:mlp_mnist}
%\end{example}
%
%
%
\begin{frame}{MNIST dataset: sample handwritten digits}
%\begin{figure}[!t]
%    \centerline{
%    \subfloat[label $0$]{%
%    %\scalebox{0.25}{%
    \resizebox{0.75in}{0.75in}{%
%    % \tikzsetnextfilename{reg_neural_mnist_ex_0}
    \begin{tikzpicture}
        \begin{axis}[
            % view={0}{90},   % not needed for `matrix plot*' variant
            colormap/blackwhite,
            enlargelimits=false,
            axis on top,
            point meta min=0,
            point meta max=255,
        ]
       \addplot [matrix plot,point meta=explicit] file [meta=index 2]
            {REG/neural/figs/mnist_digits/digit_M0_10.txt};
        \end{axis}
    \end{tikzpicture}}%}
%    \subfloat[label $1$]{%
%    % \scalebox{0.25}{%
    \resizebox{0.75in}{0.75in}{%
%    % \tikzsetnextfilename{reg_neural_mnist_ex_1}
    \begin{tikzpicture}
        \begin{axis}[
            % view={0}{90},   % not needed for `matrix plot*' variant
            colormap/blackwhite,
            enlargelimits=false,
            axis on top,
            point meta min=0,
            point meta max=255,
        ]
       \addplot [matrix plot,point meta=explicit] file [meta=index 2]
            {REG/neural/figs/mnist_digits/digit_M1_5.txt};
        \end{axis}
    \end{tikzpicture}}%}
%    \subfloat[label $2$]{%
%    % \scalebox{0.25}{%
    \resizebox{0.75in}{0.75in}{%
%    % \tikzsetnextfilename{reg_neural_mnist_ex_2}
    \begin{tikzpicture}
        \begin{axis}[
            % view={0}{90},   % not needed for `matrix plot*' variant
            colormap/blackwhite,
            enlargelimits=false,
            axis on top,
            point meta min=0,
            point meta max=255,
        ]
       \addplot [matrix plot,point meta=explicit] file [meta=index 2]
            {REG/neural/figs/mnist_digits/digit_M2_35.txt};
        \end{axis}
    \end{tikzpicture}}%}
%    \subfloat[label $3$]{%
    \resizebox{0.75in}{0.75in}{%
%    % \scalebox{0.25}{%
%    % \tikzsetnextfilename{reg_neural_mnist_ex_3}
    \begin{tikzpicture}
        \begin{axis}[
            % view={0}{90},   % not needed for `matrix plot*' variant
            colormap/blackwhite,
            enlargelimits=false,
            axis on top,
            point meta min=0,
            point meta max=255,
        ]
       \addplot [matrix plot,point meta=explicit] file [meta=index 2]
            {REG/neural/figs/mnist_digits/digit_M3_30.txt};
        \end{axis}
    \end{tikzpicture}}%}
%    \subfloat[label $4$]{%
    \resizebox{0.75in}{0.75in}{%
%    % \scalebox{0.25}{%
%    % \tikzsetnextfilename{reg_neural_mnist_ex_4}
    \begin{tikzpicture}
        \begin{axis}[
            % view={0}{90},   % not needed for `matrix plot*' variant
            colormap/blackwhite,
            enlargelimits=false,
            axis on top,
            point meta min=0,
            point meta max=255,
        ]
       \addplot [matrix plot,point meta=explicit] file [meta=index 2]
            {REG/neural/figs/mnist_digits/digit_M4_6.txt};
        \end{axis}
    \end{tikzpicture}}%}
%    }

%    \centerline{%
%    \subfloat[label $5$]{%
    \resizebox{0.75in}{0.75in}{%
%    % \scalebox{0.25}{%
%    % \tikzsetnextfilename{reg_neural_mnist_ex_5}
    \begin{tikzpicture}
        \begin{axis}[
            % view={0}{90},   % not needed for `matrix plot*' variant
            colormap/blackwhite,
            enlargelimits=false,
            axis on top,
            point meta min=0,
            point meta max=255,
        ]
       \addplot [matrix plot,point meta=explicit] file [meta=index 2]
            {REG/neural/figs/mnist_digits/digit_M5_15.txt};
        \end{axis}
    \end{tikzpicture}}%}
%    \subfloat[label $6$]{%
    \resizebox{0.75in}{0.75in}{%
%    % \scalebox{0.25}{%
%    % \tikzsetnextfilename{reg_neural_mnist_ex_6}
    \begin{tikzpicture}
        \begin{axis}[
            % view={0}{90},   % not needed for `matrix plot*' variant
            colormap/blackwhite,
            enlargelimits=false,
            axis on top,
            point meta min=0,
            point meta max=255,
        ]
       \addplot [matrix plot,point meta=explicit] file [meta=index 2]
            {REG/neural/figs/mnist_digits/digit_M6_21.txt};
        \end{axis}
    \end{tikzpicture}}%}
%    \subfloat[label $7$]{%
    \resizebox{0.75in}{0.75in}{%
%    % \scalebox{0.25}{%
%    % \tikzsetnextfilename{reg_neural_mnist_ex_7}
    \begin{tikzpicture}
        \begin{axis}[
            % view={0}{90},   % not needed for `matrix plot*' variant
            colormap/blackwhite,
            enlargelimits=false,
            axis on top,
            point meta min=0,
            point meta max=255,
        ]
       \addplot [matrix plot,point meta=explicit] file [meta=index 2]
            {REG/neural/figs/mnist_digits/digit_M7_17.txt};
        \end{axis}
    \end{tikzpicture}}%}
%    \subfloat[label $8$]{%
    \resizebox{0.75in}{0.75in}{%
%    % \scalebox{0.25}{%
%    % \tikzsetnextfilename{reg_neural_mnist_ex_8}
    \begin{tikzpicture}
        \begin{axis}[
            % view={0}{90},   % not needed for `matrix plot*' variant
            colormap/blackwhite,
            enlargelimits=false,
            axis on top,
            point meta min=0,
            point meta max=255,
        ]
       \addplot [matrix plot,point meta=explicit] file [meta=index 2]
            {REG/neural/figs/mnist_digits/digit_M8_84.txt};
        \end{axis}
    \end{tikzpicture}}%}
%    \subfloat[label $9$]{%
    \resizebox{0.75in}{0.75in}{%
%    % \scalebox{0.25}{%
%    % \tikzsetnextfilename{reg_neural_mnist_ex_9}
    \begin{tikzpicture}
        \begin{axis}[
            % view={0}{90},   % not needed for `matrix plot*' variant
            colormap/blackwhite,
            enlargelimits=false,
            axis on top,
            point meta min=0,
            point meta max=255,
        ]
       \addplot [matrix plot,point meta=explicit] file [meta=index 2]
            {REG/neural/figs/mnist_digits/digit_M9_9.txt};
        \end{axis}
    \end{tikzpicture}}%}
%    }
%    \vspace{0.2in}
%    \caption{MNIST dataset: Sample handwritten digits.}
%    \label{fig:reg:neural:mnist_ex}
%\end{figure}
\end{frame}
%\begin{figure}[!t]
%    \centering
\pgfplotscreateplotcyclelist{my black white}{%
solid, every mark/.append style={solid, fill=black}, mark=*\\%
solid, every mark/.append style={solid, fill=black}, mark=square*\\%
solid, every mark/.append style={solid, fill=gray}, mark=otimes*\\%
densely dashed, every mark/.append style={solid, fill=gray},mark=square*\\%
solid, every mark/.append style={solid, fill=gray}, mark=*\\%
solid, every mark/.append style={solid, fill=gray},mark=diamond*\\%
solid, every mark/.append style={solid,
fill=gray},mark=triangle*\\%
}
% \tikzsetnextfilename{reg_neural_mnist_errors_mlp}
\begin{frame}{MNIST: Prediction error as a function of epochs}
\begin{tikzpicture}
    \begin{axis}[
            width=3.25in,
            height=2in,
            scale only axis,
            xmin=1, xmax=15,
            ymin=100, ymax=3000,
            label style={font=\small},
            tick label style={font=\tiny},
            xlabel=epochs,
        ylabel=errors,
        xtick={1,2,3,4,5,6,7,8,9,10,11,12,13,14,15},
        % enlargelimits=false,
        axis on top,
        legend style ={ at={(1.02,1)},
            anchor=north west, draw=black,
        fill=white, align=left},
        cycle list name=my black white,
        % smooth
    ]
    \addplot table [x index=0, y index=2] %
        {REG/neural/figs/test_loss/test_loss_output_m0.txt};
        \addlegendentry{$m=0$};
        \addplot table [x index=0, y index=2] %
        {REG/neural/figs/test_loss/test_loss_output_m7.txt};
        \addlegendentry{$m=7$};
        \addplot table [x index=0, y index=2] %
        {REG/neural/figs/test_loss/test_loss_output_m10.txt};
        \addlegendentry{$m=10$};
        \addplot table [x index=0, y index=2] %
        {REG/neural/figs/test_loss/test_loss_output_m49.txt};
        \addlegendentry{$m=49$};
        \addplot table [x index=0, y index=2] %
        {REG/neural/figs/test_loss/test_loss_output_m98.txt};
        \addlegendentry{$m=98$};
        \addplot table [x index=0, y index=2] %
        {REG/neural/figs/test_loss/test_loss_output_m196.txt};
        \addlegendentry{$m=196$};
        \addplot table [x index=0, y index=2] %
        {REG/neural/figs/test_loss/test_loss_output_m392.txt};
        \addlegendentry{$m=392$};
    \end{axis}
\end{tikzpicture}
\end{frame}
%\caption{MNIST: Prediction error as a function of epochs.}
%\label{fig:reg:neural:mnist_errors_mlp}
%\end{figure}
%
%
%
%
%
%
%
%
%
%
%
%
%
%
%% \readdata{\dataRt}{REG/neural/figs/kernelSine10.txt}
%% \readdata{\dataRy}{REG/neural/figs/kernelSine20.txt}
%% \begin{figure}[t!]
%%     \captionsetup[subfloat]{captionskip=20pt}
%%     \centering
%%     \def\pshlabel#1{\scriptsize {$#1$}}
%%     \def\psvlabel#1{\scriptsize {$#1$}}
%%     \psset{arrowscale=2}
%%     \psset{xAxisLabel=$X$,yAxisLabel= $Y$,
%%         xAxisLabelPos={c,-0.35in},yAxisLabelPos={-0.35in,c}}
%%     \centerline{%
%%         \subfloat[$(-10,10)$]{%
%%         \label{fig:reg:neural:kernel10}
%%     \scalebox{0.55}{%
%%         \psgraph[tickstyle=bottom,Dx=2.5,Dy=0.5,Ox=-12.5,Oy=-1.25]%
%%                 {->}(-12.0,-1.25)(12.5,1.25){4in}{2.5in}%
%%     \psset{dotstyle=Bo,dotscale=1.5,fillcolor=lightgray}
%%     \dataplot[plotstyle=dots,showpoints=true]{\dataS}
%%     \psset{dotstyle=Bo,dotscale=0.5,fillcolor=black}
%%     \listplot[plotstyle=dots,showpoints=true,plotNoMax=2,plotNo=2]{\dataRt}
%%     \psplot[linewidth=0.5pt]{-10}{10}{x RadtoDeg sin} 
%%     \psline[linewidth=0.5pt](-12,0)(12.5,0)
%%     \endpsgraph
%%     }}
%%     \hspace{0.5in}
%%     \subfloat[$(-20,20)$]{
%%         \label{fig:reg:neural:kernel20}
%%     \scalebox{0.55}{%
%%         \psgraph[tickstyle=bottom,Dx=5,Dy=0.5,Ox=-20,Oy=-2.5]%
%%         {->}(-20.0,-2.5)(20.5,2.5){4in}{2.5in}%
%%     \psset{dotstyle=Bo,dotscale=1.5,fillcolor=lightgray}
%%     \dataplot[plotstyle=dots,showpoints=true]{\dataS}
%%     \psset{dotstyle=Bo,dotscale=0.5,fillcolor=black}
%%     \listplot[plotstyle=dots,showpoints=true,plotNoMax=2,plotNo=2]{\dataRy}
%%     \psplot[linewidth=0.5pt]{-20}{20}{x RadtoDeg sin} 
%%     \psline[linewidth=0.5pt](-20,0)(20.5,0)
%%     \endpsgraph
%%     }}
%%     }
%%     \caption{Kernel Ridge Regression for Sine Curve: Gaussian kernel
%%     ($\sigma=6.5$).}
%%     \label{fig:reg:neural:kernel_regression}
%% \end{figure}
%
%
%
%
%
\begin{frame}{Deep Multilayer Perceptron}
%\begin{figure}[!b]
%\captionsetup[subfloat]{captionskip=10pt}
%    \centering
%    \psset{tnpos=l,tnsep=2pt,colsep=1.5,rowsep=0.75,mcol=c,
%    ArrowInside=->, arrowscale=2}
\centerline{
%    \subfloat[detail view]{
%    \label{fig:reg:neural:deepMLPdetailed}
    \scalebox{0.5}{%
\psmatrix
[mnode=none]{$l=0$} 
    & [mnode=none]{$l=1$} 
    & [mnode=none]{$l=2$} 
    & [mnode=none]{$\ldots$} 
    & [mnode=none]{$l=h$} 
    & [mnode=none]{$l=h+1$}\\[-1em]
\Tcircle[name=z00,doubleline=true]{$x_0$}~[tnpos=l]{$1$} 
    & \Tcircle[name=z10,doubleline=true]{$z^1_0$}~[tnpos=l]{$1$} 
    & \Tcircle[name=z20,doubleline=true]{$z^2_0$}~[tnpos=l]{$1$} 
    & \Tcircle[name=zi0,linecolor=gray,doubleline=true]{$\cdots$}~[tnpos=l]{$1$} 
    & \Tcircle[name=zh0,doubleline=true]{$z^h_0$}~[tnpos=l]{$1$}\\[1em] 
\Tcircle[name=z01]{$x_1$} 
    & \Tcircle[name=z11]{$z^1_1$}
    & \Tcircle[name=z21]{$z^2_1$} 
    & \Tcircle[name=zi1,linecolor=gray]{$\cdots$} 
    & \Tcircle[name=zh1]{$z^h_1$} 
    & \Tcircle[name=y1]{$o_1$}\\[-1em]
[mnode=none]{$\vdots$} 
    & [mnode=none]{$\vdots$} 
    & [mnode=none]{$\vdots$} 
    & [mnode=none]{$\vdots$} 
    & [mnode=none]{$\vdots$} 
    & [mnode=none]{$\vdots$}\\[-1em]
\Tcircle[name=z0k]{$x_i$}
    & \Tcircle[name=z1k]{$z^1_i$} 
    & \Tcircle[name=z2k]{$z^2_i$} 
    & \Tcircle[name=zik,linecolor=gray]{$\cdots$} 
    & \Tcircle[name=zhk]{$z^h_i$} 
    & \Tcircle[name=yk]{$o_{i}$} \\[-1em]
[mnode=none]{$\vdots$} 
    & [mnode=none]{$\vdots$} 
    & [mnode=none]{$\vdots$} 
    & [mnode=none]{$\vdots$} 
    & [mnode=none]{$\vdots$} 
    & [mnode=none]{$\vdots$}\\[-1em]
\Tcircle[name=z0h]{$x_{d}$} 
    & \Tcircle[name=z1h]{$z^1_{n_{1}}$} 
    & \Tcircle[name=z2h]{$z^2_{n_{2}}$} 
    & \Tcircle[name=zih,linecolor=gray]{$\cdots$} 
    & \Tcircle[name=zhh]{$z^h_{n_{h}}$} 
    & \Tcircle[name=yh]{$o_{\!p}$}
\psset{ArrowInsidePos=.95}
\psset{linecolor=gray,linewidth=0.5pt}
%bias to next layer
\ncline{z20}{zi1}
\ncline{z20}{zik}
\ncline{z20}{zih}
\ncline{zi0}{zh1}
\ncline{zi0}{zhk}
\ncline{zi0}{zhh}
%hidden2 to hidden-dots
\ncline{z21}{zi1}
\ncline{z21}{zik}
\ncline{z21}{zih}
\ncline{z2k}{zi1}
\ncline{z2k}{zik}
\ncline{z2k}{zih}
\ncline{z2h}{zi1}
\ncline{z2h}{zik}
\ncline{z2h}{zih}
%hidden-dots to hidden-h
\ncline{zi1}{zh1}
\ncline{zi1}{zhk}
\ncline{zi1}{zhh}
\ncline{zik}{zh1}
\ncline{zik}{zhk}
\ncline{zik}{zhh}
\ncline{zih}{zh1}
\ncline{zih}{zhk}
\ncline{zih}{zhh}
\psset{linecolor=black,linewidth=0.5pt}
%bias to next layer
\ncline{z00}{z11}
\ncline{z00}{z1k}
\ncline{z00}{z1h}
\ncline{z10}{z21}
\ncline{z10}{z2k}
\ncline{z10}{z2h}
\ncline{zh0}{y1}
\ncline{zh0}{yk}
\ncline{zh0}{yh}
%input to hidden
\ncline{z01}{z11}
\ncline{z01}{z1k}
\ncline{z01}{z1h}
\ncline{z0k}{z11}
\ncline{z0k}{z1k}
\ncline{z0k}{z1h}
\ncline{z0h}{z11}
\ncline{z0h}{z1k}
\ncline{z0h}{z1h}
%hidden1 to hidden2
\ncline{z11}{z21}
\ncline{z11}{z2k}
\ncline{z11}{z2h}
\ncline{z1k}{z21}
\ncline{z1k}{z2k}
\ncline{z1k}{z2h}
\ncline{z1h}{z21}
\ncline{z1h}{z2k}
\ncline{z1h}{z2h}
%#hidden-h to output
\ncline{zh1}{y1}
\ncline{zh1}{yk}
\ncline{zh1}{yh}
\ncline{zhk}{y1}
\ncline{zhk}{yk}
\ncline{zhk}{yh}
\ncline{zhh}{y1}
\ncline{zhh}{yk}
\ncline{zhh}{yh}
\endpsmatrix
}}
%}
%\vspace{0.2in}
%\centerline{
%    \subfloat[layer view]{
%\label{fig:reg:neural:deepMLP_layered}

\centerline{
\scalebox{0.5}{%
\psmatrix
[mnode=none]{$l=0$} 
    & [mnode=none]{$l=1$} 
    & [mnode=none]{$l=2$} 
    & [mnode=none]{$\ldots$} 
    & [mnode=none]{$l=h$} 
    & [mnode=none]{$l=h+1$}\\[-1em]
    %\TR[name=l0]{\psframe(-1,-1)(1,1)\rput(0,0){$\bx$}}
\TR[name=l0]{\myfbox{$\bx$}}
    & \TR[name=l1]{\myfbox{$\bz^1$}}
    & \TR[name=l2]{\myfbox{$\bz^2$}}
    & \TR[name=li]{\myfboxB[linecolor=gray]{$\cdots\!$}}
    & \TR[name=lh]{\myfbox{$\bz^h$}}
    & \TR[name=lo]{\myfbox{$\bo$}}
\psset{ArrowInside=-}
\psset{arrows=->}
\psset{linewidth=2pt}
%bias to next layer
\ncline{l0}{l1}\nbput[npos=0.5]{$\bW_0, \bb_0$}
\ncline{l1}{l2}\nbput[npos=0.5]{$\bW_1, \bb_1$}
\ncline[linecolor=gray,linewidth=1pt]{l2}{li}%\nbput[npos=0.5]{$\bW_2, \bb_2$}
\ncline[linecolor=gray,linewidth=1pt]{li}{lh}%\nbput[npos=0.5]{$\bW_{h-1},\bb_{h-1}$}
\ncline{lh}{lo}\nbput[npos=0.5]{$\bW_h, \bb_{h}$}
\endpsmatrix
}}
%}
%\vspace{0.2in}
%\caption{Deep multilayer perceptron, with $h$ hidden layers.}
%\label{fig:reg:neural:deepMLP}
%\end{figure}
\end{frame}
%
%
%
%\section{Deep Multilayer Perceptrons}
%\index{deep multilayer perceptron}
%\index{deep learning}
%\index{deep MLP|see{deep multilayer perceptron}}
%We now generalize the feed-forward and backpropagation steps for
%many hidden layers, as well as arbitrary error
%and neuron activation functions.
%
%Consider an MLP with $h$ hidden layers as shown in
%\cref{fig:reg:neural:deepMLP}.
%We assume that the input to the MLP comprises $n$ points $\bx_i \in
%\setR^d$ with the corresponding true response vector $\by_i \in
%\setR^p$. 
%We denote the input neurons as
%layer $l=0$, the first hidden layer as $l=1$, the last hidden layer as
%$l=h$, and finally the output
%layer as layer $l=h+1$. 
%We use $n_l$ to denote the number of neurons in layer $l$.
%Since the input points are $d$-dimensional, this implies
%$n_0 = d$, and since the true response vector is $p$-dimensional, we
%have $n_{h+1} = p$. The hidden layers have $n_1$ neurons for the first
%hidden layer, $n_2$ for the second layer, and $n_h$ for the last hidden
%layer.
%The vector of neuron values for layer $l$ (for $l=0,\cdots,h+1$) 
%is denoted as
%\begin{align*}
%    \bz^l = \lB(z^l_{1}, \cdots, z^l_{n_l}\rB)^T
%\end{align*}
%Each layer except
%the output layer has one extra bias neuron, which is the neuron
%at index $0$. Thus, the bias neuron for layer $l$ is denoted $z^l_0$ and
%its value is fixed at $z^l_{0}=1$.
%\cref{fig:reg:neural:deepMLPdetailed} displays a detailed view of an MLP
%with $h$ hidden layers, showing the individual neurons in each layer
%including the bias neuron.
%Note that the vector of input neuron values is also written as
%\begin{align*}
%    \bx = \lB(x_1,x_2,\cdots,x_d\rB)^T 
%    = \lB(z^0_{1}, z^0_{2}, \cdots, z^0_{d}\rB)^T = \bz^0
%\end{align*}
%and the vector of output neuron values is also denoted as
%\begin{align*}
%    \bo = (o_1, o_2, \cdots, o_p)^T  = \lB(z^{h+1}_{1}, z^{h+1}_{2},
%    \cdots, z^{h+1}_{p}\rB)^T = \bz^{h+1}  
%\end{align*}
%
%The weight matrix between 
%layer $l$ and layer $l+1$ neurons
%is denoted $\bW_{\!l} \in \setR^{n_{l} \times n_{l+1}}$, 
%and the vector of bias terms from the bias neuron 
%$z^l_0$ to neurons in layer $l+1$ is
%denoted $\bb_l \in \setR^{n_{l+1}}$,
%for $l=0,1,\cdots,h$.
%Thus, $\bW_0 \in \setR^{d \times n_{1}}$ is the weight matrix between the input and first hidden
%layer, $\bW_1 \in \setR^{n_{1} \times n_{2}}$ is the weight matrix
%between the first and second hidden layer, and so on; finally,
%$\bW_{\!h} \in \setR^{n_{h} \times p}$ 
%is the weight matrix between the last hidden layer
%and the output layer. For the bias vectors, $\bb_0 \in \setR^{n_1}$
%specifies the biases for neurons in the first hidden layer, $\bb_1
%\in \setR^{n_2}$ the biases for neurons in the second hidden layer, and
%so on. Thus, $\bb_{h} \in \setR^p$ specifies the biases for the
%output neurons.
%\cref{fig:reg:neural:deepMLP_layered} shows a layer view for an MLP with
%$h$ hidden layers. This is a more compact representation that clearly 
%specifies the architecture or topology of the MLP. Each layer $l$ is
%represented as a rectangular node, and is marked with the vector of neuron values,
%$\bz^{l}$. The bias neurons are not shown, but are assumed to be present
%in each layer except the output.
%An edge between layer $l$ and
%$l+1$ is
%labeled with the weight matrix $\bW_l$ and bias vector
%$\bb_{l}$ that specify the parameters between those layers.
%
%For training deep MLPs, we will refer to several partial
%derivative vectors, described next.
%Define $\delta^{l}_i$ as the net gradient, i.e., the partial
%derivative of the error function
%with respect to the net value at $z^{l}_i$
%\begin{align}
%\delta^l_i = \frac{\partial \cE_\bx}{\partial \net_i}
%\label{eq:reg:neural:deltali}
%\end{align}
%and let $\bdelta^l$ denote the net gradient vector at layer $l$, for
%$l=1,2,\cdots,h+1$
%\begin{align}
%    \bdelta^l = \bigl(\delta^l_1, \cdots, \delta^l_{n_l} \bigr)^T 
%    \label{eq:reg:neural:bdelta}
%\end{align}
%Let $f^l$ denote the
%activation function for layer $l$, for $l=0,1,\cdots,h+1$, and further
%let $\partial\bff^{\;l}$ denote the vector of the derivatives of the activation
%function with respect to $\net_{i}$ for all neurons $z_i^l$ in layer $l$:
%\begin{align}
%    \partial\bff^{\;l} = \lB(\frac{\partial f^l(\net_{1})}{\partial \net_{1}},
%    \cdots, \frac{\partial f^l(\net_{n_l})}{\partial \net_{n_l}}\rB)^T
%    \label{eq:reg:neural:bff}
%\end{align}
%% Let $\partial\bff^{\;h+1}$ denote the vector of the derivatives of the activation
%% function with respect to the $\net_{\!i}$ for all output neurons:
%% \begin{align}
%%     \partial\bff^{\;h+1} = 
%%     \lB(\frac{\partial f^{h+1}(\net_{1})}{\partial \net_{1}},
%%         \frac{\partial f^{h+1}(\net_{2})}{\partial \net_{2}}, 
%%     \;\cdots,\; \frac{\partial f^{h+1}(\net_{p})}{\partial
%% \net_{p}}\rB)^T
%% \label{eq:reg:neural:bffO}
%% \end{align}
%Finally, let $\partial\cE_\bx$ denote the vector of partial
%derivatives of the error function with respect to the values
%$o_i$ for all output neurons:
%\begin{align}
%    \partial\bcE_\bx = 
%    \lB(\frac{\partial \cE_\bx}{\partial o_{1}},
%        \frac{\partial \cE_\bx}{\partial o_{2}},
%    \;\cdots,\; \frac{\partial \cE_\bx}{\partial o_{p}}\rB)^T
%    \label{eq:reg:neural:vE}
%\end{align}
%
%\subsection{Feed-forward Phase} 
%\index{deep multilayer perceptron!feed-forward phase}
%\index{deep learning!feed-forward phase}
\begin{frame}{Feed-forward Phase} 
%
Typically in a deep MLP, the same activation function $f^l$ is used for
all neurons in a given layer $l$.  The input layer always uses the identity
activation, so $f^0$ is the identity function. Also, all bias neurons also
use the identity function with a fixed value of $1$.  The output layer
typically uses sigmoid or softmax activations for classification tasks,
or identity activations for regression tasks. The hidden layers typically
use sigmoid, tanh, or ReLU activations.  %In our discussion, we
%assume a fixed activation function $f^l$ for all neurons in a given layer.
%However, it is easy to generalize to a neuron specific activation function
%$f^l_i$ for neuron $z_i^l$ in layer $l$.
%
For a given input pair $(\bx, \by) \in \bD$,
the deep MLP computes the output vector via the
feed-forward process:
\begin{samepage}
\begin{align*}
    \bo & = f^{h+1}\Bigl(\bb_{h} + \bW_{\!h}^T \cdot \bz^h\Bigr)\\ 
    & = f^{h+1}\Bigl(\bb_{h} + \bW_{\!h}^T \cdot 
    f^h\lB(\bb_{h-1} + \bW_{\!h-1}^T \cdot \bz^{h-1} \rB) \Bigr)\\
    & = \qquad \vdots\\
    & = f^{h+1}\Biggl(\bb_{h} + \bW_{\!h}^T \cdot 
        f^h\biggl(\bb_{h-1} + \bW_{\!h-1}^T \cdot f^{h-1}%
            \Bigl(\cdots f^2 \Bigl(\bb_{1} +  \bW_{1}^T \cdot
                f^1 \bigl(\bb_{0} +  \bW_{0}^T \cdot \bx
\bigr) \Bigr) \Bigr) \biggr) \Biggr) 
\end{align*}
\end{samepage}
\end{frame}
%Note that each $f^l$ distributes over its argument. That is,
%$$f^l\lB(\bb_{l-1} + \bW_{l-1}^T \cdot \bx\rB) 
%= \Bigl( f^l\lB(\net_{1}\rB), f^l\lB(\net_{2}\rB), \cdots,
%f^l\lB(\net_{n_l}\rB) \Bigr)^T $$
%% If each neuron has its own specific activation function $f^l_i$, 
%% we would write it as
%% \begin{empheq}[box=\tcbhighmath]{align}
%% f^1\lB(\bb_0 + \bW_0^T \cdot \bx\rB) 
%% = \Bigl( f^1_1\lB(\net_{1}\rB), f^1_2\lB(\net_{2}\rB), \cdots,
%% f^1_{n_1}\lB(\net_{n_1}\rB) \Bigr)^T 
%% \end{empheq}
%
%
%\subsection{Backpropagation Phase}
%\index{deep multilayer perceptron!backpropagation phase}
%\index{deep learning!backpropagation phase}
%\index{backpropagation!deep multilayer perceptron}
%\index{backpropagation!deep learning}
%
\begin{frame}{Backpropagation Phase}
Consider the weight update between a given layer and another, including
between the input and hidden layer, or between two hidden layers, or
between the last hidden layer and the output layer. 
Let $z^l_i$ be a neuron in layer $l$, and $z^{l+1}_{\!j}$ a neuron in the next
layer $l+1$. Let $w^l_{ij}$ be the weight between $z^l_i$ and
$z^{l+1}_j$, and let $b^l_{\!j}$ denote the bias term between
$z^l_0$ and $z^{l+1}_j$.
The weight and bias are updated using the gradient descent approach
%\index{deep multilayer perceptron!weight gradient}
%\index{deep learning!weight gradient}
%\index{deep multilayer perceptron!bias gradient}
%\index{deep learning!bias gradient}
\begin{align*}
   w^l_{i\!j} & = w^l_{ij} - \eta \cdot \grad_{w^l_{ij}} &
   b^l_{\!j} & = b^l_{\!j} - \eta \cdot \grad_{b^l_{\!j}}
\end{align*}
where $\grad_{w^l_{ij}}$ is the weight gradient and
$\grad_{b^l_{\!j}}$ is the bias gradient, i.e., the partial
derivative of the error function with
respect to the weight and bias, respectively.
\end{frame}
%\begin{align*}
%\grad_{w^l_{ij}} & = \frac{\partial \cE_\bx}{\partial w^l_{ij}} &
%\grad_{b^l_{\!j}} & = \frac{\partial \cE_\bx}{\partial b^l_{\!j}}
%\end{align*}
%
\begin{frame}{Backpropagation Phase}
%As noted earlier in \cref{eq:reg:neural:gradwij}, 
We can use the chain
rule to write the weight and bias gradient, as follows
\begin{align}
    \begin{aligned}
    \grad_{w^l_{ij}} = 
 \frac{\partial \cE_\bx}{\partial w^l_{ij}}
     & = \frac{\partial \cE_\bx}{\partial \net_{\!j}} \cdot 
    \frac{\partial \net_{\!j}}{\partial w^l_{ij}}
    = \delta^{l+1}_j  \cdot z^l_i
    = z^l_i \cdot \delta^{l+1}_j \notag\\ 
    \grad_{b^l_{\!j}} = 
 \frac{\partial \cE_\bx}{\partial b^l_{\!j}}
     & = \frac{\partial \cE_\bx}{\partial \net_{\!j}} \cdot 
    \frac{\partial \net_{\!j}}{\partial b^l_{\!j}}
    = \delta^{l+1}_j
\end{aligned}
%    \label{eq:reg:neural:gradwij_deep}
\end{align}
%where $\delta^{l+1}_j$ is the net gradient [\cref{eq:reg:neural:deltali}], 
%i.e., the partial
%derivative of the error function
%with respect to the net value at $z^{l+1}_j$,
%and we have
%\begin{align*}
%   \frac{\partial \net_{\!j}}{\partial w^l_{ij}} & = 
%   \frac{\partial}{\partial w^l_{ij}} 
%   \lB\{b^l_{\!j} + \sum_{k=0}^{n_l} w^l_{k\!j} \cdot z^l_{k} \rB\}
%     = z^l_i &
%     % = 
%   % \frac{\partial}{\partial w^l_{ij}}
%    % \lB\{\sum_{k\ne i}^m w^l_{k\!j} \cdot z^l_{k} + 
%    % w^l_{i\!j} \cdot z^l_i \rB\}
%   \frac{\partial \net_{\!j}}{\partial b^l_{\!j}} & = 
%   \frac{\partial}{\partial b^l_{\!j}} 
%   \lB\{b^l_{\!j} + \sum_{k=0}^{n_l} w^l_{k\!j} \cdot z^l_{k} \rB\}
%     = 1
%\end{align*}
%
%Given the vector of neuron values at layer $l$, namely
%    $\bz^l = \lB(z^l_{1}, \cdots, z^l_{n_l}\rB)^T$,
%we can compute the entire weight gradient matrix via an outer product
%operation
%\begin{align}
%    \tcbhighmath{
%    \bgrad_{\bW_l} = \bz^l \cdot \lB(\bdelta^{l+1}\rB)^T}
%\label{eq:reg:neural:gradW_deep}
%\end{align}
%and the bias gradient vector as:
%\begin{align}
%    \tcbhighmath{
%    \bgrad_{\bb_l} = \bdelta^{l+1}}
%\label{eq:reg:neural:gradB_deep}
%\end{align}
%with $l=0,1,\cdots,h$. Here $\bdelta^{l+1}$ is the net gradient vector
%at layer $l+1$ [\cref{eq:reg:neural:bdelta}].
%
%This also allows us to update all the weights and biases as follows
In summary, the update of the weights and biases is
\begin{empheq}[box=\tcbhighmath]{align}
    \begin{aligned}
    \bW_l & = \bW_l - \eta \cdot \bgrad_{\bW_l}\\
    \bb_l & = \bb_l - \eta \cdot \bgrad_{\bb_l}
\end{aligned}
\end{empheq}
where $\eta$ is the step size.
However, we observe that to compute
the weight and bias gradients for layer $l$ 
we need to compute
the net gradients $\bdelta^{l+1}$ at layer $l+1$. 
\end{frame}
%
%\subsection{Net Gradients at Output Layer}
%\label{sec:reg:neural:netgradients}
%\index{deep multilayer perceptron!net gradient}
%\index{deep learning!net gradient}
\begin{frame}{Net Gradients at Output Layer}
Let us consider how to compute the net gradients at the output layer
$h+1$. If all of the output neurons are independent (for example, when
using linear or sigmoid activations), the net gradient is
obtained by differentiating the error function with
respect to the net signal at the output neurons. That is,
\begin{align*}
    \delta^{h+1}_{\!j} = \frac{\partial \cE_\bx}{\partial \net_{\!j}} =
    \frac{\partial \cE_\bx}{\partial f^{h+1}(\net_{\!j})} \cdot
\frac{\partial f^{h+1}(\net_{\!j})}{\partial \net_{\!j}} = 
\frac{\partial \cE_\bx}{\partial o_{\!j}} \cdot 
\frac{\partial f^{h+1}(\net_{\!j})}{\partial \net_{\!j}} 
\end{align*}
Thus, the gradient depends on two terms, the partial derivative of the
error function with respect to the output neuron value, and the
derivative of the activation function with respect to its argument.
%The net gradient vector across all output neurons is given as
%\begin{align}
%    \tcbhighmath{
%    \bdelta^{h+1} = \partial\bff^{\;h+1} \;\odot\; \partial\bcE_\bx}
%\end{align}
%where $\odot$ is the element-wise or Hadamard product, 
%$\partial\bff^{\;h+1}$ is the vector of derivatives of the activation function
%with respect to its argument [\cref{eq:reg:neural:bff}] at the output 
%layer $l=h+1$, and $\partial\bcE_\bx$ is the vector of error derivatives with
%respect to the output neuron values [\cref{eq:reg:neural:vE}].
%
On the other hand, if the output neurons are not independent (for
example, when using a softmax activation), then we have to modify the
computation of the net gradient at each output neuron as follows:
\begin{align*}
    \delta^{h+1}_{\!j} = \frac{\partial \cE_\bx}{\partial \net_{\!j}} =
    \sum^{p}_{i=1} \frac{\partial \cE_\bx}{\partial f^{h+1}(\net_{\!i})} \cdot
\frac{\partial f^{h+1}(\net_{\!i})}{\partial \net_{\!j}}
\end{align*}
%Across all output neurons, we can write this compactly as follows:
%\begin{align}
%    \tcbhighmath{
%    \bdelta^{h+1}
%= \partial\bF^{\;h+1} \cdot \partial\bcE_\bx}
%    \label{eq:reg:neural:FE_dependent}
%\end{align}
%where $\partial\bF^{\;h+1}$ is the matrix of derivatives of $o_i =
%f^{h+1}(\net_i)$ with respect to $net_{\!j}$ for all $i,j=1,2,\cdots,p$, given as
%\begin{align*}
%    \partial\bF^{\;h+1} & = 
%    \matr{
%        \displaystyle\frac{\partial o_1}{\partial \net_{1}} &
%        \displaystyle\frac{\partial o_1}{\partial \net_{2}} &
%          \cdots & 
%          \displaystyle\frac{\partial o_1}{\partial \net_{p}}\\[1em]
%        \displaystyle\frac{\partial o_2}{\partial \net_{1}} &
%        \displaystyle\frac{\partial o_2}{\partial \net_{2}} &
%          \cdots & 
%         \displaystyle\frac{\partial o_2}{\partial \net_{p}}\\
%         \vdots & \vdots & \cdots & \vdots\\
%         \displaystyle\frac{\partial o_p}{\partial \net_{1}} &
%        \displaystyle\frac{\partial o_p}{\partial \net_{2}} &
%          \cdots & 
%         \displaystyle\frac{\partial o_p}{\partial \net_{p}}
%     } 
%\end{align*}
%
Typically, for regression tasks, we use the squared error function with
linear activation function at the output neurons, whereas for 
logistic regression and classification, we use the cross-entropy error
function with a sigmoid activation for binary classes, and softmax
activation for multiclass problems. %For these common cases, 
%the net gradient vector at the output
%layer is given as follows:
\end{frame}
%
%\begin{description}
%\index{deep multilayer perceptron!squared error}
%\index{neural networks!squared error}
%\index{deep learning!squared error}
%    \item[\bf Squared Error:] 
%        From \cref{eq:reg:neural:deriv_SE},
%        the error gradient is given as
%        \begin{align*}
%            \partial \bcE_{\bx} = \frac{\partial\bcE_\bx}{\partial \bo} = \bo - \by    
%        \end{align*}
%    The net gradient at the output layer is given as
%    \begin{align*}
%        \bdelta^{h+1} = \partial\bff^{\;h+1} \odot \partial\bcE_\bx
%    \end{align*}
%    where $\partial\bff^{\;h+1}$ depends on the activation function at the output.
%    Typically, for regression tasks, we use a linear activation at the
%    output neurons. In that case, we have $\partial\bff^{\;h+1} = \bone$
%    (see \cref{eq:reg:neural:deriv_identity}). 
%    
%
%\index{deep multilayer perceptron!binary cross-entropy error}
%\index{deep learning!binary cross-entropy error}
%\index{neural networks!binary cross-entropy error}
%    \item[\bf Cross-Entropy Error (binary output, sigmoid activation):] Consider the binary case first, with
%        a single output neuron $o$ with sigmoid activation. 
%        Recall that the binary cross-entropy error [\cref{eq:reg:neural:CEbinary}] is given as
%        \begin{align*}
%            \cE_\bx = -\bigl( y\cdot\ln(o) +
%            (1-y)\cdot\ln(1-o) \bigr)
%        \end{align*}
%        From \cref{eq:reg:neural:deriv_CE_binary} we have
%        \begin{align*}
%            \partial \bcE_\bx = \frac{\partial \cE_\bx}{\partial o} & = \frac{o-y}{o \cdot (1-o)}
%        \end{align*}
%        Further, for sigmoid activation, we have  
%        \begin{align*}
%        \partial\bff^{\;h+1} = \frac{\partial f(net_o)}{\partial \net_o}
%        = o\cdot(1-o)
%        \end{align*}
%    Therefore, the net gradient at the output neuron is
%    \begin{align*}
%        \delta^{h+1} & = \partial \bcE_\bx \cdot \partial
%        \bff^{\;h+1} = \frac{o-y}{o \cdot (1-o)} \cdot o\cdot(1-o) = o-y
%    \end{align*}
%
%
%\index{deep multilayer perceptron!cross-entropy error}
%\index{deep learning!cross-entropy error}
%\index{neural networks!cross-entropy error}
%\item[\bf Cross-Entropy Error ($K$ outputs, softmax activation):] 
%    Recall that the cross-entropy error function [\cref{eq:reg:neural:CE}] is given as
%\begin{align*}
%    \cE_\bx = - \sum_{i=1}^K y_i \cdot \ln(o_i) = 
%    - \Bigl( y_1 \cdot \ln(o_1) + \cdots 
%            + y_K \cdot \ln(o_K) \Bigr) 
%\end{align*}
%
%Using \cref{eq:reg:neural:deriv_CE}, 
%the vector of error derivatives
%with respect to the output
%neurons is given as
%\begin{align*}
%    \partial\bcE_\bx & = 
%    \lB(\frac{\partial \cE_\bx}{\partial o_{1}},
%        \frac{\partial \cE_\bx}{\partial o_{2}},
%    \;\cdots,\; \frac{\partial \cE_\bx}{\partial o_{K}}\rB)^T
%     = \lB( -\frac{y_1}{o_1}, -\frac{y_2}{o_2}, \cdots, 
%             -\frac{y_K}{o_K}
%    \rB)^T
%\end{align*}
%where $p = K$ is the number of output neurons.
%    
%    Cross-entropy error is typically used with the softmax
%    activation so that we get a (normalized) probability 
%    value for each
%    class. That is, 
%    \begin{align*}
%        o_{\!j} & = \softmax( net_{\!j}) = \frac{\exp\{ net_{\!j} \}}{ \sum_{i=1}^K \exp \{
%        net_{i} \} }
%    \end{align*}
%    so that the output
%    neuron values sum to one, $\sum_{j=1}^K o_{\!j} = 1$.
%    Since an output neuron depends on all other output
%    neurons, 
%    we need to compute the matrix of derivatives of each output with
%    respect to each of the net signals at the output neurons [see
%    \cref{eq:reg:neural:softmax_der,eq:reg:neural:pd_ef_o}]:
%    \begin{align*}
%    \partial\bF^{\;h+1} & = 
%    \begin{small}
%    \matr{
%        \displaystyle\frac{\partial o_1}{\partial \net_{1}} &
%        \displaystyle\frac{\partial o_1}{\partial \net_{2}} &
%          \cdots & 
%          \displaystyle\frac{\partial o_1}{\partial \net_{K}}\\[1em]
%        \displaystyle\frac{\partial o_2}{\partial \net_{1}} &
%        \displaystyle\frac{\partial o_2}{\partial \net_{2}} &
%          \cdots & 
%         \displaystyle\frac{\partial o_2}{\partial \net_{K}}\\
%         \vdots & \vdots & \cdots & \vdots\\
%         \displaystyle\frac{\partial o_K}{\partial \net_{1}} &
%        \displaystyle\frac{\partial o_K}{\partial \net_{2}} &
%          \cdots & 
%         \displaystyle\frac{\partial o_K}{\partial \net_{K}}
%     } 
%     = \matr{
%         o_1\cdot (1-o_1) & -o_1\cdot o_2 & \cdots & -o_1 \cdot o_K\\
%         -o_1 \cdot o_2 & o_2\cdot (1-o_2) & \cdots & -o_2 \cdot o_K\\
%         \vdots & \vdots & \cdots & \vdots\\
%         -o_1 \cdot o_K & -o_2 \cdot o_K & \cdots & o_K\cdot (1-o_K)
%     }
%    \end{small}
%%\label{eq:reg:neural:bF}
%\end{align*}
%
%Therefore, the net gradient vector at the output layer is
%\begin{align}
%    \bdelta^{h+1} & 
%    = \partial\bF^{\;h+1} \cdot \partial\bcE_\bx\\
%    & = \matr{
%         o_1\cdot (1-o_1) & -o_1\cdot o_2 & \cdots & -o_1 \cdot o_K\\
%         -o_1 \cdot o_2 & o_2\cdot (1-o_2) & \cdots & -o_2 \cdot o_K\\
%         \vdots & \vdots & \cdots & \vdots\\
%         -o_1 \cdot o_K & -o_2 \cdot o_K & \cdots & o_K\cdot (1-o_K)
%     }
%     \cdot \matr{-\frac{y_1}{o_1}\\[0.25em]
%         -\frac{y_2}{o_2}\\[0.25em]
%         \vdots\\ 
%             -\frac{y_K}{o_K}
%         }\notag\\
%    & = \matr{
%        -y_1 + y_1\cdot o_1 + \sum_{i\ne 1}^K y_i\cdot o_1\\[0.25em]
%        -y_2 + y_2\cdot o_2 + \sum_{i\ne 2}^K y_i\cdot o_2\\[0.25em]
%        \vdots\\
%        -y_K + y_K\cdot o_K + \sum_{i\ne K}^K y_i\cdot o_K\\
%    }
%    = \matr{
%        -y_1 + o_1 \cdot \sum_{i=1}^K y_i\\[0.25em]
%        -y_2 + o_2 \cdot \sum_{i=1}^K y_i\\[0.25em]
%        \vdots\\
%        -y_K + o_K \cdot \sum_{i=1}^K y_i\\
%    } \notag\\
%    & = \matr{
%        -y_1 + o_1\\
%        -y_2 + o_2\\
%        \vdots\\
%        -y_K + o_K\\
%        }, \text{since } \sum_{i=1}^K y_i = 1\notag\\
%    & =  \bo - \by\notag
%\end{align}
%
%\end{description}
%
%
%\subsection{Net Gradients at Hidden Layers}
\begin{frame}{Net Gradients at Hidden Layers}
%\index{deep multilayer perceptron!net gradient}
%\index{deep learning!net gradient}
Let us assume that we have already computed the net gradients at layer
$l+1$, namely $\bdelta^{l+1}$.
Since neuron $z^l_{\!j}$ in layer $l$ is connected to all of the neurons in
layer $l+1$ (except for the bias neuron $z^{l+1}_0$), 
to compute the net gradient at
$z^l_{\!j}$, we
have to account for the error from each neuron in layer $l+1$, as
follows:
\begin{align*}
    \delta^l_{\!j} = \frac{\partial \cE_\bx}{\partial \net_{\!j}}
    & =
    \sum^{n_{l+1}}_{k=1} \frac{\partial \cE_\bx}{\partial \net_k} \cdot
    \frac{\partial \net_{k}}{\partial f^l(\net_{\!j})} \cdot
    \frac{\partial f^l(\net_{\!j})}{\partial \net_{\!j}}\\
& = \frac{\partial f^l(\net_{\!j})}{\partial \net_{\!j}} \cdot 
\sum^{n_{l+1}}_{k=1} \delta^{l+1}_k \cdot w^l_{\!jk}
\end{align*}
So the net gradient at $z^l_{\!j}$ in layer $l$ depends on the derivative of the 
activation function with respect to its $\net_j$, and the 
weighted sum of the net gradients from
all the neurons $z^{l+1}_k$ at the next layer $l+1$.
\end{frame}
%
%We can compute the net gradients for all the
%neurons in level $l$ in one step, as follows:
%\begin{align}
%    \tcbhighmath{
%    \bdelta^l = \partial\bff^{\;l} \;\odot\; \bigl(\bW_l \cdot
%\bdelta^{l+1}\bigr)}
%    \label{eq:reg:neural:bdelta_l}
%\end{align}
%where $\odot$ is the element-wise product, and
%$\partial\bff^{\;l}$ is the vector of derivatives of the activation function
%with respect to its argument [\cref{eq:reg:neural:bff}] at layer $l$.
\begin{frame}{Net Gradients at Hidden Layers}
For the commonly used activation functions at the hidden layer, %using
%the derivatives from \cref{sec:reg:neural:deriv_activations}, 
we have
\begin{align*}
    \partial \bff^{\;l} = 
    \begin{cases}
        \bone & \text{for linear}\\
        \bz^l (\bone - \bz^l) & \text{for sigmoid}\\
        (\bone - \bz^l \odot \bz^l) & \text{for }\tanh\\
    \end{cases}
\end{align*}
%For ReLU, we have to apply \cref{eq:reg:neural:deriv_relu} to each
%neuron. Note that softmax is generally not used for hidden layers.
%
The net gradients are computed recursively, starting from the output
layer $h+1$, then hidden layer $h$, and so on, until we finally compute
the net gradients at the first hidden layer $l=1$. That is,
\begin{samepage}
\begin{align*}
\bdelta^h & = \partial\bff^{\;h} \;\odot\; \lB(\bW_h \cdot
\bdelta^{h+1}\rB) \\
\bdelta^{h-1} & = 
\partial\bff^{\;h-1} \;\odot\; \lB(\bW_{\!h-1} \cdot
\bdelta^{h}\rB) =    
\partial\bff^{\;h-1} \;\odot\; 
    \Bigl(\bW_{\!h-1} \cdot 
    \bigl( \partial\bff^{\;h} \;\odot\; \lB(\bW_h \cdot \bdelta^{h+1}\rB) 
    \bigr) 
    \Bigr) \\
  & \qquad \vdots\\
   \bdelta^1 
%    & = \partial\bff^{\;1} \;\odot\; \bigl(\bW_1 \cdot \bdelta^{2}\bigr)\\
%    & = \partial\bff^{\;1} \;\odot\; \Bigl(\bW_1 \cdot 
%    \bigl( \partial\bff^{\;2} \;\odot\; \lB(\bW_2 \cdot \bdelta^{3}\rB) \bigr)
%    \Bigr)\\
%    & = \qquad \vdots\\
   & = \partial\bff^{\;1} \;\odot\; 
   \Biggl(\bW_1 \cdot 
    \biggl( \partial\bff^{\;2} \;\odot\; 
    \Bigl(\bW_2 \cdot 
    \cdots
    \bigl( \partial\bff^{\;h} \;\odot\; \lB(\bW_h \cdot \bdelta^{h+1}\rB) 
    \bigr) 
    \Bigr) 
    \biggr)
   \Biggr)
\end{align*}
\end{samepage}
\end{frame}
%
%
\begin{frame}{Deep MLP Training: Stochastic Gradient Descent}
\begin{small}
\begin{tightalgo}[H]{\textwidth-18pt}
    \SetKwInOut{Algorithm}{\textsc{Deep-MLP-Training} ($\bD, h, \eta,
    \texttt{maxiter}, n_1, n_2, \cdots, n_h, f^1, f^2, \cdots, f^{h+1}$)}
\Algorithm{}
 $n_0 \assign d$ \tcp*[h]{input layer size}\;
 $n_{h+1} \assign p$ \tcp*[h]{output layer size}\;
\tcp{Initialize weight matrices and bias vectors}
\For{$l=0,1,2,\cdots,h$}{%
    $\bb_l \assign\; $ random $n_{l+1}$ vector with small
    values\;
    $\bW_l \assign\; $ random $n_l \times n_{l+1}$ matrix with small
    values}
$t \assign 0$ \tcp*[h]{iteration counter}\;
\end{tightalgo}
\end{small}
\end{frame}

\begin{frame}{Deep MLP Training: Stochastic Gradient Descent}
\hspace*{-1cm}
\begin{small}
\begin{tightalgo}[H]{\textwidth+18pt}
\setcounter{AlgoLine}{6}
\Repeat{$t \ge \text{\tt maxiter}$}
{%
    \ForEach{$(\bx_i,\by_i) \in \bD$ in random order}{%
        $\bz^0 \assign \bx_i$ \tcp{Feed-Forward Phase}
        \lFor{$l=0,1,2,\ldots,h$}{
            $\bz^{l+1} \assign f^{l+1}\Bigl(\bb_l + \bW_l^T \cdot \bz^l \Bigr)$}
        $\bo_i \assign \bz^{h+1}$\;
        % \[\bdelta^{h+1} \assign 
        % \begin{cases}
        % \partial\bff^{\;h+1} \;\odot\; \partial\bcE_{\bx_i} & \text {if
        % independent outputs}\\
        % \partial\bF^{\;h+1} \cdot
        % \partial\bcE_{\bx_i} & \text{otherwise}
% \end{cases}\]
        % \tcp*[h]{net gradients at output}\; 
        \eIf(\texttt{// Backpropagation phase}){independent outputs}{ %\tcp{Backpropagation Phase}
            $\bdelta^{h+1} \assign \partial\bff^{\;h+1} \;\odot\;
        \partial\bcE_{\bx_i}$\tcp*[h]{net gradients at output} \;
        }{%
       $\bdelta^{h+1} \assign  \partial\bF^{\;h+1} \cdot
        \partial\bcE_{\bx_i}$\tcp*[h]{net gradients at output}\;
        }
        \lFor{$l=h,h-1,\cdots,1$}{
            $\bdelta^l \assign \partial\bff^{\;l} \;\odot\; \bigl(\bW_l \cdot
            \bdelta^{l+1}\bigr)$
            \tcp*[h]{net gradients}}% at layer $l$}}
        \For(\texttt{// Gradient Descent Step}){$l=0,1,\cdots,h$}{ %\tcp{Gradient Descent Step}
            $\bgrad_{\bW_{l}} \assign \bz^l \cdot \bigl(\bdelta^{l+1}\bigr)^T$  
            \tcp*[h]{weight gradient matrix at layer $l$}\;
            $\bgrad_{\!\bb_{l}} \assign \bdelta^{l+1}$  
            \tcp*[h]{bias gradient vector at layer $l$}\;
        }
        \For{$l=0,1,\cdots,h$}{
            $\bW_{l} \assign \bW_{l} - \eta \cdot \bgrad_{\bW_{l}}$ 
            \tcp*[h]{update $\bW_{l}$}\;
            $\bb_{l} \assign \bb_{l} - \eta \cdot
            \bgrad_{\bb_{l}}$ 
            \tcp*[h]{update $\bb_{l}$}\;}
    }
    $t \assign t+1$\;
}%
\end{tightalgo}
\end{small}
\end{frame}
%\caption{Deep MLP Training: Stochastic Gradient Descent}
%\label{alg:reg:neural:deepmlp}
%\end{tightalgo}
%
%
%\vspace*{-0.3in}
%\subsection{Training Deep MLPs}
%\index{deep learning!training}
%\index{deep multilayer perceptron!training}
%\index{neural networks!training}
%
%\cref{alg:reg:neural:deepmlp} shows the
%pseudo-code for
%learning the weights and biases for a deep MLP.
%The inputs comprise the dataset $\bD$,
%%= \lB\{\bx_i^T, \by_i\rB\}_{i=1,\cdots,n}$, 
%the number of hidden layers $h$,
%the step size or learning rate for gradient descent $\eta$,
%an integer threshold {\tt maxiter} denoting the number of iterations for
%training, parameters $n_1, n_2, \cdots, n_h$ that denote the
%number of neurons (excluding
%bias, which will be added automatically) for each of the hidden layers
%$l=1,2,\cdots,h$, and the type of activation functions
%$f^1, f^2, \cdots, f^{h+1}$ for each of the layers (other than the input
%layer that uses identity activations).
%The size of the input ($d$) and output ($p$) layers is
%determined directly from $\bD$.
%
%
%The MLP first initializes 
%the $(n_l \times n_{l+1})$ weight matrices $\bW_l$ between layers
%$l$ and $l+1$ with small values chosen uniformly at random, e.g., in the
%range $[-0.01,0.01]$. 
%The MLP considers each input pair $(\bx_i, \by_i) \in
%\bD$, and computes the predicted response $\bo_i$ via the feed-forward
%process. The backpropagation phase begins by computing the error between
%$\bo_i$ and true response $\by_i$, and computing the net gradient vector
%$\bdelta^{h+1}$ at the output layer. These net gradients are
%backpropagated from layer $h+1$ to layer $h$, from $h$ to $h-1$, and so
%on until we obtain the net gradients at the first hidden layer $l=1$.
%These net gradients are used to compute the weight gradient matrix
%$\bgrad_{\bW_l}$ at layer $l$, which can in turn be used to update the
%weight matrix $\bW_l$. Likewise, the net gradients specify the bias
%gradient vector $\bgrad_{\!\bb_{l}}$ at layer $l$, which is used to
%update $\bb_{l}$.
%After each point has been used to update the
%weights, that completes one iteration or epoch of training.
%The training stops when {\tt maxiter} epochs have been reached.
%On the other hand, during testing, for any input $\bx$, we apply the
%feed-forward steps and print the predicted output $\bo$.
%
%\index{multilayer perceptron!stochastic gradient descent}
%\index{stochastic gradient descent}
%\index{neural networks!stochastic gradient descent}
%It is important to note that \cref{alg:reg:neural:deepmlp} follows a
%stochastic gradient descent approach, since 
%the points are considered in random order, and the weight and bias
%gradients are computed after observing each training point. 
%In practice, it is
%common to update the gradients by considering a fixed sized 
%subset of the training points called a {\em minibatch} instead of using
%single points. That is, the training data is divided into minibatches
%using an additional parameter called {\em batch size}, and a gradient
%descent step is performed after computing the bias and weight gradient
%from each minibatch.
%This helps better estimate the gradients, and also
%allows vectorized matrix operations over the minibatch of points, which
%can lead to faster convergence and substantial speedups in the
%learning. 
%% \index{deep multilayer perceptron!batch size}
%\index{deep multilayer perceptron!minibatch learning}
%% \index{deep learning!batch size}
%\index{deep learning!minibatch learning}
%\index{minibatch learning}
%
%\index{deep multilayer perceptron!vanishing gradient}
%\index{deep multilayer perceptron!exploding gradient}
%\index{deep learning!vanishing gradient}
%\index{deep learning!exploding gradient}
%\index{vanishing gradient}
%\index{exploding gradient}
\begin{frame}{Vanishing or Exploding Gradients}
%One caveat while training very deep MLPs is the problem of vanishing
%and exploding gradients. 
In the {\em vanishing gradient} problem, the norm of the net gradient can
decay exponentially with the distance from the output layer, that is, as
we backpropagate the gradients from the output layer to the input layer.
In this case the network will learn extremely slowly, if at all, since
the gradient descent method will make minuscule changes to the weights
and biases. 

\medskip

On the other hand, in the {\em exploding gradient} problem, 
the norm of the net gradient can grow exponentially
with the distance from the output layer. In this case, the weights and
biases will become exponentially large, resulting in a failure to learn.
The gradient explosion problem can be mitigated to some extent 
by {\em gradient thresholding}, that is, by resetting the value if it
exceeds an upper bound. 

\medskip

The vanishing gradients problem is more
difficult to address. Typically sigmoid activations are more susceptible
to this problem, and one solution is to use alternative activation
functions such as ReLU. In general, recurrent neural networks, which 
are deep neural networks with {\em feedback} connections, are more prone to
vanishing and exploding gradients.%; we will revisit these issues  in
\end{frame}
%\cref{sec:reg:deep:LSTM}.
%
%% \begin{figure}[!ht]
%% \centerline{
%% \scalebox{0.4}{%
%%     \Large
%% \psmatrix
%% [mnode=none]{$l=0$} 
%%     & [mnode=none]{$l=1$} 
%%     & [mnode=none]{$l=2$}\\[-3em] 
%% \TR[name=w0]{\myfbox{
%% $\bW_0 = \amatr{r}{0.4638\\
%% -0.4890\\
%% -0.3576\\
%% -1.3367\\
%%  1.0434}$\\~\\
%% $\bb_0  = \amatr{r}{
%% -1.9776\\
%% -1.2075\\
%%  0.3886\\
%% -3.2220\\
%% -2.1704}$
%%  }} &
%% \TR[name=w1]{\myfbox{
%% $\bW_1 = \amatr{r}{
%%  0.6617&  0.2607&  0.0738&  1.0666&  0.6877\\
%% -0.1105& -0.3099&  1.1666& -0.8218&  0.4280\\
%%  0.1276&  0.2138&  0.9959&  0.0097&  0.3180\\
%% -0.3977&  0.0937& -0.4843&  0.6051&  0.1378\\
%% -0.1607& -0.3974&  0.2491& -0.2231& -0.1669}$\\~\\
%% $\bb_1 = \amatr{r}{
%%     -2.8153\\
%%  0.0007\\
%% -0.3644\\
%% -0.8676\\
%%  1.6138}$
%%  }} &
%% \TR[name=w2]{\myfbox{
%% $\bW_2  = \amatr{r}{
%%  1.1419& -0.1673& -0.2480&  0.4529& -0.8849\\
%% -0.1565& -0.5049&  0.9812& -0.0152&  0.1220\\
%% -0.9481&  1.1906&  0.6961& -0.1330&  0.0390\\
%%  0.3197& -0.4164& -0.4143& -0.2397& -0.0909\\
%%  0.5029&  0.5316&  0.5418&  0.0520& -0.0547}$\\
%% $\bb_2 = \amatr{r}{
%%     -1.6861\\
%% -0.2840\\
%% -0.0666\\
%% -0.2217\\
%% -0.9404}$
%%  }}%
%%  \\
%%   & [mnode=none]{$l=3$} 
%%     & [mnode=none]{$l=4$}\\[-3em]
%% & \TR[name=w3]{\myfbox{
%% $\bW_3 = \amatr{r}{
%%  0.4947& -0.2944& -1.2717& -0.2165&  0.3168\\
%%  0.0392& -0.9671&  0.4323&  0.1788& -0.5804\\
%%  0.8031&  0.5081&  0.8192& -0.3131&  0.4112\\
%% -0.1011& -0.2265&  0.4354&  0.3456& -0.5670\\
%% -0.2308&  0.3691& -0.5308&  0.1618& -0.1812}$\\~\\
%% $\bb_3 = \amatr{r}{
%%  0.4946\\
%%  0.5053\\
%% -1.3389\\
%%  0.2651\\
%%  0.7855}$
%%  }} &
%% \TR[name=w4]{\myfbox{
%% $\bW_4 = \amatr{r}{
%%  0.2735& -0.1405&  0.2914&  0.3434&  0.3361\\
%% -0.1564& -0.4411&  0.4550&  0.1145& -0.0671\\
%%  0.1110&  0.1876& -0.1310& -0.1022&  0.0700\\
%% -1.1419&  0.8067& -0.5453&  0.5210& -0.8485\\
%%  0.5680&  0.3033&  0.6067& -0.1326& -0.3219}$\\~\\
%% $\bb_4 = \amatr{r}{
%% -0.1972\\
%% -0.8991\\
%% -0.2521\\
%%  0.6895\\
%% -1.6376}$
%%  }}\\
%% [mnode=none]{$l=5$}
%%     & [mnode=none]{$l=6$} 
%%     & [mnode=none]{$l=7$}\\[-3em]
%% \TR[name=w5]{\myfbox{
%% $\bW_5 = \amatr{r}{
%% -0.0448& -0.1771& -0.2649& -0.6728& -0.4034\\
%%  0.0184&  0.3338& -0.2538& -0.3805&  0.6332\\
%% -0.7608& -0.4046& -0.0080&  0.9755& -0.2662\\
%% -0.7241&  0.5369& -0.1765& -0.1966& -0.2071\\
%%  0.0103& -0.4657& -0.4335&  0.3140& -0.1524}$\\~\\
%% $\bb_5 = \amatr{r}{
%%  0.8455\\
%% -1.4781\\
%%  0.4122\\
%%  0.7232\\
%%  0.0418}$
%%  }} &
%% \TR[name=w6]{\myfbox{
%% $\bW_6 = \amatr{r}{
%%  0.2661& -0.6841& -0.3484&  0.1431& -0.4535\\
%%  0.3161& -0.6000& -0.5746& -0.1459&  0.0317\\
%% -0.1101&  0.0255&  0.8850&  0.5128&  0.0837\\
%% -0.0913& -0.2297& -0.1306&  0.2683& -0.0414\\
%% -0.2465& -0.3955& -1.0111& -0.8854&  0.0149}$\\~\\
%% $\bb_6 = \amatr{r}{
%%     0.5508\\
%%  1.2972\\
%% -0.3870\\
%% -0.3943\\
%%  1.7865}$
%%  }} &
%% \TR[name=w7]{\myfbox{
%% $\bW_7 = \amatr{r}{ 0.6622\\ 0.9750\\-0.6919\\0.1791\\-1.4889}^T$\\~\\
%% $\bb_7 = \matr{ 0.4266 }$
%% }}
%% \endpsmatrix
%% }}
%% \caption{Deep multilayer perceptron for sine curve.}
%% \label{fig:reg:neural:deepMLP_sine_arch}
%% \end{figure}
%
%
%% \begin{figure}[!ht]
%% \centerline{
%% \scalebox{0.45}{%
%%     \large
%% \parbox{.95\linewidth}{%
%% \begin{align*}    
%% \bW_0 & = \amatr{r}{0.4638 &
%% -0.4890 & 
%% -0.3576 &
%% -1.3367 &
%%  1.0434} &
%% \bb_0 & = \amatr{r}{
%% -1.9776\\
%% -1.2075\\
%%  0.3886\\
%% -3.2220\\
%% -2.1704} &
%% \bW_1 & = \amatr{r}{
%% 0.6617 & -0.1105 & 0.1276 & -0.3977 & -0.1607 \\
%% 0.2607 & -0.3099 & 0.2138 & 0.0937 & -0.3974 \\
%% 0.0738 & 1.1666 & 0.9959 & -0.4843 & 0.2491 \\
%% 1.0666 & -0.8218 & 0.0097 & 0.6051 & -0.2231 \\
%% 0.6877 & 0.428 & 0.318 & 0.1378 & -0.1669} &
%% \bb_1 & = \amatr{r}{
%%     -2.8153\\
%%  0.0007\\
%% -0.3644\\
%% -0.8676\\
%%  1.6138}\\
%% \bW_2 & = \amatr{r}{
%% 1.1419 & -0.1565 & -0.9481 & 0.3197 & 0.5029 \\
%% -0.1673 & -0.5049 & 1.1906 & -0.4164 & 0.5316 \\
%% -0.248 & 0.9812 & 0.6961 & -0.4143 & 0.5418 \\
%% 0.4529 & -0.0152 & -0.133 & -0.2397 & 0.052 \\
%% -0.8849 & 0.122 & 0.039 & -0.0909 & -0.0547} &
%% \bb_2 & = \amatr{r}{
%%     -1.6861\\
%% -0.2840\\
%% -0.0666\\
%% -0.2217\\
%% -0.9404} &
%% \bW_3 & = \amatr{r}{
%% 0.4947 & 0.0392 & 0.8031 & -0.1011 & -0.2308 \\
%% -0.2944 & -0.9671 & 0.5081 & -0.2265 & 0.3691 \\
%% -1.2717 & 0.4323 & 0.8192 & 0.4354 & -0.5308 \\
%% -0.2165 & 0.1788 & -0.3131 & 0.3456 & 0.1618 \\
%% 0.3168 & -0.5804 & 0.4112 & -0.567 & -0.1812} &
%% \bb_3 & = \amatr{r}{
%%  0.4946\\
%%  0.5053\\
%% -1.3389\\
%%  0.2651\\
%%  0.7855}\\
%% \bW_4  &= \amatr{r}{
%% 0.2735 & -0.1564 & 0.111 & -1.1419 & 0.568 \\
%% -0.1405 & -0.4411 & 0.1876 & 0.8067 & 0.3033 \\
%% 0.2914 & 0.455 & -0.131 & -0.5453 & 0.6067 \\
%% 0.3434 & 0.1145 & -0.1022 & 0.521 & -0.1326 \\
%% 0.3361 & -0.0671 & 0.07 & -0.8485 & -0.3219} &
%% \bb_4 & = \amatr{r}{
%% -0.1972\\
%% -0.8991\\
%% -0.2521\\
%%  0.6895\\
%% -1.6376} &
%% \bW_5  & = \amatr{r}{
%% -0.0448 & 0.0184 & -0.7608 & -0.7241 & 0.0103 \\
%% -0.1771 & 0.3338 & -0.4046 & 0.5369 & -0.4657 \\
%% -0.2649 & -0.2538 & -0.008 & -0.1765 & -0.4335 \\
%% -0.6728 & -0.3805 & 0.9755 & -0.1966 & 0.314 \\
%% -0.4034 & 0.6332 & -0.2662 & -0.2071 & -0.1524} &
%% \bb_5 & = \amatr{r}{
%%  0.8455\\
%% -1.4781\\
%%  0.4122\\
%%  0.7232\\
%%  0.0418} \\
%% \bW_6 & = \amatr{r}{
%% 0.2661 & 0.3161 & -0.1101 & -0.0913 & -0.2465 \\
%% -0.6841 & -0.6 & 0.0255 & -0.2297 & -0.3955 \\
%% -0.3484 & -0.5746 & 0.885 & -0.1306 & -1.0111 \\
%% 0.1431 & -0.1459 & 0.5128 & 0.2683 & -0.8854 \\
%% -0.4535 & 0.0317 & 0.0837 & -0.0414 & 0.0149} &
%% \bb_6 & = \amatr{r}{
%%     0.5508\\
%%  1.2972\\
%% -0.3870\\
%% -0.3943\\
%%  1.7865} &
%% \bW_7 & = \amatr{r}{ 0.6622 \\ 0.9750 \\-0.6919 \\0.1791 \\ -1.4889} &
%% \bb_7 & = \matr{ 0.4266 }
%% \end{align*}
%% }}}
%% \caption{Parameters (weight matrices and bias vectors) for the deep MLP for sine curve: 7 hidden layers
%%     with 5 hidden neurons per layer. Input and output layers have a
%% single neuron.}
%% \label{fig:reg:neural:deepMLP_sine_arch}
%% \end{figure}
%
%%\begin{figure}[!t]
%%\centerline{
%%\scalebox{0.55}{%
%%    \large
%%\parbox{.95\linewidth}{%
%%\begin{align*}    
%%\bW_0 & = \amatr{r}{0.46 &
%%-0.49 & 
%%-0.36 &
%%-1.34 &
%% 1.04} &
%%\bb_0 & = \amatr{r}{
%%-1.98\\
%%-1.21\\
%% 0.39\\
%%-3.22\\
%%-2.17} &
%%\bW_1 & = \amatr{r}{
%%0.66 & -0.11 & 0.13 & -0.39 & -0.16 \\
%%0.26 & -0.31 & 0.21 & 0.09 & -0.39 \\
%%0.07 & 1.17 & 0.99 & -0.48 & 0.25 \\
%%1.07 & -0.82 & 0.01 & 0.61 & -0.22 \\
%%0.69 & 0.43 & 0.32 & 0.14 & -0.17} &
%%\bb_1 & = \amatr{r}{
%%    -2.82\\
%% 0.00\\
%%-0.36\\
%%-0.87\\
%% 1.61}\\
%%\bW_2 & = \amatr{r}{
%%1.14 & -0.16 & -0.95 & 0.32 & 0.50 \\
%%-0.17 & -0.50 & 1.19 & -0.42 & 0.53 \\
%%-0.25 & 0.98 & 0.69 & -0.41 & 0.54 \\
%%0.45 & -0.02 & -0.13 & -0.24 & 0.05 \\
%%-0.88 & 0.12 & 0.04 & -0.09 & -0.05} &
%%\bb_2 & = \amatr{r}{
%%    -1.69\\
%%-0.28\\
%%-0.07\\
%%-0.22\\
%%-0.94} &
%%\bW_3 & = \amatr{r}{
%%0.49 & 0.04 & 0.80 & -0.10 & -0.23 \\
%%-0.29 & -0.97 & 0.51 & -0.23 & 0.37 \\
%%-1.27 & 0.43 & 0.82 & 0.44 & -0.53 \\
%%-0.22 & 0.18 & -0.31 & 0.35 & 0.16 \\
%%0.32 & -0.58 & 0.41 & -0.57 & -0.18} &
%%\bb_3 & = \amatr{r}{
%% 0.49\\
%% 0.51\\
%%-1.34\\
%% 0.27\\
%% 0.79}\\
%%\bW_4  &= \amatr{r}{
%%0.27 & -0.16 & 0.11 & -1.14 & 0.57 \\
%%-0.14 & -0.44 & 0.19 & 0.81 & 0.30 \\
%%0.29 & 0.46 & -0.13 & -0.55 & 0.61 \\
%%0.34 & 0.11 & -0.10 & 0.52 & -0.13 \\
%%0.34 & -0.07 & 0.07 & -0.85 & -0.32} &
%%\bb_4 & = \amatr{r}{
%%-0.19\\
%%-0.89\\
%%-0.25\\
%% 0.69\\
%%-1.64} &
%%\bW_5  & = \amatr{r}{
%%-0.04 & 0.02 & -0.76 & -0.72 & 0.01 \\
%%-0.18 & 0.33 & -0.40 & 0.54 & -0.47 \\
%%-0.27 & -0.25 & -0.01 & -0.18 & -0.43 \\
%%-0.67 & -0.38 & 0.98 & -0.19 & 0.31 \\
%%-0.40 & 0.63 & -0.27 & -0.21 & -0.15} &
%%\bb_5 & = \amatr{r}{
%% 0.85\\
%%-1.48\\
%% 0.41\\
%% 0.72\\
%% 0.04} \\
%%\bW_6 & = \amatr{r}{
%%0.27 & 0.32 & -0.11 & -0.09 & -0.25 \\
%%-0.68 & -0.60 & 0.03 & -0.23 & -0.39 \\
%%-0.35 & -0.57 & 0.89 & -0.13 & -1.01 \\
%%0.14 & -0.15 & 0.51 & 0.27 & -0.89 \\
%%-0.45 & 0.03 & 0.08 & -0.04 & 0.01} &
%%\bb_6 & = \amatr{r}{
%%    0.55\\
%% 1.29\\
%%-0.39\\
%%-0.39\\
%% 1.79} &
%%\bW_7 & = \amatr{r}{ 0.66 \\ 0.98 \\-0.69 \\0.18 \\ -1.49} &
%%\bb_7 & = \matr{ 0.43 }
%%\end{align*}
%%}}}
%%\caption{Parameters (weight matrices and bias vectors) for the deep MLP for sine curve: 7 hidden layers
%%    with 5 hidden neurons per layer. Input and output layers have a
%%single neuron.}
%%\label{fig:reg:neural:deepMLP_sine_arch}
%%\end{figure}
%
%
%%\readdata{\dataS}{REG/neural/figs/sine-train_n100.txt}
%%\readdata{\dataPfiveH}{REG/neural/figs/Nh5_Nl6/foo_e500.txt}
%%\readdata{\dataPoneK}{REG/neural/figs/Nh5_Nl6/foo_e1000.txt}
%%\readdata{\dataPtwofiveK}{REG/neural/figs/Nh5_Nl6/foo_e2500.txt}
%%\readdata{\dataPtenK}{REG/neural/figs/Nh5_Nl6/foo_e10000.txt}
%%\readdata{\dataPthirtyK}{REG/neural/figs/Nh5_Nl6/foo_e30000.txt}
%%\readdata{\dataPsinM}{REG/neural/figs/Nh5_Nl6/test2_predict.txt}
%%\begin{figure}[htb!]
%%    \captionsetup[subfloat]{captionskip=20pt}
%%    \centering
%%    \def\pshlabel#1{\scriptsize {$#1$}}
%%    \def\psvlabel#1{\scriptsize {$#1$}}
%%    \psset{arrowscale=2}
%%    \psset{xAxisLabel=$X$,yAxisLabel= $Y$,
%%        xAxisLabelPos={c,-0.35in},yAxisLabelPos={-0.35in,c}}
%%    \centerline{%
%%    \subfloat[$t=500$]{%
%%        \label{fig:reg:neural:sine_deep_500}
%%    \scalebox{0.55}{%
%%    \psgraph[tickstyle=bottom,Dx=2,Dy=0.25,Ox=-12,Oy=-1.25]{->}(-12.0,-1.25)(12.5,1.25){4in}{2.25in}%
%%    \psset{dotstyle=Bo,dotscale=1.5,fillcolor=lightgray}
%%    \dataplot[plotstyle=dots,showpoints=true]{\dataS}
%%    \psset{dotstyle=Bo,dotscale=0.5,fillcolor=black}
%%    \listplot[plotstyle=dots,showpoints=true,plotNoMax=2,plotNo=2]{\dataPfiveH}
%%    \psplot[linewidth=0.5pt,plotpoints=200]{-10}{10}{x RadtoDeg sin} 
%%    \psline[linewidth=0.5pt](-12,0)(12.5,0)
%%    \endpsgraph
%%    }}
%%    \hspace{0.5in}
%%    \subfloat[$t=1000$]{%
%%        \label{fig:reg:neural:sine_deep_1k}
%%        \scalebox{0.55}{%
%%    \psgraph[tickstyle=bottom,Dx=2,Dy=0.25,Ox=-12,Oy=-1.25]{->}(-12.0,-1.25)(12.5,1.25){4in}{2.25in}%
%%    \psset{dotstyle=Bo,dotscale=1.5,fillcolor=lightgray}
%%    \dataplot[plotstyle=dots,showpoints=true]{\dataS}
%%    \psset{dotstyle=Bo,dotscale=0.5,fillcolor=black}
%%    \listplot[plotstyle=dots,showpoints=true,plotNoMax=2,plotNo=2]{\dataPoneK}
%%    \psplot[linewidth=0.5pt,plotpoints=200]{-10}{10}{x RadtoDeg sin} 
%%    \psline[linewidth=0.5pt](-12,0)(12.5,0)
%%    \endpsgraph
%%    }}
%%    }
%%    \vspace{-0.1in}
%%    \centerline{%
%%    \subfloat[$t=2500$]{%
%%        \label{fig:reg:neural:sine_deep_2500}
%%    \scalebox{0.55}{%
%%    \psgraph[tickstyle=bottom,Dx=2,Dy=0.25,Ox=-12,Oy=-1.25]{->}(-12.0,-1.25)(12.5,1.25){4in}{2.25in}%
%%    \psset{dotstyle=Bo,dotscale=1.5,fillcolor=lightgray}
%%    \dataplot[plotstyle=dots,showpoints=true]{\dataS}
%%    \psset{dotstyle=Bo,dotscale=0.5,fillcolor=black}
%%    \listplot[plotstyle=dots,showpoints=true,plotNoMax=2,plotNo=2]{\dataPtwofiveK}
%%    \psplot[linewidth=0.5pt,plotpoints=200]{-10}{10}{x RadtoDeg sin} 
%%    \psline[linewidth=0.5pt](-12,0)(12.5,0)
%%    \endpsgraph
%%    }}
%%    \hspace{0.5in}
%%    \subfloat[$t=10000$]{%
%%        \label{fig:reg:neural:sine_deep_10k}
%%        \scalebox{0.55}{%
%%    \psgraph[tickstyle=bottom,Dx=2,Dy=0.25,Ox=-12,Oy=-1.25]{->}(-12.0,-1.25)(12.5,1.25){4in}{2.25in}%
%%    \psset{dotstyle=Bo,dotscale=1.5,fillcolor=lightgray}
%%    \dataplot[plotstyle=dots,showpoints=true]{\dataS}
%%    \psset{dotstyle=Bo,dotscale=0.5,fillcolor=black}
%%    \listplot[plotstyle=dots,showpoints=true,plotNoMax=2,plotNo=2]{\dataPtenK}
%%    \psplot[linewidth=0.5pt,plotpoints=200]{-10}{10}{x RadtoDeg sin} 
%%    \psline[linewidth=0.5pt](-12,0)(12.5,0)
%%    \endpsgraph
%%    }}
%%    }
%%    \vspace{-0.1in}
%%    \centerline{%
%%    \subfloat[$t=30000$]{%
%%        \label{fig:reg:neural:sine_deep_30k}
%%    \scalebox{0.55}{%
%%    \psgraph[tickstyle=bottom,Dx=2,Dy=0.25,Ox=-12,Oy=-1.25]{->}(-12.0,-1.25)(12.5,1.25){4in}{2.25in}%
%%    \psset{dotstyle=Bo,dotscale=1.5,fillcolor=lightgray}
%%    \dataplot[plotstyle=dots,showpoints=true]{\dataS}
%%    \psset{dotstyle=Bo,dotscale=0.5,fillcolor=black}
%%    \listplot[plotstyle=dots,showpoints=true,plotNoMax=2,plotNo=2]{\dataPthirtyK}
%%    \psplot[linewidth=0.5pt,plotpoints=200]{-10}{10}{x RadtoDeg sin} 
%%    \psline[linewidth=0.5pt](-12,0)(12.5,0)
%%    \endpsgraph
%%    }}
%%    \hspace{0.5in}
%%    \subfloat[Test range ${[-20,20]}$]{%
%%    \label{fig:reg:neural:sine_deep_20test}
%%    \scalebox{0.55}{%
%%        \psgraph[tickstyle=bottom,Dx=5,Dy=0.5,Ox=-20,Oy=-1.5]{->}(-20.0,-1.5)(24,1.5){4in}{2.25in}%
%%    \psset{dotstyle=Bo,dotscale=1.5,fillcolor=lightgray}
%%    \dataplot[plotstyle=dots,showpoints=true]{\dataS}
%%    \psset{dotstyle=Bo,dotscale=0.5,fillcolor=black}
%%    \listplot[plotstyle=dots,showpoints=true,plotNoMax=2,plotNo=2]{\dataPsinM}
%%    \psplot[linewidth=0.5pt,plotpoints=250]{-20}{20}{x RadtoDeg sin} 
%%    \psline[linewidth=0.5pt](-20,0)(20.5,0)
%%    %\psline[linewidth=0.5pt](-10.1,-1.1)(10.1,-1.1)
%%    %\psline[linewidth=0.5pt](10.1,-1.1)(10.1,1.1)
%%    %\psline[linewidth=0.5pt](10.1,1.1)(-10.1,1.1)
%%    %\psline[linewidth=0.5pt](-10.1,1.1)(-10.1,-1.1)
%%    \psframe[linewidth=0.5pt](-10.1,-1.1)(10.1,1.1)
%%    \endpsgraph
%%    }}
%%    }
%%%\vspace{0.4in}
%%    \caption{Deep MLP for sine curve: 7 hidden layers, 5 hidden neurons
%%        per layer, ReLU activations.
%%The gray dots represent the training data. The
%%bold line is the predicted response, whereas the thin line is the true
%%response.
%%\protect\subref{fig:reg:neural:sine_deep_500}--\protect\subref{fig:reg:neural:sine_deep_30k}:
%%Predictions after different number of iterations.
%%\protect\subref{fig:reg:neural:sine_deep_20test}: 
%%Testing outside the training range. Good fit within the training
%%    range $[-10,10]$ shown in the box.}
%%\label{fig:reg:neural:sine_deep}
%%\end{figure}
%
%
%%\begin{example}[Deep MLP]
%%    We now illustrate a deep MLP for learning the sine data.
%%    \cref{fig:reg:neural:sine_deep} shows the training data (the gray points
%%    on the curve), which comprises $n=100$ points $x_i$ sampled randomly
%%    in the range $[-10,10]$, with $y_i = \sin(x_i)$. 
%%    The testing data comprises 1000 points sampled uniformly from 
%%    the same range.
%%    The figure also shows the desired output curve (thin line).
%%    We used a deep MLP with seven hidden layers using ReLU
%%    activations. The size of each of the hidden layers is $5$, with the
%%    input and output layers comprising a single neuron.
%%    The output unit
%%    uses an identity activation, and the step size $\eta=0.001$.
%
%%    \cref{fig:reg:neural:deepMLP_sine_arch} shows the 
%%     weight matrices and bias vectors for the different layers
%%     obtained after training.
%%    \cref{fig:reg:neural:sine_deep} shows the output of the deep MLP on
%%    the test set, after different number of epochs. The final SSE is
%%    $0.98$ over
%%    the 1000 test points. 
%%    %Interestingly, with 5 hidden neurons per layer, we found that fewer
%%    %than seven ReLU layers did not result in a good fit.
%%    Furthermore, we can observe in
%%    \cref{fig:reg:neural:sine_deep_20test} that whereas the deep MLP is able to
%%    approximate the sine curve in the specified test range $[-10,10]$,
%%    it does not  yield a good fit outside this range.
%%\end{example}
%
%
%
\begin{frame}{MNIST: Deep MLPs}
\framesubtitle{Prediction error as a function of epochs.}
%
%\begin{figure}[!b]
    \centering
\pgfplotscreateplotcyclelist{my black white}{%
solid, every mark/.append style={solid, fill=gray}, mark=*\\%
solid, every mark/.append style={solid, fill=gray}, mark=square*\\%
solid, every mark/.append style={solid, fill=gray},
mark=diamond*\\%
densely dashed, every mark/.append style={solid, fill=gray},
mark=triangle*\\%
}
    % \tikzsetnextfilename{reg_neural_mnist_errors_mlp}
\begin{tikzpicture}
    \begin{axis}[
            width=3.25in,
            height=2in,
            scale only axis,
            xmin=1, xmax=15,
            ymin=100, ymax=5500,
            label style={font=\small},
            tick label style={font=\tiny},
            xlabel=epochs,
        ylabel=errors,
        xtick={1,2,3,4,5,6,7,8,9,10,11,12,13,14,15},
        % enlargelimits=false,
        axis on top,
        legend style ={ at={(0.25,0.95)},
            anchor=north west, draw=black,
        fill=white, align=left, font=\small},
        cycle list name=my black white,
        % smooth
    ]
    \addplot table [x index=0, y index=2] %
    {REG/neural/figs/test_loss/test_loss_output_deep_m392.txt};
    \addlegendentry{$n_1=392$};
    \addplot table [x index=0, y index=2] %
    {REG/neural/figs/test_loss/test_loss_output_deep_m196_49.txt};
    \addlegendentry{$n_1=196,n_2=49$};
    \addplot table [x index=0, y index=2] %
    {REG/neural/figs/test_loss/test_loss_output_deep_m392_196_49.txt};
    \addlegendentry{$n_1=392,n_2=196,n_3=49$};
    \addplot table [x index=0, y index=2] %
    {REG/neural/figs/test_loss/test_loss_output_deep_m392_196_98_49.txt};
    \addlegendentry{$n_1=392,n_2=196,n_3=98,n_4=49$};
    \end{axis}
\end{tikzpicture}
%\caption{MNIST: Deep MLPs; prediction error as a function of epochs.}
%\label{fig:reg:neural:mnist_deep_mlp}
%\end{figure}
\end{frame}
%
%
%\begin{example}[Deep MLP]
\begin{frame}{Deep MLP}
\framesubtitle{Example}
    We now examine deep MLPs for predicting the labels for the
    MNIST handwritten images dataset.

\medskip

% that we considered in \cref{ex:reg:neural:mlp_mnist}. 
Recall that this dataset has
    $n=60000$ grayscale images of size $28\times28$ that we treat as
    $d=784$ dimensional vectors. 
    The pixel values between 0 and 255 are converted to the range 0 and
    1 by dividing each value by 255.
    The target response vector is a one-hot
    encoded vector for class labels $\{0,1,\ldots,9\}$. 

\medskip

    Thus, the input to the MLP $\bx_i$ has dimensionality $d=784$, and
    the output layer has dimensionality $p=10$. We use softmax
    activation for the output layer. We use ReLU activation for the
    hidden layers, and consider several deep models with different
    number and sizes of the hidden layers. We use step size $\eta=0.3$
    and train for $t=15$ epochs.
    Training was done using minibatches, using batch size of 1000.
\end{frame}

\begin{frame}{Deep MLP}
\framesubtitle{Example}

    During the training of each of the deep MLPs, we evaluate its
    performance on the separate MNIST test datatset that contains 10,000
    images.
    %\cref{fig:reg:neural:mnist_deep_mlp} 
Figure plots the number of errors
    after each epoch for the different deep MLP models.
    The final test error at the end of training
    is given as
    \begin{center}
        \begin{tabular}{|c||c|}
            \hline
            hidden layers & errors\\
            \hline
            $n_1=392$ & 396\\
            $n_1=196,n_2=49$ & 303\\
            $n_1=392,n_2=196,n_3=49$ & 290\\
            $n_1=392,n_2=196,n_3=98,n_4=49$ & 278\\
            \hline
        \end{tabular}
    \end{center}
    We can observe that as we increase the number of layers, we do get
    performance improvements. The deep MLP with four hidden layers of
    sizes $n_1=392,n_2=196,n_3=98,n_4=49$ results in an error rate of
    2.78\% on the training set, whereas the MLP with a single hidden
    layer of size $n_1=392$ has an error rate of 3.96\%.
    Thus, the deeper MLP significantly improves the prediction accuracy.
    However, adding more layers does not reduce the error rate, and can
    also lead to performance degradation.
\end{frame}
%    \label{ex:reg:neural:mlp_mnist_deep}
%\end{example}
%
%
%\section{Further Reading}
%\label{sec:reg:neural:ref}
%\begin{refsection}
%    Artificial neural networks have their origin in the work of
%    \citet*{mcculloch1943logical}. The first application of a single
%    neuron, called a {\em perceptron}, to supervised learning was by
%    \citet*{rosenblatt1958perceptron}.
%    \citet*{minsky1969perceptron} pointed out limitations of
%    perceptrons, which were not overcome until the development of the 
%     backpropagation algorithm, which was introduced by
%    \citet*{rumelhart1986learning} to train general multilayer
%    perceptrons.
%
%\printbibliography[heading=emptyheading]
%\end{refsection}
%
%\section{Exercises}
%\label{sec:reg:neural:exercise}
%
%\begin{exercises}[Q1.]
%
%
%    \begin{figure}[!t]
%    \centering
%    \scalebox{0.9}{\scantokens{
%    % \tikzsetnextfilename{reg_neural_ex_q1}
%    \begin{tikzpicture}
%        \matrix [matrix of math nodes, nodes={circle, draw}, column
%        sep={3cm}, row sep = {1cm}]{
%        & |(z1)| z_1 & \\
%        |(x)| x & |(z2)| z_2 & |(o)| o\\
%        & |(z3)| z_3 & \\
%    };
%    \begin{scope}[
%            every path/.style={draw, thick, -Stealth[black],shorten
%            >=2pt},
%           every node/.style={font=\small\itshape,fill=white,circle}]
%        \path (x) -> node [midway] {$w_1$} (z1);
%        \draw (x) -- node [midway] {$w_2$}  (z2);
%        \draw (x) --  node [midway] {$w_3$} (z3);
%        \draw (z1) ->  node [midway] {$w'_1$} (o);
%        \draw (z2) --  node [midway] {$w'_2$} (o);
%        \draw (z3) --  node [midway] {$w'_3$} (o);
%    \end{scope}
%    \end{tikzpicture}
%    \endinput}}
%    \caption{Neural network for Q\ref{q:reg:neural:q_nn}.}
%    \label{fig:reg:neural:q_nn}
%    \end{figure}
%\item \label{q:reg:neural:q_nn} Consider the neural network in
%    \cref{fig:reg:neural:q_nn}.
%    Let bias values be fixed at 0, and let the weight matrices between the input and hidden, and hidden and
%    output layers, respectively, be:
%    \begin{align*}
%    \bW & = (w_1, w_2, w_3) = (1, 1, -1) &
%    \bW' & = (w'_1, w'_2, w'_3)^T = (0.5, 1, 2)^T    
%    \end{align*}
%            Assume that the hidden layer
%            uses ReLU, whereas the output layer uses
%            sigmoid activation. Assume SSE error.
%            Answer the following questions, when the input is $x=4$ and
%            the true response is $y=0$:
%    \begin{sexercises}
%        
%        \item Use forward propagation to compute
%            the predicted output.
%        % \item Using inverted dropout for the hidden layer,
%        %     with the dropout vector $\bu = (1, 0, 1)^T$ and the
%        %     probability of retention $r=2/3$ (i.e., a dropout rate of
%        %     $1-r = 1/3$), show the forward propagation step to compute
%        %     the predicted output.
% 
%
%        \item What is the loss or error value?
%            
%        \item Compute the net gradient vector $\bdelta^o$ for
%            the output layer.
%            
%        \item Compute the net gradient vector $\bdelta^h$ for
%            the hidden layer.
%            
%        \item Compute the weight gradient matrix
%            $\bgrad_{\bW'}$ between the hidden and output layers.
%            
%
%        \item Compute the weight gradient matrix
%            $\bgrad_{\bW}$ between the input and hidden layers.
%    \end{sexercises}
%
%
%
%\item Show that the derivative of the sigmoid function
%[\cref{eq:reg:neural:sigmoid}]  with
%    respect to its argument is given as
%    $$\frac{\partial f(z)}{\partial z} = f(z) \cdot (1-f(z))$$
%
%
%\item Show that the derivative of the hyperbolic tangent
%function [\cref{eq:reg:neural:tanh}] with
%    respect to its argument is given as
%$$\frac{\partial f(z)}{\partial z} = 1-f(z)^2$$
%
%% \begin{answer}
%% Let $h = e^z - e^{-z}$ and $g=e^z+e^{-z}$. Then,
%% \begin{align*}
%%     \frac{\partial f(z)}{\partial z} 
%%     & = \frac{h' \cdot g - h \cdot g'}{g^2}\\ 
%%     & = \frac{(e^z+e^{-z})\cdot (e^z+e^{-z}) - (e^z-e^{-z})\cdot
%%     (e^z-e^{-z})}{(e^z+e^{-z})^2}\\
%%     & = \frac{(e^z+e^{-z})^2 - (e^z-e^{-z})^2}{(e^z+e^{-z})^2}\\
%%     & = 1 - \lB(\frac{e^z-e^{-z}}{e^z+e^{-z}}\rB)^2\\
%%     & = 1 - f(z)^2
%% \end{align*}
%% \end{answer}
%
%\item Show that the derivative of the softmax
%    function is given as
%    \begin{align*}
%        \frac{\partial f(z_{\!i}|\;\bz)}{\partial z_{\!j}} = 
%    \begin{cases}
%        f(z_{\!i}) \cdot (1-f(z_{\!i})) & \text{if } j=i\\
%        -f(z_{\!i}) \cdot f(z_{\!j}) & \text{if } j\ne i
%    \end{cases}
%    \end{align*}
%    where $\bz = \{z_1, z_2, \cdots, z_p\}$.
%% \begin{answer}
%% Consider the case when $z_i = z_j$.
%% Let $h = e^{z_i}$ and $g=\sum_{k=1}^p e^{z_k}$. Then,
%%  \begin{align*}
%%      \frac{\partial f(z_i)}{\partial z_i} 
%%      & = \frac{h' \cdot g - h \cdot g'}{g^2}\\ 
%%      & = \frac{(e^{z_i})\cdot g - (e^{z_i} \cdot e^{z_i})}
%%      {g^2}\\
%%      & = \frac{e^{z_i}}{g} \cdot \lB(\frac{g - e^{z_i}}{g}\rB)
%%      = \frac{e^{z_i}}{g} \cdot \lB(1 - \frac{e^{z_i}}{g}\rB)\\
%%      & = f(z_i)\cdot (1 - f(z_i))
%%  \end{align*}
%% For the case when $z_i \ne z_j$.
%% Let $h = e^{z_i}$ and $g=\sum_{k=1}^p e^{z_k}$. Then,
%% \begin{align*}
%%     \frac{\partial f(z_i)}{\partial z_{\!j}} 
%%     & = \frac{h' \cdot g - h \cdot g'}{g^2}\\ 
%%     & = \frac{0 \cdot g - (e^{z_i} \cdot e^{z_{\!j}})}
%%     {g^2}\\
%%     & = -\frac{e^{z_i}}{g} \cdot \frac{e^{z_{\!j}}}{g}\\
%%     & = -f(z_i)\cdot f(z_{\!j})
%% \end{align*}
%% \end{answer}
%
%\item Derive an expression for the net gradient vector at the output
%    neurons when using softmax activation with the squared error
%    function.
%
%    % \begin{answer}
%    %     Since we are using the softmax activation, we will use the
%    %     matrix of gradients given in \cref{eq:reg:neural:bF}. 
%
%    %     The net gradient is given as
%    %     \begin{align*}
%    %    \frac{\partial \cE_\bx}{\partial \net_{\!j}}
%    %    & = \sum_{i=1}^K \frac{\partial \cE_\bx}{\partial o_{i}} \cdot
%    %    \frac{\partial o_{i}}{\partial \net_{\!j}} = 
%    %    \lB(\partial \bF^{h+1}\rB)^T \cdot (\bo-\by)
%    %     \end{align*}
%    %     We can simplify the expression further, as follows
%    %    \begin{align*}
%    % \frac{\partial \cE_\bx}{\partial \net_{\!j}}
%    %    & = \sum_{i=1}^K \frac{\partial \cE_\bx}{\partial o_{i}} \cdot
%    %    \frac{\partial o_{i}}{\partial \net_{\!j}}\\
%    %    & = (o_{\!j}-y_{\!j}) o_{\!j} (1-o_{\!j}) + 
%    %    \sum_{i\ne j} (o_i-y_i) \cdot (-o_i \cdot o_{\!j})\\
%    %    & = o_{\!j}(o_{\!j}-y_{\!j}) - o_{\!j} \lB( 
%    %        \sum_{i=1}^K (o_i-y_i) \cdot o_i
%    %    \rB)\\
%    %    & = o_{\!j}(o_{\!j}-y_{\!j}) - o_{\!j} \cdot (\bo-\by)^T\bo
%    %    \end{align*}
%    %    Across all the neurons, we can write this compactly as
%    %    \begin{align*}
%    %        \bdelta^{h+1} = \bo \odot (\bo-\by) - \bo \cdot (\bo-\by)^T\bo
%    %    \end{align*}
%    % \end{answer}
%
%\item Show that if the weight matrix and bias vectors are initialized to
%    zero, then all neurons in a given layer will have identical values
%    in each iteration.
%
%\item Prove that with linear activation functions, a
%    multilayered network is equivalent to a single-layered neural
%    network.
%
%\item Compute the
%    expression for the net gradient vector at the
%    output layer, $\bdelta^{h+1}$, assuming cross-entropy error,
%    $\cE_\bx = - \sum^{K}_{i=1} y_i \cdot \ln(o_i)$,
%    with $K$ independent binary output neurons that use sigmoid
%    activation, that is, $o_i = \text{sigmoid}(\net_i)$. 
%
%\item Given an MLP with one hidden layer, derive the equations for
%    $\bdelta_h$ and $\bdelta_o$ using vector derivatives, i.e., by
%    computing $\frac{\partial \cE_\bx}{\partial \bnet_{h}}$ and
%    $\frac{\partial \cE_\bx}{\partial \bnet_{o}}$, where
%$\bnet_{h}$ and $\bnet_{o}$ are the net input vectors at the hidden and
%output layers.
%\end{exercises}
%
