%WMJ Missing: Geometry and standardized residuals

\lecture{evalreg}{evalreg}

\date{Chapter 27: Regression Evaluation}

\begin{frame}
\titlepage
\end{frame}

%\chapter{Regression Evaluation}
%\label{ch:reg:eval}
%
%Given a set of predictor attributes or independent 
%variables $X_1, X_2, \cdots, X_d$,
%and given the response attribute $Y$, 
%the goal of regression is to learn a  $f$, such that
%\begin{align*}
%    Y = f(X_1, X_2, \cdots, X_d) + \varepsilon  = f(\bX) + \varepsilon
%\end{align*}
%where $\bX = (X_1, X_2, \cdots, X_d)^T$ is the $d$-dimensional 
%multivariate random variable comprised of the predictor variables.
%Here, the random variable $\varepsilon$ denotes the inherent {\em error}
%in the response that is not explained by the linear model.
%
%When estimating the regression function $f$, we make assumptions about
%the form of $f$, for example, whether $f$ is a linear
%or some non-linear function of the parameters of the model. 
%For example, in linear regression,
%we assume that 
%\begin{align*}
%    f(\bX) = \trueb + \truew_1 \cdot X_1 + \ldots + \truew_d \cdot X_d
%\end{align*}
%where $\trueb$ is the bias term, and $\truew_i$ is the regression
%coefficient for attribute $X_i$. 
%The model is {\em linear}, since $f(\bX)$ is a linear function of the
%parameters $\trueb, \truew_1, \cdots, \truew_d$.
%
%% Given a dataset $\bD = \{\bx_i, y_i\}_{i=1}^n$ comprising $n$ points
%% $\bx_i$ in a $d$-dimensional space, and the corresponding response value
%% $y_i$, we estimate a regression model $\hf$, given as follows:
%% \begin{align*}
%%     \hf(\bX) = w_0 + w_1 \cdot X_1 + \ldots + w_d \cdot X_d
%% \end{align*}
%% where $w_0$ is the estimated bias, and $w_i$ is the estimated
%% regression coefficient for attribute $X_i$.
%
%Once we have estimated the bias and coefficients, we need to formulate a
%probabilistic model of regression to evaluate the learned model in terms
%of goodness of fit, confidence intervals for the parameters, and to test
%for the regression effects, namely whether $\bX$ really helps in
%predicting $Y$. In particular, we assume that even if the value of $\bX$
%has been fixed, there can still be uncertainty in the response $Y$.
%Further, we will assume that the error $\varepsilon$ is independent of
%$\bX$ and follows a normal (or Gaussian) distribution with mean $\mu =
%0$ and variance $\sigma^2$, that is, we assume that the errors are
%independent and identically distributed with zero mean and fixed
%variance. 
%
%\index{regression!probabilistic model}
%\index{regression!deterministic component}
%\index{regression!random error component}
%
%The probabilistic regression model comprises two components --- the 
%{\em deterministic component} comprising the observed predictor attributes,
%and the {\em random error component} comprising the error term, which is
%assumed to be independent of the predictor attributes. With the
%assumptions on the form of the regression function $f$, and the error
%variable $\varepsilon$, we can answer several interesting questions such
%as how good of a fit is the estimated model to the input data? How close
%are the estimated bias and coefficients to the true but unknown
%parameters, and so on. We consider such questions in this chapter.
%
%
%\section{Univariate Regression}
\begin{frame}{Univariate Regression}
%In univariate regression we have one dependent attribute $Y$ and one
%independent attribute $X$, and we assume that the true relationship can
%be modeled as a linear function
\begin{equation*}
    Y = f(X) + \varepsilon = \trueb + \truew \cdot X + \varepsilon
\end{equation*}
where $\truew$ is the slope of the best fitting line and $\trueb$ is its
intercept, and $\varepsilon$ is the random error variable that follows a
normal distribution with mean $\mu=0$ and variance $\sigma^2$. 
%% Note also that $\hY = f(X)$ is the predicted response variable.
%
%\subsubsection{Mean and Variance of Response Variable}
%Consider a fixed value $x$ for the independent variable $X$. The 
%expected value of the response variable $Y$ given $x$ is
%\begin{align*}
%    E[Y|X=x] = E[\trueb + \truew \cdot x + \varepsilon] = \trueb+\truew \cdot x + E[\varepsilon] =
%\trueb + \truew\cdot x
%\end{align*}
%The last step follows from our assumption that $E[\varepsilon] = \mu = 0$.
%Also, since $x$ is assumed to be fixed, and $\trueb$ and $\truew$ are constants, 
%the expected value $E[\trueb + \truew\cdot x] = \trueb + \truew\cdot x$. 
%Next, consider the variance of $Y$ given $X=x$, we have
%\begin{align*}
%    \var(Y|X=x) = \var(\trueb + \truew \cdot x + \varepsilon) = \var(\trueb+\truew \cdot x) +
%    \var(\varepsilon) = 0 + \sigma^2 = \sigma^2
%\end{align*}
%Here $\var(\trueb+\truew \cdot x) = 0$, since $\trueb$, $\truew$ and $x$ are all constants.
%Thus, 
%given $X=x$, the response variable $Y$ follows a
%normal distribution with mean $E[Y|X=x] = \trueb + \truew \cdot x$, and variance
%$\var(Y|X=x) = \sigma^2$.
%
%
%\subsubsection{Estimated Parameters}
The true parameters $\trueb$, $\truew$ and $\sigma^2$ are all unknown, and have to
be estimated from the training data $\bD$ comprising $n$ points $x_i$
and corresponding response values $y_i$, for $i=1,2,\cdots,n$. Let
$b$ and $w$ denote the estimated bias and weight terms;
we can then make predictions for
any given value $x_i$ as follows:
\begin{equation*}
    \hy_i =  b + w \cdot x_i
\end{equation*}
The estimated bias $b$ and weight $w$ are obtained 
by minimizing the sum of
squared errors (SSE), given as
\begin{align*}
    SSE = \sum_{i=1}^n (y_i - \hy_i)^2 = \sum^{n}_{i=1} (y_i - b - w
    \cdot x_i)^2 
\end{align*}
\end{frame}
%with the least squares estimates given as (see
%\cref{eq:reg:linear:w-cov-var} and \cref{eq:reg:linear:b})
%\begin{align*}
%    w & = \frac{\sigma_{XY}}{\sigma_{X}^2} & b & = \mu_{Y} - w \cdot
%    \mu_{X}
%\end{align*}
%
%% We shall show below that $b$ and $w$ are unbiased estimators for the
%% true parameters $b$ and $w$. That is,
%% \begin{align*}
%%     E[b] & = b  & E[w] & = w
%% \end{align*}
%% Therefore, the expected value of the predicted response $\hy_i$ for a
%% fixed $x_i$ is given as
%% \begin{align*}
%%     E[\hy_i] = E[b + w \cdot x_i] = E[b] + E[w]\cdot x_i = b + w
%%     \cdot x_i = E[Y|X=x_i]
%% \end{align*}
%% Thus, what we are predicting is the expected value of $Y$, when $X=x_i$.
%
%\subsection{Estimating Variance ($\sigma^2$)}
\begin{frame}{Univariate Regression}
According to our model, the variance in prediction is entirely due to the random error
term $\varepsilon$.
We can estimate this variance by
considering the predicted value $\hy_i$ and its deviation from the
true response $y_i$, that is, by looking at the residual error
\begin{equation*}
    \epsilon_i = y_i - \hy_i
\end{equation*}
%
%One of the properties of the estimated values $b$ and $w$ is that
%the sum of residual errors is zero, since
%\begin{align}
%    \sum_{i=1}^n \epsilon_i & = \sum_{i=1}^n \bigl(y_i - b - w \cdot
%    x_i\bigr)\notag\\
%    & = \sum_{i=1}^n \bigl(y_i - \mu_Y + w \cdot \mu_X - w \cdot
%    x_i\bigr)\notag\\
%    & = \Bigl(\sum_{i=1}^n y_i\Bigr) - n \cdot \mu_Y + w \cdot \Bigl(n\mu_X - \sum_{i=1}^n
%    x_i\Bigr)\notag\\
%    & = n \cdot \mu_Y - n \cdot \mu_Y + w \cdot (n \cdot \mu_X - n \cdot \mu_X) = 0 
%    \label{eq:reg:eval:ssezero}
%\end{align}
%Thus, the expected value of $\epsilon_i$ is zero, since $E[\epsilon_i] =
%\tfrac{1}{n} \sum_{i=1}^n \epsilon_i = 0$.
%In other words, the sum of the errors above and below the regression line cancel
%each other. 
%
%Next, 
The estimated variance $\hsigma^2$ is given as
%\index{regression!estimated variance}
\begin{align*}
    \hsigma^2 & = \var(\epsilon_i)
    = \frac{1}{n-2} \cdot \sum_{i=1}^n \bigl(\epsilon_i -
    E[\epsilon_i]\bigr)^2
    = \frac{1}{n-2} \cdot \sum_{i=1}^n \epsilon_i^2
    = \frac{1}{n-2} \cdot \sum_{i=1}^n (y_i - \hy_i)^2
\end{align*}
Thus, the estimated variance is
\begin{align}
    \tcbhighmath{
    \hsigma^2 = \frac{SSE}{n-2} }
    \label{eq:reg:eval:sse_est}
\end{align}
We divide by $n-2$ to get an unbiased estimate, 
since $n-2$ is the number of degrees of freedom for
estimating SSE.% (see \cref{fig:reg:eval:uni_geom}). In other words, out of the $n$ training points, we need
%to estimate two parameters $\truew$ and $\trueb$, with $n-2$ remaining degrees of
%freedom.
\end{frame}
%
%The squared root of the variance is called the {\em standard error of
%regression}
%\index{regression!standard error of regression}
%\begin{align}
%    \tcbhighmath{
%    \hsigma = \sqrt{\frac{SSE}{n-2}}}
%\end{align}
%
%\subsection{Goodness of Fit}
%\label{sec:reg:eval:goodness_fit}
%\index{regression!total scatter}
%\index{regression!total sum of squares}
%\index{total sum of squares}
%\index{TSS|see{total sum of squares}}
\begin{frame}{Univariate Regression}
The SSE value gives an indication of how much of the variation in $Y$
cannot be explained by our linear model. We can compare this value with
the {\em total scatter}, also called {\em total sum of squares}, 
for the dependent variable $Y$, defined as
\begin{align*}
    TSS = \sum^{n}_{i=1} (y_i - \mu_{Y})^2
\end{align*}
Notice that in TSS, we compute the squared deviations of the true
response from the true mean for $Y$, whereas, in SSE we compute the
squared deviations of the true response from the predicted response.
%
%
\end{frame}

\begin{frame}{Univariate Regression}
The total scatter can be decomposed into two components by adding and
subtracting $\hy_i$ as follows
\begin{align*}
    TSS & = \sum^{n}_{i=1} (y_i - \mu_{Y})^2
    = \sum^{n}_{i=1} (y_i - \hy_i + \hy_i - \mu_{Y})^2 \\
& = \sum^{n}_{i=1} (y_i - \hy_i)^2 + \sum_{i=1}^n (\hy_i - \mu_{Y})^2 + 
2 \sum^{n}_{i=1} (y_i - \hy_i) \cdot (\hy_i - \mu_Y)\\
& = \sum^{n}_{i=1} (y_i - \hy_i)^2 + \sum_{i=1}^n (\hy_i - \mu_{Y})^2
= SSE + RSS
\end{align*}
where we use the fact that $\sum^{n}_{i=1} (y_i - \hy_i) \cdot (\hy_i -
\mu_Y) = 0$, and
%\index{regression!regression sum of squares}
%\index{regression sum of squares}
%\index{RSS|see{regression sum of squares}}
\begin{align*}
RSS = \sum_{i=1}^n (\hy_i - \mu_{Y})^2
\end{align*}
is a new term called
{\em regression sum of squares} that measures the squared deviation of the
predictions from the true mean.
\end{frame}

\begin{frame}{Univariate Regression}
TSS can thus be decomposed into two parts: SSE, which is the
amount of variation not explained by the model, 
and RSS, which is the amount of variance explained
by the model.
Therefore, the fraction of the variation left unexplained by the model is given
by the ratio $\frac{SSE}{TSS}$. 
Conversely, the fraction of the
variation that is explained by the model, called the {\em coefficient
of determination} or simply the {\em $R^2$ statistic}, is given as
%\index{regression!coefficient of determination}
%\index{regression!R$^2$ statistic}
%\index{R$^2$ statistic|see{coefficient of determination}}
%\index{coefficient of determination}
\begin{align}
    \tcbhighmath{
    R^2 = \frac{TSS - SSE}{TSS} = 1 - \frac{SSE}{TSS} = \frac{RSS}{TSS}}
    \label{eq:reg:eval:r2_rss}
\end{align}
The higher the $R^2$ statistic the better the estimated model, with $R^2 \in
[0,1]$.
\end{frame}
%
%\begin{example}[Variance and Goodness of Fit]
%    \label{ex:reg:eval:bi_var_gf}
\begin{frame}{Variance and Goodness of Fit}
   Consider %\cref{ex:reg:linear:bivariate} that shows 
the
   regression of {\tt petal length} ($X$; the predictor variable) on 
    {\tt petal width} ($Y$; the response variable) for the
   Iris dataset.
%    Figure~\ref{fig:reg:eval:irisplw} 
Figure shows the scatterplot between
    the two attributes.  There are a total of $n=150$ data points.
    The least squares estimates for the bias and regression coefficients
    are as follows
    \begin{align*}
        w & = 0.4164 & b & = -0.3665
    \end{align*}
    The SSE value is given as
    \begin{align*}
        SSE = \sum_{i=1}^{150} \epsilon_i^2 
        = \sum_{i=1}^{150} (y_i - \hy_i)^2 = 6.343
    \end{align*}
    
    Thus, the estimated variance and standard error of regression are given as
    \begin{align*}
        \hsigma^2 & = \frac{SSE}{n-2} = \frac{6.343}{148} = 4.286 \times 10^{-2}\\
        \hsigma & = \sqrt{\frac{SSE}{n-2}} = \sqrt{4.286 \times 10^{-2}} = 0.207
    \end{align*}
\end{frame}
    
\begin{frame}{Variance and Goodness of Fit}
    For the bivariate Iris data, the values of TSS and RSS are given as
    \begin{align*}
        TSS & = 86.78 & RSS & = 80.436
    \end{align*}
    We can observe that $TSS = SSE + RSS$.
    The fraction of variance explained by the model, that is, the $R^2$
    value, is given as
    \begin{align*}
        R^2 = \frac{RSS}{TSS} = \frac{80.436}{86.78} = 0.927
    \end{align*}
    This indicates a very good fit of the linear model.
\end{frame}
%\end{example}
%
%
%\readdata{\dataPLW}{REG/eval/figs/iris-plw.dat}
%\begin{figure}[t!]
%\vspace*{-0.1in}
%    \centering
%    \scalebox{0.9}{%
%    \psset{dotstyle=Bo,dotscale=1.5,fillcolor=lightgray,arrowscale=2,PointName=none}
%    \psset{xAxisLabel=$X$: petal length,yAxisLabel= $Y$: petal width,
%        xAxisLabelPos={c,-0.4in},yAxisLabelPos={-0.5in,c}}
%    \psgraph[tickstyle=bottom,Dx=1.0,Dy=0.5,Oy=-0.5,subticks=2]{->}(0.0,-0.5)(7.5,3){4in}{2.5in}%
%    \dataplot[plotstyle=dots,showpoints=true]{\dataPLW}
%    \psdot[dotstyle=*,dotscale=2](3.76,1.20)
%    \psplot[plotstyle=line]{0}{7.5}{x 0.4164 mul -0.3665 add}
%    \endpsgraph
%}
%\vspace{0.6in}
%\caption{Scatterplot: {\tt petal length} ($X$) versus {\tt petal
%width} ($Y$). Mean point shown as black circle.}
%\label{fig:reg:eval:irisplw}
%\vspace{-0.2in}
%\end{figure}%\vspace*{12pt}
%
%
%
%
%
%%\begin{figure}[t!]
%%    \centering
%%    \vspace{-1.5in}
%%    \psset{unit=0.4in}
%%    \begin{pspicture}(-1.5,-1.5)(4,6)
%%   \psset{Alpha=81,Beta=17,arrowscale=1.5}
%%    %\psset{nameX={$Z$}, nameY={$\bone$}, nameZ={$~$}}
%%    %\pstThreeDCoor[xMin=0, xMax=8, yMin=0, yMax=4,
%%    %    zMin=0, zMax=0, linecolor=black]
%%\psset{IIIDxTicksPlane=xy}
%%    %\psset{dotstyle=Bo,dotscale=2.5,fillcolor=lightgray}
%%    %\pstThreeDDot[drawCoor=false,linecolor=gray](5.9,3.0,4.2)
%%    \psset{nameX={$\bx_n$}, nameY={$\bx_2$}, nameZ={$\bx_1$}}
%%    \pstThreeDCoor[RotZ=-20,xMin=0,xMax=7,yMin=0,yMax=7,zMin=0,zMax=2,
%%    linecolor=black,linewidth=0.5pt,linestyle=dashed]
%%    \pstThreeDLine[linewidth=2pt,arrows=->](0,0,0)(5.9,3.0,4.2)
%%    \pstThreeDLine[linewidth=2pt](0,0,0)(0,3.0,0)
%%    \pstThreeDLine[linewidth=2pt](0,0,0)(5.9,0,0)
%%    \pstThreeDLine[linewidth=1pt,arrows=->](0,0,0)(0,4.0,0)
%%    \pstThreeDLine[linewidth=1pt,arrows=->](0,0,0)(8,0,0)
%%    \pstThreeDLine[linestyle=dotted,linecolor=gray](5.9,0,0)(5.9,3.0,0)
%%    \pstThreeDLine[linestyle=dotted,linecolor=gray](0,3.0,0)(5.9,3.0,0)
%%    \pstThreeDLine[linestyle=dotted,linecolor=gray](8,0,0)(8,2.03,0)
%%    \pstThreeDLine[linestyle=dotted,linecolor=gray](0,2.03,0)(8,2.03,0)
%%    %\pstThreeDLine[linewidth=1pt,arrows=->](0,0,0)(5.9,1.5,0)
%%    \pstThreeDLine[linewidth=1pt,arrows=->](0,0,0)(8,2.03,0)
%%    \pstThreeDSquare(-2.5,-1,0)(12,0,0)(0,6,0)
%%    \psset{dotsep=2pt}
%%    \pstThreeDLine[linewidth=2pt,arrows=->,linecolor=gray](0,0,0)(5.9,3.0,0)
%%    %\pstThreeDPut[origin=rt](3,1.7,2.7){$Y$}
%%    \pstThreeDPut[origin=rt](6.4,3.1,4.5){$Y$}
%%    %\pstThreeDPut[origin=rt](4,1.6,0){$\hY$}
%%    \pstThreeDPut[origin=rt](6.4,3.2,0){$\hY$}
%%    \pstThreeDPut[origin=c](8.5,0,0){$Z$}
%%    \pstThreeDPut[origin=r](0,4.2,0){$\bone$}
%%    \pstThreeDPut[origin=c](8.5,2.1,0){$X$}
%%    %\pstThreeDDot[drawCoor=false,linecolor=gray](5.9,3.0,0)
%%    \pstThreeDLine[linewidth=1.5pt,linestyle=dashed,arrows=->,linecolor=gray](5.9,3.0,0)(5.9,3.0,4.2)
%%    \pstThreeDPut[origin=rt](5.9,3.9,3.4){$\bepsilon = Y - \hY$}
%%    % \pstPlanePut[plane=xz,planecorr=xyrot](5.9,3.0,1){$\overbrace{\hspace{1.1in}}^{\bepsilon}$}
%%    % \rput(2.7,0.2){\begin{Rotateleft}$\underbrace{\hspace{1.6in}}_{~}$\end{Rotateleft}}
%%    % \rput(2.7,0.2){$\bepsilon$}
%%    %right angle
%%    \pstThreeDLine[linewidth=1pt](5,2.54,0)(5,2.54,0.5)
%%    \pstThreeDLine[linewidth=1pt](5.9,3.0,0.5)(5,2.54,0.5)
%%    %theta
%%    \psset{beginAngle=0,endAngle=90}
%%    \pstThreeDEllipse(0,0,0)(1,0.6,-0.02)(1,0.6,0.84)
%%    \pstThreeDPut[origin=r](1.5,1,0.8){$\theta$}
%%    %\pstPlanePut[plane=yz,planecorr=normal](1.2,0.9,0){\scriptsize $\theta$}
%%    \end{pspicture}
%%    \vspace{0.6in}
%%\caption{Geometry of univariate regression}
%%\label{fig:reg:eval:uni_geom}
%%\end{figure}
%
%% \vspace*{-0.2in}
%\subsubsection{Geometry of Goodness of Fit}
%
%Recall that $Y$ can be decomposed into two orthogonal parts as
%illustrated in \cref{fig:reg:eval:uni_geom_uc}.
%\begin{align*}
%    Y = \hY + \bepsilon
%\end{align*}
%where $\hY$ is the projection of $Y$ onto the subspace spanned by
%$\{\bone, X\}$. Using the fact that this subspace is the same as that
%spanned by the orthogonal vectors $\{\bone, \mX\}$, with 
%$\mX = X - \mu_X \cdot \bone$, we can further decompose $\hY$ as follows
%\begin{align}
%    \hY & = \proj_\bone(Y) \cdot \bone + \proj_{\mX}(Y) \cdot \mX
%     = \mu_Y \cdot \bone + \frac{Y^T \mX}{\mX^T \mX} \cdot \mX
%     = \mu_Y \cdot \bone + w \cdot \mX
%    \label{eq:reg:eval:hY_uni}
%\end{align}
%
%
%
%
%
%\begin{figure}[t!]
%    \centering
% \captionsetup[subfloat]{captionskip=40pt}
% \vspace{-0.4in}
%    \psset{unit=0.4in}
%    \centerline{
%        \hspace{0.4in}
%    \subfloat[Uncentered]{
%    \label{fig:reg:eval:uni_geom_uc}
%    \scalebox{0.725}{
%    \begin{pspicture}(-1.5,-1.5)(4,6)
%   \psset{Alpha=81,Beta=17,arrowscale=1.5}
%    %\psset{nameX={$Z$}, nameY={$\bone$}, nameZ={$~$}}
%    %\pstThreeDCoor[xMin=0, xMax=8, yMin=0, yMax=4,
%    %    zMin=0, zMax=0, linecolor=black]
%\psset{IIIDxTicksPlane=xy}
%    %\psset{dotstyle=Bo,dotscale=2.5,fillcolor=lightgray}
%    %\pstThreeDDot[drawCoor=false,linecolor=gray](5.9,3.0,4.2)
%    \psset{nameX={$\bx_n$}, nameY={$\bx_2$}, nameZ={$\bx_1$}}
%    \pstThreeDCoor[RotZ=-20,xMin=0,xMax=5,yMin=0,yMax=6,zMin=0,zMax=2,
%    linecolor=black,linewidth=0.5pt,linestyle=dashed]
%    \pstThreeDLine[linewidth=2pt,arrows=->](0,0,0)(5.9,3.0,4.2)
%    \pstThreeDLine[linewidth=2pt](0,0,0)(0,3.0,0)
%    \pstThreeDLine[linewidth=2pt](0,0,0)(5.9,0,0)
%    \pstThreeDLine[linewidth=1pt,arrows=->](0,0,0)(0,4.0,0)
%    \pstThreeDLine[linewidth=1pt,arrows=->](0,0,0)(8,0,0)
%    \pstThreeDLine[linestyle=dotted,linecolor=gray](5.9,0,0)(5.9,3.0,0)
%    \pstThreeDLine[linestyle=dotted,linecolor=gray](0,3.0,0)(5.9,3.0,0)
%    \pstThreeDLine[linestyle=dotted,linecolor=gray](8,0,0)(8,2.03,0)
%    \pstThreeDLine[linestyle=dotted,linecolor=gray](0,2.03,0)(8,2.03,0)
%    %\pstThreeDLine[linewidth=1pt,arrows=->](0,0,0)(5.9,1.5,0)
%    \pstThreeDLine[linewidth=1pt,arrows=->](0,0,0)(8,2.03,0)
%    \pstThreeDSquare(-2.5,-1,0)(12,0,0)(0,6,0)
%    \psset{dotsep=2pt}
%    \pstThreeDLine[linewidth=2pt,arrows=->,linecolor=gray](0,0,0)(5.9,3.0,0)
%    %\pstThreeDPut[origin=rt](3,1.7,2.7){$Y$}
%    \pstThreeDPut[origin=rt](6.4,3.1,4.5){$Y$}
%    %\pstThreeDPut[origin=rt](4,1.6,0){$\hY$}
%    \pstThreeDPut[origin=rt](6.4,3.2,0){$\hY$}
%    \pstThreeDPut[origin=c](8.5,0,0){$\mX$}
%    \pstThreeDPut[origin=r](0,4.2,0){$\bone$}
%    \pstThreeDPut[origin=c](8.5,2.1,0){$X$}
%    %\pstThreeDDot[drawCoor=false,linecolor=gray](5.9,3.0,0)
%    \pstThreeDLine[linewidth=1.5pt,linestyle=dashed,arrows=->,linecolor=gray](5.9,3.0,0)(5.9,3.0,4.2)
%    \pstThreeDPut[origin=rt](5.9,3.9,3.4){$\bepsilon =
%    Y - \hY$}
%    % \pstPlanePut[plane=xz,planecorr=xyrot](5.9,3.0,1){$\overbrace{\hspace{1.1in}}^{\bepsilon}$}
%    % \rput(2.7,0.2){\begin{Rotateleft}$\underbrace{\hspace{1.6in}}_{~}$\end{Rotateleft}}
%    % \rput(2.7,0.2){$\bepsilon$}
%    %right angle
%    \pstThreeDLine[linewidth=1pt](5,2.54,0)(5,2.54,0.5)
%    \pstThreeDLine[linewidth=1pt](5.9,3.0,0.5)(5,2.54,0.5)
%    %%theta
%    %\psset{beginAngle=0,endAngle=90}
%    %\pstThreeDEllipse(0,0,0)(1,0.6,-0.02)(1,0.6,0.84)
%    %\pstThreeDPut[origin=r](1.5,1,0.8){$\theta$}
%    %\pstPlanePut[plane=yz,planecorr=normal](1.2,0.9,0){\scriptsize $\theta$}
%    \end{pspicture}
%    }}
%    %}
%    %\vspace{0.6in}
%% \caption{Geometry of univariate regression: uncentered vectors}
%% \label{fig:reg:eval:uni_geom_un}
%% \end{figure}
%% \begin{figure}[h!]
%%     \centering
%    % \hspace{1.5in}
%% \captionsetup[subfloat]{captionskip=30pt}
%%\vspace{-0.75in}
%    % \psset{unit=0.4in}
%    \hspace{0.5in}
%    %\centerline{
%    \subfloat[Centered]{
%    \label{fig:reg:eval:uni_geom_c}
%    \scalebox{0.725}{
%            \hspace{1in}
%    \begin{pspicture}(-1.5,-1.5)(4,6)
%   \psset{Alpha=70,Beta=17,arrowscale=1.5}
%    %\psset{nameX={$Z$}, nameY={$\bone$}, nameZ={$~$}}
%    %\pstThreeDCoor[xMin=0, xMax=8, yMin=0, yMax=4,
%    %    zMin=0, zMax=0, linecolor=black]
%\psset{IIIDxTicksPlane=xy}
%    %\psset{dotstyle=Bo,dotscale=2.5,fillcolor=lightgray}
%    %\pstThreeDDot[drawCoor=false,linecolor=gray](5.9,3.0,4.2)
%    \psset{nameX={$\bx_n$}, nameY={$\bx_2$}, nameZ={$\bx_1$}}
%    \pstThreeDCoor[RotZ=-20,xMin=0,xMax=6,yMin=0,yMax=4,zMin=0,zMax=3,
%    linecolor=black,linewidth=0.5pt,linestyle=dashed]
%    \pstThreeDLine[linewidth=2pt,arrows=->](0,0,0)(5.9,3.0,4.2)
%    %\pstThreeDLine[linewidth=2pt](0,0,0)(0,3.0,0)
%    %\pstThreeDLine[linewidth=2pt](0,0,0)(5.9,0,0)
%    \pstThreeDLine[linewidth=1pt,arrows=->](0,0,0)(0,4.0,0)
%    \pstThreeDLine[linewidth=1pt,arrows=->](0,0,0)(8,0,0)
%    \pstThreeDLine[linestyle=dotted,linecolor=gray](5.9,0,0)(5.9,3.0,0)
%    \pstThreeDLine[linestyle=dotted,linecolor=gray](0,3.0,0)(5.9,3.0,0)
%    \pstThreeDLine[linestyle=dotted,linecolor=gray](8,0,0)(8,2.03,0)
%    \pstThreeDLine[linestyle=dotted,linecolor=gray](0,2.03,0)(8,2.03,0)
%    %\pstThreeDLine[linewidth=1pt,arrows=->](0,0,0)(5.9,1.5,0)
%    \pstThreeDLine[linewidth=1pt,arrows=->](0,0,0)(8,2.03,0)
%    \pstThreeDSquare(-1,0,-0.75)(10.5,0,0)(0,0,5.75)
%    \psset{dotsep=2pt}
%    \pstThreeDLine[linewidth=2pt,arrows=->,linecolor=gray](0,0,0)(5.9,0,0)
%    \pstThreeDLine[linewidth=1pt,arrows=->,linecolor=gray](0,0,0)(5.9,3.0,0)
%    %\pstThreeDPut[origin=rt](3,1.7,2.7){$Y$}
%    \pstThreeDPut[origin=rt](6.4,3.2,4.6){$Y$}
%    \pstThreeDPut[origin=rt](6.4,0.1,4.6){$\mY$}
%    %\pstThreeDPut[origin=rt](4,1.6,0){$\hY$}
%    \pstThreeDPut[origin=rt](6.4,3.2,0){$\hY$}
%    \pstThreeDPut[origin=rt](5.8,0,-0.3){$\hmY$}
%    \pstThreeDPut[origin=c](8.5,0,0){$\mX$}
%    \pstThreeDPut[origin=r](0,4.2,0){$\bone$}
%    \pstThreeDPut[origin=c](8.5,2.1,0){$X$}
%    %\pstThreeDDot[drawCoor=false,linecolor=gray](5.9,3.0,0)
%    \pstThreeDLine[linewidth=1.5pt,linestyle=dashed,arrows=->,linecolor=gray](5.9,0,0)(5.9,0,4.2)
%    \pstThreeDPut[origin=rt](7,0,2.5){$\bepsilon$}
%    % \pstPlanePut[plane=xz,planecorr=xyrot](5.9,3.0,1){$\overbrace{\hspace{1.1in}}^{\bepsilon}$}
%    % \rput(2.7,0.2){\begin{Rotateleft}$\underbrace{\hspace{1.6in}}_{~}$\end{Rotateleft}}
%    % \rput(2.7,0.2){$\bepsilon$}
%    \pstThreeDLine[linewidth=2pt,arrows=->](0,0,0)(5.9,0.0,4.2)
%    \pstThreeDLine[linestyle=dotted,linecolor=gray](5.9,0.0,4.2)(5.9,3.0,4.2)
%    \pstThreeDLine[linestyle=dotted,linecolor=gray](0,3.0,0)(5.9,3.0,4.2)
%    %right angle
%    \pstThreeDLine[linewidth=1pt](5.9,0,0.5)(5.2,0,0.5)
%    \pstThreeDLine[linewidth=1pt](5.2,0,0.5)(5.2,0,0)
%    %theta
%    \psset{beginAngle=0,endAngle=90}
%    \pstThreeDEllipse(0,0,0)(1,0,0)(1,0,0.78)
%    \pstThreeDPut[origin=r](1.7,0,0.7){$\theta$}
%    %\pstPlanePut[plane=yz,planecorr=normal](1.2,0.9,0){\scriptsize $\theta$}
%    \end{pspicture}
%    }}}
%    \vspace{0.2in}
%\caption{Geometry of univariate regression: uncentered and centered
%vectors. The vector space that is the complement of $\bone$ has
%dimensionality $n-1$. The error space (containing
%the vector $\bepsilon$) is orthogonal to $\mX$, and has
%dimensionality $n-2$, which also specifies the degrees of freedom for
%the estimated variance $\hsigma^2$.
%}
%\label{fig:reg:eval:uni_geom}
%\end{figure}
%
%Likewise, the vectors $Y$ and $\hY$ can be centered by subtracting their
%projections
%along the vector $\bone$
%\begin{align*}
%    \mY & = Y - \mu_Y \cdot \bone &
%    \hmY & = \hY - \mu_Y \cdot \bone = w \cdot \mX
%\end{align*}
%where the last step follows from \cref{eq:reg:eval:hY_uni}.
%The centered vectors $\mY$, $\hmY$ and $\mX$ all lie in the $n-1$
%dimensional subspace orthogonal to the vector $\bone$, as illustrated in
%\cref{fig:reg:eval:uni_geom_c}.
%In this subspace, the centered
%vectors $\mY$ and $\hmY$, and the error vector $\bepsilon$ 
%form a right triangle, since $\hmY$ is the orthogonal projection of
%$\mY$ onto the vector $\mX$.
%Noting that $\bepsilon = Y - \hY = \mY - \hmY$,
%by the Pythagoras theorem, we have
%\begin{align*}
%    \norm[\big]{\mY}^2 & = \norm[\big]{\hmY}^2 + \norm{\bepsilon}^2
%    = \norm[\big]{\hmY}^2 + \norm[\big]{Y-\hY}^2
%    %\norm{Z_Y}^2 & = \norm{w \cdot Z}^2 + \norm{Y-\hY}^2
%\end{align*}
%This equation is 
%equivalent to the decomposition of the total scatter,
%TSS, 
%into sum of squared errors, SSE, and residual sum of squares, RSS. 
%To see this, note that 
%the total scatter, TSS, is defined as follows:
%\begin{align*}
%    TSS = \sum_{i=1}^n (y_i - \mu_Y)^2 = \norm[\big]{Y - \mu_Y \cdot \bone}^2 =
%    \norm[\big]{\mY}^2
%\end{align*}
%The residual sum of squares, RSS, is defined as
%\begin{align*}
%    RSS = \sum_{i=1}^n (\hy_i - \mu_Y)^2 = \norm[\big]{\hY - \mu_Y \cdot \bone}^2 = 
%    \norm[\big]{\hmY}^2
%\end{align*}
%Finally, the sum of squared errors, SSE, is defined as
%\begin{align*}
%    SSE = \norm{\bepsilon}^2 = \norm[\big]{Y -\hY}^2
%\end{align*}
%Thus, the geometry of univariate regression makes it evident that
%\begin{align*}
%    \norm[\big]{\mY}^2 & = \norm[\big]{\hmY}^2 + \norm[\big]{Y-\hY}^2\\
%    \norm[\big]{Y - \mu_Y \cdot \bone}^2 & = \norm[\big]{\hY -
%    \mu_Y\cdot\bone}^2 + \norm[\big]{Y-\hY}^2\\
%    TSS & = RSS + SSE
%\end{align*}
%
%
%Notice further that since $\mY$, $\hmY$ and $\bepsilon$ form a
%right triangle, the cosine of the angle between $\mY$ and
%$\hmY$ is given as the ratio of the base to the hypotenuse.
%On the other hand, via \cref{eq:eda:numeric:scorrdot}, the cosine of the
%angle is also the correlation between $Y$
%and $\hY$, denoted $\rho_{Y\hY}$. Thus, we have:
%\begin{align*}
%    \rho_{Y\hY} = \cos\theta = \frac{\norm[\big]{\hmY}}{\norm[\big]{\mY}}
%\end{align*}
%We can observe that
%\begin{align*}
%    \norm[\big]{\hmY} = \rho_{Y\hY} \cdot \norm[\big]{\mY}
%\end{align*}
%Note that, whereas $\abs{\rho_{Y\hY}} \le 1$, due to the projection
%operation,  the angle between $Y$ and $\hY$ is always less than or equal
%to $90^\circ$, which means that $\rho_{Y\hY} \in [0,1]$ for univariate
%regression.
%Thus, the predicted response vector $\hmY$ is smaller than the
%true response vector $\mY$ by an amount equal to the correlation between them. 
%Furthermore, by
%\cref{eq:reg:eval:r2_rss}, the coefficient of determination is the same
%as the squared correlation between $Y$ and $\hY$
%\begin{align*}
%    R^2 = \frac{RSS}{TSS} =
%    \frac{\norm[\big]{\hmY}^2}{\norm[\big]{\mY}^2}
%     = \rho_{Y\hY}^2
%\end{align*}
%
%%This
%%is an instance of the phrase {\em regression to the mean}. 
%
%% \begin{align*}
%%     \norm{Y}^2 = \norm{\hY}^2 + \norm{\bepsilon}^2
%% \end{align*}
%% Define $Z_Y = Y - \mu_Y \cdot \bone$, and noting that $\bepsilon = Y -
%% \hY$, we have
%% \begin{align*}
%%     \norm{Z_Y + \mu_Y \cdot \bone}^2 & = \norm{w \cdot Z + 
%%         \mu_Y \cdot \bone}^2 + \norm{Y-\hY}^2\\
%%     Z_Y^T Z_Y + 2 \mu_Y \cdot Z_Y^T \bone + n \mu_Y^2 & = 
%%     w^2 \cdot Z^T Z + 2 \mu_Y \cdot Z^T \bone + n \mu_Y^2 + \norm{Y-\hY}^2\\
%%     Z_Y^T Z_Y & = w^2 \cdot Z^T Z + \norm{Y-\hY}^2\\
%%     \norm{Z_Y}^2 & = \norm{w \cdot Z}^2 + \norm{Y-\hY}^2
%% \end{align*}
%% Above, we used the fact that both $Z_Y$ and $Z$ are orthogonal to
%% $\bone$, and thus $Z_Y^T \bone = 0$ and $Z^T\bone = 0$.
%
%% We will now show that the above equation is 
%% equivalent to the decomposition of the total scatter,
%% TSS, 
%% into sum of squared errors, SSE, and residual sum of squares, RSS. 
%% Note that the total scatter, TSS, is defined as follows:
%% \begin{align*}
%%     TSS = \sum_{i=1}^n (y_i - \mu_Y)^2 = \norm{Y - \mu_Y \cdot 1}^2 =
%%     \norm{Z_Y}^2
%% \end{align*}
%% Also, the residual sum of squares, RSS, is defined as
%% \begin{align*}
%%     RSS = \sum_{i=1}^n (y_i - \mu_Y)^2 = \norm{\hY - \mu_Y \cdot 1}^2 = 
%%     \norm{w \cdot Z}^2
%% \end{align*}
%% where we used \cref{eq:reg:eval:hY_uni}, i.e., $\hY = \mu_Y \cdot \bone
%% + w \cdot Z$, which implies that $\hy - \mu_Y \cdot \bone  = w
%% \cdot Z$.
%% Finally, the sum of squared errors is defined as
%% \begin{align*}
%%     SSE = \norm{\bepsilon}^2 = \norm{Y -\hY}^2
%% \end{align*}
%% Thus, the geometry of univariate regression makes it evident that
%% \begin{align*}
%%     \norm{Z_Y}^2 & = \norm{w \cdot Z}^2 + \norm{Y-\hY}^2\\
%%     TSS & = RSS + SSE
%% \end{align*}
%
%% \enlargethispage{24pt}
%\begin{example}[Geometry of Goodness of Fit]
%    Continuing  \cref{ex:reg:eval:bi_var_gf}, the correlation between
%    $Y$ and $\hY$ is given as
%    \begin{align*}
%        \rho_{Y\hY} = \cos\theta =
%        \frac{\norm[\big]{\hmY}}{\norm[\big]{\mY}} =
%        \frac{\sqrt{RSS}}{\sqrt{TSS}}  =
%        \frac{\sqrt{80.436}}{\sqrt{86.78}} = \frac{8.969}{9.316} = 0.963
%    \end{align*}
%    The square of the correlation is equivalent to $R^2$, since 
%    \begin{align*}
%        \rho_{Y\hY}^2 = (0.963)^2 = 0.927
%    \end{align*}
%    The angle between $Y$ and $\hY$ is given as
%    \begin{align*}
%        \theta = \cos^{-1}(0.963) = 15.7^\circ
%    \end{align*}
%    The relatively small angle indicates a good linear fit.
%\end{example}
%
%\subsection{Inference about Regression Coefficient and Bias Term}
\begin{frame}{Inference about Regression Coefficient and Bias Term}

The estimated values of the bias and regression coefficient, $b$ and $w$,
are only point estimates for the true parameters $\trueb$ and $\truew$. To obtain
confidence intervals for these parameters, we treat each $y_i$ as a
random variable for the response given the corresponding fixed value
$x_i$.
These random variables are all independent and identically distributed
as $Y$, with expected value $\trueb + \truew \cdot x_i$ and variance
$\sigma^2$. 
On the other hand, the $x_i$ values are fixed {\em a priori} and
therefore $\mu_X$ and $\sigma_X^2$ are also fixed values.
%
We can now treat $b$ and $w$ as random variables,
with
\begin{align*}
    b & = \mu_{Y} - w \cdot \mu_X\\
    w & = \frac{\sum^{n}_{i=1} (x_i - \mu_X)(y_{i}
    - \mu_{Y})}{ \sum^{n}_{i=1} (x_i - \mu_X)^2 } = \frac{1}{s_X} 
    \sum^{n}_{i=1} (x_i - \mu_X) \cdot y_{i}
        = \sum^{n}_{i=1} c_i \cdot y_i
\end{align*}
where $c_i$ is a constant (since $x_i$ is fixed), given as
\begin{align}
    c_i = \frac{x_i - \mu_X}{ s_X }
    \label{eq:reg:eval:ci}
\end{align}
and $s_X = \sum^{n}_{i=1} (x_i - \mu_X)^2$ is the total
scatter for $X$, defined as the sum of squared deviations of $x_i$ from
its mean $\mu_X$.
\end{frame}
%We also use the fact that 
%$$ \sum^{n}_{i=1}  (x_i - \mu_X)\cdot \mu_{Y} =
%\mu_{Y} \cdot \sum^{n}_{i=1}  (x_i - \mu_X) = 0$$
%Note that
%\begin{align*}
%    \sum^{n}_{i=1} c_i = \frac{1}{s_X} \sum^{n}_{i=1} (x_i - \mu_X) = 0
%\end{align*}
%
%\subsubsection{Mean and Variance of Regression Coefficient}
\begin{frame}{Mean and Variance of Regression Coefficient}
The expected value of $w$ is given as
\begin{align*}
E[w] & = E\lB[ \sum^{n}_{i=1} c_i y_i \rB] = \sum^{n}_{i=1} c_i \cdot
    E[y_i] = \sum^{n}_{i=1} c_i (\trueb+\truew\cdot x_i) \\
    & = \trueb \sum^{n}_{i=1} c_i + \truew \cdot \sum^{n}_{i=1} c_i \cdot x_i
    = \frac{\truew}{s_X} \cdot \sum^{n}_{i=1} (x_i - \mu_X) \cdot x_i
    = \frac{\truew}{s_X} \cdot s_X = \truew
\end{align*}
which follows from the observation that $ \sum^{n}_{i=1} c_i = 0$, and
further
\begin{align*}
s_X & = \sum^{n}_{i=1} (x_i - \mu_X)^2  
= \Bigl(\sum_{i=1}^n x_i^2\Bigr) - n \cdot \mu_X^2
      = \sum^{n}_{i=1} (x_i - \mu_X) \cdot x_i
\end{align*}
Thus, $w$ is an unbiased estimator for the true parameter $\truew$.
\end{frame}

%
\begin{frame}{Mean and Variance of Regression Coefficient}
Using the fact that the variables $y_i$ are independent and identically
distributed as $Y$,
we can compute the variance of $w$ as follows
\begin{align}
    \var(w) = \var\lB( \sum^{n}_{i=1} c_i \cdot y_i\rB) = \sum^{n}_{i=1}
    c_i^2 \cdot \var(y_i) = 
    \sigma^2 \cdot \sum^{n}_{i=1} c_i^2 = 
    \frac{\sigma^2}{s_X}
    \label{eq:reg:eval:var_hw}
\end{align}
since $c_i$ is a constant, $\var(y_i) = \sigma^2$, and further
\begin{align*}
\sum_{i=1}^n c_i^2 = \frac{1}{ s_X^2}  \cdot \sum^{n}_{i=1}  (x_i - \mu_X)^2
    = \frac{s_X}{s_X^2} = \frac{1}{s_X}
\end{align*}
The standard deviation of $w$, also called the standard error of
$w$, is given as
\begin{align}
    \tcbhighmath{
    \se(w) = \sqrt{\var(w)} = \frac{\sigma}{\sqrt{s_X}}}
\end{align}
\end{frame}

%\subsubsection{Mean and Variance of Bias Term}
\begin{frame}{Mean and Variance of Bias Term}
The expected value of $b$ is given as
\begin{align*}
    E[b] & = E[ \mu_Y - w \cdot \mu_X]
    = E\lB[ \frac{1}{n} \sum^{n}_{i=1} y_i - w \cdot \mu_X \rB]\\
    & = \lB(\frac{1}{n} \cdot \sum_{i=1}^n E[ y_{i} ]\rB) - \mu_X \cdot E[w]
    = \lB(\frac{1}{n} \sum^{n}_{i=1} (\trueb + \truew \cdot x_i)\rB) 
    - \truew \cdot \mu_X\\
    & = \trueb + \truew \cdot \mu_X - \truew \cdot \mu_X = \trueb
\end{align*}
Thus, $b$ is an unbiased estimator for the true parameter $\trueb$.

Using the observation that all $y_i$ are independent, 
the variance of the bias term can be computed as follows
\end{frame}
%
\begin{frame}{Mean and Variance of Bias Term}
\begin{align*}
    \var(b) & = \var(\mu_Y - w \cdot \mu_X)\\
    & = \var\lB( \frac{1}{n} \sum^{n}_{i=1} y_i \rB) +
    \var(\mu_X \cdot w)\\
   &  = \frac{1}{n^2} \cdot n \sigma^2 + \mu_X^2 \cdot \var(w)
    = \frac{1}{n} \cdot \sigma^2 + \mu_X^2 \cdot \frac{\sigma^2}{s_X}\\
  & = \lB(\frac{1}{n} + \frac{\mu_X^2}{s_X}\rB) \cdot \sigma^2 
\end{align*}
% where $\beta = (1/n + \mu_X^2/s_X)$. 
where we used the
fact that for any two random variables $A$ and $B$, we have $\var(A-B) =
\var(A)+\var(B)$. That is, variances of $A$ and $B$ add, even though we
are computing the variance of $A-B$.
The standard deviation of $b$, also called the standard error of
$b$, is given as
\begin{align}
    \tcbhighmath{
    \se(b) = \sqrt{\var(b)} 
= \sigma \cdot \sqrt{\frac{1}{n} + \frac{\mu_X^2}{s_X}}}
\end{align}
\end{frame}

%
%\subsubsection{Covariance of Regression Coefficient and Bias}
\begin{frame}{Covariance of Regression Coefficient and Bias}

We can also compute the covariance of $w$ and $b$, as follows
\begin{align*}
    \ocov(w, b) & = E[w \cdot b] - E[w] \cdot E[b]
    = E[(\mu_Y - w \cdot \mu_X)\cdot w] - \truew \cdot \trueb\\
 & = \mu_Y \cdot E[w] - \mu_X \cdot E[w^2] - \truew \cdot \trueb
    = \mu_Y \cdot \truew - \mu_X \cdot \bigl(\var(w) + E[w]^2 \bigr) -
    \truew
    \cdot \trueb\\
    & = \mu_Y \cdot \truew - \mu_X \cdot \lB( \frac{\sigma^2}{s_X} - \truew^2 \rB)
    - \truew \cdot \trueb
     = \truew \cdot \underbrace{(\mu_Y - \truew \cdot \mu_X)}_{\trueb} - \frac{\mu_X \cdot \sigma^2
    }{s_X} - \truew \cdot \trueb\\
    & = - \frac{\mu_X \cdot \sigma^2}{s_X}
\end{align*}
where we use the fact that $\var(w) = E[w^2] - E[w]^2$,
which implies $E[w^2] = \var(w) + E[w]^2$, and further that 
$\mu_Y - \truew \cdot \mu_X = \trueb$.
\end{frame}

%\index{regression!confidence interval}
%\index{confidence interval}
%\subsubsection{Confidence Intervals}
\begin{frame}{Confidence Intervals}
Since the $y_i$ variables are all normally distributed, their linear combination also
follows a normal distribution. Thus, $w$ follows a normal distribution
with mean $\truew$ and variance $\sigma^2/s_X$.
Likewise, $b$ follows a normal distribution
with mean $\trueb$ and variance $(1/n + \mu_X^2/s_X) \cdot \sigma^2$.

Since the true variance $\sigma^2$ is unknown, we use the estimated 
variance $\hsigma^2$, % (from \cref{eq:reg:eval:sse_est}), 
to
define the standardized variables $Z_{w}$ and $Z_{b}$ as follows
\begin{empheq}[box=\tcbhighmath]{align}
    Z_{w} & = \frac{w - E[w]}{\se(w)} =
    \frac{w - \truew}{ \frac{\hsigma}{\sqrt{s_X}}} 
    &
    Z_{b} & = \frac{b - E[b]}{\se(b)} =
    \frac{b - \trueb}{\hsigma \sqrt{(1/n + \mu_X^2/s_X)}}
    \label{eq:reg:eval:TwTb}
\end{empheq}
These variables follow the Student's $t$ distribution with
$n-2$ degrees of freedom. 
%Let $T_{n-2}$ denote the cumulative $t$ distribution with $n-2$ degrees
%of freedom, and let $t_{\alpha/2}$ denote the critical value 
%of $T_{n-2}$ that encompasses $\alpha/2$ of the probability
%mass in the right tail. That is,
%\begin{align*}
%    P(Z \ge t_{\alpha/2}) = \tfrac{\alpha}{2}
%    \text{ or equivalently }
%    T_{n-2}(t_{\alpha/2}) = 1 - \tfrac{\alpha}{2}
%\end{align*}
%Since the $t$ distribution is symmetric, we have
%\begin{align*}
%    P(Z \ge -t_{\alpha/2}) = 1-\tfrac{\alpha}{2}
%    \text{ or equivalently }
%    T_{n-2}(-t_{\alpha/2}) = \tfrac{\alpha}{2}
%\end{align*}
Given confidence level $1-\alpha$, i.e., significance level $\alpha \in
(0,1)$, the $100(1-\alpha)$\% confidence
interval for the true values, $\truew$ and $\trueb$, are therefore
as follows
\begin{align*}
    P\lB(w -t_{\alpha/2} \cdot \se(w) \; \le
        \; \truew \;
    \le \; w + t_{\alpha/2} \cdot \se(w)\rB) =
    1-\alpha\\
    P\lB(b -t_{\alpha/2} \cdot \se(b) \; \le
        \;\trueb\;
    \le \; b + t_{\alpha/2} \cdot \se(b) \rB) =
    1-\alpha
\end{align*}
\end{frame}
%
%
%\begin{example}[Confidence Intervals]
%    \label{ex:reg:eval:bi_ci}
\begin{frame}{Confidence Intervals}
\framesubtitle{Example}
%    Continuing with \cref{ex:reg:eval:bi_var_gf}, 
We consider the
    variance of the bias and regression coefficient, and their
    covariance. However, since we do not know the true variance
    $\sigma^2$, we use the estimated variance and the standard error for
    the Iris data
    \begin{align*}
        \hsigma^2 & = \frac{SSE}{n-2} = 4.286 \times 10^{-2}\\
        \hsigma & = \sqrt{4.286 \times 10^{-2}} = 0.207
    \end{align*}
    Furthermore, we have 
    \begin{align*}
        \mu_X & = 3.7587 & s_X & = 463.864
    \end{align*}
    Therefore, the estimated variance and standard error of $w$ is given as
    \begin{align*}
        \var(w) & = \frac{\hsigma^2}{s_X} = \frac{4.286 \times
        10^{-2}}{463.864} = 9.24\times 10^{-5}\\
        \se(w) & = \sqrt{\var(w)} = \sqrt{9.24\times 10^{-5}} = 9.613
        \times 10^{-3}
    \end{align*}
\end{frame}

\begin{frame}{Confidence Intervals}
\framesubtitle{Example}
    The estimated variance and standard error of $b$ is
    \begin{align*}
        \var(b) & = \lB(\frac{1}{n} + \frac{\mu_X^2}{s_X}\rB)\cdot \hsigma^2\\
        & = \lB(\frac{1}{150} + \frac{(3.759)^2}{463.864}\rB) \cdot
        (4.286 \times 10^{-2}) \\
        & = (3.712 \times 10^{-2}) \cdot
        (4.286 \times 10^{-2}) = 1.591 \times 10^{-3}\\
        \se(b) & = \sqrt{\var(b)} = \sqrt{1.591 \times 10^{-3}} =
        3.989 \times 10^{-2}
    \end{align*}
    and the covariance between $b$ and $w$ is
    \begin{align*}
        \ocov(w, b) = - \frac{\mu_X \cdot \hsigma^2}{s_X} = -
        \frac{3.7587 \cdot (4.286 \times 10^{-2})}{463.864} = -3.473 \times
        10^{-4}
    \end{align*}
    
    For the confidence interval, we use a confidence level of
    $1-\alpha=0.95$ (or $\alpha=0.05$). 
    The critical value of the $t$-distribution, with $n-2=148$
    degrees of freedom, that
    encompasses $\alpha/2 = 0.025$ fraction of the probability mass in
    the right tail is
    $t_{\alpha/2} = 1.976$. 
    We have
    \begin{align*}
        t_{\alpha/2} \cdot \se(w) & = 1.976 
        \cdot (9.613 \times 10^{-3}) = 0.019
    \end{align*}
\end{frame}

\begin{frame}{Confidence Intervals}
\framesubtitle{Example}
    Therefore, the 95\% confidence interval
    for the true value, $\truew$, of the regression coefficient is given as
    \begin{align*}
        \lB(w -t_{\alpha/2} \cdot \se(w),\;\; 
        w + t_{\alpha/2} \cdot \se(w) \rB)&  = 
    \lB( 0.4164-0.019,\; 0.4164+0.019 \rB)\\
    & = (0.397, 0.435)
    \end{align*}
    Likewise, we have:
    \begin{align*}
        t_{\alpha/2} \cdot \se(b) & = 1.976
        \cdot (3.989 \times 10^{-2}) = 0.079
    \end{align*}
    Therefore, the 95\% confidence interval for the true bias term,
    $\trueb$, is
    \begin{align*}
        \lB(b -t_{\alpha/2} \cdot \se(b),\;\;
        b +t_{\alpha/2} \cdot \se(b) \rB)&  = 
        \lB(-0.3665-0.079,\; -0.3665+0.079 \rB)\\
       &  = (-0.446, -0.288)
    \end{align*}
\end{frame}

%\end{example}
%
%
%\subsection{Hypothesis Testing for Regression Effects}
%\index{regression effect}
%\index{regression!hypothesis testing}
%\index{hypothesis testing!regression}
\begin{frame}{Hypothesis Testing for Regression Effects}
One of the key questions in regression is whether $X$
predicts the response $Y$. In the regression model, 
$Y$ depends on $X$ through the parameter $\truew$, therefore, we can check
for the regression effect by 
assuming the null hypothesis $H_0$ that
$\truew=0$, with the alternative hypothesis $H_a$ being $\truew \ne 0$: 
\begin{align*}
    H_0\!: & \; \truew = 0 & H_a\!: & \; \truew \ne 0 
\end{align*}
When $\truew=0$, the response $Y$ depends only on the bias $\trueb$ and the random
error $\varepsilon$. 
In other words, $X$ provides no information about the response
variable $Y$. 
\end{frame}
%
%%Further, since $E[w] = w = 0$, 
%%under the null hypothesis we have $b = \mu_Y$, and is independent of
%%$X$. 
%
\begin{frame}{Hypothesis Testing for Regression Effects}
Now consider the standardized variable $Z_{w}$.% from
%\cref{eq:reg:eval:TwTb}. 
Under
the null hypothesis we have $E[w] = \truew = 0$. Thus,
\begin{align}
    \tcbhighmath{
    Z_{w} = \frac{w - E[w]}{\se(w)} = \frac{w}{ \hsigma / \sqrt{s_X}} }
\end{align}
We can therefore compute the $\pvalue$ for the $Z_{w}$ statistic
using a two-tailed test via the $t$ distribution with $n-2$ degrees of freedom.
Given significance level $\alpha$ (e.g., $\alpha=0.01$),
we reject the null hypothesis if the $\pvalue$ is below $\alpha$.
In this case, we accept the
alternative hypothesis that the estimated value of the slope parameter
is significantly different from zero.

\medskip

We can also define the $f$-statistic, which is the ratio of the
regression sum of squares, RSS, to
the estimated variance, given as
\index{regression!$f$-statistic}
\begin{align}
    \tcbhighmath{
    f = \frac{RSS}{\hsigma^2} 
    = \frac{\sum_{i=1}^n (\hy_i - \mu_Y)^2}{\sum_{i=1}^n (y_i -
\hy_i)^2 \Big/ (n-2)}}
\end{align}
\end{frame}

\begin{frame}{Hypothesis Testing for Regression Effects}
Under the null hypothesis, one can show that 
\begin{align*}
    E[RSS] = \sigma^2
\end{align*}
Further, it is also true that
\begin{align*}
    E[\hsigma^2] = \sigma^2
\end{align*}
Thus, under the null hypothesis the $f$-statistic has a value close to
1, which indicates that there is no relationship between the predictor
and response variables. On the other hand, if the alternative hypothesis
is true, then $E[RSS] \ge \sigma^2$, resulting in a larger $f$ value.

\medskip

In fact, the $f$-statistic follows a $F$-distribution with $1,(n-2)$ degrees of
freedom (for the numerator and denominator, respectively); therefore, we can reject the null hypothesis that $w = 0$ if
the $\pvalue$ of $f$ is less than the significance level $\alpha$, say $0.01$.
\end{frame}
%% \ge F_{\alpha,1,n-2}$, where $F_{\alpha,1,n-2}$ denotes the probability
%% mass encompassed by the critical value $\alpha$ for the F-distribution
%% with $1,n-2$ degrees of freedom, where $\alpha \ge 90\%$.
%
%
%Interestingly the $f$-test is equivalent to the $t$-test since
%$Z_{w}^{\;2} = f$. We can see this as follows:
%\begin{align*}
%f & = \frac{1}{\hsigma^2} \cdot \sum_{i=1}^n (\hy_i - \mu_Y)^2
% = \frac{1}{\hsigma^2} \cdot \sum_{i=1}^n (b + w \cdot x_i - \mu_Y)^2\\
%    & = \frac{1}{\hsigma^2} \cdot \sum_{i=1}^n (\mu_Y - w \cdot \mu_X + w \cdot x_i -
%    \mu_Y)^2
%    = \frac{1}{\hsigma^2} \cdot \sum_{i=1}^n \bigl(w \cdot (x_i-
%    \mu_X)\bigr)^2\\
%    & = \frac{1}{\hsigma^2} \cdot w^2 \cdot \sum_{i=1}^n (x_i-
%\mu_X)^2
%     = \frac{w^2 \cdot s_X}{\hsigma^2}\\
%    & = \frac{w^2}{\hsigma^2/s_X} = Z_{w}^{\;2}
%\end{align*}
%
%
%\subsubsection{Test for Bias Term}
\begin{frame}{Hypothesis Testing for Regression Effects}
Note that we can also test if the bias value is statistically
significant or not by setting up the null hypothesis, $H_0: \trueb=0$, versus
the alternative hypothesis $H_a: \trueb\ne 0$. We then evaluate the
$Z_{b}$ statistic %(see \cref{eq:reg:eval:TwTb}) 
under the null
hypothesis:
\begin{align}
    \tcbhighmath{
    Z_{b} = \frac{b - E[b]}{\se(b)} =
\frac{b}{\hsigma\cdot \sqrt{(1/n + \mu_X^2/s_X)}} }
\end{align}
since, under the null hypothesis $E[b] = \trueb = 0$.
Using a two-tailed $t$-test with $n-2$ degrees of freedom, we can
compute the $\pvalue$ of $Z_{b}$. We reject the null hypothesis if
this value is smaller than the significance level $\alpha$.
\end{frame}
%
%\begin{example}[Hypothesis Testing]
\begin{frame}{Hypothesis Testing}
\framesubtitle{Example}
%    We continue with \cref{ex:reg:eval:bi_ci}, but now we test for the
%    regression effect. 
Under the null hypothesis we have $\truew=0$, further
    $E[w] = \truew = 0$. Therefore, the standardized variable $Z_{w}$ is
    given as
    \begin{align*}
    Z_{w} & = \frac{w - E[w]}{\se(w)} =
    \frac{w}{\se(w)} 
    = \frac{0.4164}{9.613 \times 10^{-3}} = 43.32
    \end{align*}
    Using a two-tailed $t$-test with $n-2$ degrees of freedom, we
    find that
    \begin{align*}
        \pvalue(43.32) \approx 0
    \end{align*}
    Since this value is much less than the significance level $\alpha=0.01$, we conclude
    that observing such an extreme value of $Z_{w}$ is unlikely under the
    null hypothesis. Therefore, we reject the null hypothesis and accept
    the alternative hypothesis that $\truew\ne 0$.

    Now consider the $f$-statistic, we have
    \begin{align*}
        f = \frac{RSS}{\hsigma^2} = \frac{80.436}{4.286 \times
        10^{-2}} = 1876.71
    \end{align*}
\end{frame}


\begin{frame}{Hypothesis Testing}
\framesubtitle{Example}
%
    Using the $F$-distribution with $(1,n-2)$ degrees of freedom, we
    have
    \begin{align*}
        \pvalue(1876.71) \approx 0
    \end{align*}
    In other words, such a large value of the $f$-statistic is extremely
    rare, and we can reject the null hypothesis. We conclude that $Y$
    does indeed depend on $X$, since $\truew\ne 0$.

    Finally, we test whether the bias term is significant or not. Under
    the null hypothesis $H_0: \trueb=0$, we have
    \begin{align*}
        Z_{b} & =  \frac{b}{\se(b)} =
    \frac{-0.3665}{3.989 \times 10^{-2}} = -9.188 
    \end{align*}
    Using the two-tailed $t$-test, we find
    \begin{align*}
        \pvalue(-9.188)  = 3.35 \times 10^{-16}
    \end{align*}
    It is clear that such an extreme $Z_{b}$ value is highly unlikely
    under the null hypothesis. Therefore, we accept the alternative
    hypothesis that $H_a: \trueb\ne 0$.
\end{frame}
%\end{example}
%
%
%\subsection{Standardized Residuals}
%\index{regression!standardized residuals}
%Our assumption about the true errors $\varepsilon_i$ is that they are
%normally distributed with mean $\mu = 0$ and fixed variance $\sigma^2$. 
%After fitting the linear model, we can examine how well the residual
%errors $\epsilon_i = y_i - \hy_i$ satisfy the normality assumption.
%For this, we need to compute the mean and variance of $\epsilon_i$, by
%treating it as a random variable.
%
%The mean of $\epsilon_i$ is given as
%\begin{align*}
%    E[\epsilon_i] & = E[y_i - \hy_i] = E[y_i] - E[\hy_i]\\
%    & = \trueb + \truew \cdot x_i - E[b + w \cdot x_i] = 
%    \trueb + \truew \cdot x_i - (\trueb + \truew \cdot x_i) = 0 
%\end{align*}
%which follows from the fact that $E[b] = \trueb$ and $E[w] = \truew$.
%
%To compute the variance of $\epsilon_i$, we will express it as a linear
%combination of the $y_{\!j}$ variables, by noting that
%\begin{align*}
%    w & = \frac{1}{s_X} \lB( \sum^{n}_{j=1} x_{\!j} y_{\!j} - n \cdot \mu_X
%    \cdot \mu_Y \rB)
%    = \frac{1}{s_X} \lB(\sum^{n}_{j=1} x_{\!j} y_{\!j} - 
%    \sum^{n}_{j=1} \mu_X \cdot y_{\!j} \rB)
%     = \sum^{n}_{j=1} \frac{(x_{\!j} - \mu_X)}{s_X} \cdot y_{\!j}\\
%    b & = \mu_Y - w \cdot \mu_X
%    = \lB(\sum^{n}_{j=1} \frac{1}{n} \cdot y_{\!j}\rB) - w \cdot \mu_X 
%\end{align*}
%Therefore, we can express $\epsilon_i$, as follows
%\begin{align}
%    \epsilon_i = y_i - \hy_i & = y_i - b - w \cdot x_i
%    = y_i - \sum^{n}_{j=1} \frac{1}{n} y_{\!j} + w \cdot \mu_X - w \cdot
%    x_i\notag\\
%    & = y_i -\sum^{n}_{j=1} \frac{1}{n} y_{\!j}  - (x_i - \mu_X) \cdot w\notag\\
%    & =  y_i -\sum^{n}_{j=1} \frac{1}{n} y_{\!j} - 
%    \sum^{n}_{j=1} \frac{(x_i - \mu_X) \cdot (x_{\!j} - \mu_X)}{s_X} \cdot
%    y_{\!j} \notag\\
%    & = \lB(1 - \frac{1}{n} - \frac{(x_i - \mu_X)^2}{s_X} \rB) \cdot y_i - 
%    \sum_{j\ne i} \lB( \frac{1}{n} + \frac{(x_i - \mu_X)\cdot (x_{\!j} -
%    \mu_X)}{s_X} \rB) \cdot y_{\!j}
%    \label{eq:reg:eval:varei_tmp1}
%\end{align}
%where we have separated $y_i$ from the rest of the $y_{\!j}$'s, so that all
%terms in the summation are independent.
%Define $a_{\!j}$ as follows:
%\begin{align}
%a_{\!j} = \lB(\frac{1}{n} + \frac{(x_i - \mu_X)\cdot
%    (x_{\!j} - \mu_X)}{s_X}\rB)
%    \label{eq:reg:eval:aj}
%\end{align}
%Rewriting \cref{eq:reg:eval:varei_tmp1} in terms of $a_{\!j}$, we get
%\begin{align}
%    \var(\epsilon_i) & = \var\Bigl((1 - a_i) \cdot y_i - \sum_{j\ne i} a_{\!j} \cdot
%    y_{\!j}\Bigr) \notag\\
%    & = (1-a_i)^2 \cdot \var(y_i) + \sum_{j\ne i} a_{\!j}^2 \cdot
%    \var(y_{\!j})\notag\\
%    & =\sigma^2 \cdot \bigl(1-2 a_i + a_i^2 + \sum_{j\ne i}
%    a_{\!j}^2\bigr),\quad
%    \text{ since } \var(y_{\!i}) = \var(y_{\!j}) = \sigma^2
%    \notag\\
%    & = \sigma^2 \cdot \bigl(1 - 2 a_i + \sum_{j=1}^n a_{\!j}^2 \bigr)
%    \label{eq:reg:eval:varei_tmp2}
%\end{align}
%Consider the term $\sum_{j=1}^n a_{\!j}^2$, we have
%\begin{align}
%     \sum_{j=1}^n a_{\!j}^2 & = \sum_{j=1}^n \lB(\frac{1}{n} + \frac{(x_i - \mu_X)\cdot
%    (x_{\!j} - \mu_X)}{s_X}\rB)^2 \notag\\
%    & = \sum_{j=1}^n \lB(\frac{1}{n^2} - \frac{2 \cdot (x_i - \mu_X)\cdot
%    (x_{\!j} - \mu_X)}{n \cdot s_X} + \frac{(x_i - \mu_X)^2\cdot
%    (x_{\!j} - \mu_X)^2}{s_X^2}\rB)\notag\\
%    & = \frac{1}{n} - \frac{2\cdot (x_i-\mu_X)}{n\cdot s_X}
%    \sum^{n}_{j=1} (x_{\!j} - \mu_X) + \frac{(x_i - \mu_X)^2}{s_X^2}
%    \sum^{n}_{j=1} (x_{\!j} - \mu_X)^2\notag\\
%    \intertext{since $\sum^{n}_{j=1} (x_{\!j} - \mu_X) = 0$ and
%    $\sum^{n}_{j=1} (x_{\!j} - \mu_X)^2 = s_X$, we get}
%    \sum_{j=1}^n a_{\!j}^2 & = \frac{1}{n} + \frac{(x_i - \mu_X)^2}{s_X}
%    \label{eq:reg:eval:varei_tmp3}
%\end{align}
%Plugging \cref{eq:reg:eval:aj,eq:reg:eval:varei_tmp3} into
%\cref{eq:reg:eval:varei_tmp2}, we get
%\begin{align*}
%    \var(\epsilon_i) & = \sigma^2 \cdot \lB(
%        1 - \frac{2}{n} - \frac{2\cdot(x_i-\mu_X)^2}{s_X} +
%    \frac{1}{n} + \frac{(x_i - \mu_X)^2}{s_X}\rB)\\
%    & = \sigma^2 \cdot \lB(
%    1 - \frac{1}{n} - \frac{(x_i-\mu_X)^2}{s_X}\rB)
%\end{align*}
%
%We can now define the {\em standardized residual} $\epsilon_i^*$ by dividing $\epsilon_i$ by its
%standard deviation after replacing $\sigma^2$ by its estimated value
%$\hsigma^2$. That is,
%\begin{align}
%    \tcbhighmath{
%    \epsilon_i^* = \frac{\epsilon_i}{\sqrt{\var(\epsilon_i)}} =
%    \frac{\epsilon_i}{\displaystyle \hsigma \cdot \sqrt{1 - \frac{1}{n} -
%    \frac{(x_i-\mu_X)^2}{s_X}}}
%}
%\end{align}
%These standardized residuals should follow a standard normal
%distribution. We can thus plot the standardized residuals 
%against the quantiles of a standard
%normal distribution, and check if the normality assumption holds.
%Significant deviations would indicate that our model assumptions may not
%be correct.
%
%%\pgfplotstabletypeset[columns={Q,E}]{REG/eval/figs/QQ-iris.txt}
%\begin{figure}[!t]
%\vspace*{-0.1in}
%    \centerline{
%    \subfloat[QQ-plot]{%
%    \label{fig:reg:eval:bi_qq}
%    \begin{tikzpicture}
%        \tikzset{every mark/.append style={scale=0.7}}
%        \begin{axis}[width=2.65in,
%            label style={font=\small},
%            every axis y label/.style={
%            at={(ticklabel cs:0.5)},rotate=90,anchor=center,
%            font=\small,
%            },
%            % y label style={at={(-0.1,0.5)}}, 
%        xlabel=$Q$,ylabel=$\bepsilon^*$
%]
%\addplot [%scatter, draw=black, scatter src=none, 
%    mark = o,
%    only marks, mark=o] table [x=Q,y=E] {REG/eval/figs/QQ-iris.txt};
%\addplot [draw=gray, mark=none,line width=1pt] 
%table [ x=Q, y=Q]%
%{REG/eval/figs/QQ-iris.txt};
%% \addplot+ [draw=black, mark=none] 
%% table [ x=Q, y={create col/linear regression={y=E}}]%
%% {REG/eval/figs/QQ-iris.txt};
%\end{axis}
%\end{tikzpicture}
%}
%\hspace{0.25in}
%\subfloat[$X$ vs.\ $\bepsilon^*$]{
%    \label{fig:reg:eval:bi_Xe}
%    \begin{tikzpicture}
%        \tikzset{every mark/.append style={scale=0.7}}
%        \begin{axis}[width=2.65in,
%            label style={font=\small},
%            every axis y label/.style={
%            at={(ticklabel cs:0.5)},rotate=90,anchor=center,
%            font=\small,
%            },
%            xlabel=$X$,ylabel=$\bepsilon^*$ ]
%\addplot [%scatter, draw=black, scatter src=none, 
%    only marks, mark=o] table [x=X,y=E] {REG/eval/figs/QQ-iris.txt};
%\end{axis}
%\end{tikzpicture}
%}}
%\vspace{0.2in}
%\caption{Residual plots: \protect\subref{fig:reg:eval:bi_qq} Quantile-Quantile scatter plot of normal quantiles versus
%standardized residuals. \protect\subref{fig:reg:eval:bi_Xe} Independent variable
%versus standardized residuals.}
%\label{fig:reg:eval:bi_qqXe}
%\vspace{-0.2in}
%\end{figure}
%
%\begin{example}[Standardized Residuals]
%    Consider the Iris dataset from \cref{ex:reg:eval:bi_var_gf}, with
%    the predictor variable ({\tt petal length}) and response
%    variable ({\tt petal width}), and $n=150$.
%    \cref{fig:reg:eval:bi_qq} shows the quantile-quantile (QQ) plot. 
%    The $y$-axis is the list of standardized residuals sorted from the
%    smallest to the largest. The
%    $x$-axis is the list of the quantiles of the standard 
%    normal distribution for a sample
%    of size $n$, defined as
%    \begin{align*}
%        Q & = (q_1, q_2, \ldots, q_n)^T\\
%        q_i & = F^{-1}\lB(\frac{i-0.5}{n}\rB)
%    \end{align*}
%    where $F$ is the cumulative distribution function (CDF), 
%    and $F^{-1}$ is the inverse CDF or quantile function
%    (see \cref{eq:eda:numeric:icdf}) for the normal distribution.
%    Thus, the $Q$ values are also sorted in increasing order from
%    smallest to largest. If the standardized residuals follow a normal
%    distribution, then the QQ plot should follow a straight line.
%    \cref{fig:reg:eval:bi_qq} plots this perfect line for comparison.
%    We can observe that the residuals are essentially 
%    normally distributed.
%    %except perhaps for a few points
%    %on the top right of the plot that have slightly larger residuals.
%
%    The plot of the independent variable $X$ versus the standardized
%    residuals is also instructive. We can see in
%    \cref{fig:reg:eval:bi_Xe} that there is no particular trend or
%    pattern to the residuals, and the residual values are concentrated
%    along the mean value of $0$, with the majority of the points being
%    within two standard deviations of the mean, as expected if they were
%    sampled from a normal distribution.
%\end{example}
%
%\vspace*{-0.1in}
%\section{Multiple Regression}
%\index{multiple regression}
%
%In multiple regression there are multiple independent attributes $X_1,
%X_2, \cdots, X_d$ and a single dependent or response attribute $Y$, and
%we assume that the true relationship can be modeled as a linear function
%\begin{align*}
%    Y = \trueb + \truew_1 \cdot X_1 + \truew_2 \cdot X_2 + \ldots + \truew_d \cdot X_d +
%    \varepsilon
%\end{align*}
%where $\trueb$ is the intercept or bias term and $\truew_i$ is the regression
%coefficient for attribute $X_i$. Recall that $\truew_i$ denotes the expected
%increase in $Y$ with a unit increase in the value of $X_i$, assuming all
%other variables are held fixed.
%We assume that $\varepsilon$ is a random variable that is normally
%distributed with mean $\mu=0$
%and variance $\sigma^2$. Further, we assume that the errors for
%different observations are all independent of each other, and
%consequently the observed responses are also independent.
%
%
%
%\subsubsection{Mean and Variance of Response Variable}
%Let $\bX = (X_1, X_2, \cdots, X_d)^T \in \setR^d$ denote the multivariate random
%variable comprising the independent attributes. Let $\bx = (x_1, x_2,
%\cdots, x_d)^T$ be some {\em fixed} value of $\bX$, and let $\truebw = (\truew_1,
%\truew_2, \cdots, \truew_d)^T$.
%The expected response value is then given as
%\begin{align*}
%    E[Y| \bX=\bx] & = E[\trueb + \truew_1 \cdot x_1 + \ldots + \truew_d \cdot x_d +
%    \varepsilon] = E\lB[\trueb + \sum_{i=1}^d \truew_i \cdot x_i\rB] + E[\varepsilon]\\
%    & = \trueb + \truew_1 \cdot x_1 + \ldots + \truew_d \cdot x_d = \trueb + \truebw^T\bx
%\end{align*}
%which follows from the assumption that $E[\varepsilon] = 0$.
%The variance of the response variable is given as
%\begin{align*}
%    \var(Y|\bX=\bx) & = \var\lB(\trueb + \sum_{i=1}^d \truew_i \cdot x_i +
%    \varepsilon\rB) = \var\lB(\trueb + \sum_{i=1}^d \truew_i \cdot x_i \rB) 
%    + \var(\varepsilon) = 0 + \sigma^2 = \sigma^2
%\end{align*}
%which follows from the assumption that all $x_i$ are fixed {\it a
%priori}.
%Thus, we conclude that $Y$ also follows a normal distribution with mean
%$E[Y|\bx] = \trueb + \sum_{i=1}^d \truew_i \cdot x_i = \trueb + \truebw^T\bx$ and variance $\var(Y|\bx)
%= \sigma^2$.
%
%
%\subsubsection{Estimated Parameters}
%The true parameters $\trueb$, $\truew_1, \truew_2, \cdots, \truew_d$ and $\sigma^2$ are all unknown, and have to
%be estimated from the training data $\bD$ comprising $n$ points $\bx_i$
%and corresponding response values $y_i$, for $i=1,2,\cdots,n$. 
%We augment the data matrix by adding a new column $X_0$ with all values
%fixed at $1$, that is, $X_0 = \bone$. Thus, the augmented data $\abD \in \setR^{n \times (d+1)}$ comprises the $(d+1)$ attributes $X_0,
%X_1, X_2, \cdots, X_d$, and each augmented point is given as $\abx_i = (1, x_{i1},
%x_{i2}, \cdots, x_{id})^T$. 
%
%Let
%$b = w_0$ denote the estimated bias term, and let $w_i$
%denote the estimated regression weights. The augmented vector of estimated
%weights, including the bias term, is 
%\begin{align*}
%    \abw = (w_0, w_1, \cdots, w_d)^T 
%\end{align*}
%We then make predictions for
%any given point $\bx_i$ as follows:
%\begin{equation*}
%    \hy_i = b \cdot 1 + w_1 \cdot x_{i1} + \cdots w_d \cdot x_{id}
%    = \abw^T\abx_i
%\end{equation*}
%Recall that these estimates are obtained
%by minimizing the sum of
%squared errors (SSE), given as
%\begin{align*}
%    SSE = \sum_{i=1}^n (y_i - \hy_i)^2 
%    = \sum^{n}_{i=1} \lB(y_i - b - \sum_{j=1}^d w_j \cdot x_{ij}\rB)^2 
%\end{align*}
%with the least squares estimate [\cref{eq:reg:linear:multiple_w}] 
%given as
%\begin{align*}
%    \abw & = \lB(\abD^T\abD\rB)^{-1} \abD^T Y
%\end{align*}
%
%The estimated variance $\hsigma^2$ is then given as
%\index{multiple regression!estimated variance}
%\begin{align}
%    \tcbhighmath{
%    \hsigma^2 = \frac{SSE}{n-(d+1)} = \frac{1}{n-d-1} \cdot \sum_{i=1}^n (y_i
%- \hy_i)^2}
%    \label{eq:reg:eval:sse_est_multivar}
%\end{align}
%We divide by $n-(d+1)$ to get an unbiased estimate, 
%since $n-(d+1)$ is the number of degrees of freedom for
%estimating SSE (see \cref{fig:reg:eval:multi_geom}). In other words, out of the $n$ training points, we need
%to estimate $d+1$ parameters, $\trueb$ and the $\truew_i$'s, 
%with $n-(d+1)$ remaining degrees of
%freedom.
%
%\paragraph{Estimated Variance is Unbiased}
%We now show that $\hsigma^2$ is an unbiased estimator of the true (but
%unknown) variance $\sigma^2$. 
%Recall from \cref{eq:reg:linear:multiple_hY} that
%\begin{align*}
%    \hY = \abD \abw = \abD(\abD^T\abD)^{-1} \abD^T Y = \bH Y
%\end{align*}
%where $\bH$ is the $n \times n$ hat matrix (assuming that
%$(\abD^T\abD)^{-1}$ exists). Note that $\bH$ is an {\em orthogonal
%projection matrix}, since it is symmetric ($\bH^T = \bH$) and idempotent
%($\bH^2 = \bH$). The hat matrix $\bH$ is symmetric since
%\begin{align*}
%    \bH^T = \bigl(\abD(\abD^T\abD)^{-1} \abD^T\bigr)^T = \bigl(\abD^T\bigr)^T 
%    \bigl((\abD^T\abD)^{T}\bigr)^{-1} \abD^T = \bH
%\end{align*}
%and it is idempotent since 
%\begin{align*}
%    \bH^2 = \abD(\abD^T\abD)^{-1} \abD^T\abD (\abD^T\abD)^{-1} \abD^T =
%    \abD(\abD^T\abD)^{-1} \abD^T = \bH
%\end{align*}
%Furthermore, the trace of the hat matrix is
%given as
%\begin{align*}
%    \tr(\bH) = \tr(\abD(\abD^T\abD)^{-1} \abD^T) =
%    \tr(\abD^T\abD(\abD^T\abD)^{-1}) = \tr(\bI_{(d+1)}) = d+1
%\end{align*}
%where $\bI_{(d+1)}$ is the $(d+1) \times (d+1)$ identity matrix, and 
%we used the fact that the trace of a product of matrices is
%invariant under cyclic permutations.
%
%Finally, note that the matrix $\bI - \bH$ is also symmetric and
%idempotent, where $\bI$ is the $n \times n$ identity matrix, since
%\begin{align*}
%    (\bI - \bH)^T & = \bI^T - \bH^T = \bI - \bH\\
%    (\bI - \bH)^2 & = (\bI - \bH) (\bI - \bH) =
%    \bI - \bH -\bH + \bH^2  = \bI - \bH
%\end{align*}
%
%Now consider the squared error; we have
%\begin{align}
%    SSE & = \norm[\big]{Y - \hY}^2 = \norm[\big]{Y - \bH Y}^2 = 
%    \norm[\big]{(\bI - \bH)Y}^2 \notag\\
%    & = Y^T (\bI -\bH) (\bI - \bH) Y = Y^T (\bI - \bH) Y
%    \label{eq:reg:eval:sse_yIHy}
%\end{align}
%However, note that the response vector $Y$ is given as
%\begin{align*}
%    Y = \abD\trueabw + \bvepsilon
%\end{align*}
%where $\trueabw = (\truew_0, \truew_1, \cdots, \truew_d)^T$ is the true
%(augmented)
%vector of parameters of
%the model, and $\bvepsilon = (\varepsilon_1, \varepsilon_2, \cdots,
%\varepsilon_n)^T$ is the true error vector, which is assumed to
%be normally distributed with mean $E[\bvepsilon] = \bzero$ and with fixed
%variance $\varepsilon_i = \sigma^2$ for each point, so that
%$\ocov(\bvepsilon) = \sigma^2 \bI$.
%Plugging the expression of $Y$ into \cref{eq:reg:eval:sse_yIHy}, we get
%\begin{align*}
%    SSE & = Y^T (\bI - \bH) Y = (\abD\trueabw + \bvepsilon)^T (\bI - \bH)
%    (\abD\trueabw + \bvepsilon)
%     = (\abD\trueabw + \bvepsilon)^T \bigl( \underbrace{(\bI - \bH)
%        \abD\trueabw}_{\bzero} + (\bI - \bH) \bvepsilon \bigr)\\
%        & = \bigl((\bI - \bH) \bvepsilon\bigr)^T(\abD\trueabw + \bvepsilon) 
%    = \bvepsilon^T (\bI - \bH) (\abD\trueabw + \bvepsilon)\\
%    & = \bvepsilon^T \underbrace{(\bI - \bH)\abD\trueabw}_{\bzero} + \bvepsilon^T (\bI - \bH)\bvepsilon
%     = \bvepsilon^T (\bI - \bH) \bvepsilon
%\end{align*}
%where we use the observation that
%$$(\bI - \bH)\abD \trueabw = \abD \trueabw - \bH
%\abD \trueabw = \abD \trueabw - (\abD(\abD^T\abD)^{-1} \abD^{T})\abD \trueabw = \abD \trueabw - \abD
%\trueabw = \bzero$$
%
%Let us consider the expected value of SSE; we have
%\begin{align*}
%    E[SSE] & = E\lB[\bvepsilon^T (\bI - \bH) \bvepsilon\rB]\\
%     & = E\lB[ \sum^{n}_{i=1} \varepsilon_i^2 - \sum^{n}_{i=1} \sum^{n}_{j=1} h_{ij} \varepsilon_i
%    \varepsilon_j\rB]
%    = \sum^{n}_{i=1} E[\varepsilon_i^2] - \sum^{n}_{i=1} \sum^{n}_{j=1}
%    h_{ij} E[\varepsilon_i \varepsilon_j]\\
%    & = \sum^{n}_{i=1} (1 - h_{ii}) E[\varepsilon_i^2],\; \text{since
%        $\varepsilon_i$ are independent, and therefore
%    $E[\varepsilon_i\varepsilon_j] = 0$}\\
%    & = \Bigl(n - \sum^{n}_{i=1} h_{ii}\Bigr) \sigma^2 = \bigl(n -
%    \tr(\bH)\bigr)
%    \sigma^2 = (n-d-1) \cdot \sigma^2
%\end{align*}
%where we used the fact that $\sigma^2 = \var(\varepsilon_i) = E[\varepsilon_i^2] -
%\bigl(E[\varepsilon_i]\bigr)^2 = E[\varepsilon_i^2]$, since $E[\varepsilon_i] =
%0$. It follows that
%\begin{align}
%    \hsigma^2 = E\lB[\frac{SSE}{(n-d-1)} \rB] = \frac{1}{(n-d-1)} E[SSE] =
%    \frac{1}{(n-d-1)} \cdot (n-d-1) \cdot \sigma^2 = \sigma^2
%    \label{eq:reg:eval:sse_est_unbiased}
%\end{align}
%
%\index{multiple regression!goodness of fit}
%\subsection{Goodness of Fit}
%Following the derivation in \cref{sec:reg:eval:goodness_fit},
%the decomposition of the total sum of squares, TSS, into the sum of
%squared errors, SSE, and the residual sum of squares, RSS, holds true
%for multiple regression as well:
%\begin{align*}
%    TSS & = SSE + RSS\\
%    \sum^{n}_{i=1} (y_i - \mu_Y)^2 = &
%    \sum^{n}_{i=1} (y_i - \hy_i)^2 +
%    \sum^{n}_{i=1} (\hy_i - \mu_Y)^2
%\end{align*}
%
%The {\em coefficient of multiple determination}, $R^2$, gives the goodness of fit,
%measured as the fraction of the variation explained by the linear model:
%\index{regression!coefficient of multiple determination}
%\index{regression!R$^2$ statistic}
%\index{coefficient of multiple determination}
%\begin{align}
%    \tcbhighmath{
%    R^2 = 1 - \frac{SSE}{TSS} = \frac{TSS-SSE}{TSS} =
%\frac{RSS}{TSS}}
%\end{align}
%
%One of the potential problems with the $R^2$ measure is that it is
%susceptible to increase as the number of attributes increase, even
%though the additional attributes may be uninformative. To counter this,
%we can consider the {\em adjusted coefficient of determination}, which
%takes into account the degrees of freedom in both TSS and SSE
%\index{regression!adjusted coefficient of determination}
%\index{regression!adjusted R$^{2}$ statistic}
%\index{adjusted coefficient of determination}
%\begin{align}
%    \tcbhighmath{
%    R_a^2 = 1 - \frac{SSE \big/ (n-d-1)}{TSS \big/ (n-1)} = 
%1 - \frac{(n-1) \cdot SSE}{ (n-d-1) \cdot TSS}}
%\end{align}
%We can observe that the adjusted $R_a^2$ measure is always less than
%$R^2$, since the ratio $\tfrac{n-1}{n-d-1} > 1$. If there is too much of
%a difference between $R^2$ and $R_a^2$, it might indicate that there are
%potentially many, possibly irrelevant, attributes being used to fit the
%model.
%
%
%
%\begin{figure}[t!]
%    \vspace*{-0.3in}
%\centering
%\psset{unit=0.25in}
%\scalebox{0.95}{%
%%\psset{viewpoint=20 7 7 rtp2xyz,Decran=30}
%\psset{viewpoint=20 10 10 rtp2xyz,Decran=30}
%\psset{lightsrc=viewpoint}
%\psset{incolor=white}
%\psset{opacity=0.2}
%\psset{fillcolor=white}
%\begin{pspicture}(3,-5)(8,7)
%    \axesIIID[axisnames={X_1, X_2, Y}](0,0,0)(9,8,3)
%%\psSolid[object=cube, 
%%    intersectiontype=0,
%%    intersectionplan={[-0.0819 0.45 -1 -0.0139]},
%%    intersectioncolor=(black)]
%% \psSolid[object=plan, 
%%     definition=equation,
%%     args={[0.0819 -0.45 1 0.0139]},
%%     base=0 8.5 0 7.5, 
%%     fillcolor=lightgray,
%%     %plangrid,
%%     action=draw*]
%\psset{linewidth=0.5pt, dotsize=0.3}
%\psset{dotstyle=Bo,fillcolor=gray}
%\psPoint(6.9,4.9,1.5){p134}
\psPoint(6.9,4.9,1.62599){p135}
\psline(p134)(p135)
\psdot(p134)
\psPoint(6.6,4.6,1.3){p136}
\psPoint(6.6,4.6,1.51556){p137}
\psline(p136)(p137)
\psdot(p136)
\psPoint(4.6,1.4,0.2){p138}
\psPoint(4.6,1.4,0.23936){p139}
\psline(p138)(p139)
\psdot(p138)
\psPoint(6.0,4.0,1.0){p140}
\psPoint(6.0,4.0,1.2947){p141}
\psline(p140)(p141)
\psdot(p140)
\psPoint(6.7,5.8,1.8){p142}
\psPoint(6.7,5.8,2.04737){p143}
\psline(p142)(p143)
\psdot(p142)
\psPoint(5.7,4.2,1.2){p144}
\psPoint(5.7,4.2,1.40927){p145}
\psline(p144)(p145)
\psdot(p144)
\psPoint(4.9,1.5,0.1){p146}
\psPoint(4.9,1.5,0.25979){p147}
\psline(p146)(p147)
\psdot(p146)
\psPoint(5.0,1.5,0.2){p148}
\psPoint(5.0,1.5,0.2516){p149}
\psline(p148)(p149)
\psdot(p148)
\psPoint(5.7,4.1,1.3){p150}
\psPoint(5.7,4.1,1.36427){p151}
\psline(p150)(p151)
\psdot(p150)
\psPoint(5.0,1.4,0.2){p152}
\psPoint(5.0,1.4,0.2066){p153}
\psline(p152)(p153)
\psdot(p152)
\psPoint(7.2,6.0,1.8){p154}
\psPoint(7.2,6.0,2.09642){p155}
\psline(p154)(p155)
\psdot(p154)
\psPoint(5.5,4.0,1.3){p156}
\psPoint(5.5,4.0,1.33565){p157}
\psline(p156)(p157)
\psdot(p156)
\psPoint(5.5,4.0,1.3){p158}
\psPoint(5.5,4.0,1.33565){p159}
\psline(p158)(p159)
\psdot(p158)
\psPoint(4.6,1.5,0.2){p160}
\psPoint(4.6,1.5,0.28436){p161}
\psline(p160)(p161)
\psdot(p160)
\psPoint(7.2,5.8,1.6){p162}
\psPoint(7.2,5.8,2.00642){p163}
\psline(p162)(p163)
\psdot(p162)
\psPoint(5.8,4.0,1.2){p164}
\psPoint(5.8,4.0,1.31108){p165}
\psline(p164)(p165)
\psdot(p164)
\psPoint(5.6,3.9,1.1){p166}
\psPoint(5.6,3.9,1.28246){p167}
\psline(p166)(p167)
\psdot(p166)
\psPoint(5.8,4.1,1.0){p168}
\psPoint(5.8,4.1,1.35608){p169}
\psline(p168)(p169)
\psdot(p168)
\psPoint(5.1,1.9,0.4){p170}
\psPoint(5.1,1.9,0.42341){p171}
\psline(p170)(p171)
\psdot(p170)
\psPoint(6.3,4.4,1.3){p172}
\psPoint(6.3,4.4,1.45013){p173}
\psline(p172)(p173)
\psdot(p172)
\psPoint(5.6,4.5,1.5){p174}
\psPoint(5.6,4.5,1.55246){p175}
\psline(p174)(p175)
\psdot(p174)
\psPoint(6.1,4.6,1.4){p176}
\psPoint(6.1,4.6,1.55651){p177}
\psline(p176)(p177)
\psdot(p176)
\psPoint(7.3,6.3,1.8){p178}
\psPoint(7.3,6.3,2.22323){p179}
\psline(p178)(p179)
\psdot(p178)
\psPoint(5.6,4.2,1.3){p180}
\psPoint(5.6,4.2,1.41746){p181}
\psline(p180)(p181)
\psdot(p180)
\psPoint(4.8,1.4,0.1){p182}
\psPoint(4.8,1.4,0.22298){p183}
\psline(p182)(p183)
\psdot(p182)
\psPoint(5.7,3.5,1.0){p184}
\psPoint(5.7,3.5,1.09427){p185}
\psline(p184)(p185)
\psdot(p184)
\psPoint(5.3,1.5,0.2){p186}
\psPoint(5.3,1.5,0.22703){p187}
\psline(p186)(p187)
\psdot(p186)
\psPoint(5.7,4.2,1.3){p188}
\psPoint(5.7,4.2,1.40927){p189}
\psline(p188)(p189)
\psdot(p188)
\psPoint(4.4,1.3,0.2){p190}
\psPoint(4.4,1.3,0.21074){p191}
\psline(p190)(p191)
\psdot(p190)
\psPoint(6.1,4.7,1.2){p192}
\psPoint(6.1,4.7,1.60151){p193}
\psline(p192)(p193)
\psdot(p192)
\psPoint(6.1,5.6,1.4){p194}
\psPoint(6.1,5.6,2.00651){p195}
\psline(p194)(p195)
\psdot(p194)
\psPoint(5.6,4.1,1.3){p196}
\psPoint(5.6,4.1,1.37246){p197}
\psline(p196)(p197)
\psdot(p196)
\psPoint(5.4,4.5,1.5){p198}
\psPoint(5.4,4.5,1.56884){p199}
\psline(p198)(p199)
\psdot(p198)
\psPoint(4.9,1.5,0.1){p200}
\psPoint(4.9,1.5,0.25979){p201}
\psline(p200)(p201)
\psdot(p200)
\psPoint(7.9,6.4,2.0){p202}
\psPoint(7.9,6.4,2.21909){p203}
\psline(p202)(p203)
\psdot(p202)
\psPoint(7.7,6.9,2.3){p204}
\psPoint(7.7,6.9,2.46047){p205}
\psline(p204)(p205)
\psdot(p204)
\psPoint(4.7,1.6,0.2){p206}
\psPoint(4.7,1.6,0.32117){p207}
\psline(p206)(p207)
\psdot(p206)
\psPoint(4.4,1.4,0.2){p208}
\psPoint(4.4,1.4,0.25574){p209}
\psline(p208)(p209)
\psdot(p208)
\psPoint(6.0,4.5,1.5){p210}
\psPoint(6.0,4.5,1.5197){p211}
\psline(p210)(p211)
\psdot(p210)
\psPoint(5.0,3.3,1.0){p212}
\psPoint(5.0,3.3,1.0616){p213}
\psline(p212)(p213)
\psdot(p212)
\psPoint(6.3,5.6,1.8){p214}
\psPoint(6.3,5.6,1.99013){p215}
\psline(p214)(p215)
\psdot(p214)
\psPoint(5.0,3.5,1.0){p216}
\psPoint(5.0,3.5,1.1516){p217}
\psline(p216)(p217)
\psdot(p216)
\psPoint(4.8,1.6,0.2){p218}
\psPoint(4.8,1.6,0.31298){p219}
\psline(p218)(p219)
\psdot(p218)
\psPoint(4.4,1.3,0.2){p220}
\psPoint(4.4,1.3,0.21074){p221}
\psline(p220)(p221)
\psdot(p220)
\psPoint(6.2,4.5,1.5){p222}
\psPoint(6.2,4.5,1.50332){p223}
\psline(p222)(p223)
\psdot(p222)
\psPoint(6.0,5.0,1.5){p224}
\psPoint(6.0,5.0,1.7447){p225}
\psline(p224)(p225)
\psdot(p224)
\psPoint(7.4,6.1,1.9){p226}
\psPoint(7.4,6.1,2.12504){p227}
\psline(p226)(p227)
\psdot(p226)
\psPoint(4.9,3.3,1.0){p228}
\psPoint(4.9,3.3,1.06979){p229}
\psline(p228)(p229)
\psdot(p228)
\psPoint(7.0,4.7,1.4){p230}
\psPoint(7.0,4.7,1.5278){p231}
\psline(p230)(p231)
\psdot(p230)
\psPoint(5.5,3.7,1.0){p232}
\psPoint(5.5,3.7,1.20065){p233}
\psline(p232)(p233)
\psdot(p232)
\psPoint(6.8,4.8,1.4){p234}
\psPoint(6.8,4.8,1.58918){p235}
\psline(p234)(p235)
\psdot(p234)
\psPoint(6.1,4.7,1.4){p236}
\psPoint(6.1,4.7,1.60151){p237}
\psline(p236)(p237)
\psdot(p236)
\psPoint(6.7,4.4,1.4){p238}
\psPoint(6.7,4.4,1.41737){p239}
\psline(p238)(p239)
\psdot(p238)
\psPoint(4.8,1.9,0.2){p240}
\psPoint(4.8,1.9,0.44798){p241}
\psline(p240)(p241)
\psdot(p240)
\psPoint(4.9,1.4,0.2){p242}
\psPoint(4.9,1.4,0.21479){p243}
\psline(p242)(p243)
\psdot(p242)
\psPoint(4.3,1.1,0.1){p244}
\psPoint(4.3,1.1,0.12893){p245}
\psline(p244)(p245)
\psdot(p244)
\psPoint(5.0,1.4,0.2){p246}
\psPoint(5.0,1.4,0.2066){p247}
\psline(p246)(p247)
\psdot(p246)
\psPoint(6.4,4.3,1.3){p248}
\psPoint(6.4,4.3,1.39694){p249}
\psline(p248)(p249)
\psdot(p248)
\psPoint(5.2,1.5,0.2){p250}
\psPoint(5.2,1.5,0.23522){p251}
\psline(p250)(p251)
\psdot(p250)
\psPoint(5.8,3.9,1.2){p252}
\psPoint(5.8,3.9,1.26608){p253}
\psline(p252)(p253)
\psdot(p252)
\psPoint(7.6,6.6,2.1){p254}
\psPoint(7.6,6.6,2.33366){p255}
\psline(p254)(p255)
\psdot(p254)
\psPoint(6.3,5.1,1.5){p256}
\psPoint(6.3,5.1,1.76513){p257}
\psline(p256)(p257)
\psdot(p256)
\psPoint(6.4,5.5,1.8){p258}
\psPoint(6.4,5.5,1.93694){p259}
\psline(p258)(p259)
\psdot(p258)
\psPoint(6.3,4.9,1.5){p260}
\psPoint(6.3,4.9,1.67513){p261}
\psline(p260)(p261)
\psdot(p260)
\psPoint(5.0,1.6,0.2){p262}
\psPoint(5.0,1.6,0.2966){p263}
\psline(p262)(p263)
\psdot(p262)
\psPoint(6.7,4.7,1.5){p264}
\psPoint(6.7,4.7,1.55237){p265}
\psline(p264)(p265)
\psdot(p264)
\psPoint(6.0,5.1,1.6){p266}
\psPoint(6.0,5.1,1.7897){p267}
\psline(p266)(p267)
\psdot(p266)
\psPoint(4.8,1.6,0.2){p268}
\psPoint(4.8,1.6,0.31298){p269}
\psline(p268)(p269)
\psdot(p268)
\psPoint(5.7,4.5,1.3){p270}
\psPoint(5.7,4.5,1.54427){p271}
\psline(p270)(p271)
\psdot(p270)
\psPoint(5.1,1.6,0.2){p272}
\psPoint(5.1,1.6,0.28841){p273}
\psline(p272)(p273)
\psdot(p272)
\psPoint(6.6,4.4,1.4){p274}
\psPoint(6.6,4.4,1.42556){p275}
\psline(p274)(p275)
\psdot(p274)
\psPoint(5.2,1.5,0.1){p276}
\psPoint(5.2,1.5,0.23522){p277}
\psline(p276)(p277)
\psdot(p276)
\psPoint(7.7,6.7,2.0){p278}
\psPoint(7.7,6.7,2.37047){p279}
\psline(p278)(p279)
\psdot(p278)
\psPoint(4.9,1.5,0.1){p280}
\psPoint(4.9,1.5,0.25979){p281}
\psline(p280)(p281)
\psdot(p280)
\psPoint(5.4,1.5,0.2){p282}
\psPoint(5.4,1.5,0.21884){p283}
\psline(p282)(p283)
\psdot(p282)
\psPoint(6.5,5.5,1.8){p284}
\psPoint(6.5,5.5,1.92875){p285}
\psline(p284)(p285)
\psdot(p284)
\psPoint(5.5,3.8,1.1){p286}
\psPoint(5.5,3.8,1.24565){p287}
\psline(p286)(p287)
\psdot(p286)
\psPoint(6.2,4.3,1.3){p288}
\psPoint(6.2,4.3,1.41332){p289}
\psline(p288)(p289)
\psdot(p288)
\psPoint(6.5,4.6,1.5){p290}
\psPoint(6.5,4.6,1.52375){p291}
\psline(p290)(p291)
\psdot(p290)
\psPoint(5.4,1.7,0.2){p292}
\psPoint(5.4,1.7,0.30884){p293}
\psline(p292)(p293)
\psdot(p292)
\psPoint(5.5,4.4,1.2){p294}
\psPoint(5.5,4.4,1.51565){p295}
\psline(p294)(p295)
\psdot(p294)
\psPoint(7.7,6.7,2.2){p296}
\psPoint(7.7,6.7,2.37047){p297}
\psline(p296)(p297)
\psdot(p296)
\psPoint(5.1,1.5,0.2){p298}
\psPoint(5.1,1.5,0.24341){p299}
\psline(p298)(p299)
\psdot(p298)

%\psSurface[showAxes=false,ngrid=.2 .2,fillcolor=lightgray,%axesboxed,
%        linewidth=0.5\pslinewidth,
%        color1 = {[rgb]{0 0 0}},
%        color2 = {[rgb]{1 1 1}},
%        hue=(color1) (color2), lightintensity=5,
%    algebraic,Zmin=0,Zmax=3](0,0)(8,7){%
%    (-0.0819*x) + (0.45*y) - 0.0139 }
%\psset{dotstyle=Bo,fillcolor=white}
%\psPoint(5.9,4.2,1.5){p0}
\psPoint(5.9,4.2,1.39289){p1}
\psline(p0)(p1)
\psdot(p0)
\psPoint(4.7,1.3,0.2){p2}
\psPoint(4.7,1.3,0.18617){p3}
\psline(p2)(p3)
\psdot(p2)
\psPoint(6.5,5.8,2.2){p4}
\psPoint(6.5,5.8,2.06375){p5}
\psline(p4)(p5)
\psdot(p4)
\psPoint(5.8,5.1,1.9){p6}
\psPoint(5.8,5.1,1.80608){p7}
\psline(p6)(p7)
\psdot(p6)
\psPoint(6.7,5.6,2.4){p8}
\psPoint(6.7,5.6,1.95737){p9}
\psline(p8)(p9)
\psdot(p8)
\psPoint(5.1,1.5,0.4){p10}
\psPoint(5.1,1.5,0.24341){p11}
\psline(p10)(p11)
\psdot(p10)
\psPoint(5.1,1.5,0.3){p12}
\psPoint(5.1,1.5,0.24341){p13}
\psline(p12)(p13)
\psdot(p12)
\psPoint(6.1,4.9,1.8){p14}
\psPoint(6.1,4.9,1.69151){p15}
\psline(p14)(p15)
\psdot(p14)
\psPoint(5.0,1.6,0.4){p16}
\psPoint(5.0,1.6,0.2966){p17}
\psline(p16)(p17)
\psdot(p16)
\psPoint(5.9,5.1,1.8){p18}
\psPoint(5.9,5.1,1.79789){p19}
\psline(p18)(p19)
\psdot(p18)
\psPoint(6.5,5.2,2.0){p20}
\psPoint(6.5,5.2,1.79375){p21}
\psline(p20)(p21)
\psdot(p20)
\psPoint(5.7,1.5,0.4){p22}
\psPoint(5.7,1.5,0.19427){p23}
\psline(p22)(p23)
\psdot(p22)
\psPoint(4.9,4.5,1.7){p24}
\psPoint(4.9,4.5,1.60979){p25}
\psline(p24)(p25)
\psdot(p24)
\psPoint(5.0,1.6,0.6){p26}
\psPoint(5.0,1.6,0.2966){p27}
\psline(p26)(p27)
\psdot(p26)
\psPoint(6.8,5.9,2.3){p28}
\psPoint(6.8,5.9,2.08418){p29}
\psline(p28)(p29)
\psdot(p28)
\psPoint(5.4,1.3,0.4){p30}
\psPoint(5.4,1.3,0.12884){p31}
\psline(p30)(p31)
\psdot(p30)
\psPoint(5.0,1.2,0.2){p32}
\psPoint(5.0,1.2,0.1166){p33}
\psline(p32)(p33)
\psdot(p32)
\psPoint(5.7,5.0,2.0){p34}
\psPoint(5.7,5.0,1.76927){p35}
\psline(p34)(p35)
\psdot(p34)
\psPoint(5.1,3.0,1.1){p36}
\psPoint(5.1,3.0,0.91841){p37}
\psline(p36)(p37)
\psdot(p36)
\psPoint(6.3,5.0,1.9){p38}
\psPoint(6.3,5.0,1.72013){p39}
\psline(p38)(p39)
\psdot(p38)
\psPoint(6.8,5.5,2.1){p40}
\psPoint(6.8,5.5,1.90418){p41}
\psline(p40)(p41)
\psdot(p40)
\psPoint(7.1,5.9,2.1){p42}
\psPoint(7.1,5.9,2.05961){p43}
\psline(p42)(p43)
\psdot(p42)
\psPoint(5.7,1.7,0.3){p44}
\psPoint(5.7,1.7,0.28427){p45}
\psline(p44)(p45)
\psdot(p44)
\psPoint(5.6,4.9,2.0){p46}
\psPoint(5.6,4.9,1.73246){p47}
\psline(p46)(p47)
\psdot(p46)
\psPoint(6.3,6.0,2.5){p48}
\psPoint(6.3,6.0,2.17013){p49}
\psline(p48)(p49)
\psdot(p48)
\psPoint(5.4,1.5,0.4){p50}
\psPoint(5.4,1.5,0.21884){p51}
\psline(p50)(p51)
\psdot(p50)
\psPoint(6.3,5.6,2.4){p52}
\psPoint(6.3,5.6,1.99013){p53}
\psline(p52)(p53)
\psdot(p52)
\psPoint(6.9,5.4,2.1){p54}
\psPoint(6.9,5.4,1.85099){p55}
\psline(p54)(p55)
\psdot(p54)
\psPoint(7.7,6.1,2.3){p56}
\psPoint(7.7,6.1,2.10047){p57}
\psline(p56)(p57)
\psdot(p56)
\psPoint(5.6,3.6,1.3){p58}
\psPoint(5.6,3.6,1.14746){p59}
\psline(p58)(p59)
\psdot(p58)
\psPoint(6.4,5.3,1.9){p60}
\psPoint(6.4,5.3,1.84694){p61}
\psline(p60)(p61)
\psdot(p60)
\psPoint(5.0,1.3,0.3){p62}
\psPoint(5.0,1.3,0.1616){p63}
\psline(p62)(p63)
\psdot(p62)
\psPoint(5.1,1.7,0.5){p64}
\psPoint(5.1,1.7,0.33341){p65}
\psline(p64)(p65)
\psdot(p64)
\psPoint(5.8,5.1,2.4){p66}
\psPoint(5.8,5.1,1.80608){p67}
\psline(p66)(p67)
\psdot(p66)
\psPoint(4.6,1.0,0.2){p68}
\psPoint(4.6,1.0,0.05936){p69}
\psline(p68)(p69)
\psdot(p68)
\psPoint(5.2,1.4,0.2){p70}
\psPoint(5.2,1.4,0.19022){p71}
\psline(p70)(p71)
\psdot(p70)
\psPoint(6.1,4.0,1.3){p72}
\psPoint(6.1,4.0,1.28651){p73}
\psline(p72)(p73)
\psdot(p72)
\psPoint(5.5,1.3,0.2){p74}
\psPoint(5.5,1.3,0.12065){p75}
\psline(p74)(p75)
\psdot(p74)
\psPoint(4.6,1.4,0.3){p76}
\psPoint(4.6,1.4,0.23936){p77}
\psline(p76)(p77)
\psdot(p76)
\psPoint(6.2,4.8,1.8){p78}
\psPoint(6.2,4.8,1.63832){p79}
\psline(p78)(p79)
\psdot(p78)
\psPoint(4.8,1.4,0.3){p80}
\psPoint(4.8,1.4,0.22298){p81}
\psline(p80)(p81)
\psdot(p80)
\psPoint(6.2,5.4,2.3){p82}
\psPoint(6.2,5.4,1.90832){p83}
\psline(p82)(p83)
\psdot(p82)
\psPoint(6.4,4.5,1.5){p84}
\psPoint(6.4,4.5,1.48694){p85}
\psline(p84)(p85)
\psdot(p84)
\psPoint(6.7,5.0,1.7){p86}
\psPoint(6.7,5.0,1.68737){p87}
\psline(p86)(p87)
\psdot(p86)
\psPoint(5.9,4.8,1.8){p88}
\psPoint(5.9,4.8,1.66289){p89}
\psline(p88)(p89)
\psdot(p88)
\psPoint(6.7,5.7,2.5){p90}
\psPoint(6.7,5.7,2.00237){p91}
\psline(p90)(p91)
\psdot(p90)
\psPoint(5.4,1.7,0.4){p92}
\psPoint(5.4,1.7,0.30884){p93}
\psline(p92)(p93)
\psdot(p92)
\psPoint(6.3,4.9,1.8){p94}
\psPoint(6.3,4.9,1.67513){p95}
\psline(p94)(p95)
\psdot(p94)
\psPoint(6.4,5.3,2.3){p96}
\psPoint(6.4,5.3,1.84694){p97}
\psline(p96)(p97)
\psdot(p96)
\psPoint(6.3,4.7,1.6){p98}
\psPoint(6.3,4.7,1.58513){p99}
\psline(p98)(p99)
\psdot(p98)
\psPoint(6.5,5.1,2.0){p100}
\psPoint(6.5,5.1,1.74875){p101}
\psline(p100)(p101)
\psdot(p100)
\psPoint(6.7,5.7,2.1){p102}
\psPoint(6.7,5.7,2.00237){p103}
\psline(p102)(p103)
\psdot(p102)
\psPoint(6.9,5.7,2.3){p104}
\psPoint(6.9,5.7,1.98599){p105}
\psline(p104)(p105)
\psdot(p104)
\psPoint(4.5,1.3,0.3){p106}
\psPoint(4.5,1.3,0.20255){p107}
\psline(p106)(p107)
\psdot(p106)
\psPoint(5.2,3.9,1.4){p108}
\psPoint(5.2,3.9,1.31522){p109}
\psline(p108)(p109)
\psdot(p108)
\psPoint(5.5,1.4,0.2){p110}
\psPoint(5.5,1.4,0.16565){p111}
\psline(p110)(p111)
\psdot(p110)
\psPoint(5.8,5.1,1.9){p112}
\psPoint(5.8,5.1,1.80608){p113}
\psline(p112)(p113)
\psdot(p112)
\psPoint(5.1,1.4,0.2){p114}
\psPoint(5.1,1.4,0.19841){p115}
\psline(p114)(p115)
\psdot(p114)
\psPoint(6.4,5.6,2.1){p116}
\psPoint(6.4,5.6,1.98194){p117}
\psline(p116)(p117)
\psdot(p116)
\psPoint(6.4,5.6,2.2){p118}
\psPoint(6.4,5.6,1.98194){p119}
\psline(p118)(p119)
\psdot(p118)
\psPoint(5.8,1.2,0.2){p120}
\psPoint(5.8,1.2,0.05108){p121}
\psline(p120)(p121)
\psdot(p120)
\psPoint(5.1,1.4,0.3){p122}
\psPoint(5.1,1.4,0.19841){p123}
\psline(p122)(p123)
\psdot(p122)
\psPoint(6.0,4.5,1.6){p124}
\psPoint(6.0,4.5,1.5197){p125}
\psline(p124)(p125)
\psdot(p124)
\psPoint(7.2,6.1,2.5){p126}
\psPoint(7.2,6.1,2.14142){p127}
\psline(p126)(p127)
\psdot(p126)
\psPoint(6.9,5.1,2.3){p128}
\psPoint(6.9,5.1,1.71599){p129}
\psline(p128)(p129)
\psdot(p128)
\psPoint(6.0,4.8,1.8){p130}
\psPoint(6.0,4.8,1.6547){p131}
\psline(p130)(p131)
\psdot(p130)
\psPoint(6.7,5.2,2.3){p132}
\psPoint(6.7,5.2,1.77737){p133}
\psline(p132)(p133)
\psdot(p132)

%% \multido{\nx=1+1}{8}{%
%%         \psPoint(\nx\space,0,0){X1}
%%         \psPoint(\nx\space,-0.1,0){X2}
%%         \psline(X1)(X2)
%%     \uput[l](X2){\scalebox{0.8}{\scriptsize \nx}}}
%% \multido{\ny=1+1}{8}{%
%%         \psPoint(0,\ny\space,0){Y1}
%%         \psPoint(0,\ny\space,0.1){Y2}
%%         \psline(Y1)(Y2)\uput[u](Y2){\scalebox{0.8}{\scriptsize \ny}}}
%% \multido{\nz=1+1}{2}{%
%%         \psPoint(0,0,\nz\space){Z1}
%%         \psPoint(0,-0.1,\nz\space){Z2}
%%         \psline(Z1)(Z2)\uput[l](Z2){\scalebox{0.8}{\scriptsize \nz}}}
%\end{pspicture}
%}
%\vspace{0.2in}
%\caption{Multiple regression: {\tt sepal length} ($X_1$) and {\tt petal
%    length} ($X_2$) with response attribute {\tt petal width} ($Y$).
%The vertical bars show the residual errors for the points. Points in
%white are above the plane, whereas points in gray are below the plane.}
%    \label{fig:reg:eval:multi}
%    % \vspace*{-0.2in}
%\end{figure}
%
%\begin{example}[Multiple Regression: Goodness of Fit]
%    \label{ex:reg:eval:multiple}
%    Continuing with multiple regression from
%    \cref{ex:reg:linear:multiple},
%    \cref{fig:reg:eval:multi} shows the multiple regression of 
%{\tt sepal length} ($X_1$) and {\tt petal
%    length} ($X_2$) on the response attribute {\tt petal width} ($Y$)
%for the Iris dataset with $n=150$ points. 
%We also add an extra attribute $X_0 = \bone_{150}$, which is a vector
%of all ones in $\setR^{150}$. The augmented dataset $\abD \in
%\setR^{150\times 3}$ comprises $n=150$ points, and
%three attributes $X_0$, $X_1$ and $X_2$.
%
%The uncentered $3 \times 3$ scatter matrix
%$\abD^T\abD$ and its inverse are given as
%\begin{align*}
%    \abD^T\abD & = \matr{
%150.0 & 876.50  & 563.80\\
%876.5 & 5223.85 & 3484.25\\
%563.8 & 3484.25 & 2583.00 }
%&
%(\abD^T\abD)^{-1} & = \amatr{r}{
%0.793 & -0.176 & 0.064\\
%-0.176 &  0.041 & -0.017\\
%0.064 & -0.017 &  0.009}
%\end{align*}
%The augmented estimated weight vector $\abw$ is given as
%\begin{align*}
%    \abw = \matr{w_0\\w_1\\w_2} = (\abD^T\abD)^{-1} \cdot (\abD^T Y) = 
%    \amatr{r}{-0.014\\ -0.082\\ 0.45}
%\end{align*}
%The bias term is therefore $b=w_0=-0.014$, and
%the fitted model is 
%\begin{align*}
%    \hY = -0.014 -0.082 \cdot X_1 + 0.45\cdot X_2
%\end{align*}
%\cref{fig:reg:eval:multi} shows the fitted hyperplane. It also shows
%the residual error for each point. The white colored points have
%positive residuals (i.e., $\epsilon_i > 0$ or $\hy_i > y_i$), whereas the gray points have negative residual
%values (i.e., $\epsilon_i < 0$ or $\hy_i < y$).
%
%    The SSE value is given as
%    \begin{align*}
%        SSE = \sum_{i=1}^{150} \epsilon_i^2 
%        = \sum_{i=1}^{150} (y_i - \hy_i)^2 = 6.179
%    \end{align*}
%    
%    Thus, the estimated variance and standard error of regression are given as
%    \begin{align*}
%        \hsigma^2 & = \frac{SSE}{n-d-1} = \frac{6.179}{147} = 4.203 \times 10^{-2}\\
%        \hsigma & = \sqrt{\frac{SSE}{n-d-1}} = \sqrt{4.203 \times
%        10^{-2}} = 0.205
%    \end{align*}
%    
%    The values of total and residual sum of squares are given as
%    \begin{align*}
%        TSS & = 86.78 & RSS & = 80.60
%    \end{align*}
%    We can observe that $TSS = SSE + RSS$.
%    The fraction of variance explained by the model, that is the $R^2$
%    value, is given as
%    \begin{align*}
%        R^2 = \frac{RSS}{TSS} = \frac{80.60}{86.78} = 0.929
%    \end{align*}
%    This indicates a very good fit of the multiple linear regression model.
%    Nevertheless, it makes sense to also consider the adjusted $R_a^2$
%    value
%    \begin{align*}
%    R_a^2 = 1 - \frac{(n-1) \cdot SSE}{ (n-d-1) \cdot TSS} 
%    =  1 - \frac{149 \times 6.179}{ 147 \times 86.78} = 0.928 
%    \end{align*}
%    The adjusted value is almost the same as the $R^2$ value.
%
%\end{example}
%
%
%
%\begin{figure}[t!]
%    \centering
%    \vspace*{-1.75in}
%    \psset{unit=0.4in}
%    \scalebox{1.2}{%
%    \begin{pspicture}(-1.5,-1.5)(4,6)
%   \psset{Alpha=81,Beta=17,arrowscale=1.5}
%    %\psset{nameX={$Z$}, nameY={$\bone$}, nameZ={$~$}}
%    %\pstThreeDCoor[xMin=0, xMax=8, yMin=0, yMax=4,
%    %    zMin=0, zMax=0, linecolor=black]
%\psset{IIIDxTicksPlane=xy}
%    %\psset{dotstyle=Bo,dotscale=2.5,fillcolor=lightgray}
%    %\pstThreeDDot[drawCoor=false,linecolor=gray](5.9,3.0,4.2)
%    \psset{nameX={$\bx_n$}, nameY={$\bx_2$}, nameZ={$\bx_1$}}
%    \pstThreeDCoor[RotZ=-20,xMin=0,xMax=7,yMin=0,yMax=7,zMin=0,zMax=2,
%    linecolor=black,linewidth=0.5pt,linestyle=dashed]
%    \pstThreeDLine[linewidth=2pt,arrows=->](0,0,0)(5.9,3.0,4.2)
%    \pstThreeDLine[linewidth=2pt](0,0,0)(0,3.0,0)
%    \pstThreeDLine[linewidth=2pt](0,0,0)(5.9,0,0)
%    \pstThreeDLine[linewidth=1pt,arrows=->](0,0,0)(0,4.0,0)
%    \pstThreeDLine[linewidth=1pt,arrows=->](0,0,0)(8,0,0)
%    %\pstThreeDLine[linewidth=1pt,arrows=->](0,0,0)(5.9,1.5,0)
%    \pstThreeDLine[linewidth=1pt,arrows=->](0,0,0)(4.8,4,0) %X1
%    \pstThreeDLine[linewidth=1pt,arrows=->](0,0,0)(8,2.03,0) %X2
%    \pstThreeDSquare(-2.5,-1,0)(12,0,0)(0,6,0)
%    \psset{dotsep=2pt}
%    \pstThreeDLine[linewidth=2pt,arrows=->,linecolor=gray](0,0,0)(5.9,3.0,0)
%    %\pstThreeDPut[origin=rt](3,1.7,2.7){$Y$}
%    \pstThreeDPut[origin=rt](6.4,3.1,4.65){$\mY$}
%    %\pstThreeDPut[origin=rt](4,1.6,0){$\hY$}
%    \pstThreeDPut[origin=rt](6.45,3.25,0){$\hmY$}
%    \pstThreeDPut[origin=c](8.6,0,0){$U_2$}
%    \pstThreeDPut[origin=r](0,4.2,0){$U_1$}
%    \pstThreeDPut[origin=c](5,4.2,0){$\mX_1$}
%    \pstThreeDPut[origin=c](8.6,2.2,0){$\mX_2$}
%    %\pstThreeDDot[drawCoor=false,linecolor=gray](5.9,3.0,0)
%    \pstThreeDLine[linewidth=1.5pt,linestyle=dashed,arrows=->,linecolor=gray](5.9,3.0,0)(5.9,3.0,4.2)
%    \pstThreeDPut[origin=rt](6.5,3.3,3.4){$\bepsilon$}
%    % \pstPlanePut[plane=xz,planecorr=xyrot](5.9,3.0,1){$\overbrace{\hspace{1.1in}}^{\bepsilon}$}
%    % \rput(2.7,0.2){\begin{Rotateleft}$\underbrace{\hspace{1.6in}}_{~}$\end{Rotateleft}}
%    % \rput(2.7,0.2){$\bepsilon$}
%    % dotted lines
%    \pstThreeDLine[linestyle=dotted,linecolor=gray](5.9,0,0)(5.9,3.0,0)
%    \pstThreeDLine[linestyle=dotted,linecolor=gray](0,3.0,0)(5.9,3.0,0)
%    %right angle
%    \pstThreeDLine[linewidth=1pt](5,2.54,0)(5,2.54,0.5)
%    \pstThreeDLine[linewidth=1pt](5.9,3.0,0.5)(5,2.54,0.5)
%    %theta
%    \psset{beginAngle=0,endAngle=90}
%    \pstThreeDEllipse(0,0,0)(1,0.6,-0.02)(1,0.6,0.84)
%    \pstThreeDPut[origin=r](1.5,1,0.8){$\theta$}
%    %\pstPlanePut[plane=yz,planecorr=normal](1.2,0.9,0){\scriptsize $\theta$}
%    \end{pspicture}
%}
%    \vspace{0.8in}
%    \caption{Geometry of multiple regression. The figure shows two centered predictor
%        variables $\mX_1$ and $\mX_2$, along with the corresponding
%        orthogonal basis vectors $U_1$ and $U_2$. The subspace $\bone$ is not shown.
%    The dimensionality of the error space, containing the vector
%$\bepsilon$, is $n-d-1$, which also specifies the degrees of freedom for
%the estimated variance $\hsigma^2$.}
%\label{fig:reg:eval:multi_geom}
%\end{figure}
%
%
%\subsubsection{Geometry of Goodness of Fit}
%In multiple regression there are $d$ predictor attributes $X_1, X_2,
%\cdots, X_d$. We can center them by subtracting their projection along
%the vector $\bone$ to obtain the centered predictor vectors $\mX_{\!i}$. Likewise,
%we can center the response vector $Y$ and the predicted vector $\hY$. Thus, we have 
%\begin{align*}
%    \mX_{\!i} & = X_i - \mu_{X_i} \cdot \bone &
%    \mY & = Y - \mu_Y \cdot \bone &
%    \hmY & = \hY - \mu_Y \cdot \bone
%\end{align*}
%
%Once $Y$, $\hY$ and $X_{\!i}$'s have been centered, they all lie in the $n-1$
%dimensional subspace orthogonal to the vector $\bone$.
%\cref{fig:reg:eval:multi_geom} shows this $n-1$ dimensional subspace.
%In this subspace, we first extract an orthogonal basis $\{U_1, U_2,
%\cdots, U_d\}$ via the Gram-Schmidt orthogonalization process outlined
%in \cref{sec:reg:linear:orthogonalization}, and the predicted response
%vector is the sum of the projections of $\mY$ onto each of the new basis
%vectors (see \cref{eq:reg:linear:multiple_proj_hY}).
%
%The centered
%vectors $\mY$ and $\hmY$, and the error vector $\bepsilon$ 
%form a right triangle, and thus, by the Pythagoras theorem, we have
%\begin{align}
%    \begin{aligned}
%        \norm[\big]{\mY}^2 & = \norm[\big]{\hmY}^2 + \norm{\bepsilon}^2
%        = \norm[\big]{\hmY}^2 + \norm[\big]{Y-\hY}^2\\
%    TSS & = RSS + SSE
%\end{aligned}
%\label{eq:reg:eval:decomp}
%\end{align}
%
%The correlation between $Y$ and $\hY$ is
%the cosine of the angle between $\mY$ and
%$\hmY$, which is also 
%given as the ratio of the base to the hypotenuse
%\begin{align*}
%    \rho_{Y\hY} = \cos\theta = \frac{\norm[\big]{\hmY}}{\norm[\big]{\mY}}
%\end{align*}
%
%Furthermore, by
%\cref{eq:reg:eval:r2_rss}, the coefficient of multiple determination is given as
%\begin{align*}
%    R^2 & = \frac{RSS}{TSS} =
%    \frac{\norm[\big]{\hmY}^2}{\norm[\big]{\mY}^2} = \rho_{Y\hY}^2
%\end{align*}
%
%
%
%
%\begin{example}[Geometry of Goodness of Fit]
%    Continuing  \cref{ex:reg:eval:multiple}, the correlation between
%    $Y$ and $\hY$ is given as
%    \begin{align*}
%        \rho_{Y\hY} = \cos\theta =
%        \frac{\norm[\big]{\hmY}}{\norm[\big]{\mY}} =
%        \frac{\sqrt{RSS}}{\sqrt{TSS}}  =
%        \frac{\sqrt{80.60}}{\sqrt{86.78}} = 0.964
%    \end{align*}
%    The angle between $Y$ and $\hY$ is given as
%    \begin{align*}
%        \theta = \cos^{-1}(0.964) = 15.5^\circ
%    \end{align*}
%    The relatively small angle indicates a good linear fit.
%\end{example}
%
%\subsection{Inferences about Regression Coefficients}
%Let $Y$ be the response vector over all observations.
%Let $\abw = (w_0, w_1, w_2, \cdots, w_d)^T$ be the estimated vector of 
%regression coefficients, computed as 
%\begin{align*}
%    \abw & = \lB(\abD^T\abD\rB)^{-1} \abD^T Y
%\end{align*}
%
%The expected value of $\abw$ is given as follows:
%\begin{align*}
%    E[\abw] & = E\lB[(\abD^T\abD)^{-1} \abD^T Y\rB] 
%    = (\abD^T\abD)^{-1} \abD^T \cdot E[Y]\\
%& = (\abD^T\abD)^{-1} \abD^T \cdot E[\abD\trueabw + \bvepsilon]
%= (\abD^T\abD)^{-1} (\abD^T \abD) \trueabw = \trueabw
%\end{align*}
%since $E[\bvepsilon] = \bzero$.
%Thus, $\abw$ is an unbiased estimator for the true regression
%coefficients vector $\trueabw$.
%
%Next, we compute the covariance matrix for $\abw$, as follows
%\begin{align}
%    \ocov(\abw) &= \ocov\lB((\abD^T\abD)^{-1} \abD^T Y\rB), \;
%    \text{letting $\bA = (\abD^T\abD)^{-1} \abD^T$, we get} \notag\\
%& = \ocov(\bA Y) = \bA \ocov(Y) \bA^T \notag\\
%& = \bA \cdot \lB(\sigma^2 \cdot \bI\rB) \cdot \bA^T \notag\\
%& = (\abD^T\abD)^{-1} \abD^T \lB(\sigma^2 \cdot \bI\rB) 
%\abD (\abD^T\abD)^{-1} \notag\\ 
%& = \sigma^2 \cdot (\abD^T\abD)^{-1} (\abD^T\abD) (\abD^T\abD)^{-1} \notag\\
%& = \sigma^2 (\abD^T\abD)^{-1} \label{eq:reg:eval:cov_hbw}
%\end{align}
%Here, we made use of the fact that $\bA = (\abD^T\abD)^{-1} \abD^T$ is a
%matrix of fixed values, and therefore $\ocov(\bA Y) = \bA \ocov(Y)
%\bA^T$. Also, we have $\ocov(Y) = \sigma^2 \cdot \bI$, where $\bI$ is
%the $n\times n$ identity matrix. This follows from the fact that the
%observed responses  
%$y_i$'s are all independent and have the same variance $\sigma^2$. 
%
%Note that $\abD^T\abD \in \setR^{(d+1) \times (d+1)}$ is the uncentered scatter matrix for the augmented data. Let
%$\bC$ denote the inverse of $\abD^T\abD$. That is
%\begin{align*}
%    (\abD^T\abD)^{-1} = \bC
%\end{align*}
%Therefore, the covariance matrix for $\abw$ can be written as
%\begin{align*}
%    \ocov(\abw) = \sigma^2 \bC
%\end{align*}
%In particular, the diagonal entries $\sigma^2 \cdot c_{ii}$ give the
%variance for each of the regression coefficient estimates (including for
%$b = w_0$), and their squared roots specify the standard errors.
%That is
%\begin{align*}
%    \var(w_i) & = \sigma^2 \cdot c_{ii} & \se(w_i) & =
%    \sqrt{\var(w_i)} = \sigma \cdot \sqrt{c_{ii}}
%\end{align*}
%
%We can now define the standardized variable $Z_{w_{i}}$ that can be
%used to derive the confidence intervals for $\truew_i$ as follows
%\begin{align}
%    \tcbhighmath{
%    Z_{w_{i}} = \frac{w_i - E[w_i]}{\se(w_i)} = 
%\frac{w_i - \truew_i}{\hsigma \sqrt{c_{ii}}}}
%        \label{eq:reg:eval:Twi}
%\end{align}
%where we have replaced the unknown true variance $\sigma^2$ by
%$\hsigma^2$. Each of the variables $Z_{w_i}$ follows a $t$-distribution
%with $n-d-1$ degrees of freedom, from which we can obtain the
%$100(1-\alpha)$\% confidence
%interval of the true value $\truew_i$ as follows:
%\begin{align*}
%    P\lB(w_i -t_{\alpha/2} \cdot \se(w_i) \; \le
%        \; \truew_i \;
%    \le \; w_i + t_{\alpha/2} \cdot \se(w_i)\rB) =
%    1-\alpha
%\end{align*}
%Here, $t_{\alpha/2}$ is the critical value of the $t$ distribution, with
%$n-d-1$ degrees of freedom, that encompasses $\alpha/2$ fraction of the
%probability mass in the right tail, given as
%\begin{align*}
%    P(Z \ge t_{\alpha/2}) = \tfrac{\alpha}{2}
%    \text{ or equivalently }
%    T_{n-d-1}(t_{\alpha/2}) = 1 - \tfrac{\alpha}{2}
%\end{align*}
%
%
%\begin{example}[Confidence Intervals]
%    \label{ex:reg:eval:multiple_ci}
%    Continuing with multiple regression from
%    \cref{ex:reg:eval:multiple}, we have
%    \begin{align*}
%        \hsigma^2 & = 4.203 \times 10^{-2}\\
%        \bC & = (\abD^T\abD)^{-1} = 
%        \amatr{r}{
%        0.793 & -0.176 & 0.064\\
%        -0.176 &  0.041 & -0.017\\
%        0.064 & -0.017 &  0.009}
%    \end{align*}
%    Therefore, the covariance matrix of the estimated regression
%    parameters, including the bias term, is given as
%    \begin{align*}
%        \ocov(\abw) = \hsigma^2 \cdot \bC =
%        \amatr{r}{
%            3.333 \times 10^{-2} & -7.379 \times 10^{-3} & 2.678\times 10^{-3}\\
%            -7.379 \times 10^{-3} & 1.714 \times 10^{-3} & -7.012\times
%            10^{-4}\\
%            2.678 \times 10^{-3} & -7.012 \times 10^{-4} & 3.775\times
%        10^{-4}}
%    \end{align*}
%    The diagonal entries specify the variances and standard errors of
%    each of the estimated parameters
%    \begin{align*}
%        \var(b) & = 3.333 \times 10^{-2} & 
%        \se(b) & = \sqrt{3.333 \times 10^{-2}} = 0.183\\
%        \var(w_1) & = 1.714 \times 10^{-3} & 
%        \se(w_1) & = \sqrt{1.714 \times 10^{-3}} = 0.0414\\
%        \var(w_2) & = 3.775 \times 10^{-4} &
%        \se(w_2) & = \sqrt{3.775 \times 10^{-4}} = 0.0194
%    \end{align*}
%    where $b = w_0$.
%
%    Using confidence level $1-\alpha=0.95$ (or significance level
%    $\alpha=0.05$), the critical value of the $t$-distribution that
%    encompasses $\tfrac{\alpha}{2}=0.025$ 
%    fraction of the probability mass in
%    the right tail is given as
%    $t_{\alpha/2} = 1.976$.
%    Thus, the 95\% confidence intervals for the true bias term $\trueb$, and
%    the true regression coefficients $\truew_1$ and $\truew_2$, are:
%    \begin{align*}
%        \trueb \in \lB(b \pm t_{\alpha/2} \cdot \se(b)\rB)
%         & = \lB( -0.014-0.074, -0.014+0.074 \rB)\\
%         & = (-0.088,0.06)\\
%        \truew_1 \in \lB(w_1 \pm t_{\alpha/2} \cdot \se(w_1)\rB)
%        & = \lB( -0.082-0.0168, -0.082+0.0168 \rB)\\
%        & = (-0.099,-0.065)\\
%        \truew_2 \in \lB(w_2 \pm t_{\alpha/2} \cdot \se(w_2)\rB)
%         & = \lB( 0.45-0.00787, 0.45+0.00787 \rB)\\
%         & = (0.442,0.458)
%    \end{align*}
%\end{example}
%
%
%\subsection{Hypothesis Testing}
%\index{multiple regression!hypothesis testing}
%\index{hypothesis testing}
%Once the parameters have been estimated, it is beneficial to test
%whether the regression coefficients are close to zero or substantially
%different. 
%For this we set up the null hypothesis that all the true weights are
%zero, except for the
%bias term ($\trueb = \truew_0$). 
%%We exclude the bias since it simply specifies the
%%intercept along each dimension, and there is no reason to believe that
%%it should be zero.
%We contrast the null hypothesis with the alternative hypothesis that at least one
%of the weights is not zero
%\begin{align*}
%    H_0\!: & \quad \truew_1 = 0, \truew_2 = 0, \ldots, \truew_d = 0 \\
%    H_a\!: & \quad \exists i, \text{ such that } \truew_i \ne 0
%\end{align*}
%The null hypothesis can also be written as $H_0: \truebw = \bzero$,
%where $\truebw = (\truew_1, \truew_2, \cdots, \truew_d)^T$.
%
%
%We use the $F$-test that compares the ratio of the adjusted RSS value to the
%estimated variance $\hsigma^2$, defined via the $f$-statistic
%\index{multiple regression!$f$-statistic}
%\index{multiple regression!$F$-test}
%\begin{align}
%    \tcbhighmath{
%    f = \frac{RSS/d}{\hsigma^2} = \frac{RSS\big/d}{SSE\big/(n-d-1)}}
%    \label{eq:reg:eval:Ftest}
%\end{align}
%
%Under the null hypothesis, we have
%\begin{align*}
%    E[RSS/d] = \sigma^2
%\end{align*}
%To see this, let us examine the regression equations in vector terms,
%namely
%\begin{align*}
%    \hY & = b \cdot \bone + w_1 \cdot X_1 + \ldots + w_d \cdot X_d\\
%    \hY & = (\mu_Y - w_1 \mu_{X_1} - \ldots - w_d
%    \mu_{X_d}) \cdot \bone + w_1 \cdot X_1 + \ldots + w_d \cdot
%    X_d\\
%    \hY- \mu_Y \cdot \bone & = w_1 (X_1 - \mu_{X_1} \cdot \bone) + \ldots
%    + w_d (X_d - \mu_{X_d} \cdot \bone), \; \text{which
%            implies}\\
%    \hmY & = w_1\; \mX_1 + w_2\; \mX_2 + \ldots + w_d\; \mX_d =
%    \sum^{d}_{i=1} w_i \; \mX_i
%\end{align*}
%Let us consider the RSS value; we have
%\begin{align*}
%    RSS & = \norm[\big]{\hY - \mu_Y \cdot \bone}^2 = \norm[\big]{\hmY}^2 = \hmY^T
%    \hmY\\
%    & = \Bigl(\sum^{d}_{i=1} w_i \; \mX_{\!i} \Bigr)^T \Bigl(\sum^{d}_{j=1}
%    w_{\!j}\;  \mX_{\!j} \Bigr) 
%    = \sum^{d}_{i=1} \sum^{d}_{j=1} w_{\!i}w_{\!j}\; \mX_{\!i}^T\mX_{\!j}
%    = \bw^T (\mbD^T\mbD) \bw
%\end{align*}
%where $\bw = (w_1, w_2, \cdots, w_d)^T$ is the $d$-dimensional
%vector of regression coefficients (without the bias term), and $\mbD \in
%\setR^{n \times d}$ is
%the centered data matrix (without augmentation by the $X_0 = \bone$ attribute).
%The expected value of RSS is thus given as
%\begin{align}
%    E[RSS] & = E\lB[\bw^T (\mbD^T\mbD) \bw\rB] \notag\\
%    & = \tr\lB(E\lB[\bw^T (\mbD^T\mbD) \bw\rB]\rB), \; \text{since,
%    $E\lB[\bw^T (\mbD^T\mbD) \bw\rB]$ is a scalar} \notag\\
%    & = E\lB[\tr\lB(\bw^T (\mbD^T\mbD) \bw\rB)\rB] \notag\\
%    & = E\lB[\tr\lB((\mbD^T\mbD)\bw\bw^T\rB)\rB], \; \text{since
%    trace is invariant under cyclic permutation} \notag\\
%    & = \tr\lB((\mbD^T\mbD) \cdot E[\bw\bw^T] \rB) \notag\\
%    & = \tr\lB((\mbD^T\mbD) \cdot \lB(\ocov(\bw) + E[\bw] \cdot
%    E[\bw]^T\rB)\rB) \label{eq:reg:eval:Erss}\\
%    & = \tr\lB((\mbD^T\mbD) \cdot \ocov(\bw) \rB),\;
%    \text{since under the null hypothesis $E[\bw] = \truebw = \bzero$}
%    \notag\\
%    & = \tr\lB( (\mbD^T\mbD) \cdot \sigma^2 (\mbD^T\mbD)^{-1} \rB) =
%    \sigma^2 \tr\lB( \bI_d \rB) = d \cdot \sigma^2 \notag 
%\end{align}
%where $\bI_d$ is the $d \times d$ identity matrix. We also used the fact that
%\begin{align*}
%    \ocov(\bw) & = E[\bw\bw^T] - E[\bw] \cdot E[\bw]^T, \text{ and
%    therefore}\\
%    E[\bw\bw^T] & =\ocov(\bw) + E[\bw] \cdot E[\bw]^T 
%\end{align*}
%Notice that from \cref{eq:reg:eval:cov_hbw}, 
%the covariance matrix for the augmented weight vector $\abw$, that
%includes the bias term, is given
%as $\sigma^2 (\abD^T\abD)^{-1}$. However, since we are ignoring the bias
%$b=w_0$ in the hypothesis test, 
%we are interested only in the lower right $d \times d$ submatrix of
%$(\abD^T\abD)^{-1}$, which excludes the values related to $w_0$.
%It can be shown that this 
%submatrix is precisely the inverse of the centered scatter matrix
%$(\mbD^T\mbD)^{-1}$ for the unaugmented data. We used this fact in the
%derivation above.
%Therefore, it follows that 
%\begin{align*}
%    E\lB[\frac{RSS}{d}\rB] = \frac{1}{d} E[RSS] = \frac{1}{d} \cdot d
%    \cdot \sigma^2 = \sigma^2
%\end{align*}
%
%Further, as per \cref{eq:reg:eval:sse_est_unbiased}, the estimated
%variance is an unbiased estimator, so that
%\begin{align*}
%    E[\hsigma^2] = \sigma^2
%\end{align*}
%Thus, under the null hypothesis the $f$-statistic has a value close to
%1, which indicates that there is no relationship between the predictor
%and response variables. On the other hand, if the alternative hypothesis
%is true, then $E[RSS/d] \ge \sigma^2$, resulting in a larger $f$ value.
%
%The ratio $f$ follows a $F$-distribution with $d, (n-d-1)$ degrees of
%freedom for the numerator and denominator, respectively. Therefore,
%we can reject the null hypothesis if the $\pvalue$ is less than the
%chosen significance level, say $\alpha=0.01$. 
%% That is, we reject the null
%% hypothesis if 
%% $f > F_{\alpha,d,n-d-1}$ for critical value $\alpha$ (e.g.,
%% $\alpha=0.01$).
%
%Notice that, since $R^2 = 1 - \tfrac{SSE}{TSS} = \frac{RSS}{TSS}$, we
%have 
%\begin{align*}
%SSE & = (1-R^2) \cdot TSS & RSS & = R^2 \cdot TSS
%\end{align*}
%Therefore, we
%can rewrite the $f$ ratio as follows
%\begin{empheq}[box=\tcbhighmath]{align}
%    f = \frac{RSS\big/d}{SSE\big/(n-d-1)} = \frac{n-d-1}{d} \cdot \frac{R^2}{1-R^2}
%\end{empheq}
%In other words, the $F$-test compares the adjusted fraction of explained
%variation to the unexplained variation. If $R^2$ is high, it means the
%model can fit the data well, and that is more evidence to reject the null
%hypothesis.
%
%% \vspace*{-0.1in}
%\subsubsection{Hypothesis Testing for Individual Parameters}
%We can also test whether each independent attribute $X_i$, 
%contributes significantly for the prediction of $Y$ or not, assuming
%that all the attributes are still retained in the model.
%
%For attribute $X_i$, we set up the null hypothesis $H_0: \truew_i = 0$ and
%contrast it with the alternative hypothesis $H_a: \truew_i \ne 0$.
%Using \cref{eq:reg:eval:Twi}, the standardized variable $Z_{w_{i}}$
%under the null hypothesis is given as
%\begin{empheq}[box=\tcbhighmath]{align}
%    Z_{w_{i}} = \frac{w_i - E[w_i]}{\se(w_i)} = 
%    \frac{w_i}{\se(w_i)} =
%        \frac{w_i}{\hsigma \sqrt{c_{ii}}}
%\end{empheq}
%since $E[w_i] = \truew_i = 0]$.
%Next, using a two-tailed $t$-test with $n-d-1$ degrees of freedom, we
%compute $\pvalue(Z_{w_i})$. If this probability is smaller than the
%significance level $\alpha$ (say 0.01), 
%we can reject the null hypothesis. 
%Otherwise, we accept the null hypothesis, which would imply that $X_i$
%does not add significant value in predicting the response in light of
%other attributes already used to fit the model. The $t$-test can also be
%used to test whether the bias term is significantly different from $0$
%or not.
%
%\vspace*{-0.1in}
%\begin{example}[Hypothesis Testing]
%    We continue with \cref{ex:reg:eval:multiple_ci}, but now we test for the
%    regression effect. Under the null hypothesis that $\truew_1 = \truew_2 = 0$,
%    the expected value of RSS is $\sigma^2$. Thus, we expect the
%    $f$-statistic to be close to 1. Let us check if that is the case; we
%    have
%    \begin{align*}
%        f = \frac{RSS/d}{\hsigma^2} = \frac{80.60/2}{4.203 \times
%        10^{-2}} = 958.8
%    \end{align*}
%    Using the $F$-distribution with $(d,n-d-1) = (2,147)$ degrees of freedom, we
%    have
%    \begin{align*}
%        \pvalue(958.8) \approx 0
%    \end{align*}
%    In other words, such a large value of the $f$-statistic is extremely
%    rare, and therefore, we can reject the null hypothesis. We conclude that $Y$
%    does indeed depend on at least one of the predictor attributes $X_1$
%    or $X_2$.
%
%    We can also test for each of the regression coefficients
%    individually using the $t$-test. For example, for $\truew_1$, let the
%    null hypothesis be $H_0: \truew_1 = 0$, so that the alternative
%    hypothesis is $H_a: \truew_1 \ne 0$. Assuming that the model still has
%    both $X_1$ and $X_2$ as the predictor variables, we can compute the
%    $t$-statistic using \cref{eq:reg:eval:Twi}:
%    \begin{align*}
%        Z_{w_1} = \frac{w_1}{\se(w_1)}  =
%        \frac{-0.082}{0.0414} = -1.98
%    \end{align*}
%    Using a two-tailed $t$-test with $n-d-1=147$ degrees of freedom, we
%    find that
%    \begin{align*}
%        \pvalue(-1.98) = 0.0496
%    \end{align*}
%    Since the $\pvalue$ is only marginally less than a significance
%    level of $\alpha=0.05$ (i.e., a 95\% confidence level), 
%    this means that $X_1$ is only weakly 
%    relevant for predicting $Y$ in the
%    presence of $X_2$. In fact, if we use the more stringent
%    significance level $\alpha=0.01$, 
%    we would conclude that $X_1$ is not significantly
%    predictive of $Y$, given $X_2$.
%
%    On the other hand, individually for $\truew_2$, if we test whether $H_0: \truew_2 = 0$
%    versus $H_a: \truew_2 \ne 0$, we have:
%    \begin{align*}
%        Z_{w_2} = \frac{w_2}{\se(w_2)}  =
%        \frac{0.45}{0.0194} = 23.2
%    \end{align*}
%    Using a two-tailed $t$-test with $n-d-1=147$ degrees of freedom, we
%    find that
%    \begin{align*}
%        \pvalue(23.2) \approx 0
%    \end{align*}
%    This means, that individually $X_2$ is significantly predictive of
%    $Y$ even in the presence of $X_1$. 
%    
%    Using the $t$-test, we can also compute the $\pvalue$ for the bias
%    term:
%    \begin{align*}
%        Z_{b} = \frac{b}{\se(b)}  =
%        \frac{-0.014}{0.183} = -0.077
%    \end{align*}
%    which has a $\pvalue=0.94$ for a two-tailed test. This means, we
%    accept the null hypothesis that $\trueb=0$, and reject the alternative
%    hypothesis that $\trueb\ne 0$.
%\end{example}
%
%% \enlargethispage{1\baselineskip}
%\vspace*{-0.1in}
%\begin{example}[Centered Scatter Matrix]
%    Here we show that the inverse of the centered scatter matrix
%    $(\mbD^T\mbD)^{-1}$ for the unaugmented data is the lower right
%    submatrix of the inverse of the uncentered scatter matrix
%    $(\abD^T\abD)^{-1}$ for the augmented data.
%
%    For the augmented data comprising $X_0$,
%    $X_1$ and $X_2$, from \cref{ex:reg:eval:multiple}, 
%    the uncentered $3 \times 3$ scatter matrix
%    $\abD^T\abD$ and its inverse, are given as
%\begin{align*}
%    \abD^T\abD & = \matr{
%150.0 & 876.50  & 563.80\\
%876.5 & 5223.85 & 3484.25\\
%563.8 & 3484.25 & 2583.00 }
%&
%(\abD^T\abD)^{-1} & = \amatr{r}{
%0.793 & -0.176 & 0.064\\
%-0.176 &  0.041 & -0.017\\
%0.064 & -0.017 &  0.009}
%\end{align*}
%
%For the unaugmented data comprising only $X_1$ and $X_2$, the centered
%scatter matrix and its inverse are given as:
%\begin{align*}
%    \mbD^T\mbD & = \matr{
%        102.17 & 189.78\\
%        189.78 & 463.86
%    }
%&
%(\mbD^T\mbD)^{-1} & = \amatr{r}{
%0.041 & -0.017\\
%-0.017 &  0.009}
%\end{align*}
%We can observe that $(\mbD^T\mbD)^{-1}$ is exactly the lower right
%$2 \times 2$ submatrix of $(\abD^T\abD)^{-1}$. 
%\end{example}
%
%
%\subsection{Geometric Approach to Statistical Testing}
%The geometry of multiple regression provides further insight into the
%hypothesis testing approach for the regression effect.
%Let $\mX_i = X_i - \mu_{X_i} \cdot \bone$ denote the centered attribute
%vector, and let $\mbX = (\mX_1, \mX_2, \cdots, \mX_d)^T$ denote the
%multivariate centered vector of predictor variables.
%The $n$-dimensional
%space over the points is divided into three mutually orthogonal 
%subspaces, namely the 1-dimensional {\em mean space} 
%$\cS_\mu = span(\bone)$, 
%the $d$ dimensional {\em centered variable space} 
%$\cS_{\mX} = span(\mbX)$, and the
%$n-d-1$ dimensional {\em error space} $\cS_\epsilon$, which contains the error
%vector $\bepsilon$. 
%The response vector $Y$ can thus be decomposed into three components
%(see \cref{fig:reg:eval:uni_geom,fig:reg:eval:multi_geom})
%\begin{align*}
%    Y = \mu_Y \cdot \bone + \hmY + \bepsilon
%\end{align*}
%Recall that the {\em degrees of freedom}
%of a random vector is defined
%as the dimensionality of its enclosing subspace.
%Since the original dimensionality of the point space is $n$, we have a
%total of $n$ degrees of freedom. The mean space has dimensionality $dim(\cS_\mu)=1$,
%the centered variable space has
%$dim(\cS_{\mX}) = d$, and the error space has $dim(\cS_\epsilon) =
%n-d-1$, so that we have
%\begin{align*}
%    dim(\cS_\mu) + dim(\cS_{\mX}) + dim(\cS_\epsilon) = 1 + d + (n-d-1) = n
%\end{align*}
%
%\subsubsection{Population Regression Model}
%Recall that the regression model posits that for a fixed value $\bx_i =
%(x_{i1}, x_{i2}, \cdots, x_{id})^T$, 
%the true response $y_i$ is
%given as
%\begin{align*}
%y_i = \trueb + \truew_1 \cdot x_{i1} + \ldots + \truew_d \cdot x_{id} + \varepsilon_i
%\end{align*}
%where the systematic part of the model $\trueb+ \sum_{j=1}^d \truew_j\cdot x_{ij}$ is fixed, and the
%error term $\varepsilon_i$ varies randomly, with the
%assumption that $\varepsilon_i$ follows a normal distribution with mean
%$\mu= 0$ and variance $\sigma^2$. We also assume that the
%$\varepsilon_i$ values are all independent of each other.
%
% %\begin{figure}[h!]
% %       \hspace{3in}
% %       \centerline{
% %   \begin{pspicture}(-1.5,-1.5)(4,6)
% %  \psset{Alpha=30,Beta=17,arrowscale=1.5}
%%\psset{IIIDxTicksPlane=xy}
% %   \psset{nameX={$\bx_n$}, nameY={$\bx_2$}, nameZ={$\bx_1$}}
% %   \pstThreeDCoor[RotZ=-20,xMin=0,xMax=6,yMin=0,yMax=4,zMin=0,zMax=3,
% %   linecolor=black,linewidth=0.5pt,linestyle=dashed]
% %   \pstThreeDSphere[SegmentColor={[cmyk]{0,0,0,0}}](5.9,0,4.2){1}
% %   %\pstThreeDLine[linewidth=2pt,arrows=->](0,0,0)(5.9,3.0,4.2)%Y
% %   \pstThreeDLine[linewidth=1pt,arrows=->](0,0,0)(0,4.0,0)%1
% %   \pstThreeDLine[linewidth=1pt,arrows=->](0,0,0)(8,0,0)%Z
% %   %\pstThreeDLine[linewidth=1pt,arrows=->](0,0,0)(8,2.03,0)%X
% %   \pstThreeDLine[linewidth=2pt,arrows=->,linecolor=gray](0,0,0)(5.9,0,0)%E[Yc|Z]
% %   \pstThreeDLine[linewidth=2pt,arrows=->](0,0,0)(5.9,0.0,4.2)%Yc
% %   % \pstThreeDLine[linestyle=dotted,linecolor=gray](5.9,0,0)(5.9,3.0,0)
% %   % \pstThreeDLine[linestyle=dotted,linecolor=gray](0,3.0,0)(5.9,3.0,0)
% %   % \pstThreeDLine[linestyle=dotted,linecolor=gray](8,0,0)(8,2.03,0)
% %   % \pstThreeDLine[linestyle=dotted,linecolor=gray](0,2.03,0)(8,2.03,0)
% %   % \pstThreeDLine[linestyle=dotted,linecolor=gray](5.9,0.0,4.2)(5.9,3.0,4.2)
% %   % \pstThreeDLine[linestyle=dotted,linecolor=gray](0,3.0,0)(5.9,3.0,4.2)
% %   \pstThreeDSquare(-1,0,-0.75)(10.5,0,0)(0,0,5.75)
% %   \psset{dotsep=2pt}
% %   %\pstThreeDPut[origin=rt](6.4,3.2,4.6){$Y$}
% %   \pstThreeDPut[origin=rt](6.4,0.1,4.6){$Y_{\!c}$}
% %   \pstThreeDPut[origin=rt](5.8,0,-0.3){$E[Y_{\!c}|Z]$}
% %   \pstThreeDPut[origin=c](8.5,0,0){$Z$}
% %   \pstThreeDPut[origin=r](0,4.2,0){$\bone$}
% %   %\pstThreeDPut[origin=c](8.5,2.1,0){$X$}
% %   \pstThreeDPut[origin=rt](7,0,2.5){$\bepsilon$}
% %   \end{pspicture}
%%}
% %   \caption{Geometry of population regression model}
%%\label{fig:reg:eval:popreg}
%%\end{figure}
%
%
%
%\begin{figure}[!t]
%    \vspace{-0.25in}
%    \hspace{-1.25in}
%\centerline{
%\definecolor{mygray}{gray}{0.95}
%\psset{unit=0.5in,arrowscale=1.5,labelsep=2pt}
%\psset{viewpoint=30 20 20 rtp2xyz,Decran=100,lightsrc=viewpoint}
%\begin{pspicture}(-3,-3)(3,3.5)
%\psSolid[object=parallelepiped,
%a=0.5,b=2.4,c=0.9,fillcolor=white,opacity=0.2,action=draw*](0.1,0.8,0.2)
%\psPoint[](0,0,0){O}
%\psPoint[](0,1.5,0){C}
%\psline[linewidth=2pt,linecolor=gray]{->}(O)(C)
%\axesIIID[axisnames={\bone,\cS_{\mX},\cS_\epsilon}](-0.1,-0.1,-0.1)(1.5,2.2,0.8)
%\psPoint[](0,1.75,0.2){A}
%\psset{dotsep=0.5pt}
%\psline[linewidth=1.5pt,linestyle=dotted]{->}(C)(A)
%\psdot[dotscale=1,fillcolor=lightgray,dotstyle=Bo](A)
%\psPoint[](0,0.75,-0.01){mC}
%\uput[-90](mC){$E[\mY|\mbX]$}
%\psPoint[](0,0.75,0.1){mA}
%\uput[90](mA){$\mY$}
%\psPoint[](0,1.6,0.15){mE}
%\uput[-90](mE){$\bvepsilon$}
%\psline[linewidth=1pt,linestyle=dashed]{->}(O)(A)
%\psSolid[object=sphere,hollow,linecolor=lightgray,
%    opacity=0.75,r=0.3,incolor=white,opacity=0.15,%fillcolor=gray!25,
%ngrid=20 20](0,1.5,0)%
%% \psSolid[object=sphere,hollow,linecolor=lightgray,
%%     opacity=0.75,r=0.3,incolor=white,opacity=0.15,%fillcolor=gray!25,
%% ngrid=20 20](0,0,0)%
%\end{pspicture}
%}
%    \vspace{-0.3in}
%% \vspace{0.1in}
%    \caption{Population regression model. The $n-1$ dimensional
%        subspace, comprising $\cS_{\mX} + \cS_\epsilon$,
%    orthogonal to $\bone$ is shown as a hyperrectangle. The true error vector is
%    a random vector of length $\norm{\bvepsilon}$; the $(n-1)$-dimensional
%hypersphere shows the possible orientations of $\bvepsilon$ for a fixed
%length value.}
%\label{fig:reg:eval:pop_reg}
%\end{figure}
%
%Plugging $\trueb = \mu_Y - \sum^{d}_{j=1} \truew_{\!j} \cdot \mu_{X_{\!j}}$ into the
%equation above, we obtain
%\begin{align*}
%    y_i & = \mu_Y + \truew_1 \cdot (x_{i1}-\mu_{X_1}) + \ldots + \truew_d \cdot
%    (x_{id} - \mu_{X_d}) +
%    \varepsilon_i\\
%    & = \mu_Y + \truew_1 \cdot \mx_{i1} + \ldots + \truew_d \cdot \mx_{id} + \varepsilon_i
%\end{align*}
%where $\mx_{ij} = x_{ij} - \mu_{X_{\!j}}$ is the centered value for
%attribute $X_{\!j}$. 
%Across all the points, we can rewrite the above equation in vector form
%\begin{align*}
%    Y = \mu_Y \cdot \bone + \truew_1 \cdot \mX_1 + \ldots + \truew_d \cdot \mX_d + \bvepsilon
%\end{align*}
%We can also center the vector $Y$, so that we obtain a regression model
%over the centered response and predictor variables, as illustrated in
%\cref{fig:reg:eval:pop_reg}:
%\begin{align*}
%    \mY = Y - \mu_Y \cdot \bone = \truew_1 \cdot \mX_1 + \truew_2 \cdot \mX_2 + \ldots
%    + \truew_d \cdot \mX_d + \bvepsilon = E[\mY|\mbX] +
%    \bvepsilon
%\end{align*}
%In this equation, $\sum_{i=1}^d \truew_i \cdot \mX_i$ is a fixed vector that
%denotes the expected value $E[\mY|\mbX]$ and $\bvepsilon$ is an
%$n$-dimensional random vector that is distributed according to a 
%$n$-dimensional multivariate normal vector with mean $\bmu = \bzero$,
%and a fixed variance $\sigma^2$ along all dimensions, so that its
%covariance matrix is $\cov = \sigma^2 \cdot \bI$.
%The distribution of $\bvepsilon$ is therefore given as
%\begin{align*}
%    f(\bvepsilon) = \frac{1}{(\sqrt{2\pi})^n \cdot \sqrt{|\cov|}} \cdot 
%    \exp \lB\{ -\frac{\bvepsilon^T \cov^{-1} \bvepsilon}{2} \rB\} =
%\frac{1}{(\sqrt{2\pi})^n \cdot \sigma^n} \cdot 
%\exp \lB\{ -\frac{\norm{\bvepsilon}^2}{2 \cdot \sigma^2} \rB\}
%\end{align*}
%which follows from the fact that $|\cov| = \det(\cov) = \det(\sigma^2 \bI) =
%\lB(\sigma^2\rB)^n$ and $\cov^{-1} = \frac{1}{\sigma^2} \bI$.
%
%The density of $\bvepsilon$ is thus a function of its
%squared length $\norm{\bvepsilon}^2$, independent of its angle. 
%In other words, the vector $\bvepsilon$ is distributed uniformly over
%all angles and is equally likely to point in any direction. 
%% On the other
%% hand, each component $\varepsilon_i$ of $\bvepsilon$ is independent, 
%% and is distributed in length according to a
%% normal distribution with mean $\mu=0$ and variance $\sigma^2$.
%\cref{fig:reg:eval:pop_reg} illustrates the population regression model.
%The fixed vector $E[\mY|\mbX]$ is shown, 
%and one orientation of
%$\bvepsilon$ is shown. 
%It is important to note that the $n-1$ dimensional hypersphere denotes
%the fact that the random vector $\bvepsilon$ can be in any orientation in this
%hypersphere of radius $\norm{\bvepsilon}$.
%%Moreover, the radius also varies according to the normal
%%distribution.
%Notice how the population regression model differs from the fitted
%model. 
%The residual error vector $\bepsilon$ is
%orthogonal to the predicted mean response vector $\hmY$, which 
%acts as the estimate for $E[\mY|\mbX]$. 
%On the other hand, in the population regression model, 
%the random error vector $\bvepsilon$ can be in any
%orientation compared to $E[\mY|\mbX]$.
%
%\subsubsection{Hypothesis Testing}
%Consider the population regression model
%\begin{align*}
%    Y = \mu_Y \cdot \bone + \truew_1 \cdot \mX_1 + \ldots + \truew_d \cdot \mX_d +
%    \bvepsilon = \mu_Y \cdot \bone + E[\mY|\mbX] + \bvepsilon
%\end{align*}
%
%To test whether $X_1, X_2, \cdots, X_d$ are useful in predicting $Y$, we
%consider what would happen if all of the regression coefficients were
%zero, which forms our null hypothesis
%\begin{align*}
%H_0\!: & \quad \truew_1 = 0, \truew_2 = 0, \ldots, \truew_d = 0    
%\end{align*}
%In this case, we have 
%\begin{align*}
%Y = \mu_Y \cdot \bone + \bvepsilon\; \implies Y - \mu_Y\cdot\bone =
%\bvepsilon\; \implies \mY = \bvepsilon  
%\end{align*}
%Since $\bvepsilon$ is normally distributed with mean $\bzero$ and
%covariance matrix $\sigma^2\cdot \bI$, under the null hypothesis,
%the variation in $\mY$ for a given value of $\bx$ will therefore
%be centered around the origin $\bzero$.
%
%On the other hand, under the alternative hypothesis $H_a$ that at least
%one of the $\truew_i$ is non-zero, we have
%\begin{align*}
%\mY = E[\mY|\mbX] + \bvepsilon 
%\end{align*}
%Thus, the variation in
%$\mY$ is shifted away from the origin $\bzero$ in the direction 
%$E[\mY|\mbX]$.
%
%In practice, we obviously do not know the true value of $E[\mY|\mbX]$,
%but we can estimate it by projecting the 
%centered observation vector $\mY$ onto the
%subspaces
%$\cS_{\mX}$ and $\cS_\epsilon$, as follows
%\begin{align*}
%    \mY = w_1 \cdot \mX_1 + w_2 \cdot \mX_2 + \ldots + w_d \cdot
%\mX_d + \bepsilon = \hmY  + \bepsilon
%\end{align*}
%
%Now, under the null hypothesis, the true centered response vector is $\mY =
%\bvepsilon$, and therefore,
%$\hmY$ and $\bepsilon$ are simply the projections of the random
%error vector $\bvepsilon$ onto the subspaces $\cS_{\mX}$ and
%$\cS_\epsilon$, as shown in \cref{fig:reg:eval:hypotest_0}. In this
%case, we also 
%expect the length of $\bepsilon$ and
%$\hmY$ to be roughly comparable.
%On the other hand, under the alternative hypothesis, we have
%$\mY = E[\mY|\mbX] + \bvepsilon$, and so $\hmY$ will be
%relatively much longer compared to $\bepsilon$, as shown in
%\cref{fig:reg:eval:hypotest_a}.
%
%\begin{figure}[!t]
%    \centering
% \captionsetup[subfloat]{captionskip=30pt}
%    \vspace{-1in}
%    \centerline{
%        \subfloat[Null hypothesis: $\mY = \bvepsilon$]{
%\label{fig:reg:eval:hypotest_0}
%        \scalebox{1}{
%            \psset{labelsep=2pt}
%    \definecolor{mygray}{gray}{0.95}
%\psset{unit=0.5in,arrowscale=1.5}
%\psset{viewpoint=30 20 20 rtp2xyz,Decran=100,lightsrc=viewpoint}
%\begin{pspicture}(0,-1)(6,5)
%\psSolid[object=parallelepiped,
%a=0.5,b=2.4,c=0.9,fillcolor=white,opacity=0.2,action=draw*](0.1,0.8,0.2)
%\psPoint[](0,0,0){O}
%\psPoint[](0,1.5,0){C}
%\axesIIID[axisnames={\bone,\cS_{\mX},\cS_\epsilon}](-0.2,-0.2,-0.2)(1.5,2.2,0.8)
%\psPoint[](0,0.25,0.2){A}
%\psPoint[](0,0.25,0){B}
%\psset{dotsep=0.5pt}
%\psline[linewidth=1.5pt,linestyle=dashed]{->}(B)(A)
%\psline[linewidth=1.5pt,linestyle=dotted]{->}(O)(A)
%\psline[linewidth=2pt,linecolor=gray]{->}(O)(B)
%\psdot[dotscale=1,fillcolor=lightgray,dotstyle=Bo](A)
%\psPoint[](0,0.1,-0.01){mC}
%\uput[-90](mC){$\hmY$}
%\psPoint[](0,0.125,0.1){mE}
%\uput[90](mE){$\bvepsilon$}
%\psPoint[](0,0.3,0.05){mB}
%\uput[90](mB){$\bepsilon$}
%%\psSolid[object=sphere,hollow,linecolor=lightgray,
%%    opacity=0.75,r=0.3,incolor=white,opacity=0.15,%fillcolor=gray!25,
%%ngrid=20 20](0,1.5,0)%
%\psSolid[object=sphere,hollow,linecolor=lightgray,
%    opacity=0.75,r=0.3,incolor=white,opacity=0.15,%fillcolor=gray!25,
%ngrid=20 20](0,0,0)%
%\end{pspicture}
%% \begin{pspicture}(-1,-2)(3,2.5)
%%     \axesIIID[axisnames={\bone,\bZ,\bepsilon}](-0.1,-0.1,-0.1)(1.5,0.5,0.75)
%% % \psSolid[object=parallelepiped,
%% % a=0.5,b=1,c=1,fillcolor=white,opacity=0.2](-0.1,0.2,-0.01)
%% \psSolid[object=parallelepiped,
%% a=0.5,b=1,c=0.8,fillcolor=white,opacity=0.2,action=draw*](-0.05,0.2,0.05)
%% \psPoint[](0,0,0){O}
%% \psPoint[](0,0.1,0.3){A}
%% \psPoint[](0,0.1,0){B}
%% \psPoint[](0,0.5,0.17){C}
%% \psset{dotsep=0.5pt}
%% \psline[linewidth=1.5pt,linestyle=dotted]{->}(O)(A)
%% \psdot[dotscale=1,fillcolor=lightgray,dotstyle=Bo](A)
%% \psline[linewidth=2pt,linecolor=gray]{->}(O)(B)
%% \psline[linewidth=2pt,linestyle=dashed]{->}(B)(A)
%% \psline[linewidth=2pt]{->}(O)(C)
%% \psPoint[](0,0.75,0.05){mC}
%% \psPoint[](0,0.1,-0.05){mE}
%% \uput[60](mE){$\bvepsilon$}
%% % \psPoint[](0,0.75,0.35){mA}
%% % \uput[90](mA){$Y_{\!c}$}
%% \psSolid[object=sphere,hollow,linecolor=lightgray,
%%     opacity=0.75,r=0.3,incolor=white,opacity=0.15,%fillcolor=gray!25,
%% ngrid=20 20](0,0,0)%
%% \end{pspicture}
%}}}
%\vspace{-1in}
%\centerline{
%    \subfloat[Alternative hypothesis]{
%\label{fig:reg:eval:hypotest_a}
%        \scalebox{1}{
%            \psset{labelsep=2pt}
%    \definecolor{mygray}{gray}{0.95}
%\psset{unit=0.5in,arrowscale=1.5}
%\psset{viewpoint=30 20 20 rtp2xyz,Decran=100,lightsrc=viewpoint}
%\begin{pspicture}(0,-1)(6,5)
%\psSolid[object=parallelepiped,
%a=0.5,b=2.4,c=0.9,fillcolor=white,opacity=0.2,action=draw*](0.1,0.8,0.2)
%\psPoint[](0,0,0){O}
%\psPoint[](0,1.5,0){C}
%\psPoint[](0,1.75,0){B}
%% \psline[linewidth=2pt,linecolor=gray]{->}(O)(C)
%\axesIIID[axisnames={\bone,\cS_{\mX},\cS_\epsilon}](-0.2,-0.2,-0.2)(1.5,2.2,0.8)
%\psPoint[](0,1.75,0.2){A}
%\psset{dotsep=0.5pt}
%\psline[linewidth=1.5pt,linestyle=dashed]{->}(B)(A)
%\psline[linewidth=1.5pt,linestyle=dotted]{->}(C)(A)
%\psdot[dotscale=1,fillcolor=lightgray,dotstyle=Bo](A)
%\psdot[dotscale=1,fillcolor=lightgray,dotstyle=Bo](C)
%\psPoint[](0,0.75,-0.01){mC}
%\uput[-90](mC){$\hmY$}
%\psPoint[](0,0.75,0.1){mA}
%\uput[90](mA){$\mY$}
%\psPoint[](0,1.6,0.15){mE}
%\uput[-90](mE){$\bvepsilon$}
%\psPoint[](0,1.8,0.05){mB}
%\uput[90](mB){$\bepsilon$}
%\psline[linewidth=2pt,linecolor=gray]{->}(O)(B)
%\psline[linewidth=1pt,linestyle=dashed]{->}(O)(A)
%\psSolid[object=sphere,hollow,linecolor=lightgray,
%    opacity=0.75,r=0.3,incolor=white,opacity=0.15,%fillcolor=gray!25,
%ngrid=20 20](0,1.5,0)%
%% \psSolid[object=sphere,hollow,linecolor=lightgray,
%%     opacity=0.75,r=0.3,incolor=white,opacity=0.15,%fillcolor=gray!25,
%% ngrid=20 20](0,0,0)%
%\end{pspicture}
%% \begin{pspicture}(1,-1)(6,5)
%%     \axesIIID[axisnames={\bone,\bZ,\bepsilon}](-0.1,-0.1,-0.1)(0.5,1.75,0.75)
%% \psSolid[object=parallelepiped,
%% a=0.5,b=2,c=1,fillcolor=white,opacity=0.2,action=draw*](0.2,1.1,0.6)
%% \psPoint[](0,0,0){O}
%% \psPoint[](0,1.5,0.5){C}
%% \psline[linewidth=2pt]{->}(O)(C)
%% \psPoint[](0,1.6,0.8){A}
%% \psset{dotsep=0.5pt}
%% \psline[linewidth=1.5pt,linestyle=dotted]{->}(C)(A)
%% \psdot[dotscale=1,fillcolor=lightgray,dotstyle=Bo](A)
%% \psPoint[](0,0.75,0.05){mC}
%% \uput[90](mC){$E[Y_{\!c}|\bZ]$}
%% \psPoint[](0,0.75,0.35){mA}
%% \uput[90](mA){$Y_{\!c}$}
%% \psPoint[](0,1.6,0.55){mE}
%% \uput[60](mE){$\bvepsilon$}
%% \psline[linewidth=1pt,linestyle=dashed]{->}(O)(A)
%% \psSolid[object=sphere,hollow,linecolor=lightgray,
%%     opacity=0.75,r=0.3,incolor=white,opacity=0.15,%fillcolor=gray!25,
%% ngrid=20 20](0,1.5,0.5)%
%% \end{pspicture}
%}}}
%    \vspace{0.3in}
%% \vspace{0.1in}
%    \caption{Geometric test for regression effect: Null versus alternative
%    hypothesis.}
%\label{fig:reg:eval:hypotest}
%\end{figure}
%
%Given that we expect to see a difference between $\hmY$ under the null
%and alternative hypotheses, this suggests a geometric test based on the relative lengths of $\hmY$ and
%$\bepsilon$ (since we do not know the true $E[\mY|\mbX]$ or
%$\bvepsilon$).
%However, there is one difficulty;
%we cannot compare their lengths directly, since $\hmY$
%lies in a $d$ dimensional space, whereas $\bepsilon$ lies in a $n-d-1$
%dimensional space. Instead, we can compare their lengths after
%normalizing by
%the number of dimensions. Define the mean squared length of per
%dimension for the two vectors $\hmY$ and $\bepsilon$, as follows
%\begin{align*}
%    M\bigl(\hmY\bigr) & = \frac{\norm[\big]{\hmY}^2}{dim(\cS_{\mX})} = 
%    \frac{\norm[\big]{\hmY}^2}{d}\\
%    M(\bepsilon) & = \frac{\norm{\bepsilon}^2}{dim(\cS_\epsilon)} = 
%   \frac{\norm{\bepsilon}^2}{n-d-1}
%\end{align*}
%Thus, the geometric test for the regression effect is the ratio of the
%normalized mean squared length of $\hmY$ to $\bepsilon$, given as
%\begin{align*}
%    \frac{M\bigl(\hmY\bigr)}{M(\bepsilon)} = 
%    \frac{\norm[\big]{\hmY}^2\big/ d}{\norm{\bepsilon}^2 \big/ (n-d-1)}
%\end{align*}
%It is interesting to note that from \cref{eq:reg:eval:decomp} we have
%$\norm[\big]{\hmY}^2 = RSS$ and
%$\norm{\bepsilon}^2 = \norm[\big]{Y-\hY}^2 = SSE$, and  therefore, the
%geometric ratio test is
%identical to the F-test in \cref{eq:reg:eval:Ftest}, since 
%\begin{align*}
%    \frac{M\bigl(\hmY\bigr)}{M(\bepsilon)} = 
%    \frac{\norm[\big]{\hmY}^2\big/ d}{\norm{\bepsilon}^2 \big/ (n-d-1)}
%    = \frac{RSS\big/d}{SSE\big/(n-d-1)} = f
%\end{align*}
%The geometric approach, illustrated in \cref{fig:reg:eval:hypotest}, makes it clear that if $f \approx 1$ then the
%null hypothesis holds, and we conclude that $Y$ does not depend on the
%predictor variables $X_1, X_2, \cdots, X_d$. On the other hand, if $f$
%is large, with a $\pvalue$ less than the significance level (say,
%$\alpha=0.01$), then we can reject
%the null hypothesis and accept the alternative hypothesis that $Y$
%depends on at least one predictor variable $X_i$.
%
%\section{Further Reading}
%\label{sec:reg:eval:ref}
%\begin{refsection}
%
%For a geometrical approach to multivariate statistics see
%\citet{wickens2014geometry}, and \citet{saville2012statistical}. 
%For an excellent treatment of modern statistical inference in
%the context of regression see \citet{devore2012modern}.
%
%\printbibliography[heading=emptyheading]
%\end{refsection}
%
%\section{Exercises}
%\label{sec:reg:eval:exercise}
%
%\begin{exercises}[Q1.]
%
%\item Show that for bivariate regression,
%    we have $\trueb = \mu_Y - \truew \cdot \mu_X$, where $\trueb$ and
%    $\truew$ are the true model parameters.
%
%% \item Prove that among all possible lines the least squares line has the
%%     smallest SSE value.
%
%% \item Prove that SSE $\le$ TSS.
%
%\item Show that $\sum^{n}_{i=1} (y_i - \hy_i) \cdot (\hy_i - \mu_Y) =
%    0$.
% 
%\item Prove that $\bepsilon$ is orthogonal to $\hY$ and $\bX$. Show this
%    for bivariate regression and then for multiple regression.
%
%\item Show that for bivariate regression, the $R^2$ statistic is
%    equivalent to the squared correlation between the independent
%    attribute vector $X$ and response vector $Y$.
%    That is, show that $R^2 = 
%    \rho^2_{XY}$.
%
%\item Show that $\hmY = \hY - \mu_Y \cdot \bone$.
%
%
%\item Show that $\norm{\bepsilon} = \norm[\big]{Y-\hY} = \norm[\big]{\mY-\hmY}$.
%
%\item In bivariate regression, show that under the null hypothesis,
%    $E[RSS] = \sigma^2$.
%    
%\item In bivariate regression, show that $E[\hsigma^2] = \sigma^2$.
%
%% \item Show that $\ocov(\bw) = E[\bw\bw^T] - E[\bw] \cdot E[\bw]^T$,
%%     where $\bw = (w_1, w_2, \cdots, w_d)$ is the $d$-dimensional
%%     random variable comprising the estimated regression coefficients.
%
%\item Show that $E[RSS/d] \ge \sigma^2$.
%
%\item \label{q:reg:eval:varei} Treating each residual $\epsilon_i = y_i - \hy_i$ as a random
%    variable, show that 
%    $$\var(\epsilon_i) = \sigma^2(1-h_{ii})$$
%    where
%    $h_{ii}$ is the $i$th diagonal of the $(d+1) \times (d+1)$ hat
%    matrix $\bH$ for the augmented data $\abD$. 
%    Next, using the above expression for $\var(\epsilon_i)$, show that for
%    bivariate regression, the variance of the $i$th residual is given
%    as
%    \begin{align*}
%        \var(\epsilon_i) = \sigma^2 \cdot \lB(1 - \frac{1}{n} -
%        \frac{1}{s_X} \cdot (x_i - \mu_X)^2 \rB)
%    \end{align*}
%
%\item Given data matrix $\bD$, let  $\mbD$ the centered data matrix, and
%    $\abD$ the augmented data matrix (with the extra column $X_0 =
%    \bone$). 
%    Let $(\abD^T\abD)$ be the uncentered scatter matrix for the
%    augmented data, and let $\mbD^T\mbD$ be the scatter matrix for the
%    centered data.
%    Show that the lower right $d \times d$
%    submatrix of $(\abD^T\abD)^{-1}$ is
%    $(\mbD^T\mbD)^{-1}$.
%
%\end{exercises}
%
