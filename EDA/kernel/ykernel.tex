\lecture{kernel}{kernel}

\date{Chapter 5: Kernel Methods}

\begin{frame}
\titlepage
\end{frame}


\begin{frame}{Input and Feature Space}

For mining and analysis, it is important to f\/{i}nd a suitable data
representation.  
For example, for complex
data such as text, sequences, images, and so on, we must typically extract
or construct a set of attributes or features, so that we can represent
the data instances as multivariate vectors.  

\medskip
Given a data
instance $\bx$ (e.g., a sequence), we need to f\/{i}nd a mapping $\phi$, so
that $\phi(\bx)$ is the vector representation of $\bx$.  

\medskip
Even when 
the input data is a numeric data matrix
a nonlinear mapping $\phi$ may be used to 
discover nonlinear relationships.

\medskip
The term {\em input space} refers to the data space for the input data
$\bx$ and {\em feature space} refers to the space of mapped vectors $\phi(\bx)$.
\end{frame}



\begin{frame}{Sequence-based Features}

  Consider a dataset of DNA sequences over the alphabet $\Sigma =
\{A,C,G,T\}$. 

\medskip
One simple feature space is to represent each sequence in
terms of the probability distribution over symbols in $\Sigma$. That is,
given a sequence $\bx$ with length $|\bx| = m$, the mapping into feature
space is given as
$$\phi(\bx) = \{P(A), P(C), P(G), P(T) \}$$
where
$P(s) = \tfrac{n_s}{m}$
is the probability of observing symbol $s\in\Sigma$,
and $n_s$ is the number of
times $s$ appears in sequence $\bx$. 

\medskip
For example, if $\bx = \hbox{\it ACAGCAGTA}$, with $m=|\bx|=9$, since $A$ occurs four
times, $C$ and $G$ occur twice, and $T$ occurs once, we have
$$\phi(\bx)
= (4/9, 2/9, 2/9, 1/9)= (0.44,0.22,0.22,0.11)$$

\medskip
We can compute larger feature spaces by considering, for example, the
probability distribution over all substrings or words of size up to $k$
over the alphabet $\Sigma$.

\end{frame}


\begin{frame}{Nonlinear Features}
%As an example of a nonlinear mapping 
Consider the mapping $\phi$ that
takes as input a vector $\bx=(x_1,x_2)^T \in \setR^2$ and maps it to a
``quadratic''  feature space via the nonlinear mapping
\begin{align*}
  \phi(\bx) = (x_1^2, x_2^2, \sqrt{2} x_1x_2)^T \in \setR^3
\end{align*}

\medskip
For example, the point $\bx = (5.9,3)^T$ is mapped to the vector
\begin{align*}
  \phi(\bx) = (5.9^2, 3^2, \sqrt{2}\cdot5.9\cdot3)^T = (34.81,9,25.03)^T
\end{align*}

\medskip
We can then apply well-known
linear analysis methods in the feature space. 
\end{frame}


\begin{frame}{Kernel Method}

  \small
Let $\cI$ denote the input space, which can
comprise any arbitrary set of objects, and let $\bD = \{\bx_i\}_{i=1}^n
\subset \cI$ be a dataset comprising $n$ objects in the input space.
Let $\phi\!: \cI \to \cF$ be a mapping from the
input space $\cI$ to the feature space $\cF$. 

\medskip
Kernel methods avoid explicitly transforming each point $\bx$ in the
input space into the mapped point $\phi(\bx)$ in the feature space.
Instead, the input objects are represented via their
pairwise similarity values comprising 
the $n \times n$ {\em kernel matrix}, def\/{i}ned as
\begin{align*}
  \bK = \matr{
      K(\bx_1,\bx_1) & K(\bx_1,\bx_2) & \cdots & K(\bx_1,\bx_n)\\
      K(\bx_2,\bx_1) & K(\bx_2,\bx_2) & \cdots & K(\bx_2,\bx_n)\\
      \vdots & \vdots & \ddots & \vdots\\
      K(\bx_n,\bx_1) & K(\bx_n,\bx_2) & \cdots & K(\bx_n,\bx_n)\\
    }
\end{align*}

\medskip
$K\!: \cI \times \cI \to \setR$
is a {\em kernel function} on any two points in input space, which 
should satisfy the
condition
\begin{align*}
K(\bx_i, \bx_{j})  = \phi(\bx_i)^T \phi(\bx_{j})
\end{align*}

\medskip
Intuitively, we need to
be able to compute the value of the dot product using the
original input representation $\bx$, without having recourse to the
mapping $\phi(\bx)$.  
\end{frame}


\begin{frame}{Linear Kernel}

  Let $\phi(\bx) \to \bx$ be the {\em identity kernel}. 
  This leads to the {\em linear kernel},
  which
  is simply the dot product between two input vectors:
  \begin{align*}
    \phi(\bx)^T\phi(\by) = \bx^T\by = K(\bx, \by)
  \end{align*}

  For example, if 
    $\bx_1  = \matr{5.9 & 3}^T$ and 
    $\bx_2  = \matr{6.9 & 3.1}^T$, then we have
  \begin{align*}
    K(\bx_1, \bx_2) = \bx_1^T\bx_2=
    5.9 \times 6.9 + 3 \times 3.1 = 40.71 + 9.3 =
    50.01
  \end{align*}

  \centerline{
  \hspace{0.2in}
\scalebox{0.9}{
  \psset{dotstyle=Bo,dotscale=1.5,fillcolor=lightgray}
  \psset{xAxisLabel=$X_1$,yAxisLabel= $X_2$}
  \centering
  \pspicture[](-2,-0.75)(2,3)
  \psgraph[Dx=0.5,Dy=0.5,Ox=4.5,Oy=2]{->}(4.5,2.0)(7.0,3.5){1.5in}{1in}%
\psdot[](5.9, 3)
\uput[u](5.9,3){$\bx_1$}
\psdot[](6.9, 3.1)
\uput[u](6.9,3.1){$\bx_2$}
\psdot[](6.6, 2.9)
\uput[u](6.6,2.9){$\bx_3$}
\psdot[](4.6, 3.2)
\uput[ur](4.6,3.2){$\bx_4$}
\psdot[](6, 2.2)
\uput[u](6,2.2){$\bx_5$}
\endpsgraph
\endpspicture
}
\hspace{-0.3in}
\pspicture[](0,-1.75)(2,1)
\begin{small}
\begin{tabular}{c|ccccc}
$\bK$& $\bx_1$ & $\bx_2$ & $\bx_3$ & $\bx_4$ & $\bx_5$\\
\hline
$\bx_1$ &43.81 &50.01 &47.64 &36.74 &42.00\\
$\bx_2$ &50.01 &57.22 &54.53 &41.66 &48.22\\
$\bx_3$ &47.64 &54.53 &51.97 &39.64 &45.98\\
$\bx_4$ &36.74 &41.66 &39.64 &31.40 &34.64\\
$\bx_5$ &42.00 &48.22 &45.98 &34.64 &40.84\\
\end{tabular}
\end{small}
\endpspicture
}
\end{frame}


\begin{frame}{Kernel Trick}

  \smallskip
  Many data mining methods can be {\em kernelized}
that is, instead of mapping the input points into feature space, the data
can be represented via the $n \times n$ kernel matrix $\bK$, and all
relevant analysis can be performed over $\bK$. 

\medskip
This is done via the
{\em kernel trick}, that is, show that the analysis task
requires only dot products $\phi(\bx_i)^T \phi(\bx_{j})$ in feature space,
which can be replaced by the corresponding kernel $K(\bx_i, \bx_{j}) =
\phi(\bx_i)^T \phi(\bx_{j})$ that can be computed eff\/{i}ciently in input
space.  

\medskip
Once the kernel matrix has been computed, we no longer even need
the input points $\bx_i$, as all operations involving only dot products
in the feature space can be performed over the $n \times n$ kernel
matrix $\bK$. 

\end{frame}


\begin{frame}{Kernel Matrix}
A function $K$ is called a {\bf positive semidef\/{i}nite kernel} if and
only if it is symmetric: $$K(\bx_i, \bx_{j}) = K(\bx_{j}, \bx_i)$$ and the
corresponding kernel matrix $\bK$ for any subset $\bD \subset \cI$ is
positive semidef\/{i}nite, that is,
\begin{align*}
    & \ba^T \bK \ba \ge 0, \text{ for all vectors } \ba \in
    \setR^n\nonumber\\
    \shortintertext{which implies that}
     & \dsum_{i=1}^n \dsum_{j=1}^n a_i a_{j} K(\bx_i, \bx_{j}) \ge
    0, \text{ for all } a_i \in \setR, i \in [1,n]
\end{align*}
\end{frame}


\begin{frame}{Dot Products and Positive Semi-definite Kernels}
\small

  \begin{block}{Positive Semidef\/{i}nite Kernel}
If $K(\bx_i, \bx_{j})$ represents the dot
product $\phi(\bx_i)^T\phi(\bx_{j})$ in some feature space, then $K$ is a
positive semidef\/{i}nite kernel. 
\end{block}

\bigskip
F{i}rst, $K$ is symmetric since the dot product is
symmetric, which also implies that $\bK$ is symmetric.

\bigskip
Second, $\bK$ is positive semidef\/{i}nite because
\begin{align*}
  \ba^T \bK \ba & = \sum_{i=1}^n \sum_{j=1}^n a_i a_{j} K(\bx_i, \bx_{j})\\[-2pt]
  & =  \sum_{i=1}^n \sum_{j=1}^n a_i a_{j} \phi(\bx_i)^T \phi(\bx_{j})\\[-2pt]
& = \lB(\sum_{i=1}^n a_i \phi(\bx_i)\rB)^T
    \lB(\sum_{j=1}^n a_{j}\phi(\bx_{j})\rB)\\[-2pt]
& = \norm{\sum_{i=1}^n a_i \phi(\bx_i)}^2 \ge 0
\end{align*}
Thus, $K$ is a positive semidef\/{i}nite kernel.
\end{frame}



\begin{frame}{Empirical Kernel Map}
We now show that if we are given a positive semidef\/{i}nite kernel
$K\!: \cI \times \cI \to \setR$, then it corresponds to a dot product in
some feature space $\cF$.

\bigskip
Def\/{i}ne the
map $\phi$ as follows:
\begin{align*}
  \phi(\bx) =
  \Big((K(\bx_1, \bx), K(\bx_2, \bx), \ldots,  K(\bx_n, \bx)
  \Big)^T \in \setR^n
\end{align*}

The {\em empirical kernel map} is
def\/{i}ned as
\begin{align*}
  \phi(\bx) = \bK^{-1/2} \cdot  \Big((K(\bx_1, \bx), K(\bx_2, \bx), \ldots,  K(\bx_n, \bx) \Big)^T \in \setR^n
\end{align*}
so that the dot product yields
\begin{align*}
  \phi(\bx_i)^T\phi(\bx_{j}) & =
  \Bigl(\bK^{-1/2} \; \bK_i\Bigr)^T
  \Bigl(\bK^{-1/2} \; \bK_{j}\Bigr) \\
  & = \bK_i^T\; \bigl( \bK^{-1/2}\bK^{-1/2} \bigr) \; \bK_{j}\\
  & = \bK_i^T\; \bK^{-1}\; \bK_{j}
\end{align*}
where $\bK_i$ is the $i$th column of $\bK$.
Over all pairs of mapped points, we have
\begin{align*}
  \lB\{ \bK_i^T \bK^{-1}\; \bK_{j}\; \rB\}_{i,j=1}^n = \bK\; \bK^{-1}\; \bK = \bK
\end{align*}
\end{frame}


\begin{frame}{Data-specif\/{i}c Mercer Kernel Map}
\small
The Mercer kernel map also corresponds to a dot product in feature
space.

\medskip
Since $\bK$ is a symmetric
positive semidef\/{i}nite matrix, it has real and 
non-negative eigenvalues.
It can be decomposed as follows:
\begin{align*}
  \bK = \bU \bLambda \bU^T
\end{align*}
where $\bU$ is the orthonormal matrix of eigenvectors
$\bu_i = (u_{i1}, u_{i2}, \ldots, u_{in})^T \in \setR^n$ (for~$i=1,\ldots,n$), and $\bLambda$ is
the diagonal matrix of eigenvalues, with both arranged in non-increasing
order of the eigenvalues
$\lambda_1 \ge \lambda_2 \ge \ldots \ge \lambda_n \ge 0$:

The Mercer map $\phi$ is given as
\begin{align*}
  \phi(\bx_i) = \sqrt{\bLambda} \bU_i
\end{align*}
where $\bU_i$ is the $i$th row of $\bU$.

The kernel value is simply the dot product between
scaled rows of $\bU$:
\begin{align*}
   \phi(\bx_i)^T\phi(\bx_{j}) = \lB(\sqrt{\bLambda} \bU_i\rB)^T
   \lB(\sqrt{\bLambda} \bU_{j}\rB) = \bU_i^T \bLambda \bU_{j}
\end{align*}
\end{frame}


\begin{frame}{Polynomial Kernel}
  \small
Polynomial kernels are of two types: homogeneous or
inhomogeneous.


\medskip
Let $\bx, \by \in \setR^d$.
The (inhomogeneous) {\em polynomial
kernel} is def\/{i}ned as
  \begin{align*}
    K_q(\bx, \by)= \phi(\bx)^T \phi(\by) =
    (c+\bx^{T}\by)^q
  \end{align*}
  where $q$ is the degree of the polynomial, and $c \ge 0$ is some
  constant. When $c=0$ we obtain the homogeneous kernel,
  comprising only degree $q$ terms. When
  $c>0$, the feature space is spanned
  by all products of at most $q$ attributes.
  
  This can be seen
  from the binomial expansion
  \begin{align*}
      K_q(\bx, \by) = (c+\bx^T\by)^q
      = \sum_{k=1}^q {q \choose k} c^{q-k} \lB( \bx^T\by \rB)^k
  \end{align*}

\medskip
The most typical cases are the {\em linear} (with $q=1$) and
  {\em quadratic} (with $q=2$) kernels, given as
  \begin{align*}
      K_1(\bx, \by) &= c+ \bx^T\by\\
     K_2(\bx, \by) &= (c+ \bx^T\by)^2
  \end{align*}

\end{frame}


\begin{frame}{Gaussian Kernel}
The Gaussian kernel, also called the Gaussian radial basis
function (RBF) kernel,
is def\/{i}ned as
  \begin{align*}
      K(\bx, \by) =
    \exp\lB\{-\frac{\norm{\bx-\by}^2}{2\sigma^2} \rB\}
  \end{align*}
where $\sigma > 0$ is the spread parameter that plays the same
role as the standard deviation in a normal density function.

\medskip
Note that $K(\bx,\bx)=1$, and further that the kernel
value is inversely related to the distance between the two
points $\bx$ and $\by$.

\medskip
A feature space for the Gaussian
kernel has inf\/{i}nite dimensionality. 
\end{frame}


\begin{frame}{Basic Kernel Operations in Feature Space}

  Basic data analysis tasks that can be
  performed solely via kernels, without instantiating $\phi(\bx)$.

\medskip
{\bf Norm of a Point:} We can compute the
norm of a point $\phi(\bx)$ in feature space as follows:
\begin{align*}
  \norm{\phi(\bx)}^2 = \phi(\bx)^T\phi(\bx) = K(\bx, \bx)
\end{align*}
which implies that $\norm{\phi(\bx)} = \sqrt{K(\bx, \bx)}$.

\medskip
{\bf Distance between Points:}
The distance between $\phi(\bx_i)$ and
$\phi(\bx_{j})$ is
\begin{align*}
  \norm{\phi(\bx_i) -\phi(\bx_{j})}^2 & = \norm{\phi(\bx_i)}^2 +
  \norm{\phi(\bx_{j})}^2 - 2
  \phi(\bx_i)^T\phi(\bx_{j})\\
    & = K(\bx_i,\bx_i) + K (\bx_{j}, \bx_{j}) - 2 K(\bx_i,\bx_{j})
\end{align*}
which implies that
\begin{align*}
  \norm{\phi(\bx_i) -\phi(\bx_{j})} =
  \sqrt{K(\bx_i,\bx_i) + K (\bx_{j}, \bx_{j}) - 2 K(\bx_i,\bx_{j})}
\end{align*}
\end{frame}


\begin{frame}{Basic Kernel Operations in Feature Space}

  {\bf Kernel Value as Similarity:} We can rearrange the terms in
  $$\norm{\phi(\bx_i) -\phi(\bx_{j})}^2 
  = K(\bx_i,\bx_i) + K (\bx_{j}, \bx_{j}) - 2
  K(\bx_i,\bx_{j})$$
  to obtain
\begin{align*}
{1 \over 2} \lB(\|\phi(\bx_i)\|^2 + \|\phi(\bx_{j})\|^2 - \|
    \phi(\bx_i)-\phi(\bx_{j}) \|^2\rB) =
K(\bx_i,\bx_{j}) = \phi(\bx_i)^T\phi(\bx_{j})
\end{align*}
\smallskip
The more the distance $\|\phi(\bx_i)-\phi(\bx_{j})\|$
between the two points in
feature space, the less the kernel
value, that is, the less the similarity.

\medskip
{\bf Mean in Feature Space:} 
The mean of the points in
feature space is given as $\bmu_\phi = 1/n \sum_{i=1}^n \phi(\bx_i)$.
Thus, we cannot compute it explicitly. However, the 
the squared norm of the mean is:
\begin{align}
    \|\bmu_\phi\|^2 & = \bmu_\phi^T\bmu_\phi = 
    {1\over n^2} \sum_{i=1}^n \sum_{j=1}^n K(\bx_i, \bx_{j})
\end{align}
The squared norm of the mean in
feature space is simply the average of the values in the kernel
matrix $\bK$.
\end{frame}


\begin{frame}{Basic Kernel Operations in Feature Space}
\small

{\bf Total Variance in Feature Space:}
The total variance in feature space is
obtained by taking the average squared
deviation of points from the mean in feature space:
\begin{align*}
        \sigma_\phi^2 & = {1 \over n} \sum_{i=1}^n \|\phi(\bx_i)
        - \bmu_\phi \|^2 = 
 {1 \over n} \sum_{i=1}^n K(\bx_i, \bx_i) - {1\over n^2}
 \sum_{i=1}^n \sum_{j=1}^n K(\bx_i, \bx_{j})
\end{align*}

\medskip
{\bf Centering in Feature Space}
We can center each point in feature space by subtracting the mean
from it, as follows:
\begin{align*}
  \hat{\phi}(\bx_i) = \phi(\bx_i) - \bmu_\phi
\end{align*}
The kernel between centered points is given as
\begin{align*}
  \hat{\kern-3pt K}(\bx_i,\bx_{j}) & = \hat{\phi}(\bx_i)^T\hat{\phi}(\bx_{j})\\
  & = K(\bx_i,\bx_{j}) - {1 \over n} \dsum_{k=1}^n K(\bx_i,\bx_k)  -
  {1 \over n} \dsum_{k=1}^n K(\bx_{j},\bx_k) + {1 \over n^2}
  \dsum_{a=1}^n \dsum_{b=1}^n K(\bx_a,\bx_b)
\end{align*}

More compactly, we have:
\begin{align*}
    \hat{\bK} &= 
    \lB(\bI - {1\over n}\bone_{n\times n}\rB) \bK
    \lB(\bI - {1\over n}\bone_{n\times n}\rB)
\end{align*}
where $\bone_{n\times n}$ is the  $n \times n$ matrix of ones.
\end{frame}


\begin{frame}{Basic Kernel Operations in Feature Space}

{\bf Normalizing in Feature Space:}
The dot product between normalized points in feature space 
corresponds to the cosine of the angle between them
\begin{align*}
  \phi_n(\bx_i)^T\phi_n(\bx_{j}) =
  {\phi(\bx_i)^T\phi(\bx_{j}) \over \norm{\vphantom{\norm{\phi(\bx_{j})}}\phi(\bx_i)}
  \cdot \norm{\phi(\bx_{j})}} = \cos \theta
\end{align*}

\medskip
If the mapped points are both centered and normalized,
then a dot product corresponds to the correlation
between the two points in feature space.

\medskip
The normalized kernel matrix, $\bK_n$,
can be computed using only the kernel function $K$, as
\begin{align*}
  \bK_n(\bx_i, \bx_{j}) = {\phi(\bx_i)^T\phi(\bx_{j}) \over \norm{\vphantom{\norm{\phi(\bx_{j})}}\phi(\bx_i)}
  \cdot \norm{\phi(\bx_{j})}} =
  {K(\bx_i, \bx_{j}) \over \sqrt{K(\bx_i,\bx_i) \cdot K(\bx_{j},\bx_{j})}}
\end{align*}
$\bK_n$ has all diagonal elements as 1.
\end{frame}


\begin{frame}{Spectrum Kernel for Strings}

\small
Given alphabet $\Sigma$,  the $l$-spectrum feature map is
the mapping $\phi\!: \Sigma^* \to \setR^{|\Sigma|^l}$ from the set of
substrings over $\Sigma$ to the $|\Sigma|^l$-dimensional space
representing the number of occurrences of all possible
substrings of length $l$, def\/{i}ned as
\begin{align*}
  \phi(\bx) = \Bigl( \cdots, \#(\alpha), \cdots \Bigr)^T_{\alpha \in \Sigma^l}
\end{align*}
where $\#(\alpha)$ is the number of occurrences of the $l$-length string
$\alpha$ in $\bx$.

\medskip
The (full) spectrum map 
considers all lengths from $l=0$ to $l=\infty$, leading to an 
inf\/{i}nite
dimensional feature map $\phi: \Sigma^* \to \setR^\infty$:
\begin{align*}
  \phi(\bx) = \Bigl( \cdots,
  \#(\alpha), \cdots \Bigr)^T_{\alpha \in \Sigma^*}
\end{align*}
where $\#(\alpha)$ is the number of occurrences of the string
$\alpha$ in $\bx$.

\medskip
The ($l$-)spectrum kernel between two strings $\bx_i, \bx_{j}$
is simply the dot product between their ($l$-)spectrum maps:
\begin{align*}
  K(\bx_i, \bx_{j}) = \phi(\bx_i)^T\phi(\bx_{j})
\end{align*}

\medskip
The (full) spectrum kernel can be computed
eff\/{i}ciently via suff\/{i}x trees 
in $O(n + m)$ time for two strings of length $n$ and $m$.
\end{frame}


\begin{frame}{Diffusion Kernels on Graph Nodes}
Let $\bS$ be some symmetric similarity matrix between nodes of a graph
$G=(V,E)$.  For instance, $\bS$
can be the (weighted) adjacency matrix $\bA$ or
the Laplacian matrix $\bL = \bA - \bDelta$ (or its negation),
where $\bDelta$ is the degree matrix for
an undirected graph $G$, def\/{i}ned as $\bDelta(i,i) = d_i$ and
$\bDelta(i,j) = 0$ for all $i\ne j$, and $d_i$ is the degree of node~$i$.


\medskip
{\bf Power Kernels:}
Summing up the
product of the base similarities over all $l$-length paths between two
nodes, we obtain the $l$-length similarity matrix $\bS^{(l)}$,
which is simply the $l$th power of $\bS$, that is,
\begin{align*}
  \bS^{(l)} = \bS^l
\end{align*}

\medskip
Even path lengths lead to positive semidef\/{i}nite kernels, but odd path
lengths are not guaranteed to do so, unless the base matrix $\bS$ is
itself a positive semidef\/{i}nite matrix.

\medskip
Power kernel $\bK$ can be obtained via the eigen-decomposition of $\bS^l$:
\begin{align*}
  \bK = \bS^l = \bigl(\bU \bLambda \bU^T\bigr)^l
  = \bU \bigl(\bLambda^l\bigr) \bU^T
\end{align*}

\end{frame}


\begin{frame}{Exponential Diffusion Kernel}
\small

The exponential diffusion kernel 
we can obtain a new kernel between nodes of a graph by 
paths of all possible lengths, but damps the contribution of longer
paths
\begin{align*}
  \bK & = \sum_{l=0}^\infty {1 \over l!} \beta^l \bS^l\\
  & = \bI + \beta \bS + {1 \over 2!} \beta^2 \bS^2 + {1 \over 3!}
  \beta^3 \bS^3 + \cdots\\
  & = \exp\bigl\{ \beta \bS \bigr\}
\end{align*}
where $\beta$ is a damping factor, and $\exp\{\beta\bS\}$ is the
matrix exponential. The series on the right hand side above converges
for all $\beta \ge 0$.

%
\medskip
Substituting $\bS \,{=}\, \bU \bLambda \bU^T$ the kernel can be computed
as
\begin{align*}
  \bK
  & = \bI + \beta \bS + {1 \over 2!} \beta^2 \bS^2 + \cdots\\
  & = \bU
  \matr{
  \exp\{\beta\lambda_1\} & 0 & \cdots & 0\\
  0 & \exp\{\beta\lambda_2\} & \cdots & 0\\
  \vdots & \vdots & \ddots & 0\\
  0 & 0 & \cdots & \exp\{\beta\lambda_n\}
  } \bU^T
\end{align*}
where $\lambda_i$ is an eigenvalue of $\bS$.
\end{frame}



\begin{frame}{Von Neumann Diffusion Kernel}

The {\em von Neumann diffusion kernel} is def\/{i}ned as
\begin{align*}
  \bK = \sum_{l=0}^\infty \beta^l \bS^l
\end{align*}
where $\beta \ge 0$.
Expanding and rearranging the terms, we obtain 
\begin{align*}
   \bK &= (\bI - \beta \bS)^{-1}
\end{align*}

\medskip
The kernel is guaranteed to be
positive semidef\/{i}nite if
$\abs{\beta} < 1/\rho(\bS)$, where $\rho(\bS) = \max_i
\{\abs{\lambda_i}\}$ is called the {\em spectral radius} of $\bS$,
def\/{i}ned as the largest eigenvalue of $\bS$ in
absolute value.
\end{frame}


\begin{frame}{Graph Diffusion Kernel: Example}
\centerline{
  \psset{unit=1in}
  \psset{treesep=0.75,levelsep=1,treemode=R}
  \centering
    \pstree[]{\Tcircle[name=u1]{$v_1$}}{
      \pstree[]{\Tcircle[name=u4]{$v_4$}}{
        \Tcircle[name=u5]{$v_5$}
      }
      \pstree[]{\def\psedge{\ncline[]}
        \Tcircle[name=u3]{$v_3$}}{
        \Tcircle[name=u2]{$v_2$}
      }
    }
    \ncline[]{u5}{u2}
    \ncline[]{u4}{u3}
    }

  Adjacency and degree matrices are given as
  \begin{align*}
    \bA & = \matr{
    0 & 0 & 1 & 1 & 0\\
    0 & 0 & 1 & 0 & 1\\
    1 & 1 & 0 & 1 & 0\\
    1 & 0 & 1 & 0 & 1\\
    0 & 1 & 0 & 1 & 0
    } &
    \bDelta = \matr{
    2 & 0 & 0 & 0 & 0\\
    0 & 2 & 0 & 0 & 0\\
    0 & 0 & 3 & 0 & 0\\
    0 & 0 & 0 & 3 & 0\\
    0 & 0 & 0 & 0 & 2
    }
  \end{align*}
\end{frame}

\begin{frame}{Graph Diffusion Kernel: Example}

  Let the base similarity matrix $\bS$ be the 
	negated Laplacian matrix
  \begin{align*}
    \bS = -\bL = \bA - \bD = \amatr{r}{
    -2 & 0 & 1 & 1 & 0\\
    0 & -2 & 1 & 0 & 1\\
    1 & 1 & -3 & 1 & 0\\
    1 & 0 & 1 & -3 & 1\\
    0 & 1 & 0 & 1 & -2
    }
  \end{align*}
  The eigenvalues of $\bS$ are as
  follows:
  \begin{align*}
    \lambda_1 & = 0 & \lambda_2 & = -1.38 & \lambda_3 & = -2.38 &
    \lambda_4 & = -3.62 & \lambda_5 & = -4.62
  \end{align*}
  and the eigenvectors of $\bS$ are
  \begin{align*}
 \bU = \amatr{r}{
 \bu_1 & \bu_2 & \bu_3 & \bu_4 & \bu_5\\
 \hline
0.45& -0.63&    0.00&   0.63&   0.00\\
0.45&  0.51&   -0.60&   0.20&  -0.37\\
0.45& -0.20&   -0.37&  -0.51&   0.60\\
0.45& -0.20&    0.37&  -0.51&  -0.60\\
0.45&  0.51&    0.60&   0.20&   0.37\\
}
  \end{align*}
\end{frame} 
 

\begin{frame}{Graph Diffusion Kernel: Example}

  Assuming $\beta = 0.2$, the exponential diffusion kernel
  matrix is given as
  \begin{align*}
    \bK = \exp\bigl\{ 0.2 \bS \bigr\} & =
  \bU \matr{
  \exp\{0.2\lambda_1\} & 0 & \cdots & 0\\
  0 & \exp\{0.2\lambda_2\} & \cdots & 0\\
  \vdots & \vdots & \ddots & 0\\
  0 & 0 & \cdots & \exp\{0.2\lambda_n\}
  } \bU^T\\
  & =
    \matr{
0.70& 0.01& 0.14& 0.14& 0.01\\
0.01& 0.70& 0.13& 0.03& 0.14\\
0.14& 0.13& 0.59& 0.13& 0.03\\
0.14& 0.03& 0.13& 0.59& 0.13\\
0.01& 0.14& 0.03& 0.13& 0.70\\
    }
  \end{align*}

Assuming $\beta = 0.2$, the von Neumann kernel is given as
  \begin{align*}
    \bK = \bU(\bI - 0.2\bLambda)^{-1} \bU^T =
    \matr{
0.75& 0.02& 0.11& 0.11& 0.02\\
0.02& 0.74& 0.10& 0.03& 0.11\\
0.11& 0.10& 0.66& 0.10& 0.03\\
0.11& 0.03& 0.10& 0.66& 0.10\\
0.02& 0.11& 0.03& 0.10& 0.74\\
}
  \end{align*}
\end{frame}

