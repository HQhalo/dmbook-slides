\lecture{data}{data}
\date{Chapter 1: Data Mining and Analysis}

\begin{frame}
\titlepage
\end{frame}

\begin{frame}{Data Matrix}
Data can often be represented or abstracted as an $n
\times d$ {\em data matrix}\index{data matrix},
with $n$ rows and $d$ columns, given as
\begin{align*}
        \bD = &
        \matr{%
            & \vline & X_1 & X_2 & \cdots & X_d\\
            \hline
            \bx_1 & \vline & x_{11} & x_{12} & \cdots & x_{1d}\\
            \bx_2 & \vline& x_{21} & x_{22} & \cdots & x_{2d}\\
            \vdots & \vline& \vdots & \vdots & \ddots & \vdots\\
            \bx_n & \vline& x_{n1} & x_{n2} & \cdots & x_{nd}
        }
\end{align*}
\begin{itemize}
  \item {\bf Rows}: Also called 
{\em instances}, {\em examples}, {\em records}, {\em
transactions}, {\em objects}, {\em points}, {\em feature-vectors}, etc.
Given as a $d$-tuple 
$$\bx_i = (x_{i1}, x_{i2}, \ldots, x_{id})$$
\item {\bf Columns}: Also called 
{\em attributes}, {\em properties}, {\em features}, {\em dimensions}, {\em
variables}, {\em f\/{i}elds}, etc. Given as an $n$-tuple
$$X_{j} = (x_{1j}, x_{2j}, \ldots, x_{nj})$$
\end{itemize}
\end{frame}

\begin{frame}{Iris Dataset Extract}

\begin{center}
\begin{tabular}{c}
    $\matr{%
    & \vline & \textbf{Sepal}  & \textbf{Sepal}
    & \textbf{Petal} & \textbf{Petal}
    & \multirow{2}{*}{\textbf{Class}}\\
    & \vline & \textbf{length}  & \textbf{width}
    & \textbf{length} & \textbf{width} & \\
    & \vline & X_1 & X_2 & X_3 & X_4 & X_5\\
        \hline
        \bx_1 &\vline &  5.9 &3.0 &4.2 &1.5 &\text{Iris-versicolor}\\
        \bx_2 &\vline &  6.9 &3.1 &4.9 &1.5 &\text{Iris-versicolor}\\
        \bx_3 &\vline &  6.6 &2.9 &4.6 &1.3 &\text{Iris-versicolor}\\
        \bx_4 &\vline &  4.6 &3.2 &1.4 &0.2 &\text{Iris-setosa} \\
        \bx_5 &\vline &  6.0 &2.2 &4.0 &1.0 &\text{Iris-versicolor}\\
        \bx_6 &\vline &  4.7 &3.2 &1.3 &0.2 &\text{Iris-setosa}  \\
        \bx_7 &\vline &  6.5 &3.0 &5.8 &2.2 &\text{Iris-virginica} \\
        \bx_8 &\vline &  5.8 &2.7 &5.1 &1.9 &\text{Iris-virginica} \\
        \vdots & \vline & \vdots & \vdots & \vdots & \vdots &
        \vdots\\
        \bx_{149} & \vline&7.7 &3.8 &6.7 &2.2 & \text{Iris-virginica}\\
        \bx_{150} & \vline&5.1 &3.4 &1.5 &0.2 & \text{Iris-setosa}\\
    }%
    $
\end{tabular}
\end{center}
\end{frame}

\section[]{Attributes}
\begin{frame}{Attributes}
Attributes may be classif\/{i}ed into two main types
\begin{itemize}
  \item {\bf Numeric Attributes}: real-valued or integer-valued domain
	  \begin{itemize}
		\item {\em Interval-scaled}: only differences are meaningful\\
		  e.g., {\tt temperature}
		\item {\em Ratio-scaled}: differences and ratios are
		  meaningful\\
		  e..g, {\tt Age}
	  \end{itemize}
	\item {\bf Categorical Attributes}: set-valued domain composed of a
	  set of symbols
	  \begin{itemize}
		\item {\em Nominal}: only equality is meaningful\\
		  e.g., domain({\tt Sex}) = \{ {\tt M, F}\}
		\item {\em Ordinal}: both equality (are two values the same?) 
		  and inequality (is one value less than another?) are
		  meaningful\\
		  e.g., domain({\tt Education}) 
		  = \{ {\tt High School}, {\tt BS}, {\tt MS}, {\tt PhD}\}
	  \end{itemize}
\end{itemize}
\end{frame}


\begin{frame}{Data: Algebraic and Geometric View}
  \small
For numeric data matrix $\bD$, 
each row or point is a $d$-dimensional 
column vector:
$$\bx_i = \matr{%
    x_{i1}\\
    x_{i2}\\
    \vdots\\
    x_{id}
} =
\matr{x_{i1} & x_{i2} & \cdots & x_{id}}^T \in \setR^d$$
whereas each column or attribute is a $n$-dimensional column vector:
$$X_{j} = \matr{x_{1j} & x_{2j} & \cdots & x_{nj}}^T \in \setR^n$$
  \begin{figure}
	\vspace{-0.4in}
  \centerline{
  \subfloat[$\setR^2$]{
    \scalebox{0.7}{
    \psset{nameX={$X_1$}, nameY={$X_2$}}
    \psset{dotstyle=Bo,dotscale=2.5,fillcolor=lightgray}
    \psset{xAxisLabel=$X_1$,yAxisLabel=$X_2$}
    \pspicture(-1.5,-0.5)(2,4)
    \psgraph[]{->}(0,0)(7,4.5){1.75in}{1.25in}%
    \psdots[dotstyle=o](5.9,3.0)
    \uput[u](5.9,3.0){$~\bx_1 = (5.9, 3.0)^T$}
    \psline[linewidth=2pt,arrowscale=1.5]{->}(0,0)(5.9,3.0)
    \psline[linestyle=dashed]{-}(5.9,0)(5.9,3.0)
    \psline[linestyle=dashed]{-}(0,3.0)(5.9,3.0)
    \endpsgraph
    \endpspicture
	}}
	\subfloat[$\setR^3$]{
    \scalebox{0.5}{
    \psset{nameX={$X_1$}, nameY={$X_2$}, nameZ={$X_3$}}
    \psset{unit=0.4in}
    \begin{pspicture}(-1.5,-1.5)(4,6)
   \psset{Alpha=75,Beta=15,IIIDticks,IIIDlabels,IIIDxTicksPlane=yx}
    \pstThreeDCoor[xMin=0, xMax=7, yMin=0, yMax=4,
        zMin=0, zMax=5, linecolor=black]
    \psset{dotstyle=Bo,dotscale=2.5,fillcolor=lightgray}
    \pstThreeDDot[drawCoor=true](5.9,3.0,4.2)
    \pstThreeDPut[origin=rt](5,4.5,4.5){$\bx_1 = (5.9,3.0,4.2)^T$}
    \pstThreeDLine[linewidth=2pt,arrows=->,arrowscale=1.5](0,0,0)(5.9,3.0,4.2)
    \psset{dotsep=2pt}
    \pstThreeDLine[linewidth=1pt,linestyle=dotted](0,0,0)(5.9,3.0,0)
    \pstThreeDLine[linewidth=1pt,linestyle=dotted](0,0,4.2)(5.9,3.0,4.2)
    \end{pspicture}
	}}
	}
	\caption{Projections of $\bx_1 = (5.9, 3.0, 4.2, 1.5)^T$ in 2D and 3D}
  \end{figure}
\end{frame}


\begin{frame}{Scatterplot: 2D Iris Dataset\\
  {\tt sepal length} versus {\tt sepal width}.} 
\readdata{\dataSLW}{EDA/data/figs/iris-slw.dat}
\readdata{\dataSLWPL}{EDA/data/figs/iris-slwpl.dat}
    \centering
  Visualizing Iris dataset as points/vectors in 2D\\
	Solid circle shows the mean point
    \scalebox{0.8}{
    \psset{dotstyle=Bo,dotscale=1.5,fillcolor=lightgray,arrowscale=2,PointName=none}
    \psset{xAxisLabel=$X_1$: sepal length,yAxisLabel= $X_2$: sepal width,
        xAxisLabelPos={c,-0.4in},yAxisLabelPos={-0.5in,c}}
    \psgraph[tickstyle=bottom,Dx=0.5,Dy=0.5,Ox=4,Oy=2,subticks=2]{->}(4.0,2.0)(8.5,5){4in}{2.5in}%
    \dataplot[plotstyle=dots,showpoints=true]{\dataSLW}
    \psdot[dotstyle=*,dotscale=2](5.84,3.05)
    \psline[linestyle=dashed](5.84,2)(5.84,3.05)
    \psline[linestyle=dashed](4,3.05)(5.84,3.05)
    \endpsgraph
    }
\end{frame}


\begin{frame}{Numeric Data Matrix}
  \small
  If all attributes are numeric, then the data matrix
$\bD$ is an $n \times d$ matrix,
 or equivalently a set of $n$ row vectors 
$\bx_i^T
\in \setR^d$ 
or a set of $d$ column vectors 
$X_{j} \in \setR^n$
\begin{align*}
        \bD = &
        \matr{%
            x_{11} & x_{12} & \cdots & x_{1d}\\
            x_{21} & x_{22} & \cdots & x_{2d}\\
            \vdots & \vdots & \ddots & \vdots\\
            x_{n1} & x_{n2} & \cdots & x_{nd}
        }
        = \matr{\mbox{---} \; \bx_1^T \; \mbox{---}\\[0.3em]
                \mbox{---} \; \bx_2^T \; \mbox{---}\\[0.3em]
                \vdots\\
                \mbox{---} \; \bx_n^T \; \mbox{---}}
        = \matr{
            | & | & & |\\
            X_1 & X_2 & \cdots & X_d\\
            | & | & & |
            }
\end{align*}

The {\em mean} of the data matrix $\bD$ is the average of all the points:
    $\displaystyle mean(\bD) = \bmu = {1 \over n} \sum_{i=1}^n \bx_i$

The {\em centered
data matrix} is obtained by
subtracting the mean from all the points:
\begin{align}
\bZ = \bD - \bone \cdot \bmu^T
= \matr{\bx_1^T\\[0.3em] \bx_2^T\\[0.3em] \vdots \\ \bx_n^T} -
  \matr{\bmu^T\\[0.3em] \bmu^T \\[0.3em] \vdots \\\bmu^T}
        = \matr{\bx_1^T - \bmu^T\\[0.3em] \bx_2^T - \bmu^T \\[0.3em] \vdots \\
        \bx_n^T-\bmu^T}
        = \matr{\bz_1^T \\[0.3em] \bz_2^T \\[0.3em] \vdots \\\bz_n^T}
    \label{eq:eda:data:centered}
\end{align}
where $\bz_i = \bx_i - \bmu$ is a centered point, 
and $\bone \in \setR^n$
is the vector of ones.
\end{frame}

\begin{frame}{Norm, Distance and Angle}
  \begin{columns}
  \begin{column}[t]{.45\textwidth}
	\small
  Given two points $\ba, \bb \in \setR^m$, their {\em dot product} 
  is def\/{i}ned as the scalar
\begin{empheq}[box=\tcbhighmath]{align*}
    \ba^T\bb & = a_1b_1 + a_2b_2 + \cdots + a_mb_m\nonumber\\
    & = \sum_{i=1}^m a_i b_i
\end{empheq}

The {\em Euclidean norm} or {\em length}
 of a vector $\ba$ is def\/{i}ned as
\begin{empheq}[box=\tcbhighmath]{align*}
    \norm{\ba} = \sqrt{\ba^T\ba} = \sqrt{\sum_{i=1}^m a_i^2}
\end{empheq}
The {\em unit vector} in the direction of $\ba$ is
  $\bu = {\ba \over \norm{\ba}}$
with $\norm{\ba} = 1$.
  \end{column}

  \begin{column}[t]{.5\textwidth}
	\small
	{\em Distance} between $\ba$ and $\bb$ is given as
\begin{empheq}[box=\tcbhighmath]{align*}
    \norm{\ba - \bb} = \sqrt{\sum_{i=1}^m (a_i-b_i)^2}
\end{empheq}

{\em Angle} between $\ba$ and $\bb$ is given as
\begin{empheq}[box=\tcbhighmath]{align*}
    \cos \theta = {\ba^T\bb \over \norm{\ba}\norm{\bb}}
    = \lB({\ba\over \norm{\ba}}\rB)^T \lB({\bb\over \norm{\bb}}\rB)
\end{empheq}

    \centerline{
\scalebox{0.4}{%
    \psset{dotstyle=Bo,dotscale=2.5,fillcolor=lightgray}
    \psset{xAxisLabel=$X_1$,yAxisLabel=$X_2$}
    \psgraph[]{->}(0,0)(6,5){2.5in}{2.0in}%
    \psdots[dotstyle=o](5,3)
    \uput[ur](5,3){$(5, 3)$}
    \psdots[dotstyle=o](1,4)
    \uput[u](1,4){$(1, 4)$}
    \pcline[arrowscale=1.5,linewidth=1pt]{->}(0,0)(5,3)
    \naput[nrot=:U]{$\ba$}
    \pcline[arrowscale=1.5,linewidth=2pt,linecolor=gray]{->}(0,0)(0.83,0.5)
    \pcline[arrowscale=1.5,linewidth=1pt]{->}(0,0)(1,4)
    \nbput[nrot=:U]{$\bb$}
    \pcline[arrowscale=1.5,linewidth=2pt,linecolor=gray]{->}(0,0)(0.243,0.97)
    \pcline[arrowscale=1.5,linewidth=2pt]{->}(1,4)(5,3)
    \naput[nrot=:U]{$\ba-\bb$}
    \psset{PointName=none,PointSymbol=none}
    \pstGeonode(0,0){O}(1,4){A}(5,3){B}
    \pstMarkAngle[MarkAngleRadius=0.75]{B}{O}{A}{$\theta$}
    \endpsgraph
    }
  }
  \end{column}
\end{columns}
\end{frame}


\begin{frame}{Orthogonal Projection}
  Two vectors $\ba$ and $\bb$ are {\em orthogonal} iff $\ba^T\bb = 0$,
  i.e., the angle between them is $90^\circ$.
  Orthogonal projection of $\bb$ on $\ba$ comprises the vector
  $\bp=\bb_\parallel$ parallel to
$\ba$, and $\br=\bb_\perp$ perpendicular or orthogonal
to $\ba$, given as
$$\bb = \bb_\parallel + \bb_\perp = \bp + \br$$ 
where
\begin{empheq}[box=\tcbhighmath]{align*}
\bp = \bb_\parallel = \lB( {\ba^T\bb \over \ba^T\ba}
    \rB) \ba
\end{empheq}

    \centerline{
	\scalebox{0.6}{
    \psset{dotstyle=Bo,dotscale=2.5,fillcolor=lightgray,unit=0.4in}
    \psset{xAxisLabel=$X_1$,yAxisLabel=$X_2$}
    \psgraph[]{->}(0,0)(6,5){2.5in}{2.0in}%
    \uput[ur](5,3){$\ba$}
    \uput[u](1,4){$\bb$}
    \pcline[arrowscale=1.5,linewidth=1pt]{->}(0,0)(5,3)
    \pcline[arrowscale=1.5,linewidth=1pt]{->}(0,0)(1,4)
    \psset{PointName=none,PointSymbol=none}
    \pstGeonode(0,0){O}(1,4){B}(5,3){A}
    \pstProjection{O}{A}{B}[P]
    \pstLineAB[arrows=->,arrowscale=1.5,linewidth=2pt,linecolor=gray]{P}{B}
    \nbput[nrot=:R]{$\br = \bb_\perp$}
    \pstLineAB[arrows=->,arrowscale=1.5,linewidth=2pt,linecolor=gray]{O}{P}
    \naput[nrot=:U,labelsep=1pt]{$\bp = \bb_\parallel$}
    \endpsgraph
	}}
\end{frame}


\begin{frame}{Projection of Centered Iris Data Onto a Line $\ell$.}
    \centering
    \scalebox{0.8}{
    \psset{dotscale=1.5,arrowscale=2,PointName=none,unit=1.0in}
    \psset{dotsep=2pt}
    \begin{pspicture}(-2,-1.1)(2,1.5)
    \psaxes[Dx=0.5,Dy=0.5,labels=none]{->}(0,0)(-2,-1.2)(2.3,1.7)
    \psaxes[Dx=0.5,Dy=0.5,labels=none,axesstyle=frame]{->}(0,0)(-2,-1.2)(2.1,1.5)
\pstGeonode[PointSymbol=none](1.0738,-1.3766){na}
\pstGeonode[PointSymbol=none](-1.0738,1.3766){nb}
    \pstGeonode[PointSymbol=none](-2,1.5){t1}
    \pstGeonode[PointSymbol=none](2,1.5){t2}
    \pstGeonode[PointSymbol=none](-2,-1.2){b1}
    \pstGeonode[PointSymbol=none](2,-1.2){b2}
    \pstInterLL[PointSymbol=none]{t1}{t2}{na}{nb}{ta}
    \pstInterLL[PointSymbol=none]{b1}{b2}{na}{nb}{tb}
    \pstLineAB{ta}{tb}
    \pstGeonode[PointSymbol=Btriangle,fillcolor=lightgray](0.66,-0.05){n1}
\pstProjection[PointSymbol=triangle,dotscale=1.25]{na}{nb}{n1}[p1]
\pstLineAB[linestyle=dotted,linecolor=gray]{n1}{p1}
\pstGeonode[PointSymbol=Btriangle,fillcolor=lightgray](-0.04,-0.35){n2}
\pstProjection[PointSymbol=triangle,dotscale=1.25]{na}{nb}{n2}[p2]
\pstLineAB[linestyle=dotted,linecolor=gray]{n2}{p2}
\pstGeonode[PointSymbol=Btriangle,fillcolor=lightgray](0.86,0.05){n3}
\pstProjection[PointSymbol=triangle,dotscale=1.25]{na}{nb}{n3}[p3]
\pstLineAB[linestyle=dotted,linecolor=gray]{n3}{p3}
\pstGeonode[PointSymbol=Btriangle,fillcolor=lightgray](0.86,-0.55){n4}
\pstProjection[PointSymbol=triangle,dotscale=1.25]{na}{nb}{n4}[p4]
\pstLineAB[linestyle=dotted,linecolor=gray]{n4}{p4}
\pstGeonode[PointSymbol=Btriangle,fillcolor=lightgray](0.26,-0.05){n5}
\pstProjection[PointSymbol=triangle,dotscale=1.25]{na}{nb}{n5}[p5]
\pstLineAB[linestyle=dotted,linecolor=gray]{n5}{p5}
\pstGeonode[PointSymbol=Btriangle,fillcolor=lightgray](1.36,0.15){n6}
\pstProjection[PointSymbol=triangle,dotscale=1.25]{na}{nb}{n6}[p6]
\pstLineAB[linestyle=dotted,linecolor=gray]{n6}{p6}
\pstGeonode[PointSymbol=Btriangle,fillcolor=lightgray](0.06,-0.05){n7}
\pstProjection[PointSymbol=triangle,dotscale=1.25]{na}{nb}{n7}[p7]
\pstLineAB[linestyle=dotted,linecolor=gray]{n7}{p7}
\pstGeonode[PointSymbol=Btriangle,fillcolor=lightgray](0.66,-0.05){n8}
\pstProjection[PointSymbol=triangle,dotscale=1.25]{na}{nb}{n8}[p8]
\pstLineAB[linestyle=dotted,linecolor=gray]{n8}{p8}
\pstGeonode[PointSymbol=Btriangle,fillcolor=lightgray](-0.94,-0.55){n9}
\pstProjection[PointSymbol=triangle,dotscale=1.25]{na}{nb}{n9}[p9]
\pstLineAB[linestyle=dotted,linecolor=gray]{n9}{p9}
\pstGeonode[PointSymbol=Btriangle,fillcolor=lightgray](1.36,-0.05){n10}
\pstProjection[PointSymbol=triangle,dotscale=1.25]{na}{nb}{n10}[p10]
\pstLineAB[linestyle=dotted,linecolor=gray]{n10}{p10}
\pstGeonode[PointSymbol=Btriangle,fillcolor=lightgray](0.96,0.15){n11}
\pstProjection[PointSymbol=triangle,dotscale=1.25]{na}{nb}{n11}[p11]
\pstLineAB[linestyle=dotted,linecolor=gray]{n11}{p11}
\pstGeonode[PointSymbol=Btriangle,fillcolor=lightgray](-0.14,-0.55){n12}
\pstProjection[PointSymbol=triangle,dotscale=1.25]{na}{nb}{n12}[p12]
\pstLineAB[linestyle=dotted,linecolor=gray]{n12}{p12}
\pstGeonode[PointSymbol=Btriangle,fillcolor=lightgray](0.46,-0.55){n13}
\pstProjection[PointSymbol=triangle,dotscale=1.25]{na}{nb}{n13}[p13]
\pstLineAB[linestyle=dotted,linecolor=gray]{n13}{p13}
\pstGeonode[PointSymbol=Btriangle,fillcolor=lightgray](0.96,-0.05){n14}
\pstProjection[PointSymbol=triangle,dotscale=1.25]{na}{nb}{n14}[p14]
\pstLineAB[linestyle=dotted,linecolor=gray]{n14}{p14}
\pstGeonode[PointSymbol=Btriangle,fillcolor=lightgray](1.46,-0.15){n15}
\pstProjection[PointSymbol=triangle,dotscale=1.25]{na}{nb}{n15}[p15]
\pstLineAB[linestyle=dotted,linecolor=gray]{n15}{p15}
\pstGeonode[PointSymbol=Btriangle,fillcolor=lightgray](1.26,-0.05){n16}
\pstProjection[PointSymbol=triangle,dotscale=1.25]{na}{nb}{n16}[p16]
\pstLineAB[linestyle=dotted,linecolor=gray]{n16}{p16}
\pstGeonode[PointSymbol=Btriangle,fillcolor=lightgray](-0.24,-0.25){n17}
\pstProjection[PointSymbol=triangle,dotscale=1.25]{na}{nb}{n17}[p17]
\pstLineAB[linestyle=dotted,linecolor=gray]{n17}{p17}
\pstGeonode[PointSymbol=Btriangle,fillcolor=lightgray](0.46,0.25){n18}
\pstProjection[PointSymbol=triangle,dotscale=1.25]{na}{nb}{n18}[p18]
\pstLineAB[linestyle=dotted,linecolor=gray]{n18}{p18}
\pstGeonode[PointSymbol=Btriangle,fillcolor=lightgray](0.46,0.35){n19}
\pstProjection[PointSymbol=triangle,dotscale=1.25]{na}{nb}{n19}[p19]
\pstLineAB[linestyle=dotted,linecolor=gray]{n19}{p19}
\pstGeonode[PointSymbol=Btriangle,fillcolor=lightgray](1.06,0.05){n20}
\pstProjection[PointSymbol=triangle,dotscale=1.25]{na}{nb}{n20}[p20]
\pstLineAB[linestyle=dotted,linecolor=gray]{n20}{p20}
\pstGeonode[PointSymbol=Btriangle,fillcolor=lightgray](1.86,-0.05){n21}
\pstProjection[PointSymbol=triangle,dotscale=1.25]{na}{nb}{n21}[p21]
\pstLineAB[linestyle=dotted,linecolor=gray]{n21}{p21}
\pstGeonode[PointSymbol=Btriangle,fillcolor=lightgray](0.26,-0.45){n22}
\pstProjection[PointSymbol=triangle,dotscale=1.25]{na}{nb}{n22}[p22]
\pstLineAB[linestyle=dotted,linecolor=gray]{n22}{p22}
\pstGeonode[PointSymbol=Btriangle,fillcolor=lightgray](0.56,-0.35){n23}
\pstProjection[PointSymbol=triangle,dotscale=1.25]{na}{nb}{n23}[p23]
\pstLineAB[linestyle=dotted,linecolor=gray]{n23}{p23}
\pstGeonode[PointSymbol=Btriangle,fillcolor=lightgray](-0.04,-0.25){n24}
\pstProjection[PointSymbol=triangle,dotscale=1.25]{na}{nb}{n24}[p24]
\pstLineAB[linestyle=dotted,linecolor=gray]{n24}{p24}
\pstGeonode[PointSymbol=Btriangle,fillcolor=lightgray](2.06,0.75){n25}
\pstProjection[PointSymbol=triangle,dotscale=1.25]{na}{nb}{n25}[p25]
\pstLineAB[linestyle=dotted,linecolor=gray]{n25}{p25}
\pstGeonode[PointSymbol=Btriangle,fillcolor=lightgray](1.86,-0.45){n26}
\pstProjection[PointSymbol=triangle,dotscale=1.25]{na}{nb}{n26}[p26]
\pstLineAB[linestyle=dotted,linecolor=gray]{n26}{p26}
\pstGeonode[PointSymbol=Btriangle,fillcolor=lightgray](0.36,-0.25){n27}
\pstProjection[PointSymbol=triangle,dotscale=1.25]{na}{nb}{n27}[p27]
\pstLineAB[linestyle=dotted,linecolor=gray]{n27}{p27}
\pstGeonode[PointSymbol=Btriangle,fillcolor=lightgray](0.36,0.35){n28}
\pstProjection[PointSymbol=triangle,dotscale=1.25]{na}{nb}{n28}[p28]
\pstLineAB[linestyle=dotted,linecolor=gray]{n28}{p28}
\pstGeonode[PointSymbol=Btriangle,fillcolor=lightgray](0.46,-0.15){n29}
\pstProjection[PointSymbol=triangle,dotscale=1.25]{na}{nb}{n29}[p29]
\pstLineAB[linestyle=dotted,linecolor=gray]{n29}{p29}
\pstGeonode[PointSymbol=Btriangle,fillcolor=lightgray](0.86,0.25){n30}
\pstProjection[PointSymbol=triangle,dotscale=1.25]{na}{nb}{n30}[p30]
\pstLineAB[linestyle=dotted,linecolor=gray]{n30}{p30}
\pstGeonode[PointSymbol=Btriangle,fillcolor=lightgray](0.46,-0.35){n31}
\pstProjection[PointSymbol=triangle,dotscale=1.25]{na}{nb}{n31}[p31]
\pstLineAB[linestyle=dotted,linecolor=gray]{n31}{p31}
\pstGeonode[PointSymbol=Btriangle,fillcolor=lightgray](0.56,0.15){n32}
\pstProjection[PointSymbol=triangle,dotscale=1.25]{na}{nb}{n32}[p32]
\pstLineAB[linestyle=dotted,linecolor=gray]{n32}{p32}
\pstGeonode[PointSymbol=Btriangle,fillcolor=lightgray](0.16,-0.85){n33}
\pstProjection[PointSymbol=triangle,dotscale=1.25]{na}{nb}{n33}[p33]
\pstLineAB[linestyle=dotted,linecolor=gray]{n33}{p33}
\pstGeonode[PointSymbol=Btriangle,fillcolor=lightgray](1.56,-0.25){n34}
\pstProjection[PointSymbol=triangle,dotscale=1.25]{na}{nb}{n34}[p34]
\pstLineAB[linestyle=dotted,linecolor=gray]{n34}{p34}
\pstGeonode[PointSymbol=Btriangle,fillcolor=lightgray](0.66,0.15){n35}
\pstProjection[PointSymbol=triangle,dotscale=1.25]{na}{nb}{n35}[p35]
\pstLineAB[linestyle=dotted,linecolor=gray]{n35}{p35}
\pstGeonode[PointSymbol=Btriangle,fillcolor=lightgray](0.86,0.25){n36}
\pstProjection[PointSymbol=triangle,dotscale=1.25]{na}{nb}{n36}[p36]
\pstLineAB[linestyle=dotted,linecolor=gray]{n36}{p36}
\pstGeonode[PointSymbol=Btriangle,fillcolor=lightgray](1.06,0.15){n37}
\pstProjection[PointSymbol=triangle,dotscale=1.25]{na}{nb}{n37}[p37]
\pstLineAB[linestyle=dotted,linecolor=gray]{n37}{p37}
\pstGeonode[PointSymbol=Btriangle,fillcolor=lightgray](1.76,-0.05){n38}
\pstProjection[PointSymbol=triangle,dotscale=1.25]{na}{nb}{n38}[p38]
\pstLineAB[linestyle=dotted,linecolor=gray]{n38}{p38}
\pstGeonode[PointSymbol=Btriangle,fillcolor=lightgray](0.46,-0.25){n39}
\pstProjection[PointSymbol=triangle,dotscale=1.25]{na}{nb}{n39}[p39]
\pstLineAB[linestyle=dotted,linecolor=gray]{n39}{p39}
\pstGeonode[PointSymbol=Btriangle,fillcolor=lightgray](0.56,0.05){n40}
\pstProjection[PointSymbol=triangle,dotscale=1.25]{na}{nb}{n40}[p40]
\pstLineAB[linestyle=dotted,linecolor=gray]{n40}{p40}
\pstGeonode[PointSymbol=Btriangle,fillcolor=lightgray](-0.04,-0.35){n41}
\pstProjection[PointSymbol=triangle,dotscale=1.25]{na}{nb}{n41}[p41]
\pstLineAB[linestyle=dotted,linecolor=gray]{n41}{p41}
\pstGeonode[PointSymbol=Btriangle,fillcolor=lightgray](0.56,-0.25){n42}
\pstProjection[PointSymbol=triangle,dotscale=1.25]{na}{nb}{n42}[p42]
\pstLineAB[linestyle=dotted,linecolor=gray]{n42}{p42}
\pstGeonode[PointSymbol=Btriangle,fillcolor=lightgray](0.56,-0.25){n43}
\pstProjection[PointSymbol=triangle,dotscale=1.25]{na}{nb}{n43}[p43]
\pstLineAB[linestyle=dotted,linecolor=gray]{n43}{p43}
\pstGeonode[PointSymbol=Btriangle,fillcolor=lightgray](1.86,-0.25){n44}
\pstProjection[PointSymbol=triangle,dotscale=1.25]{na}{nb}{n44}[p44]
\pstLineAB[linestyle=dotted,linecolor=gray]{n44}{p44}
\pstGeonode[PointSymbol=Btriangle,fillcolor=lightgray](0.66,-0.05){n45}
\pstProjection[PointSymbol=triangle,dotscale=1.25]{na}{nb}{n45}[p45]
\pstLineAB[linestyle=dotted,linecolor=gray]{n45}{p45}
\pstGeonode[PointSymbol=Btriangle,fillcolor=lightgray](1.36,0.55){n46}
\pstProjection[PointSymbol=triangle,dotscale=1.25]{na}{nb}{n46}[p46]
\pstLineAB[linestyle=dotted,linecolor=gray]{n46}{p46}
\pstGeonode[PointSymbol=Btriangle,fillcolor=lightgray](1.06,0.05){n47}
\pstProjection[PointSymbol=triangle,dotscale=1.25]{na}{nb}{n47}[p47]
\pstLineAB[linestyle=dotted,linecolor=gray]{n47}{p47}
\pstGeonode[PointSymbol=Btriangle,fillcolor=lightgray](0.16,-0.05){n48}
\pstProjection[PointSymbol=triangle,dotscale=1.25]{na}{nb}{n48}[p48]
\pstLineAB[linestyle=dotted,linecolor=gray]{n48}{p48}
\pstGeonode[PointSymbol=Btriangle,fillcolor=lightgray](0.86,-0.05){n49}
\pstProjection[PointSymbol=triangle,dotscale=1.25]{na}{nb}{n49}[p49]
\pstLineAB[linestyle=dotted,linecolor=gray]{n49}{p49}
\pstGeonode[PointSymbol=Btriangle,fillcolor=lightgray](1.86,0.75){n50}
\pstProjection[PointSymbol=triangle,dotscale=1.25]{na}{nb}{n50}[p50]
\pstLineAB[linestyle=dotted,linecolor=gray]{n50}{p50}
\pstGeonode[PointSymbol=Bsquare,fillcolor=lightgray](-1.24,0.15){n51}
\pstProjection[PointSymbol=square,dotscale=1.25]{na}{nb}{n51}[p51]
\pstLineAB[linestyle=dotted,linecolor=gray]{n51}{p51}
\pstGeonode[PointSymbol=Bsquare,fillcolor=lightgray](-1.14,0.15){n52}
\pstProjection[PointSymbol=square,dotscale=1.25]{na}{nb}{n52}[p52]
\pstLineAB[linestyle=dotted,linecolor=gray]{n52}{p52}
\pstGeonode[PointSymbol=Bsquare,fillcolor=lightgray](-0.74,0.65){n53}
\pstProjection[PointSymbol=square,dotscale=1.25]{na}{nb}{n53}[p53]
\pstLineAB[linestyle=dotted,linecolor=gray]{n53}{p53}
\pstGeonode[PointSymbol=Bsquare,fillcolor=lightgray](-0.74,0.75){n54}
\pstProjection[PointSymbol=square,dotscale=1.25]{na}{nb}{n54}[p54]
\pstLineAB[linestyle=dotted,linecolor=gray]{n54}{p54}
\pstGeonode[PointSymbol=Bsquare,fillcolor=lightgray](-0.94,0.05){n55}
\pstProjection[PointSymbol=square,dotscale=1.25]{na}{nb}{n55}[p55]
\pstLineAB[linestyle=dotted,linecolor=gray]{n55}{p55}
\pstGeonode[PointSymbol=Bsquare,fillcolor=lightgray](-0.84,0.35){n56}
\pstProjection[PointSymbol=square,dotscale=1.25]{na}{nb}{n56}[p56]
\pstLineAB[linestyle=dotted,linecolor=gray]{n56}{p56}
\pstGeonode[PointSymbol=Bsquare,fillcolor=lightgray](-0.84,0.35){n57}
\pstProjection[PointSymbol=square,dotscale=1.25]{na}{nb}{n57}[p57]
\pstLineAB[linestyle=dotted,linecolor=gray]{n57}{p57}
\pstGeonode[PointSymbol=Bsquare,fillcolor=lightgray](-0.84,0.25){n58}
\pstProjection[PointSymbol=square,dotscale=1.25]{na}{nb}{n58}[p58]
\pstLineAB[linestyle=dotted,linecolor=gray]{n58}{p58}
\pstGeonode[PointSymbol=Bsquare,fillcolor=lightgray](-0.14,1.35){n59}
\pstProjection[PointSymbol=square,dotscale=1.25]{na}{nb}{n59}[p59]
\pstLineAB[linestyle=dotted,linecolor=gray]{n59}{p59}
\pstGeonode[PointSymbol=Bsquare,fillcolor=lightgray](-0.84,0.45){n60}
\pstProjection[PointSymbol=square,dotscale=1.25]{na}{nb}{n60}[p60]
\pstLineAB[linestyle=dotted,linecolor=gray]{n60}{p60}
\pstGeonode[PointSymbol=Bsquare,fillcolor=lightgray](-1.24,0.05){n61}
\pstProjection[PointSymbol=square,dotscale=1.25]{na}{nb}{n61}[p61]
\pstLineAB[linestyle=dotted,linecolor=gray]{n61}{p61}
\pstGeonode[PointSymbol=Bsquare,fillcolor=lightgray](-0.44,0.85){n62}
\pstProjection[PointSymbol=square,dotscale=1.25]{na}{nb}{n62}[p62]
\pstLineAB[linestyle=dotted,linecolor=gray]{n62}{p62}
\pstGeonode[PointSymbol=Bsquare,fillcolor=lightgray](-0.84,0.15){n63}
\pstProjection[PointSymbol=square,dotscale=1.25]{na}{nb}{n63}[p63]
\pstLineAB[linestyle=dotted,linecolor=gray]{n63}{p63}
\pstGeonode[PointSymbol=Bsquare,fillcolor=lightgray](-0.74,0.75){n64}
\pstProjection[PointSymbol=square,dotscale=1.25]{na}{nb}{n64}[p64]
\pstLineAB[linestyle=dotted,linecolor=gray]{n64}{p64}
\pstGeonode[PointSymbol=Bsquare,fillcolor=lightgray](-1.04,-0.05){n65}
\pstProjection[PointSymbol=square,dotscale=1.25]{na}{nb}{n65}[p65]
\pstLineAB[linestyle=dotted,linecolor=gray]{n65}{p65}
\pstGeonode[PointSymbol=Bsquare,fillcolor=lightgray](-0.54,0.65){n66}
\pstProjection[PointSymbol=square,dotscale=1.25]{na}{nb}{n66}[p66]
\pstLineAB[linestyle=dotted,linecolor=gray]{n66}{p66}
\pstGeonode[PointSymbol=Bsquare,fillcolor=lightgray](-0.14,0.75){n67}
\pstProjection[PointSymbol=square,dotscale=1.25]{na}{nb}{n67}[p67]
\pstLineAB[linestyle=dotted,linecolor=gray]{n67}{p67}
\pstGeonode[PointSymbol=Bsquare,fillcolor=lightgray](-1.44,-0.05){n68}
\pstProjection[PointSymbol=square,dotscale=1.25]{na}{nb}{n68}[p68]
\pstLineAB[linestyle=dotted,linecolor=gray]{n68}{p68}
\pstGeonode[PointSymbol=Bsquare,fillcolor=lightgray](-0.44,0.35){n69}
\pstProjection[PointSymbol=square,dotscale=1.25]{na}{nb}{n69}[p69]
\pstLineAB[linestyle=dotted,linecolor=gray]{n69}{p69}
\pstGeonode[PointSymbol=Bsquare,fillcolor=lightgray](-0.84,0.45){n70}
\pstProjection[PointSymbol=square,dotscale=1.25]{na}{nb}{n70}[p70]
\pstLineAB[linestyle=dotted,linecolor=gray]{n70}{p70}
\pstGeonode[PointSymbol=Bsquare,fillcolor=lightgray](-0.74,0.25){n71}
\pstProjection[PointSymbol=square,dotscale=1.25]{na}{nb}{n71}[p71]
\pstLineAB[linestyle=dotted,linecolor=gray]{n71}{p71}
\pstGeonode[PointSymbol=Bsquare,fillcolor=lightgray](-0.94,0.05){n72}
\pstProjection[PointSymbol=square,dotscale=1.25]{na}{nb}{n72}[p72]
\pstLineAB[linestyle=dotted,linecolor=gray]{n72}{p72}
\pstGeonode[PointSymbol=Bsquare,fillcolor=lightgray](-1.24,0.55){n73}
\pstProjection[PointSymbol=square,dotscale=1.25]{na}{nb}{n73}[p73]
\pstLineAB[linestyle=dotted,linecolor=gray]{n73}{p73}
\pstGeonode[PointSymbol=Bsquare,fillcolor=lightgray](-0.64,0.35){n74}
\pstProjection[PointSymbol=square,dotscale=1.25]{na}{nb}{n74}[p74]
\pstLineAB[linestyle=dotted,linecolor=gray]{n74}{p74}
\pstGeonode[PointSymbol=Bsquare,fillcolor=lightgray](-0.34,0.45){n75}
\pstProjection[PointSymbol=square,dotscale=1.25]{na}{nb}{n75}[p75]
\pstLineAB[linestyle=dotted,linecolor=gray]{n75}{p75}
\pstGeonode[PointSymbol=Bsquare,fillcolor=lightgray](-1.24,0.35){n76}
\pstProjection[PointSymbol=square,dotscale=1.25]{na}{nb}{n76}[p76]
\pstLineAB[linestyle=dotted,linecolor=gray]{n76}{p76}
\pstGeonode[PointSymbol=Bsquare,fillcolor=lightgray](-1.14,0.15){n77}
\pstProjection[PointSymbol=square,dotscale=1.25]{na}{nb}{n77}[p77]
\pstLineAB[linestyle=dotted,linecolor=gray]{n77}{p77}
\pstGeonode[PointSymbol=Bsquare,fillcolor=lightgray](-1.44,-0.15){n78}
\pstProjection[PointSymbol=square,dotscale=1.25]{na}{nb}{n78}[p78]
\pstLineAB[linestyle=dotted,linecolor=gray]{n78}{p78}
\pstGeonode[PointSymbol=Bsquare,fillcolor=lightgray](-1.04,-0.05){n79}
\pstProjection[PointSymbol=square,dotscale=1.25]{na}{nb}{n79}[p79]
\pstLineAB[linestyle=dotted,linecolor=gray]{n79}{p79}
\pstGeonode[PointSymbol=Bsquare,fillcolor=lightgray](-0.44,0.85){n80}
\pstProjection[PointSymbol=square,dotscale=1.25]{na}{nb}{n80}[p80]
\pstLineAB[linestyle=dotted,linecolor=gray]{n80}{p80}
\pstGeonode[PointSymbol=Bsquare,fillcolor=lightgray](-1.04,0.35){n81}
\pstProjection[PointSymbol=square,dotscale=1.25]{na}{nb}{n81}[p81]
\pstLineAB[linestyle=dotted,linecolor=gray]{n81}{p81}
\pstGeonode[PointSymbol=Bsquare,fillcolor=lightgray](-1.44,0.15){n82}
\pstProjection[PointSymbol=square,dotscale=1.25]{na}{nb}{n82}[p82]
\pstLineAB[linestyle=dotted,linecolor=gray]{n82}{p82}
\pstGeonode[PointSymbol=Bsquare,fillcolor=lightgray](-1.04,0.35){n83}
\pstProjection[PointSymbol=square,dotscale=1.25]{na}{nb}{n83}[p83]
\pstLineAB[linestyle=dotted,linecolor=gray]{n83}{p83}
\pstGeonode[PointSymbol=Bsquare,fillcolor=lightgray](-0.94,-0.05){n84}
\pstProjection[PointSymbol=square,dotscale=1.25]{na}{nb}{n84}[p84]
\pstLineAB[linestyle=dotted,linecolor=gray]{n84}{p84}
\pstGeonode[PointSymbol=Bsquare,fillcolor=lightgray](-1.34,-0.75){n85}
\pstProjection[PointSymbol=square,dotscale=1.25]{na}{nb}{n85}[p85]
\pstLineAB[linestyle=dotted,linecolor=gray]{n85}{p85}
\pstGeonode[PointSymbol=Bsquare,fillcolor=lightgray](-1.54,-0.05){n86}
\pstProjection[PointSymbol=square,dotscale=1.25]{na}{nb}{n86}[p86]
\pstLineAB[linestyle=dotted,linecolor=gray]{n86}{p86}
\pstGeonode[PointSymbol=Bsquare,fillcolor=lightgray](-0.84,0.55){n87}
\pstProjection[PointSymbol=square,dotscale=1.25]{na}{nb}{n87}[p87]
\pstLineAB[linestyle=dotted,linecolor=gray]{n87}{p87}
\pstGeonode[PointSymbol=Bsquare,fillcolor=lightgray](-0.64,0.45){n88}
\pstProjection[PointSymbol=square,dotscale=1.25]{na}{nb}{n88}[p88]
\pstLineAB[linestyle=dotted,linecolor=gray]{n88}{p88}
\pstGeonode[PointSymbol=Bsquare,fillcolor=lightgray](-0.34,1.15){n89}
\pstProjection[PointSymbol=square,dotscale=1.25]{na}{nb}{n89}[p89]
\pstLineAB[linestyle=dotted,linecolor=gray]{n89}{p89}
\pstGeonode[PointSymbol=Bsquare,fillcolor=lightgray](-0.84,-0.05){n90}
\pstProjection[PointSymbol=square,dotscale=1.25]{na}{nb}{n90}[p90]
\pstLineAB[linestyle=dotted,linecolor=gray]{n90}{p90}
\pstGeonode[PointSymbol=Bsquare,fillcolor=lightgray](-0.74,0.45){n91}
\pstProjection[PointSymbol=square,dotscale=1.25]{na}{nb}{n91}[p91]
\pstLineAB[linestyle=dotted,linecolor=gray]{n91}{p91}
\pstGeonode[PointSymbol=Bsquare,fillcolor=lightgray](-1.04,0.05){n92}
\pstProjection[PointSymbol=square,dotscale=1.25]{na}{nb}{n92}[p92]
\pstLineAB[linestyle=dotted,linecolor=gray]{n92}{p92}
\pstGeonode[PointSymbol=Bsquare,fillcolor=lightgray](-0.74,0.75){n93}
\pstProjection[PointSymbol=square,dotscale=1.25]{na}{nb}{n93}[p93]
\pstLineAB[linestyle=dotted,linecolor=gray]{n93}{p93}
\pstGeonode[PointSymbol=Bsquare,fillcolor=lightgray](-0.64,1.05){n94}
\pstProjection[PointSymbol=square,dotscale=1.25]{na}{nb}{n94}[p94]
\pstLineAB[linestyle=dotted,linecolor=gray]{n94}{p94}
\pstGeonode[PointSymbol=Bsquare,fillcolor=lightgray](-0.04,0.95){n95}
\pstProjection[PointSymbol=square,dotscale=1.25]{na}{nb}{n95}[p95]
\pstLineAB[linestyle=dotted,linecolor=gray]{n95}{p95}
\pstGeonode[PointSymbol=Bsquare,fillcolor=lightgray](-0.94,0.05){n96}
\pstProjection[PointSymbol=square,dotscale=1.25]{na}{nb}{n96}[p96]
\pstLineAB[linestyle=dotted,linecolor=gray]{n96}{p96}
\pstGeonode[PointSymbol=Bsquare,fillcolor=lightgray](-0.44,0.65){n97}
\pstProjection[PointSymbol=square,dotscale=1.25]{na}{nb}{n97}[p97]
\pstLineAB[linestyle=dotted,linecolor=gray]{n97}{p97}
\pstGeonode[PointSymbol=Bsquare,fillcolor=lightgray](-0.74,0.45){n98}
\pstProjection[PointSymbol=square,dotscale=1.25]{na}{nb}{n98}[p98]
\pstLineAB[linestyle=dotted,linecolor=gray]{n98}{p98}
\pstGeonode[PointSymbol=Bsquare,fillcolor=lightgray](-0.44,0.35){n99}
\pstProjection[PointSymbol=square,dotscale=1.25]{na}{nb}{n99}[p99]
\pstLineAB[linestyle=dotted,linecolor=gray]{n99}{p99}
\pstGeonode[PointSymbol=Bsquare,fillcolor=lightgray](-0.74,0.35){n100}
\pstProjection[PointSymbol=square,dotscale=1.25]{na}{nb}{n100}[p100]
\pstLineAB[linestyle=dotted,linecolor=gray]{n100}{p100}
\pstGeonode[PointSymbol=Bo,fillcolor=lightgray](0.06,-0.05){n101}
\pstProjection[PointSymbol=o,dotscale=1.25]{na}{nb}{n101}[p101]
\pstLineAB[linestyle=dotted,linecolor=gray]{n101}{p101}
\pstGeonode[PointSymbol=Bo,fillcolor=lightgray](1.06,0.05){n102}
\pstProjection[PointSymbol=o,dotscale=1.25]{na}{nb}{n102}[p102]
\pstLineAB[linestyle=dotted,linecolor=gray]{n102}{p102}
\pstGeonode[PointSymbol=Bo,fillcolor=lightgray](0.76,-0.15){n103}
\pstProjection[PointSymbol=o,dotscale=1.25]{na}{nb}{n103}[p103]
\pstLineAB[linestyle=dotted,linecolor=gray]{n103}{p103}
\pstGeonode[PointSymbol=Bo,fillcolor=lightgray](0.16,-0.85){n104}
\pstProjection[PointSymbol=o,dotscale=1.25]{na}{nb}{n104}[p104]
\pstLineAB[linestyle=dotted,linecolor=gray]{n104}{p104}
\pstGeonode[PointSymbol=Bo,fillcolor=lightgray](-0.14,-0.05){n105}
\pstProjection[PointSymbol=o,dotscale=1.25]{na}{nb}{n105}[p105]
\pstLineAB[linestyle=dotted,linecolor=gray]{n105}{p105}
\pstGeonode[PointSymbol=Bo,fillcolor=lightgray](-0.14,-0.25){n106}
\pstProjection[PointSymbol=o,dotscale=1.25]{na}{nb}{n106}[p106]
\pstLineAB[linestyle=dotted,linecolor=gray]{n106}{p106}
\pstGeonode[PointSymbol=Bo,fillcolor=lightgray](-0.34,-0.55){n107}
\pstProjection[PointSymbol=o,dotscale=1.25]{na}{nb}{n107}[p107]
\pstLineAB[linestyle=dotted,linecolor=gray]{n107}{p107}
\pstGeonode[PointSymbol=Bo,fillcolor=lightgray](-0.34,-0.75){n108}
\pstProjection[PointSymbol=o,dotscale=1.25]{na}{nb}{n108}[p108]
\pstLineAB[linestyle=dotted,linecolor=gray]{n108}{p108}
\pstGeonode[PointSymbol=Bo,fillcolor=lightgray](-0.04,-0.45){n109}
\pstProjection[PointSymbol=o,dotscale=1.25]{na}{nb}{n109}[p109]
\pstLineAB[linestyle=dotted,linecolor=gray]{n109}{p109}
\pstGeonode[PointSymbol=Bo,fillcolor=lightgray](-0.74,-0.55){n110}
\pstProjection[PointSymbol=o,dotscale=1.25]{na}{nb}{n110}[p110]
\pstLineAB[linestyle=dotted,linecolor=gray]{n110}{p110}
\pstGeonode[PointSymbol=Bo,fillcolor=lightgray](-0.24,-0.55){n111}
\pstProjection[PointSymbol=o,dotscale=1.25]{na}{nb}{n111}[p111]
\pstLineAB[linestyle=dotted,linecolor=gray]{n111}{p111}
\pstGeonode[PointSymbol=Bo,fillcolor=lightgray](-0.04,-0.35){n112}
\pstProjection[PointSymbol=o,dotscale=1.25]{na}{nb}{n112}[p112]
\pstLineAB[linestyle=dotted,linecolor=gray]{n112}{p112}
\pstGeonode[PointSymbol=Bo,fillcolor=lightgray](0.46,-0.75){n113}
\pstProjection[PointSymbol=o,dotscale=1.25]{na}{nb}{n113}[p113]
\pstLineAB[linestyle=dotted,linecolor=gray]{n113}{p113}
\pstGeonode[PointSymbol=Bo,fillcolor=lightgray](-0.24,-0.05){n114}
\pstProjection[PointSymbol=o,dotscale=1.25]{na}{nb}{n114}[p114]
\pstLineAB[linestyle=dotted,linecolor=gray]{n114}{p114}
\pstGeonode[PointSymbol=Bo,fillcolor=lightgray](0.26,-0.05){n115}
\pstProjection[PointSymbol=o,dotscale=1.25]{na}{nb}{n115}[p115]
\pstLineAB[linestyle=dotted,linecolor=gray]{n115}{p115}
\pstGeonode[PointSymbol=Bo,fillcolor=lightgray](-0.24,-0.35){n116}
\pstProjection[PointSymbol=o,dotscale=1.25]{na}{nb}{n116}[p116]
\pstLineAB[linestyle=dotted,linecolor=gray]{n116}{p116}
\pstGeonode[PointSymbol=Bo,fillcolor=lightgray](-0.14,-0.45){n117}
\pstProjection[PointSymbol=o,dotscale=1.25]{na}{nb}{n117}[p117]
\pstLineAB[linestyle=dotted,linecolor=gray]{n117}{p117}
\pstGeonode[PointSymbol=Bo,fillcolor=lightgray](-0.14,-0.15){n118}
\pstProjection[PointSymbol=o,dotscale=1.25]{na}{nb}{n118}[p118]
\pstLineAB[linestyle=dotted,linecolor=gray]{n118}{p118}
\pstGeonode[PointSymbol=Bo,fillcolor=lightgray](0.26,-0.25){n119}
\pstProjection[PointSymbol=o,dotscale=1.25]{na}{nb}{n119}[p119]
\pstLineAB[linestyle=dotted,linecolor=gray]{n119}{p119}
\pstGeonode[PointSymbol=Bo,fillcolor=lightgray](-0.24,-0.15){n120}
\pstProjection[PointSymbol=o,dotscale=1.25]{na}{nb}{n120}[p120]
\pstLineAB[linestyle=dotted,linecolor=gray]{n120}{p120}
\pstGeonode[PointSymbol=Bo,fillcolor=lightgray](-0.24,-0.05){n121}
\pstProjection[PointSymbol=o,dotscale=1.25]{na}{nb}{n121}[p121]
\pstLineAB[linestyle=dotted,linecolor=gray]{n121}{p121}
\pstGeonode[PointSymbol=Bo,fillcolor=lightgray](-0.44,-0.05){n122}
\pstProjection[PointSymbol=o,dotscale=1.25]{na}{nb}{n122}[p122]
\pstLineAB[linestyle=dotted,linecolor=gray]{n122}{p122}
\pstGeonode[PointSymbol=Bo,fillcolor=lightgray](0.26,-0.25){n123}
\pstProjection[PointSymbol=o,dotscale=1.25]{na}{nb}{n123}[p123]
\pstLineAB[linestyle=dotted,linecolor=gray]{n123}{p123}
\pstGeonode[PointSymbol=Bo,fillcolor=lightgray](0.16,-0.15){n124}
\pstProjection[PointSymbol=o,dotscale=1.25]{na}{nb}{n124}[p124]
\pstLineAB[linestyle=dotted,linecolor=gray]{n124}{p124}
\pstGeonode[PointSymbol=Bo,fillcolor=lightgray](-0.84,-0.75){n125}
\pstProjection[PointSymbol=o,dotscale=1.25]{na}{nb}{n125}[p125]
\pstLineAB[linestyle=dotted,linecolor=gray]{n125}{p125}
\pstGeonode[PointSymbol=Bo,fillcolor=lightgray](0.56,0.15){n126}
\pstProjection[PointSymbol=o,dotscale=1.25]{na}{nb}{n126}[p126]
\pstLineAB[linestyle=dotted,linecolor=gray]{n126}{p126}
\pstGeonode[PointSymbol=Bo,fillcolor=lightgray](0.86,-0.05){n127}
\pstProjection[PointSymbol=o,dotscale=1.25]{na}{nb}{n127}[p127]
\pstLineAB[linestyle=dotted,linecolor=gray]{n127}{p127}
\pstGeonode[PointSymbol=Bo,fillcolor=lightgray](-0.84,-1.05){n128}
\pstProjection[PointSymbol=o,dotscale=1.25]{na}{nb}{n128}[p128]
\pstLineAB[linestyle=dotted,linecolor=gray]{n128}{p128}
\pstGeonode[PointSymbol=Bo,fillcolor=lightgray](0.06,0.15){n129}
\pstProjection[PointSymbol=o,dotscale=1.25]{na}{nb}{n129}[p129]
\pstLineAB[linestyle=dotted,linecolor=gray]{n129}{p129}
\pstGeonode[PointSymbol=Bo,fillcolor=lightgray](0.36,-0.85){n130}
\pstProjection[PointSymbol=o,dotscale=1.25]{na}{nb}{n130}[p130]
\pstLineAB[linestyle=dotted,linecolor=gray]{n130}{p130}
\pstGeonode[PointSymbol=Bo,fillcolor=lightgray](-0.94,-0.65){n131}
\pstProjection[PointSymbol=o,dotscale=1.25]{na}{nb}{n131}[p131]
\pstLineAB[linestyle=dotted,linecolor=gray]{n131}{p131}
\pstGeonode[PointSymbol=Bo,fillcolor=lightgray](1.16,0.15){n132}
\pstProjection[PointSymbol=o,dotscale=1.25]{na}{nb}{n132}[p132]
\pstLineAB[linestyle=dotted,linecolor=gray]{n132}{p132}
\pstGeonode[PointSymbol=Bo,fillcolor=lightgray](-0.34,-0.65){n133}
\pstProjection[PointSymbol=o,dotscale=1.25]{na}{nb}{n133}[p133]
\pstLineAB[linestyle=dotted,linecolor=gray]{n133}{p133}
\pstGeonode[PointSymbol=Bo,fillcolor=lightgray](0.46,0.25){n134}
\pstProjection[PointSymbol=o,dotscale=1.25]{na}{nb}{n134}[p134]
\pstLineAB[linestyle=dotted,linecolor=gray]{n134}{p134}
\pstGeonode[PointSymbol=Bo,fillcolor=lightgray](0.96,-0.25){n135}
\pstProjection[PointSymbol=o,dotscale=1.25]{na}{nb}{n135}[p135]
\pstLineAB[linestyle=dotted,linecolor=gray]{n135}{p135}
\pstGeonode[PointSymbol=Bo,fillcolor=lightgray](0.26,-0.15){n136}
\pstProjection[PointSymbol=o,dotscale=1.25]{na}{nb}{n136}[p136]
\pstLineAB[linestyle=dotted,linecolor=gray]{n136}{p136}
\pstGeonode[PointSymbol=Bo,fillcolor=lightgray](0.86,0.05){n137}
\pstProjection[PointSymbol=o,dotscale=1.25]{na}{nb}{n137}[p137]
\pstLineAB[linestyle=dotted,linecolor=gray]{n137}{p137}
\pstGeonode[PointSymbol=Bo,fillcolor=lightgray](-0.64,-0.35){n138}
\pstProjection[PointSymbol=o,dotscale=1.25]{na}{nb}{n138}[p138]
\pstLineAB[linestyle=dotted,linecolor=gray]{n138}{p138}
\pstGeonode[PointSymbol=Bo,fillcolor=lightgray](0.56,-0.15){n139}
\pstProjection[PointSymbol=o,dotscale=1.25]{na}{nb}{n139}[p139]
\pstLineAB[linestyle=dotted,linecolor=gray]{n139}{p139}
\pstGeonode[PointSymbol=Bo,fillcolor=lightgray](-0.04,-0.35){n140}
\pstProjection[PointSymbol=o,dotscale=1.25]{na}{nb}{n140}[p140]
\pstLineAB[linestyle=dotted,linecolor=gray]{n140}{p140}
\pstGeonode[PointSymbol=Bo,fillcolor=lightgray](0.46,-0.55){n141}
\pstProjection[PointSymbol=o,dotscale=1.25]{na}{nb}{n141}[p141]
\pstLineAB[linestyle=dotted,linecolor=gray]{n141}{p141}
\pstGeonode[PointSymbol=Bo,fillcolor=lightgray](0.86,0.05){n142}
\pstProjection[PointSymbol=o,dotscale=1.25]{na}{nb}{n142}[p142]
\pstLineAB[linestyle=dotted,linecolor=gray]{n142}{p142}
\pstGeonode[PointSymbol=Bo,fillcolor=lightgray](0.16,-0.35){n143}
\pstProjection[PointSymbol=o,dotscale=1.25]{na}{nb}{n143}[p143]
\pstLineAB[linestyle=dotted,linecolor=gray]{n143}{p143}
\pstGeonode[PointSymbol=Bo,fillcolor=lightgray](-0.14,-0.25){n144}
\pstProjection[PointSymbol=o,dotscale=1.25]{na}{nb}{n144}[p144]
\pstLineAB[linestyle=dotted,linecolor=gray]{n144}{p144}
\pstGeonode[PointSymbol=Bo,fillcolor=lightgray](0.76,-0.05){n145}
\pstProjection[PointSymbol=o,dotscale=1.25]{na}{nb}{n145}[p145]
\pstLineAB[linestyle=dotted,linecolor=gray]{n145}{p145}
\pstGeonode[PointSymbol=Bo,fillcolor=lightgray](0.16,0.35){n146}
\pstProjection[PointSymbol=o,dotscale=1.25]{na}{nb}{n146}[p146]
\pstLineAB[linestyle=dotted,linecolor=gray]{n146}{p146}
\pstGeonode[PointSymbol=Bo,fillcolor=lightgray](-0.34,-0.65){n147}
\pstProjection[PointSymbol=o,dotscale=1.25]{na}{nb}{n147}[p147]
\pstLineAB[linestyle=dotted,linecolor=gray]{n147}{p147}
\pstGeonode[PointSymbol=Bo,fillcolor=lightgray](0.36,-0.15){n148}
\pstProjection[PointSymbol=o,dotscale=1.25]{na}{nb}{n148}[p148]
\pstLineAB[linestyle=dotted,linecolor=gray]{n148}{p148}
\pstGeonode[PointSymbol=Bo,fillcolor=lightgray](0.66,-0.25){n149}
\pstProjection[PointSymbol=o,dotscale=1.25]{na}{nb}{n149}[p149]
\pstLineAB[linestyle=dotted,linecolor=gray]{n149}{p149}
\pstGeonode[PointSymbol=Bo,fillcolor=lightgray](-0.34,-0.45){n150}
\pstProjection[PointSymbol=o,dotscale=1.25]{na}{nb}{n150}[p150]
\pstLineAB[linestyle=dotted,linecolor=gray]{n150}{p150}

    \rput(2.4,0){$X_1$}
    \rput(0,1.8){$X_2$}
    \nput{90}{ta}{$\ell$}
    \multido{\nl=-2.0+0.5}{9}{%
        \uput[d](\nl\space,-1.2){\small $\nl$}
        }
    \multido{\nl=-1.0+0.5}{6}{%
        \uput[l](-2,\nl\space){\small $\nl$}
        }
    \end{pspicture}
    }
\end{frame}


\begin{frame}{Data: Probabilistic View}

A {\em random variable} $X$ is a function
$X\!: \cO \to \setR$, where $\cO$
is the set of all possible outcomes of the experiment, also called the
{\em sample space}.

\smallskip A {\em discrete random variable}
takes on only a f\/{i}nite
or countably inf\/{i}nite number of values, 
whereas a {\em continuous
random variable}
if it can take on any value in $\setR$.

\medskip
By default, a numeric attribute $X_j$ 
is considered as the identity random variable given as
$$X(v) = v$$
for all $v \in \cO$. Here $\cO = \setR$.

\begin{block}{Discrete Variable: Long Sepal Length}
Define random variable $A$, denoting long sepal length (7cm or more) 
as follows:
    \begin{align*}
        A(v) =
        \begin{cases}
            0 & \text{if } v < 7\\
            1 & \text{if } v \ge 7
        \end{cases}
    \end{align*}
    The sample space of $A$ is $\cO = [4.3, 7.9]$, 
	and its range is $\{0,1\}$. Thus, $A$ is discrete.
\end{block}


\end{frame}

\begin{frame}{Probability Mass Function}
If $X$ is
discrete, the {\em probability mass function} of $X$ is
def\/{i}ned as
\begin{empheq}[box=\tcbhighmath]{align*}
    f(x) = P(X = x) \qquad \text{for all } x \in \setR
\end{empheq}
$f$ must obey the basic rules of probability.
That is, $f$ must be non-negative:
$$f(x) \ge 0$$
and the sum of all
probabilities should add to $1$:
$$\sum_{x} f(x) = 1$$
Intuitively, for a discrete variable $X$, 
the probability is
concentrated or massed at only discrete values in the range of $X$,
and is zero for all other values.
\end{frame}




\begin{frame}{Sepal Length: Bernoulli Distribution}
\begin{center}
Iris Dataset Extract: {\tt sepal length} (in centimeters)
{\fbox{
\tiny
\begin{tabular}{ccccccccccccccc}
    5.9 & 6.9 & 6.6 & 4.6 & 6.0 & 4.7 & 6.5 & 5.8 & 6.7 & 6.7 &     5.1 & 5.1 & 5.7 & 6.1 & 4.9\\
    5.0 & 5.0 & 5.7 & 5.0 & 7.2 & 5.9 & 6.5 & 5.7 & 5.5 & 4.9 &
    5.0 &
    5.5 & 4.6 & 7.2 & 6.8\\
    5.4 & 5.0 & 5.7 & 5.8 & 5.1 & 5.6 & 5.8 & 5.1 & 6.3 & 6.3 & 5.6 &
    6.1 & 6.8 & 7.3 & 5.6\\
    4.8 & 7.1 & 5.7 & 5.3 & 5.7 & 5.7 & 5.6 & 4.4 & 6.3 & 5.4 & 6.3 &
    6.9 & 7.7 & 6.1 & 5.6\\
    6.1 & 6.4 & 5.0 & 5.1 & 5.6 & 5.4 & 5.8 & 4.9 & 4.6 & 5.2 & 7.9 &
    7.7 & 6.1 & 5.5 & 4.6\\
    4.7 & 4.4 & 6.2 & 4.8 & 6.0 & 6.2 & 5.0 & 6.4 & 6.3 & 6.7 & 5.0 &
    5.9 & 6.7 & 5.4 & 6.3\\
    4.8 & 4.4 & 6.4 & 6.2 & 6.0 & 7.4 & 4.9 & 7.0 & 5.5 & 6.3 & 6.8 &
    6.1 & 6.5 & 6.7 & 6.7\\
    4.8 & 4.9 & 6.9 & 4.5 & 4.3 & 5.2 & 5.0 & 6.4 & 5.2 & 5.8 & 5.5 &
    7.6 & 6.3 & 6.4 & 6.3\\
    5.8 & 5.0 & 6.7 & 6.0 & 5.1 & 4.8 & 5.7 & 5.1 & 6.6 & 6.4 & 5.2 &
    6.4 & 7.7 & 5.8 & 4.9\\
    5.4 & 5.1 & 6.0 & 6.5 & 5.5 & 7.2 & 6.9 & 6.2 & 6.5 & 6.0 & 5.4 &
    5.5 & 6.7 & 7.7 & 5.1\\
\end{tabular}}}
\end{center}
\small
Define random variable $A$ as follows:
$A(v) =
        \begin{cases}
            0 & \text{if } v < 7\\
            1 & \text{if } v \ge 7
        \end{cases}$

\medskip
We f\/{i}nd that only 13 Irises have
    sepal length of at least 7 cm. Thus,
    the probability mass function of $A$ can be estimated as:
    $$f(1) = P(A=1) = {13 \over 150} = 0.087 = p$$
    and
    $$f(0) = P(A=0) = {137 \over 150} = 0.913 = 1-p$$

\medskip
$A$ has a {\em Bernoulli
    distribution} with parameter
    $p \in [0,1]$, which denotes the
    probability of a {\em success}, that is, the probability of
    picking an Iris with
    a long sepal length at random from the set of all points.
\end{frame}


\begin{frame}{Sepal Length: Binomial Distribution}
Define discrete random variable $B$, denoting the
    number of Irises with long sepal length in $m$ independent
    Bernoulli trials with probability of success $p$.
    In this case, $B$ takes on the discrete values $[0,m]$,
    and its probability mass
    function is given by the {\em Binomial
    distribution}
    \begin{align*}
        f(k) = P(B=k) = {m \choose k} p^k (1-p)^{m-k}
    \end{align*}
 
	Binomial distribution for long sepal length ($p=0.087$) for $m=10$
	trials

	\bigskip
    \centerline{
	\scalebox{0.5}{
    \psset{xAxisLabel=$k$,yAxisLabel=$P(B\text{=}k)$}%,xAxisLabelPos={c,-0.07}
\begin{psgraph}[dy=0.1,Dy=0.1]{->}(0,0)(-0.5,0)(11,0.45){4in}{2.5in}
\psBinomial[markZeros,fillstyle=solid,fillcolor=black,
    barwidth=0.075]{10}{0.087}
    \psset{radius=3pt}
    \Cnode*(0,0.4039179){n0}
    \Cnode*(1,0.3832797){n1}
    \Cnode*(2,0.1636632){n2}
    \Cnode*(3,0.04141357){n3}
    \Cnode*(4,0.006877071){n4}
    \Cnode*(5,0.0007830826){n5}
    \Cnode*(6,0){n6}
    \Cnode*(7,0){n7}
    \Cnode*(8,0){n8}
    \Cnode*(9,0){n9}
    \Cnode*(10,0){n10}
  \end{psgraph}
  }}
\end{frame}

\begin{frame}{Probability Density Function}
If $X$ is
continuous, the {\em probability density function} of $X$ is
def\/{i}ned as
\begin{empheq}[box=\tcbhighmath]{align*}
    P\bigl(X \in [a,b]\bigr) = \int_a^b f(x)\ dx
\end{empheq}
$f$ must obey the basic rules of probability.
That is, $f$ must be non-negative:
$$f(x) \ge 0$$
and the sum of all
probabilities should add to $1$:
$$\int_{-\infty}^\infty f(x)\ dx = 1$$

\bigskip Note that $P(X=v) = 0$ for all $v \in \setR$ since there are
infinite possible values in the sample space. What it means is
that the probability mass is spread so thinly over the range of
values that it can be measured only over intervals $[a,b] \subset
\setR$, rather than at specif\/{i}c points.
\end{frame}


\begin{frame}{Sepal Length: Normal Distribution}
  We model sepal length via the {\em Gaussian} or {\em normal} 
    density function, given as
    \begin{align*}
    f(x) = {1 \over \sqrt{2\pi\sigma^2}}
    \exp\left\{ {-(x-\mu)^2 \over
    2 \sigma^2}\right\}
    \end{align*}
	where $\mu = {1 \over n} \sum_{i=1}^n x_i$ is the mean value, and
	$\sigma^2 = {1 \over n} \sum_{i=1}^n (x_i-\mu)^2$ is the variance.


	\medskip
	Normal distribution for sepal length: $\mu=5.84$, $\sigma^2=0.681$

	\bigskip
  \centering
  \scalebox{0.6}{
    \psset{xAxisLabel=$x$,yAxisLabel=$f(x)$}
    \psgraph[Dy=0.1,Ox=2]{->}(2,0)(10,0.55){4in}{2.5in}%
    \psset{linewidth=2pt}
    %\mpsGauss{5.84}{0.825}{2.5}{9.5}
    \psset{plotpoints=200}
    \psplot[]{2.5}{9.5}{%
        Euler x 5.84 sub dup mul 2 div 0.825 dup mul div neg exp
        1.0 0.825 div TwoPi sqrt div mul%
    }%
    \psset{linewidth=1pt}
    \listplot[plotstyle=bar,barwidth=0.2cm,fillcolor=lightgray,fillstyle=solid]{5.84 0.48}
    \psline[](5.84,0)(5.84,0.48)
    \uput{0pt}[u](5.84,0.5){$\mu \pm \epsilon$}
    \endpsgraph
	}
\end{frame}


\begin{frame}{Cumulative Distribution Function}
  \small
  \begin{columns}
	\column{0.45\textwidth}
For random variable $X$, its 
{\em cumulative distribution function (CDF)}
$F:\setR \to [0,1]$,
gives the probability of observing a value at most
some given value $x$:
\begin{empheq}[box=\tcbhighmath]{align*}
    F(x) = P(X \le x) \qquad \text{for all } -\infty < x < \infty
\end{empheq}
When $X$ is discrete, $F$ is given as
\begin{align*}
    F(x) = P(X \le x) = \sum_{u \le x} f(u)
\end{align*}
When $X$ is continuous, $F$ is given as
\begin{align*}
    F(x) = P(X \le x) = \int_{-\infty}^x f(u)\ du
\end{align*}


\column{0.5\textwidth}
  CDF for binomial distribution ($p=0.087, m=10$)
    \begin{center}
	  \scalebox{0.45}{
    \psset{xAxisLabel=$x$,yAxisLabel=$F(x)$, radius=3pt}
    \psgraph[Dy=0.1,Ox=-1]{->}(-1,0)(12,1.1){4in}{2.25in}%
    \pnode(-1,0){n00}
    \Cnode[fillcolor=white,fillstyle=solid](0,0){n00b}
    \Cnode*(0,0.404){n0}
    \Cnode(1,0.404){n0b}
    \Cnode*(1,0.787){n1}
    \Cnode(2,0.787){n1b}
    \Cnode*(2,0.951){n2}
    \Cnode(3,0.951){n2b}
    \Cnode*(3,0.992){n3}
    \Cnode(4,0.992){n3b}
    \Cnode*(4,0.999){n4}
    \Cnode(5,0.999){n4b}
    \Cnode*(5,0.9999){n5}
    \Cnode(6,0.9999){n5b}
    \Cnode*(6,0.99999){n6}
    \Cnode(7,0.99999){n6b}
    \Cnode*(7,1){n7}
    \Cnode*(8,1){n8}
    \Cnode*(9,1){n9}
    \Cnode*(10,1){n10}
    \pnode(11,1){n11}
    \psset{linewidth=2pt}
    \ncline{<-}{n00}{n00b}
    \ncline[]{n0}{n0b}
    \ncline[]{n1}{n1b}
    \ncline[]{n2}{n2b}
    \ncline[]{n3}{n3b}
    \ncline[]{n4}{n4b}
    \ncline[]{n5}{n5b}
    \ncline[]{n6}{n6b}
    \ncline[]{n7}{n8}
    \ncline[]{n8}{n9}
    \ncline[]{n9}{n10}
    \ncline{->}{n10}{n11}
    \endpsgraph
	}
    \end{center}


CDF for the normal distribution ($\mu=5.84, \sigma^2=0.681$)
    \begin{center}
	  \scalebox{0.45}{
    \psset{xAxisLabel=$x$,yAxisLabel=$F(x)$, radius=3pt}
    \psgraph[Dy=0.1]{->}(0,0)(11,1.1){4in}{2.25in}%
    \psset{linewidth=2pt}
    \psCumIntegral[plotpoints=200,Simpson=10]{-0}{10}{5.84
    0.825 GAUSS}
    \psline{<-}(0,0)(1,0)
    \psline{->}(9,1)(10,1)
    \psline[linewidth=1pt,linestyle=dotted](0,0.5)(5.84,0.5)
    \psline[linewidth=1pt,linestyle=dotted](5.84,0)(5.84,0.5)
    \cnode*(5.84,0.5){3pt}{mu}
    \uput[r](5.84,0.5){$(\mu, F(\mu)) = (5.84,0.5)$}
    \endpsgraph
	}
    \end{center}
  \end{columns}
\end{frame}


\begin{frame}{Bivariate Random Variable: Joint Probability Mass
  Function}
  \small
  \begin{columns}
	\column{0.45\textwidth}
  Define discrete random variables 
  \begin{align*}
	\text{long sepal length:} X_1(v) = 
  \begin{cases}
	1 & \text{if} v \ge 7\\
	0 & \text{otherwise}
  \end{cases}\\ 
  \text{long sepal width:} X_2(v) = 
  \begin{cases}
	1 & \text{if} v \ge 3.5\\
	0 & \text{otherwise}
  \end{cases}
  \end{align*}
  
  \smallskip
  The bivariate random variable
  $$\bX = \matr{X_1\\ X_2}$$
  has the joint probability mass function
\begin{align*}
    f(\bx) & = P(\bX = \bx)\\
	\text{i.e., } f(x_1,x_2) & = P(X_1=x_1, X_2=x_2)
\end{align*}
  
	\column{0.5\textwidth}
    \centering
	\small
	Iris: joint PMF for long sepal length and sepal width
	    \begin{align*}
		  \tiny
        f\lB(0, 0\rB) & = P(X_1=0, X_2=0) = {116/150} = 0.773\\
        f\lB(0, 1\rB) & = P(X_1=0, X_2=1) = {21/150} = 0.140\\
        f\lB(1, 0\rB) & = P(X_1=1, X_2=0) = {10/150} = 0.067\\
        f\lB(1, 1\rB) & = P(X_1=1, X_2=1) = {3/150} = 0.020
    \end{align*}
	\scalebox{0.65}{
    \psset{unit=0.75in}
        \psset{viewpoint=50 40 20 rtp2xyz,Decran=50}
        \psset{lightsrc=viewpoint}
    \begin{pspicture}(-2,-1)(2,2.25)
        \axesIIID[axisnames={X_1,X_2,f(\bx)}](0,0,0)(2.5,2.5,2.2)
        \psSolid[object=grille, base=0 2 0 2, fillcolor=white,
        action=draw]
        \psPoint(0,0,2){x00}
        \psPoint(0,1,1){x01}
        \psPoint(1,0,0.5){x10}
        \psPoint(1,1,0.1){x11}
        \psPoint(-0.1,-0.1,0){O}
        \psPoint(-0.1,0.9,0){A}
        \psPoint(0.9,-0.1,0){B}
        \psdots[dotstyle=*,dotsize=0.1](x00)(x01)(x10)(x11)
        \uput[r](x00){0.773}
        \uput[u](x01){0.14}
        \uput[u](x10){0.067}
        \uput[u](x11){0.02}
        \uput[r](O){0}
        \uput[r](A){1}
        \uput[l](B){1}
        \psset{linewidth=2pt}
        \psSolid[object=line,args=0 0 0 0 0 2]
        \psSolid[object=line,args=0 1 0 0 1 1]
        \psSolid[object=line,args=1 0 0 1 0 0.5]
        \psSolid[object=line,args=1 1 0 1 1 0.1]
    \end{pspicture}
	}
  \end{columns}
\end{frame}


    \def\mye{2.7183}
    \def\mypi{3.1416}
    \def\msigma#1{(sqrt(#1))}%
    \def\rho#1#2#3{((#3)/(\msigma{#1}*\msigma{#2}))}%sigma1, sigma2, sigma12
    \def\ccA#1#2#3{((1.0)/(2*\mypi*\msigma{#1}*\msigma{#2}*sqrt(1-\rho{#1}{#2}{#3}^2)))} %sigma1, sigma2,rho
    \def\ccB#1#2#3{((-1.0)/(2*(1-(\rho{#1}{#2}{#3})^2)))}%rho
    \def\fA#1#2#3#4#5{(\ccA{#3}{#4}{#5})*\mye^(
    (\ccB{#3}{#4}{#5})*(((x-(#1))/(\msigma{#3}))^2 +
    ((y-(#2))/(\msigma{#4}))^2 -
    (2*(\rho{#3}{#4}{#5})*(x-(#1))*(y-(#2)))/(
    (\msigma{#3})*(\msigma{#4})) ))}%

\begin{frame}{Bivariate Random Variable: Probability Density Function}
  \small
  \begin{columns}
	\column{0.5\textwidth}
  Bivariate Normal: modeling joint distribution for 
  long sepal length ($X_1$) and sepal width ($X_2$)
    \begin{align*}
\tiny
      f(\bx | \bmu, \cov) = \frac{1}{2\pi\sqrt{|\cov|}}
      \;\;\exp\lB\{-\frac{(\bx-\bmu)^T\;\cov^{-1}\;(\bx-\bmu)}{2}\rB\}
    \end{align*}
where $\bmu$ and $\cov$ specify the 2D mean and covariance matrix:
\begin{align*}
\bmu & = (\mu_1, \mu_2)^T &
\cov & = \matr{ \sigma^2_{1} & \sigma_{12}\\
\sigma_{21} & \sigma^2_{2}}
\end{align*}
with mean $\mu_i = {1\over n} \sum_{k=1}^n x_{ki}$ 
and covariance $\sigma_{ij} = {1\over n} \sum_{k=1} (x_{ki} - \mu_i)
(x_{kj} - \mu_j)$. Also, $\sigma_i^2 = \sigma_{ii}$.

\column{0.45\textwidth}
\centering
Bivariate Normal
    $$\bmu = (5.843, 3.054)^T $$
    $$\cov = \amatr{r}{0.681 & -0.039\\ -0.039 & 0.187}$$
\scalebox{0.5}{
\hspace{-0.7in}
\psset{unit=0.35in}
\psset{viewpoint=50 40 20 rtp2xyz,Decran=50}
\psset{lightsrc=viewpoint,opacity=0.8,incolor=white}
\hspace*{-6pt}\begin{pspicture}(-9,-5)(7,3)
\axesIIID[axisnames={X_1,X_2,f(\bx)}](0,0,0)(11,7,2.5)
\multido{\ix=0+1}{10}{%
    \psPoint(\ix\space,6,0){X1}
    \psPoint(\ix\space,6.2,0){X2}
    \psline(X1)(X2)\uput[dr](X1){\ix}}
\multido{\iy=0+1}{6}{%
    \psPoint(10,\iy\space,0){Y1}
    \psPoint(10.2,\iy\space,0){Y2}
    \psline(Y1)(Y2)\uput[dl](Y1){\iy}}
\multido{\nz=0+1,\nl=0+0.2}{3}{%
    \psPoint(0,0,\nz\space){Z1}
    \psPoint(0,-0.2,\nz\space){Z2}
    \psline(Z1)(Z2)\uput[l](Z1){\nl}}
\psSurface[ngrid=.25 .25, linewidth=0.1pt,
transform={1 1 5 scaleOpoint3d},algebraic](0,0)(10,6){
  \fA{5.843}{3.054}{0.681}{0.187}{-0.039}
  }
  \psPoint(5.843,6,0){X1}
  \psPoint(10,3.054,0){Y1}
  \psPoint(5.843,3.054,0){O1}
  \psdot[dotsize=0.2](O1)
  \psline[linestyle=dashed](X1)(O1)
  \psline[linestyle=dashed](Y1)(O1)
\end{pspicture}
}
\end{columns}
\end{frame}


\begin{frame}{Random Sample and Statistics}
  Given a random variable $X$, a {\em random sample} of size $n$
from $X$ is def\/{i}ned as a set of $n$ {\em independent and
identically distributed (IID)}
random variables 
$$S_1, S_2, \ldots, S_n$$
The $S_i$'s have the same probability distribution as $X$, and are
statistically independent.

Two random variables $X_1$ and $X_2$ are
(statistically) {\em independent}
if, for every
$W_1 \subset \setR$ and $W_2 \subset \setR$, we have
\begin{align*}
    P(X_1 \in W_1 \text{ and } X_2 \in W_2) = P(X_1 \in W_1) \cdot
    P(X_2 \in W_2)
\end{align*}
which also implies that 
\begin{align*}
        F(\bx) & = F(x_1, x_2) = F_{1}(x_1) \cdot F_{2}(x_2)\\
        f(\bx) & = f(x_1, x_2) = f_{1}(x_1) \cdot f_{2}(x_2)
\end{align*}
where $F_{i}$ is the cumulative distribution function,
and $f_{i}$ is the probability mass or density function for random
variable $X_i$.
\end{frame}


\begin{frame}{Multivariate Sample}
Given dataset $\bD$, the $n$ data
points $\bx_i$ (with $1 \le i \le n$)
constitute a $d$-dimensional {\em multivariate random sample} drawn
from the vector random variable $\bX = (X_1, X_2, \ldots, X_d)$.

\smallskip
Since the $\bx_i$ are assumed to be
independent and
identically distributed, their joint distribution is given as
\begin{empheq}[box=\tcbhighmath]{align*}
f(\bx_1, \bx_2, \ldots, \bx_n) = \prod_{i=1}^n f_\bX(\bx_i)
\end{empheq}
where $f_{\bX}$ is the probability mass or density function for $\bX$.

\smallskip
Assuming that the $d$
attributes $X_1, X_2, \ldots, X_d$ are statistically independent, the 
joint distribution for the entire dataset is given as:
\begin{align*}
f(\bx_1, \bx_2, \ldots, \bx_n) =
\prod_{i=1}^n f(\bx_i) = \prod_{i=1}^n \prod_{j=1}^d
f_{X_{j}}(x_{ij})
\end{align*}

\end{frame}


\begin{frame}{Sample Statistics}
  \small
Let
$\{\bS_i\}_{i=1}^m$
be a random sample of size $m$ drawn from a (multivariate) random
variable $\bX$. A {\em statistic} $\htheta$ is a function 
$$\htheta\!: (\bS_1, \bS_2, \ldots, \bS_m) \to \setR$$

The statistic is an estimate of the corresponding population parameter
$\theta$, where the {\em population} refers to the entire universe of
entities under study. The statistic is itself a random variable.

\bigskip
The {\em sample mean} is a statistic, def\/{i}ned as the average
    $$\hmu = {1 \over n} \sum_{i=1}^n x_i$$ 
	For {\tt sepal length}, we have 
	$\hmu = 5.84$,
	which is an estimator for the (unknown) true population mean sepal
	length.
\end{frame}

\begin{frame}{Sample Statistics: Variance}

The {\em sample variance} is a statistic
	$$\hsigma^2 = {1 \over n} \sum_{i=1}^n (x_i-\mu)^2$$
For sepal length, we have $\hsigma^2 =0.681$.

\medskip
The {\em total variance} is a multivariate statistic
$$var(\bD) = {1 \over n} \sum_{i=1}^n \norm{\bx_i-\bmu}^2$$
For the Iris data (with 4 attributes: sepal length and width, petal
length and width), we have $var(\bD) =0.868$.

\end{frame}
